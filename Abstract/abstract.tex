%%=========
%% Abstract
%%=========
\documentclass[../dissertation.tex]{subfiles}
\begin{document}

\abstract{
In the early 1980's, Kodama, Ablowitz and Satsuma, together with Santini, Ablowitz and Fokas, 
developed the formal inverse scattering theory of the 
Intermediate Long Wave (ILW) equation and explored its connections with 
the Benjamin-Ono (BO) and KdV equations. 
The ILW equation
\begin{align*}
	u_t + \frac{1}{\delta} u_x + 2 u u_x + Tu_{xx} = 0, 
\end{align*}
models the behavior of long internal gravitational waves in stratified fluids of 
depth
$0< \delta < \infty$, where $T$ is a singular operator which dependes 
on the depth $\delta$.
In the limit 
$\delta \to 0$, the ILW reduces to the Korteweg de Vries (KdV) equation, and in the 
limit $\delta \to \infty$, the ILW (at least formally) reduces to the Benjamin-Ono (BO) 
equation. 

While the KdV
equation is very well understood, a rigorous analysis of inverse scattering for the 
ILW equation remains to be accomplished. There
is currently no rigorous proof that the Inverse Scattering Transform
outlined by Kodama \textit{et al.} solves the ILW, even for small data.
In this dissertation, we seek to help ameliorate this gap in knowledge
by presenting a mathematically rigorous construction
of the Direct Map for the ILW's Inverse Scattering Transform (IST) and using 
our analyses of this Direct Map to determine the 
Riemann-Hilbert Problem that can be used to construct the Inverse Map
for the ILW's IST.
}


\end{document}