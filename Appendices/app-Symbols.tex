%%============================
%% Appendix 4: List of Symbols
%%============================

\documentclass[../dissertation.tex]{subfiles}

\begin{document}

\newcommand{\indexsection}[1]{\noindent\textbf{#1}}

\newcolumntype{R}[1]{>{\raggedleft\let\newline\\\arraybackslash\hspace{0pt}}p{#1}}
\newcolumntype{L}[1]{>{\raggedright\let\newline\\\arraybackslash\hspace{0pt}}p{#1}}
\newcolumntype{C}[1]{>{\centering\let\newline\\\arraybackslash\hspace{0pt}}p{#1}}

\newenvironment{talign}
	{
		\newline
		\begin{center}
			$
				\begin{aligned}
	}
	{
				\end{aligned}
			$
		\end{center}
		% \vspace{-1\baselineskip}
	}


\indexsection{Fundamental Notation}

\begin{centering}
	\begin{longtable}
		{R{.115\textwidth}|L{.74\textwidth}|C{.145\textwidth}}
		\textbf{Symbol} & \textbf{Meaning} & \textbf{Reference} \\
		\hline                                      
		\endfirsthead
		\multicolumn{3}{c}%
		{Fundamental Notation -- \textit{Continued from previous page}} \\
		\hline
		\textbf{Symbol} & \textbf{Meaning} & \textbf{Reference} \\
		\hline
		\endhead
		\hline \multicolumn{3}{r}{\textit{Continued on next page}} \\
		\endfoot
		\hline
		\endlastfoot
		$:=$ & defined to be; for example $a := b$ means ``$a$ is defined to be 
			$b$''
			& p.\pageref{sym0:def} \\
		$\pv$ & Cauchy principle value; given by 
			{
				\begin{align*}
					\pv\int_{\mathbb R} f(x) \, \mathrm{d}x
						:= \lim_{\varepsilon \searrow 0} \int_{|x|>\varepsilon} f(x) \, \mathrm{d}x
				\end{align*}
			}
			& p.\pageref{sym0:pv} \\
		\textit{a.e.} & abbreviation for almost everywhere & p.\pageref{sym:ae}\\
		$\mathbb R$ & the set of all real numbers & p.\pageref{sym:Reals} \\
		$\mathbb C$ & the set of all complex numbers & p.\pageref{sym:Complex} \\
		$\mathbb Z$ & the set of all integers &  \\
		$m(\dotarg)$ & Lebesgue measure on $\mathbb R$ & p.\pageref{sym:lebesguemeasure} \\
		$\mathcal F$, $\hat{(\dotarg)}$ & Fourier transform defined as 
			\begin{talign}
				\hat f(\xi)
						:= \(\mathcal F f\)(\xi)
						= \int_{\mathbb R} e^{-i x \xi} f(x) \, \mathrm{d}x
			\end{talign}
				& p.\pageref{sym:fourier} \\[-1\baselineskip]
		$\mathcal F^{-1}$, $\check{(\dotarg)}$ & inverse Fourier transform defined 
			as
			\begin{talign}
					\check g(x)
						:= \(\mathcal F^{-1}g\)(x)
						= \frac{1}{2 \pi} \int_{\mathbb R} e^{i x \xi} g(\xi) \, \mathrm{d}\xi
			\end{talign}
			& p.\pageref{sym:fourier} \\
		$(\dotarg\pm i0)$ & implied limit of $(\dotarg \pm i \varepsilon)$ as 
				$\varepsilon \searrow 0$ 
			& p.\pageref{sym:i0} \\
		$f^+$ & lower boundary $f^+(x) := 
				\lim_{y\searrow0} f(x+ i y)$ of a function $f$ analytic on $S_\delta$,
				where $x, y \in \mathbb R$ 
			& p.\pageref{sym:bndries} \\
		$f^-$ & upper boundary $f^-(x) := \lim_{y\nearrow0} f(x+ i 2y)$ of a 
				function $f$ analytic on $S_\delta$, where $x, y \in \mathbb R$
			& p.\pageref{sym:bndries} \\
		$\lesssim$ & $q \lesssim s$ means there exists some fixed constant $C$ so 
				that $q \leq C\,s$; the constant $C$ is commonly referred to as 
				``the implied constant'' & p.\pageref{sym:lesssim} \\
		$\lesssim_k$ & $q \lesssim s$ means there exists some constant
				$C := C(k)$ depending only on the parameter $k$ so that
				$q \leq C \, s$; the constant $C$ is commonly referred as ``the 
				implied constant''  
			& p.\pageref{sym:lesssimdep} \\
		$\log_+ t$ & the function given by $\max\big\{ 0, \, \log(t) \big\}$
			& p.\pageref{sym:logplus} \\
		$\inn{x}$ & short-hand notation for $\big(1 + |x|^2\big)^{1/2}$
			& p.\pageref{sym:xbracket} \\
		$L^{p,s}(\mathbb R)$ & space of measurable functions with 
			\[
				\|f\|_{L^{p,s}}
					:= \left(\int_{\mathbb R} \inn{x}^{sp} |f(x)|^p \right)^{1/p}
					< \infty
			\]
			& p.\pageref{defn2:Lps} \\
		$\langle\dotarg\rangle L^\infty(\mathbb R)$ & space of measurable functions with 
			\begin{talign}
				\|f\|_{\inn{\dotarg} L^\infty}
					:= \esssup_{x\in \mathbb R} \left| \inn{x}^{-1} f(x)  \right|
			\end{talign}
			finite
			& p.\pageref{defn2:wLp} \\
		$B_Y(y_0, r)$ & the open ball $\{ y \in Y ~:~ \|y - y_0\|_Y < r  \}$ in 
			the metric space $Y$ with radius $r$ centered at $y_0 \in Y$
			& p.\pageref{sym:ball} \\
		$Y\to Z$ & a map from a space $Y$ to a space $Z$
			& p.\pageref{sym:mapsto} \\
		$Y \toitself$ & a map from a space $Y$ into itself
			& p.\pageref{sym:toitself} \\
		$\|\dotarg\|_{Y\to Z}$ & the induced operator norm for an operator with domain 
			$Y$ and co-domain $Z$
			& p.\pageref{sym:opnorm} \\
		$\chi_A$ & characteristic function on a set $A$; given by 
			{\begin{align*}
				\chi_A(x) := 
					\begin{cases}
						1, & x \in A \\
						0, & \text{otherwise}
					\end{cases}
			\end{align*}}
			& p.\pageref{sym:chi} \\
	\end{longtable}
\end{centering}


\newpage

\indexsection{Chapter 0 Notation}

\begin{centering}
	\begin{longtable}{R{.115\textwidth}|L{.74\textwidth}|C{.145\textwidth}}
	\textbf{Symbol} & \textbf{Meaning} & \textbf{Reference} \\
	\hline                                      
	\endfirsthead
	\multicolumn{3}{c}%
	{Chapter 0 Notation -- \textit{Continued from previous page}} \\
	\hline
	\textbf{Symbol} & \textbf{Meaning} & \textbf{Reference} \\
	\hline
	\endhead
	\hline \multicolumn{3}{r}{\textit{Continued on next page}} \\
	\endfoot
	\hline
	\endlastfoot
	
	
	$\delta$ & depth of stratified fluids\textemdash{}typically taken 
			to be $\delta=1$ 
		& p.\pageref{sym:delta} \\
		
	$\mathcal S_\delta$ & the complex strip
			$\{z \in \mathbb C ~:~ 0 < \im z < 2\delta\}$ 
		& p.\pageref{sym:Sdelta} \\
	$f^+$ & lower boundary $f^+(x) := 
			\lim_{y\searrow0} f(x+ i y)$ of a function $f$ analytic on $S_\delta$,
			where $x, y \in \mathbb R$ 
		& p.\pageref{sym:bndries} \\
	$f^-$ & upper boundary $f^-(x) := \lim_{y\nearrow0} f(x+ i 2y)$ of a 
			function $f$ analytic on $S_\delta$, where $x, y \in \mathbb R$
		& p.\pageref{sym:bndries} \\
	$L_\delta$ & operator on functions analytic in the complex strip 
		$\mathcal S_\delta$; given by 
		{
			\begin{align*}
				L_\delta (\Psi) 
					:= \frac{1}{i} \frac{\partial}{\partial x} \Psi^+ 
					- \zeta \left(\Psi^+ - \Psi^-\right) = u \Psi^+
			\end{align*}
		}
		& p.\pageref{eq0:SpecProb} \\
	$\lambda$ & a spectral parameter for the linear spectral problem 
			\eqref{eq0:SpecProb} 
		& p.\pageref{sym:zeta} \\
	$\zeta$ & a spectral parameter for \eqref{eq0:SpecProb} commonly
			parameterized by $\lambda$	as 
			$\displaystyle \zeta(\lambda; \delta) 
				= \frac{\lambda}{1-e^{-2\delta\lambda}}$ 
		& p.\pageref{sym:zeta} \\
	$\lambda(\zeta)$ &  inverse of the map $\lambda \to \zeta(\lambda)$ 
		& p.\pageref{sym:lambda} \\
	$\inn{x}$ & short-hand notation for $\big(1 + |x|^2\big)^{1/2}$
		& p.\pageref{sym:xbracket} \\
	$B_Y(y_0, r)$ & the open ball $\{ y \in Y ~:~ \|y - y_0\|_Y < r  \}$ in 
		the metric space $Y$ with radius $r$ centered at $y_0 \in Y$
		& p.\pageref{sym:ball} \\	
	$M_1$, $M_e$, $N_1$, $N_e$ & depending on context, either Jost solutions 
		or analytic extensions of solutions to the integral equations
		\eqref{eq0:JostIE}
		& p.\pageref{defn0:jost}, p.\pageref{eq0:JostIE} \\
	$M_1^+$, $M_e^+$, $N_1^+$, $N_e^+$ & depending on context, either solutions
		to the integral equations \eqref{eq0:JostIE} or lower boundary 
		values of the Jost solutions
		& p.\pageref{eq0:JostIE}, p.\pageref{defn0:jost} \\
	$r(\lambda; \delta)$ & reflection coefficent; given by
		{
			\begin{align*}
				r(\lambda; \delta) = \frac{b(\lambda; \delta)}{a(\lambda; \delta)},
			\end{align*}
			where
			\begin{align*}
				b(\lambda)
				&:= 
					\frac{i}{1-2\delta\zeta(-\lambda)} 
					\int_{\mathbb R} e^{-i\lambda x} 
						u(x) \, M_1^+(x; \lambda,\delta) 
					\, \mathrm{d}x
					\\
			a(\lambda)
				&:=
					1 
					+ \frac{i}{1-2\delta \zeta(\lambda)}
						\int_{\mathbb R} 
							u(x) \, M_1^+(x; \lambda,\delta) 
						\, \mathrm{d}x
			\end{align*}
		} 
		& p.\pageref{sym0:reflection} \\
	$\mathscr D$ & the direct scattering map for the Intermediate Long Wave (ILW) equation; maps
		ILW initial data $u$ to the corresponding reflection coefficient $r$
		& p.\pageref{sym0:DSM} \\
	$G_L$, $G_R$ & formal Green's functions corresponding to the linear spectral problem 
		\eqref{eq0:SpecProb}
		& p.\pageref{eq0:GFL},  p.\pageref{eq0:GFR}\\
	$\alpha(\lambda; \delta)$ & residue of $e^{iz\xi}/p(\xi)$ at the $\xi=0$
		pole; given by 
		{\begin{align*}
			\alpha(\lambda; \delta)
				= \frac{1}{1-2\delta\zeta(\lambda; \delta)}
		\end{align*}}
		& p.\pageref{sym0:residues} \\
	$\beta(\lambda; \delta)$ & $e^{iz\lambda}$ times the residue of 
		$e^{iz\xi}/p(\xi)$ at the $\xi=\lambda$ pole; given by 
		{\begin{align*}
			\alpha(\lambda; \delta)
				= \frac{1}{1-2\delta\zeta(-\lambda; \delta)}
		\end{align*}}
		& p.\pageref{sym0:residues} \\
	$K^+$ & non-residue term resulting from shifting the integration contours of 
		$G_L^+$ and $G_R^+$
		& p.\pageref{sym0:K} \\
	$T_{\star, \lambda, u}$ & bounded operators on 
		$\inn{\dotarg}L^\infty(\mathbb R)$ given by
		{
		\begin{align*}
			T_{\star, \lambda, u} f(x) 
				:= \big[ G_\star^+(\dotarg; \lambda) \big] * (u\,f)(x),
		\end{align*}
		where $\star = L$ or $R$
		}
		& p.\pageref{eq0:Tstar} \\
	\end{longtable}
\end{centering}


\newpage

\indexsection{Chapter 1 Notation}



\begin{centering}
	\begin{longtable}{R{.115\textwidth}|L{.74\textwidth}|C{.145\textwidth}}
		\textbf{Symbol} & \textbf{Meaning} & \textbf{Reference} \\
		\hline                                      
		\endfirsthead
		\multicolumn{3}{c}%
		{Chapter 1 Notation -- \textit{Continued from previous page}} \\
		\hline
		\textbf{Symbol} & \textbf{Meaning} & \textbf{Reference} \\
		\hline
		\endhead
		\hline \multicolumn{3}{r}{\textit{Continued on next page}} \\
		\endfoot
		\hline
		\endlastfoot
		$\mathscr D$ & the direct scattering map for the Intermediate Long Wave (ILW) equation; maps
			ILW initial data $u$ to the corresponding reflection coefficient $r$
			& p.\pageref{sym0:DSM} \\
		$\star$ & used as a placeholder for both $L$ and $R$; for example, if a 
			statement contains the notation 
			``$G_\star$ ($\star = L \text{, or } R$),'' then it is equally 
			true (or not true) for both $G_L$ and $G_R$
			& p.\pageref{rmk1:StarNotation} \\
		$f^+$ & lower boundary $f^+(x) := 
				\lim_{y\searrow0} f(x+ i y)$ of a function $f$ analytic on $S_\delta$,
				where $x, y \in \mathbb R$ 
			& p.\pageref{sym:bndries} \\
		$f^-$ & upper boundary $f^-(x) := \lim_{y\nearrow0} f(x+ i 2y)$ of a 
				function $f$ analytic on $S_\delta$, where $x, y \in \mathbb R$
			& p.\pageref{sym:bndries} \\
		$\delta$ & depth of stratified fluids\textemdash{}typically taken 
				to be $\delta=1$ 
			& p.\pageref{sym:delta} \\
		$\lambda$ & a spectral parameter for the linear spectral problem 
				\eqref{eq0:SpecProb} 
			& p.\pageref{sym:zeta} \\
		$\zeta$ & a spectral parameter for \eqref{eq0:SpecProb} commonly
				parameterized by $\lambda$	as 
				$\displaystyle \zeta(\lambda; \delta) 
					= \frac{\lambda}{1-e^{-2\delta\lambda}}$ 
			& p.\pageref{sym:zeta} \\
		$\lambda(\zeta)$ &  inverse of the map $\lambda \to \zeta(\lambda)$ 
			& p.\pageref{sym:lambda} \\
		$\zeta^*$ & nonlinear reflection $\zeta\big(-\lambda(\zeta)\big)$ 
			& p.\pageref{sym:zetastar} \\
		$G_\star^+$ & lower boundary value of the Greens' function whose contour
			of integration is $\Gamma_\star$; given by 
				{
				\begin{align*}
						G_\star^+(x; \lambda, \delta)
							:=
								\frac{1}{2\pi} 
								\int_{\Gamma_\star}
									e^{i x \xi} \,
									\frac{1}{p(\xi; \lambda, \delta)}
								\, \mathrm{d}\xi
					\end{align*}
				}
			& p.\pageref{sym:GFbndry} \\
		${\Gamma_L}$ & contour along the real line which bypasses the roots of $p$
				from below
			& p.\pageref{sym:Gamma} \\
		${\Gamma_R}$ & contour along the real line which bypasses the roots of $p$
				from above
			& p.\pageref{sym:Gamma} \\
		$p$ & Fourier symbol of the Green's functions; 
				given by
				\[
					p(\xi; \lambda, \delta)
						= \xi - \zeta(\lambda) \( 1- e^{-2\delta \xi} \)
				\]
				and commonly denoted as $p(\xi; \lambda, \delta)$, 
				$p(\xi; \zeta, \delta)$, $p(\xi; \lambda)$, 
				$p(\xi; \zeta)$, and $p(\xi)$.
			& p.\pageref{sym:GFintegrand} \\
		$\alpha(\lambda; \delta)$ & residue of $e^{iz\xi}/p(\xi)$ at the $\xi=0$
			pole; given by 
			{\begin{align*}
				\alpha(\lambda; \delta)
					= \frac{1}{1-2\delta\zeta(\lambda; \delta)}
			\end{align*}}
			& p.\pageref{sym:alphabeta} \\
		$\beta(\lambda; \delta)$ & $e^{iz\lambda}$ times the residue of 
			$e^{iz\xi}/p(\xi)$ at the $\xi=\lambda$ pole; given by 
			{\begin{align*}
				\alpha(\lambda; \delta)
					= \frac{1}{1-2\delta\zeta(-\lambda; \delta)}
			\end{align*}}
			& p.\pageref{sym:alphabeta} \\
		$K^+$ & non-residue term resulting from shifting the integration contours of 
			$G_L^+$ and $G_R^+$
			& p.\pageref{sym1:K} \\
		$\log_+ t$ & the function given by $\max\big\{ 0, \, \log(t) \big\}$
			& p.\pageref{sym:logplus} \\
		$\mathcal R_\delta$ & the complex strip about the real axis given by
				{
					\begin{align*}
						:= \{z \in \mathbb C ~:~ -\pi/\delta \leq \im z \leq \pi/\delta \}	
					\end{align*}
				}
			& p.\pageref{sym1:Rcal} \\
		$\mathpzc R_\star$ & sum residues of $e^{iz\xi}/p(\xi)$ at the $\xi=0$ and $\xi=\lambda$ 
				poles 
			& p.\pageref{sym1:ressum} \\
		$\delta_c$ & Dirac delta-function centered at $x=c$ 
			& p.\pageref{sym:dirac} \\
		$W_k$ & $k^{th}$ branch ($k \in \mathbb Z$) of the complex Lambert $W$ function
			& p.\pageref{sym1:Wk} \\
		$\Sigma_{c}$ & the integration contour
				$\mathbb R + i c \, \pi$
			& p.\pageref{sym1:SigRealLine} \\
		$\Sigma(R,~c)$ & the integration contour $(-R, R) + i c \,\pi$,
				where $R > 0$ and $(-R, R):= \{x \in \mathbb R ~:~ -R < x < R\}$
			& p.\pageref{sym1:SigR} \\
		$K_\zeta$ & the integral function given by 
				{
					\begin{align*}
						K_\zeta(x) 
							:= \int_{\mathbb R} 
								\frac{e^{ix \xi}}{\xi - \zeta\left(1-e^{-2\xi}\right)+i\pi}
							\, \mathrm{d}\xi
					\end{align*}
				}
			& p.\pageref{sym1:Kzeta} \\
		$K_q$ & the integral function given by 
				{
					\begin{align*}
						K_q(x) 
							:= \int_{\mathbb R} 
								\frac{e^{ix\xi} \chi\left( 2^{-q} x \xi\right)}
									{\xi - \zeta\left(1-e^{-2\xi}\right)+i\pi}
							\, \mathrm{d}\xi
					\end{align*}
				}
			& p.\pageref{sym1:Kq} \\
	\end{longtable}
\end{centering}



\newpage

\indexsection{Chapter 2 Notation}
\begin{centering}
	\begin{longtable}{R{.115\textwidth}|L{.74\textwidth}|C{.145\textwidth}}
		\textbf{Symbol} & \textbf{Meaning} & \textbf{Reference} \\
		\hline                                      
		\endfirsthead
		\multicolumn{3}{c}%
		{Chapter 2 Notation -- \textit{Continued from previous page}} \\
		\hline
		\textbf{Symbol} & \textbf{Meaning} & \textbf{Reference} \\
		\hline
		\endhead
		\hline \multicolumn{3}{r}{\textit{Continued on next page}} \\
		\endfoot
		\hline
		\endlastfoot
		$\inn{x}$ & short-hand notation for $\big(1 + |x|^2\big)^{1/2}$
			& p.\pageref{sym2:xbracket} \\
		$L^{p,s}(\mathbb R)$ & space of measurable functions with 
				\[
					\|f\|_{L^{p,s}}
						:= \left(\int_{\mathbb R} \inn{x}^{sp} |f(x)|^p \right)^{1/p}
						< \infty
				\]
			& p.\pageref{defn2:Lps} \\
		$\langle\dotarg\rangle L^\infty(\mathbb R)$ & space of measurable functions with 
				\begin{talign}
					\|f\|_{\inn{\dotarg} L^\infty}
						:= \esssup_{x\in \mathbb R} \left| \inn{x}^{-1} f(x)  \right|
				\end{talign}
				finite
			& p.\pageref{defn2:wLp} \\
		$L_\xi^p(\mathbb R)$ & space of measurable functions which are $L^p$ integrable
				with respect to the variable $\xi$; similar subscript notation
				is used for other function spaces
			& p.\pageref{sym2:Lpxi} \\
		$\star$ & used as a placeholder for both $L$ and $R$; for example, if a 
				statement contains the notation 
				``$G_\star$ ($\star = L \text{, or } R$),'' then it is equally 
				true (or not true) for both $G_L$ and $G_R$
			& p.\pageref{rmk1:StarNotation} \\
		$T_{\star, \lambda, u}$ & bounded operator on 
				$\inn{\dotarg}L^\infty(\mathbb R)$ given by
				{
				\begin{align*}
					T_{\star, \lambda, u} f(x) 
						:= \big[ G_\star^+(\dotarg; \lambda) \big] * (u\,f)(x)
				\end{align*}
				}
				based on context, $T_{\star, \lambda, u}$ is sometimes denoted 
				by $T_\star$, $T_{\star, \lambda}$, or $T_\lambda$
			& p.\pageref{eqn2:Tdefn} \\
		$X$ & space of potentials $u$ with $\nm{\inn{\dotarg}^{4} u}_{L^2} < \infty$
			& p.\pageref{defn2:X} \\
		$\lesssim_k$ & $q \lesssim s$ means there exists some constant
				$C := C(k)$ depending only on the parameter $k$ so that
				$q \leq C \, s$; the constant $C$ is commonly referred as ``the 
				implied constant''  
			& p.\pageref{ILW.GL.bdp} \\
		$\chi_\pm$ & the characteristic functions $\chi_-:= \chi_{(-\infty, 0)}$, 
				$\chi_+:= \chi_{(0, \infty)}$ on the respective intervals
				$(-\infty, 0)$ and $(0, \infty)$
			& p.\pageref{sym2:chipm} \\
		$G$ & as specified in Remark \ref{rmk2:notation}, $G(x,\lambda)$ 
				and $G(\lambda)$ are occasionally used to as shorthand 
				notations for $G_\star^+(x; \lambda)$
			& p.\pageref{rmk2:notation} \\
		$G_h(\lambda)$ & the difference quotient of $G_\star^+$ with respect to 
				$\lambda$; given by 
				\begin{align*}
					G_h(\lambda) := \frac{G(\lambda+h) - G(\lambda)}{h}
				\end{align*}
			& p.\pageref{sym2:Gh} \\
		$\ds\left(\frac{1}{p_\lambda(\xi)}\right)_{h}$
			& the difference quotient of $1/p$ with respect to $\lambda$;
				\begin{align*}
					\left(\frac{1}{p_\lambda(\xi)}\right)_{h}
						:= \frac{1}{h}\,
							\left[
								\frac{1}{p(\xi; \lambda +h)}
								-\frac{1}{p(\xi; \lambda)}
							\right]
				\end{align*}
			& p.\pageref{sym2:pdiffquot} \\
		$\Res_{z=c}f$ & complex residue of a function $f$ at the pole $z = c$
			& p.\pageref{eq2:gzerores}
	\end{longtable}
\end{centering}



\indexsection{Chapter 3 Notation}
\begin{centering}
\begin{longtable}{R{.115\textwidth}|L{.74\textwidth}|C{.145\textwidth}}
	\textbf{Symbol} & \textbf{Meaning} & \textbf{Reference} \\
	\hline                                      
	\endfirsthead
	\multicolumn{3}{c}%
	{Chapter 3 Notation -- \textit{Continued from previous page}} \\
	\hline
	\textbf{Symbol} & \textbf{Meaning} & \textbf{Reference} \\
	\hline
	\endhead
	\hline \multicolumn{3}{r}{\textit{Continued on next page}} \\
	\endfoot
	\hline
	\endlastfoot

\end{longtable}
\end{centering}


\indexsection{Chapter 4 Notation}
\begin{centering}
\begin{longtable}{R{.115\textwidth}|L{.74\textwidth}|C{.145\textwidth}}
	\textbf{Symbol} & \textbf{Meaning} & \textbf{Reference} \\
	\hline                                      
	\endfirsthead
	\multicolumn{3}{c}%
	{Chapter 4 Notation -- \textit{Continued from previous page}} \\
	\hline
	\textbf{Symbol} & \textbf{Meaning} & \textbf{Reference} \\
	\hline
	\endhead
	\hline \multicolumn{3}{r}{\textit{Continued on next page}} \\
	\endfoot
	\hline
	\endlastfoot

\end{longtable}
\end{centering}


% \begin{centering}
% \begin{longtable}{R{.115\textwidth}|L{.74\textwidth}|C{.145\textwidth}}
% 	\textbf{Symbol} & \textbf{Meaning} & \textbf{Reference} \\
% 	\hline                                      
% 	\endfirsthead
% 	\multicolumn{3}{c}%
% 	{Notation -- \textit{Continued from previous page}} \\
% 	\hline
% 	\textbf{Symbol} & \textbf{Meaning} & \textbf{Reference} \\
% 	\hline
% 	\endhead
% 	\hline \multicolumn{3}{r}{\textit{Continued on next page}} \\
% 	\endfoot
% 	\hline
% 	\endlastfoot
% 	% Notation & Meaning & First Use \\
% 	% \hline
% 	$:=$ & defined to be; for example $a := b$ means ``$a$ is defined to be 
% 			$b$
% 		& p.\pageref{sym:def} \\
% 	$\delta$ & depth of stratified fluids\textemdash{}typically taken 
% 			to be $\delta=1$ 
% 		& p.\pageref{sym:delta} \\
% 	$\pv$ & Cauchy principle value; given by 
% 		{
% 			\begin{align*}
% 				\pv\int_{\mathbb R} f(x) \, \mathrm{d}x
% 					:= \lim_{\varepsilon \searrow 0} \int_{|x|>\varepsilon} f(x) \, \mathrm{d}x
% 			\end{align*}
% 		}
% 		& p.\pageref{sym:pv} \\
% 	\textit{a.e.} & abbreviation for almost everywhere & p.\pageref{sym:ae}\\
% 	$\mathbb R$ & the set of all real numbers & p.\pageref{sym:Reals} \\
% 	$\mathbb C$ & the set of all complex numbers & p.\pageref{sym:Complex} \\
% 	$m(\dotarg)$ & Lebesgue measure on $\mathbb R$ & p.\pageref{sym:lebesguemeasure} \\
% 	$\mathcal F$, $\hat{(\dotarg)}$ & Fourier transform defined as 
% 		\begin{talign}
% 			\hat f(\xi)
% 					:= \(\mathcal F f\)(\xi)
% 					= \int_{\mathbb R} e^{-i x \xi} f(x) \, \mathrm{d}x
% 		\end{talign}
% 			& p.\pageref{sym:fourier} \\[-1\baselineskip]
% 	$\mathcal F^{-1}$, $\check{(\dotarg)}$ & inverse Fourier transform defined 
% 		as
% 		\begin{talign}
% 				\check g(x)
% 					:= \(\mathcal F^{-1}g\)(x)
% 					= \frac{1}{2 \pi} \int_{\mathbb R} e^{i x \xi} g(\xi) \, \mathrm{d}\xi
% 		\end{talign}
% 		& p.\pageref{sym:fourier} \\
% 	$\delta_a$ & Dirac delta-function centered at $x=a$ 
% 		& p.\pageref{sym:dirac} \\
% 	$(\dotarg\pm i0)$ & implied limit of $(\dotarg \pm i \varepsilon)$ as 
% 			$\varepsilon \searrow 0$ 
% 		& p.\pageref{sym:i0} \\
% 	$\mathcal S_\delta$ & the complex strip
% 			$\{z \in \mathbb C ~:~ 0 < \im z < 2\delta\}$ 
% 		& p.\pageref{sym:Sdelta} \\
% 	$f^+$ & lower boundary $f^+(x) := 
% 			\lim_{y\searrow0} f(x+ i y)$ of a function $f$ analytic on $S_\delta$,
% 			where $x, y \in \mathbb R$ 
% 		& p.\pageref{sym:bndries} \\
% 	$f^-$ & upper boundary $f^-(x) := \lim_{y\nearrow0} f(x+ i 2y)$ of a 
% 			function $f$ analytic on $S_\delta$, where $x, y \in \mathbb R$
% 		& p.\pageref{sym:bndries} \\
% 	$\lambda$ & a spectral parameter for the linear spectral problem 
% 			\eqref{eq0:SpecProb} 
% 		& p.\pageref{sym:zeta} \\
% 	$\zeta$ & a spectral parameter for \eqref{eq0:SpecProb} commonly
% 			parameterized by $\lambda$	as 
% 			$\displaystyle \zeta(\lambda; \delta) 
% 				= \frac{\lambda}{1-e^{-2\delta\lambda}}$ 
% 		& p.\pageref{sym:zeta} \\
% 	$\lambda(\zeta)$ &  inverse of the map $\lambda \to \zeta(\lambda)$ 
% 		& p.\pageref{sym:lambda} \\
% 	$\zeta^*$ & nonlinear reflection $\zeta\big(-\lambda(\zeta)\big)$ 
% 		& p.\pageref{sym:zetastar} \\
% 	$L_\delta$ & operator on functions analytic in the complex strip 
% 		$\mathcal S_\delta$; given by 
% 		{
% 			\begin{align*}
% 				L_\delta (\Psi) 
% 					:= \frac{1}{i} \frac{\partial}{\partial x} \Psi^+ 
% 					- \zeta \left(\Psi^+ - \Psi^-\right) = u \Psi^+
% 			\end{align*}
% 		}
% 		& p.\pageref{eq0:SpecProb} \\
% 	$\log_+ t$ & the function given by $\max\big\{ 0, \, \log(t) \big\}$
% 		& p.\pageref{sym:logplus} \\
% 	$p$ & Fourier symbol of the Green's functions; 
% 			given by
% 			\[
% 				p(\xi; \lambda, \delta)
% 					= \xi - \zeta(\lambda) \( 1- e^{-2\delta \xi} \)
% 			\]
% 			and commonly denoted as $p(\xi; \lambda, \delta)$, 
% 			$p(\xi; \zeta, \delta)$, $p(\xi; \lambda)$, 
% 			$p(\xi; \zeta)$, and $p(\xi)$.
% 		& p.\pageref{sym:GFintegrand} \\
% 	${\Gamma_L}$ & contour along the real line which bypasses the roots of $p$
% 		from below
% 		& p.\pageref{sym:Gamma} \\
% 	${\Gamma_R}$ & contour along the real line which bypasses the roots of $p$
% 		from above
% 		& p.\pageref{sym:Gamma} \\
% 	$\inn{x}$ & short-hand notation for $\big(1 + |x|^2\big)^{1/2}$
% 		& p.\pageref{sym:xbracket} \\
% 	$L^{p,s}(\mathbb R)$ & space of measurable functions with 
% 		\[
% 			\|f\|_{L^{p,s}}
% 				:= \left(\int_{\mathbb R} \inn{x}^{sp} |f(x)|^p \right)^{1/p}
% 				< \infty
% 		\]
% 		& p.\pageref{defn2:Lps} \\
% 	$\langle\dotarg\rangle L^\infty(\mathbb R)$ & space of measurable functions with 
% 		\begin{talign}
% 			\|f\|_{\inn{\dotarg} L^\infty}
% 				:= \esssup_{x\in \mathbb R} \left| \inn{x}^{-1} f(x)  \right|
% 		\end{talign}
% 		finite
% 		& p.\pageref{defn2:wLp} \\
% 	$X$ & space of potentials $u$ with $\nm{\inn{\dotarg}^{4} u}_{L^2} < \infty$
% 		% \begin{talign}
% 		% 	\|u\|_X 
% 		% 		:= \| \inn{\dotarg}^2 \, u(\dotarg) \|_{L^1} 
% 		% 			+ 
% 		% 			\left\|
% 		% 				\log_+\left(\frac{1}{(\dotarg)}\right)
% 		% 				* \inn{\dotarg} |u|
% 		% 			\right\|_{L^\infty}
% 		% \end{talign}
% 		% finite
% 		& p.\pageref{defn2:X} \\
% 	$L^{p,\infty}$ & weak $L^p$ space; also denoted $weak$-$L^p$; please see
% 		the ``Harmonic Analysis Results'' appendix for a definition
% 		& p.\pageref{sym:weakLp}\\
% 	$B_Y(y_0, r)$ & the open ball $\{ y \in Y ~:~ \|y - y_0\|_Y < r  \}$ in 
% 		the metric space $Y$ with radius $r$ centered at $y_0 \in Y$
% 		& p.\pageref{sym:ball} \\
% 	$Y\to Z$ & a map from a space $Y$ to a space $Z$
% 		& p.\pageref{sym:mapsto} \\
% 	$Y \toitself$ & a map from a space $Y$ into itself
% 		& p.\pageref{sym:toitself} \\
% 	$\|\dotarg\|_{Y\to Z}$ & the induced operator norm for an operator with domain 
% 		$Y$ and co-domain $Z$
% 		& p.\pageref{sym:opnorm} \\
% 	$G_\star^+$ & lower boundary value of the Greens' function whose contour
% 		of integration is $\Gamma_\star$; given by 
% 			{
% 			\begin{align*}
% 					G_\star^+(x; \lambda, \delta)
% 						:=
% 							\frac{1}{2\pi} 
% 							\int_{\Gamma_\star}
% 								e^{i x \xi} \,
% 								\frac{1}{p(\xi; \lambda, \delta)}
% 							\, \mathrm{d}\xi
% 				\end{align*}
% 			}
% 		& p.\pageref{sym:GFbndry} \\
% 	$T_{\star, \lambda, u}$ & bounded operator on 
% 		$\inn{\dotarg}L^\infty(\mathbb R)$ given by
% 		{
% 		\begin{align*}
% 			T_{\star, \lambda, u} f(x) 
% 				:= \big[ G_\star^+(\dotarg; \lambda) \big] * (u\,f)(x)
% 		\end{align*}
% 		}
% 		& p.\pageref{eqn2:Tdefn} \\
% 	$G_\star$ & Green's function whose contour of integration is $\Gamma_\star$ 
% 				($\star = L \text{, or } R$); given by 
% 				\begin{align*}
% 					G_\star(z; \lambda, \delta)
% 						:=
% 							\frac{1}{2\pi} 
% 							\int_{\Gamma_\star}
% 								e^{i z \xi} \,
% 								\frac{1}{p(\xi; \lambda, \delta)}
% 							\, \mathrm{d}\xi
% 				\end{align*}
% 		& p.\pageref{eq3:ILW.GFz} \\
% 	$\mathpzc E_\varepsilon, ~\mathpzc P_\varepsilon$ & two families of convolution
% 		operators given by
% 		\[
% 			\mathpzc E_\varepsilon(y)
% 				:= \frac{1}{2\pi i} \frac{y}{y^2 + \varepsilon^2} e^{-\pi|y|},
% 			\quad \text{and} \quad
% 			\mathpzc P_\varepsilon(y)
% 				:= \frac{1}{\pi} \frac{\varepsilon}{y^2 + \varepsilon^2} e^{-\pi|y|}
% 		\]
% 		& p.\pageref{sym:badCauchy} \\
% 	$\mathcal E_{\varepsilon}$ & a family of convolution operators
% 		given by
% 		{\begin{align*}
% 				\left( \mathcal E_\varepsilon f \right)(x)
% 				&:=  \frac{e^{-i \pi (2-\varepsilon)}}{2 \pi i}
% 					\int_{-\infty}^x
% 						\frac{e^{-\pi|x-x'|}}{(x-x') - i\varepsilon} f(x')
% 					\, \mathrm{d}x' \\
% 				&\quad + \frac{e^{i \pi (2-\varepsilon)}}{2 \pi i}
% 					\int_x^{\infty}
% 						\frac{e^{-\pi|x-x'|} }{(x-x') - i\varepsilon} f(x')
% 					\, \mathrm{d}x'
% 		\end{align*}}
% 		& p.\pageref{sym:almostExpCauchy} \\
% 	$E_{\varepsilon}$ & exponentially weighted Cauchy transform; given by
% 		\begin{align*}
% 			E_{\varepsilon} f(x)
% 				:= \frac{1}{2\pi i} \int_{\mathbb R}
% 					\frac{e^{-\pi|x-x'|}}{(x-x') - i \varepsilon} f(x') \, \mathrm{d}x'
% 		\end{align*}
% 		& p.\pageref{sym:expCauchy} \\
% 	$E^*$ & the maximal operator associated with the Cauchy transform
% 		$E_{\varepsilon}$; given by
% 		\begin{align*}
% 			E^*f(x)
% 				:= \( E_{\varepsilon} \)^* f(x)
% 				:= \sup_{\varepsilon>0} \left\{\left|E_{\varepsilon} f(x)\right|\right\}.
% 		\end{align*}
% 		& p.\pageref{sym:maxCauchyT} \\
% 	$M$ & The Hardy-Littlewood maximal operator; given by
% 		\begin{align*}
% 			M f(x)
%         		= \sup_{r > 0}
%         			\left\{
%         				\frac{1}{B(0, r)} \int_{B(0, r)} |f(x - x')| \, \mathrm{d}x'
% 	        		\right\}.
% 		\end{align*}
% 		& p.\pageref{sym:hardy} \\
% 	$E$ & exponentially weighted Hilbert Transform;
% 		given by
% 		\begin{align*}
% 			Ef(x)
% 				:= \frac{1}{2\pi i}
% 					\pv \int_{\mathbb R} \frac{e^{-\pi|x-x'|}}{x-x'} f(x') \, \mathrm{d}x',
% 		\end{align*}
% 		where $\pv \int (\dotarg) \, \mathrm{d}\mu$ denotes a principle value integral.
% 		& p.\pageref{sym:ExpHil} \\
% 	$E^{(\varepsilon)}$ &  truncated exponentially weighted Hilbert Transform; given
% 		by
% 		\begin{align*}
% 			E^{(\varepsilon)}f(x)
% 				:= \frac{1}{2\pi i}
% 					 \int_{|x'|\geq \varepsilon} \frac{e^{-\pi|x'|}}{x'} f(x-x') \, \mathrm{d}x'
% 		\end{align*}
% 		by definition,
% 		$(Ef)(x) = \lim_{\varepsilon \searrow 0} E^{(\varepsilon)}f(x)$
% 		& p.\pageref{sym:truncExpHil} \\
% 	$\chi_A$ & characteristic function on a set $A$; given by 
% 		{\begin{align*}
% 			\chi_A(x) := 
% 				\begin{cases}
% 					1, & x \in A \\
% 					0, & \text{otherwise}
% 				\end{cases}
% 		\end{align*}}
% 		& p.\pageref{sym:chi} \\
% 	$\star$ & used as a placeholder for both $L$ and $R$; for example, if a 
% 		statement contains the notation 
% 		``$G_\star$ ($\star = L \text{, or } R$),'' then it is equally 
% 		true (or not true) for both $G_L$ and $G_R$
% 		& p.\pageref{rmk1:StarNotation} \\
% 	$M_1$, $M_e$, $N_1$, $N_e$ & depending on context, either Jost solutions 
% 		or analytic extensions of solutions to the integral equations
% 		\eqref{eq0:JostIE}
% 		& p.\pageref{defn0:jost}, p.\pageref{eq0:JostIE} \\
% 	$M_1^+$, $M_e^+$, $N_1^+$, $N_e^+$ & depending on context, either solutions
% 		to the integral equations \eqref{eq0:JostIE} or lower boundary 
% 		values of the Jost solutions
% 		& p.\pageref{eq0:JostIE}, p.\pageref{defn0:jost} \\
% 	$\alpha(\lambda; \delta)$ & residue of $e^{iz\xi}/p(\xi)$ at the $\xi=0$
% 		pole; given by 
% 		{\begin{align*}
% 			\alpha(\lambda; \delta)
% 				= \frac{1}{1-2\delta\zeta(\lambda; \delta)}
% 		\end{align*}}
% 		& p.\pageref{sym:alphabeta} \\
% 	$\beta(\lambda; \delta)$ & $e^{iz\lambda}$ times the residue of 
% 		$e^{iz\xi}/p(\xi)$ at the $\xi=\lambda$ pole; given by 
% 		{\begin{align*}
% 			\alpha(\lambda; \delta)
% 				= \frac{1}{1-2\delta\zeta(-\lambda; \delta)}
% 		\end{align*}}
% 		& p.\pageref{sym:alphabeta} \\
% 	$a$ & & \\
% 	$b$ & & \\
% 	$\mathpzc R_\star$ & sum residues of $e^{iz\xi}/p(\xi)$ at the $\xi=0$ and $\xi=\lambda$ 
% 		poles & \\
% 	$K$ & & \\
% 	$\mathfrak C$ & the portion of $K$ with continuous limit to the upper 
% 		boundary of $\mathcal S_\delta$; given by 
% 		{
% 			\begin{align*}
% 				\mathfrak C(x,y)
% 					:= \frac{1}{2\pi} e^{-\pi|x|} \, e^{- \sign(x) \, i \pi y }
% 						\int_{\mathbb R} e^{ix \xi} \rho\big(\xi, y, \sign(x)\big) \, \mathrm{d}\xi,
% 			\end{align*}
% 		}
% 		where $x \in \mathbb R$ and $y\in [0, 2]$
% 		& p.\pageref{sym:mathfrakC} \\
% 	$r$ & reflection coefficient; given by $r(\lambda) = b(\lambda) / a(\lambda)$ 
% 		& \\
% 	$\mathscr D$ 
% 		& direct scattering map; given by  
% 		$\mathscr D: B_X(0,c_0) \ni u \mapsto r \in \inn{\dotarg} L^\infty(\mathbb R)$
% 		& p.\pageref{sym:DM} \\
% 	$c_0$ & a strictly positive constant for which the integral equations \ref{eq0:JostIE}
% 		admit unique solutions for every $u \in B_X(0, c_0)$ 
% 		& p.\pageref{prop4:exist} \\
% 	$\lesssim$ & $a \lesssim b$ means there exists some fixed constant $C$ so 
% 		that $a \leq C\,b$ & p.\pageref{sym:lesssim} \\
% \end{longtable}
% \end{centering}


\end{document}