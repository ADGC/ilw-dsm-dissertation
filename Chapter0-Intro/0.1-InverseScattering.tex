%%=================================================
%% Section 0.01: Introduction to Inverse Scattering
%%=================================================

\documentclass[../dissertation.tex]{subfiles}

\begin{document}
\section{Introduction to Inverse Scattering}\label{sec0:IST}

First introduced in the late 1960's and early 1970's as a technique for studying 
the Korteweg-de Vries (KdV) equation, the inverse scattering method has since 
been adapted to find solutions to a number of other non-linear dispersive 
equations\textemdash{}equations whose solutions model dispersive wave 
phenomenon. Since non-linear dispersive equations arise naturally in many fields 
of science and engineering as a means to model the behavior of many important 
phenomenon, understanding non-linear dispersive equations is a very active 
area of research. 
% Unfortunately, non-linear dispersive
% equations\textemdash{}particularly, partial differential and diffeo-integral 
% equations\textemdash{}have proven especially difficult objects to study. 
The inverse scattering method has been very successful in solving certain types 
of non-linear dispersive equations.

The structure of the inverse scattering method can be thought of as a model 
for how problem solving often works. When confronted with a difficult problem
about which very little is know, it is often helpful to find a way to relate
that problem to another problem which you know how to solve. So, by solving 
the related problem and connecting its solution back to the original problem, 
you can find a solution for the original problem. This process is essentially
how the inverse scattering method works. 

For us in this dissertation, solving a non-linear dispersive equation means 
finding a mathematically rigorous procedure by which if someone tells you what 
a non-linear wave described by a particular non-linear dispersive equation looks 
like initially, you can determine how that wave looks in the future. We use the 
term ``initial data'' to refer to the initial state of a wave described by a 
given non-linear dispersive equation and the term ``solution'' to refer to 
a formula which provides a description for what the initial data looks like 
for each relevant time $t$.
As mathematicians, our goal in solving a 
non-linear dispersive equation is to not only describe a map (\textit{i.e.} 
procedure or function) which takes as input initial data for that non-linear 
dispersive equation and returns as output the corresponding description for how 
that initial data evolves in time, but to also understand this map deeply enough 
to be able to guarantee\textemdash{}at least under certain 
conditions\textemdash{}this map actually works. Such a map is often 
referred to as a solution map, as it provides the solution corresponding a given
initial data (\textit{i.e.} a solution map maps initial data to solutions). 

In the inverse scattering method, 
we use certain properties briefly described below of given non-linear dispersive 
equation to construct a map $\mathscr D$, called the ``direct 
scattering map,'' which maps arbitrary 
initial data of 
the equation to the initial data's 
so-called ``scattering data'' describing certain physical properties of the 
initial data. Using properties of the direct scattering map, we have tools to determine 
how the scattering data changes in time. As demonstrated below in Figure 
\ref{fig:IST}, by also constructing a second map $\mathcal I$, called the 
``inverse scattering map''
which serves as the inverse for the direct scattering map \textemdash{}\textit{i.e.} maps 
scattering data to its corresponding initial data\textemdash{}we can then find 
a solution to a non-linear dispersive equation by applying the inverse scattering map to 
the time dependent representation of the scattering data.

\begin{figure}[h!]
	\centering
    \begin{tikzpicture}
        \def\xspace{4.5}
        \def\yspace{1.5}
        \node (A) at ({-\xspace},{\yspace}) {
            $
                \begin{pmatrix}
                    \text{Initial data}\\
                    \text{$u_0(x) = u(x, t=0)$}    
                \end{pmatrix}
            $
        };
        
        \node (B) at ({\xspace},{\yspace}) {
            $
                \begin{pmatrix}
                    \text{Scattering data} \\
                    \text{$r(\lambda, t=0)$}    
                \end{pmatrix}
            $
        };

        \node (C) at ({-\xspace}, {-\yspace}) 
            {$u(x, t)=\mathscr{I}\big( r(\lambda, t)\big)$};

        \node (D) at ({\xspace}, {-\yspace}) {$r(\lambda, t)$};

        \node (text) at (0,0) {Time Evolution};

        \path[-stealth]
            (A) edge node[above] {Direct Scattering Map $\mathscr D$} (B)
            % (B) edge node[right] {Time evolution} (D)
            (B) edge node[right] {} (D)
            (D) edge node[below] {Inverse Scattering Map $\mathscr{I}$} (C);

        \path[-stealth, dashed] 
            (A) edge (C);
    \end{tikzpicture}
    \caption{Diagram of the Inverse Scattering Method}
    \label{fig:IST}
\end{figure}

The combination of the direct scattering map and its corresponding inverse 
scattering map is called an 
Inverse Scattering Transform, or IST for short. As such, using the Inverse 
Scattering Method to solve a given non-linear dispersive equation is often 
referred to as solving that non-linear dispersive equation by its Inverse 
Scattering Transform. Readers familiar with the method of solving linear partial 
differential equations by the Fourier Transform may recognize the Inverse 
Scattering Method as an analogous method for solving non-linear dispersive 
equations. However, an IST differs from the Fourier Transform in one very 
key way: while the Fourier Transform can be defined in a single set way that 
does not depend on the linear equation one wishes to solve, the definition for 
an IST depends \textit{entirely} on the non-linear dispersive equation one hopes 
to solve using the Inverse Scattering Method. For this reason, the crux of 
solving a non-linear dispersive equation using the Inverse Scattering Method 
lies in the construction of that particular dispersive equation's 
corresponding IST, and proving that that IST is both well-defined and 
bi-Lipschitz continuous\textemdash{}\textit{i.e.} both the direct scattering map
and the inverse scattering map for a given IST are Lipschitz 
continuous.\footnote{Informally, Lipschitz continuity is essentially the requirement
that the distance between two outputs of a given function or map is directly 
proportional to the distance between their corresponding inputs. Mathematically,
we say that for any two spaces $Y$ and $Z$, a map $f : Y \to Z$ is Lipschitz 
continuous if there exists some constant $C>0$ so that 
$\nm{f(y_1) - f(y_2)}_Z \leq C \nm{y_1 - y_2}_Y$ for every $y_1, y_2 \in Y$.}
Bi-Lipschitz continuity is desired as a property of an IST as it helps to ensure 
the solution map\textemdash{}\textit{i.e.} map from initial data to the 
corresponding solution\textemdash{}is continuous in initial data, and hence, 
in the language of the theory of partial differential equations, well-posed.
In other words, proving that an IST for a particular dispersive wave is 
Lipschitz continuous ensures that changing a given initial data slightly 
results in a similarly small change in the solution. 


One can create an IST for a given dispersive equation if the equation
has what is called a ``\textit{Lax representation}'' of the equation. A Lax 
representation for a dispersive equation is 
a pair of time dependent operators $L$, $B$ parametrized by a 
function $u(x,t)$ which satisfy the identity
\begin{align}\label{eq0:compatibility}
    \frac{d}{dt} L = BL - LB
\end{align}
if and only if $u$ solves the given equation. In the literature, such $L$ and $B$ 
are sometimes refered to as a ``\textit{Lax pair}.''
Whenever $L$, $B$ satisfy 
\eqref{eq0:compatibility}, one can show that the spectrum of $L$ is time invariant.
As such, \eqref{eq0:compatibility} further implies
\begin{align}\label{eq0:timeprop}
    \left\{
        \begin{aligned}
             \kappa(t) &= \kappa(0) \\
             \frac{\partial}{\partial t}\psi &= B \psi
        \end{aligned}
    \right.
\end{align}
for all $\kappa$, $\Psi$ satisfying $L\psi = \kappa \psi$, and the given 
equation is said to be an isospectral flow for the linear spectral problem
$L\psi = \kappa \psi$. The linear spectral 
problem can be therefore solved at arbitrary time $t$ by first solving it 
for an initial time $t_0$ (typically $t_0 = 0$) and using \eqref{eq0:timeprop}
to propagate in time. The model, then, for solving an equation by the Inverse
Scattering Method is to construct the direct scattering map by solving linear
spectral problem $L\Psi = \kappa \Psi$ for time $t_0$, and then use the operator
$B$ to propagate the scattering data in time. By constructing an inverse 
to the direct scattering map (\textit{i.e.} the corresponding inverse scattering
map), one can then recover a solution to the original dispersive equation 
from the time evolved scattering data.

\end{document}