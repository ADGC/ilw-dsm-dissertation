%%========================================================
%% Section 0.02: Overview of the ILW Direct Scattering Map
%%========================================================

\documentclass[../dissertation.tex]{subfiles}

\begin{document}
\section{Overview of the ILW Direct Scattering Map}\label{sec0:DM}

While there are a series of papers from the late 1970's and early 1980's 
culminating in a paper by Y. Kodama, M.J. Ablowitz and J. Satsuma 
\cite{Kodama1982} and a subsequent paper by P.M. Santini, M.J. Ablowitz and A.S. 
Fokas \cite{Santini1984} which formally describe the Inverse Scattering 
Transform (IST) for the ILW, little research has been done to place the Inverse 
Scattering Method for the ILW equation on a rigorous mathematical footing. 
At the time of writing this dissertation, the author found no results in the 
literature showing that the IST for the ILW equation is actually well-defined
or bi-Lipschitz continuous\textemdash{}even for small data.

As described in Section \ref{sec0:IST}, using the Inverse
Scattering Method to solve the ILW entails constructing 
an invertible, bi-Lipschitz
continuous map from initial data to the corresponding scattering data in such 
a way that linearizes the flow\textemdash{}\textit{i.e.} the time dependence 
of the output of this map applied to initial data is determined by a linear 
differential equation. The ``forward
direction'' of this map which takes initial data to scattering data is referred to 
as the ``direct scattering map,'' and its inverse
is referred to as the ``inverse scattering map.'' This distinction is made as the process for 
constructing the direct scattering map for a given equation is often very different from
the process for constructing the corresponding inverse scattering map. As previously mentioned,
the combination of this direct scattering map with the corresponding inverse 
scattering map and the linear differential equation used to propagate the scattering data in 
time is called an Inverse Scattering Transform for the ILW.

What allows us to construct an IST for the ILW
is
fact that the ILW is an isospectral flow for the linear spectral problem
\begin{align}\label{eq0:SpecProb}
	L_\delta (\Psi) 
		:= \frac{1}{i} \frac{\partial}{\partial x} \Psi^+ 
			- \zeta \left(\Psi^+ - \Psi^-\right) = u \Psi^+,
\end{align}
which is a part of the Lax pair\footnote{Please see the appendix titled
\hyperref[app:Lax]{\textit{Lax Representation}} for a comparison of 
\eqref{eq0:Lax} with the ILW Lax pair typically given in the literature.}
\begin{subequations}\label{eq0:Lax}
	\begin{align}
		\frac{1}{i} \frac{\partial}{\partial x} \Psi^+ 
				- \zeta \left(\Psi^+ - \Psi^-\right)
			 &= u \Psi^+ 
			 \label{eq0:LaxX} \\
		\frac{1}{i} \frac{\bd}{\bd t} \Psi^\pm + 2 i 
				\( \zeta-\frac{1}{2\delta} \) + \Psi_{xx}
			&= \left[ \pm i u_x  - T u_x + \eta  \right] \Psi^\pm,
			\label{eq0:LaxT}
\end{align}
\end{subequations}
where $\Psi$ is a function analytic in the complex strip\label{sym:Sdelta} 
\[
	\mathcal S_\delta := \{ z \in \mathbb C \, : \, 0 < \im z < 2 \delta\}
\]
with respective lower and upper boundary values\label{sym:bndries}
\begin{align}\label{eq0:bndryvaluedefn}
	\Psi^+(x):= \lim_{y \searrow 0} \Psi(x + i y), \qquad 
	\text{and}\qquad
	\Psi^-(x):= \lim_{y \nearrow 2\delta} \Psi(x + i y),
\end{align}
where we use the superscript notation $f^\pm$ throughout this dissertation to 
indicate lower and upper boundary values as shown above of functions $f$ 
analytic on the complex strip $\mathcal S_\delta$.
The spectral parameter $\zeta \in (0, \infty)$\label{sym:zeta} is itself parametrized by a second spectral 
parameter $\lambda \in \mathbb R$\label{sym:lambda} as 
\[
	\zeta(\lambda; \delta) := \frac{\lambda}{1 - e^{-2 \delta \lambda}},
\]
where we use the notation $f(\dotarg; t_1, \ldots, t_n):=f_{t_1, \ldots, t_n}$ to denote a 
family of functions ``indexed'' in a (possibly) uncountable sense by $t_1, \ldots, t_n$. 
The output of direct scattering map $\mathscr D$ for the ILW is determined by the 
lower boundary values
$M_1^+$, $M_e^+$, $N_1^+$, $N_e^+$ of eigenfunctions $M_1$, $M_e$, $N_1$, $N_e$
of $L_\delta$ which satisfy the asymptotic conditions given in \eqref{eq0:JostDEasymp}.
Such eigenfunctions are referred to in the literature as ``Jost solutions,'' and, given their
importance in the construction of the direct scattering map $\mathscr D$, we explicitly
define Jost solutions as follows:
\begin{defn}[Jost solutions]\label{defn0:jost}
	The Jost solutions $M_1$, $M_e$, $N_1$, $N_e$ are solutions to the linear 
	spectral problem \eqref{eq0:SpecProb} whose lower boundary values
	$M_1^+$, $M_e^+$, $N_1^+$, $N_e^+$ as defined in \eqref{eq0:bndryvaluedefn}
	obey the following asymptotic conditions
	\begin{subequations}\label{eq0:JostDEasymp}
		\begin{align}
			\lim_{x\to -\infty} 
					\inn{x} 
					\left( 
						M_1^+(x; \lambda, \delta) - 1 
					\right)
				&= \lim_{x\to \infty} 
						\inn{x} 
						\left( 
							N_1^+(x; \lambda, \delta) - 1
						\right)
				= 0 \\
			\lim_{x\to -\infty} 
					\inn{x} \left( 
						M_e^+(x; \lambda, \delta) - e^{i\lambda x}
					\right)
				&= \lim_{x\to \infty} 
						\inn{x} 
						\left( 
							N_e^+(x; \lambda, \delta) - e^{i\lambda x}
						\right)
				= 0,
		\end{align}
	\end{subequations}
	where we use the notation $\inn{x}:=\sqrt{1+|x|^2}$\label{sym:xbracket} to 
	indicate a linear weight.

	Additionally, we require the upper boundary values $M_{(\dotarg)}^-$, $N_{(\dotarg)}^-$
	(where $(\dotarg)$ represents either the subscript $1$ or $e$) of 
	$M_{(\dotarg)}$, $N_{(\dotarg)}$ to have a decomposition 
	\begin{align*}
		M_1^- - 1 &= M_1^{(1)} + M_1^{(2)} \qquad &\text{and}& \qquad
		&N_1^- - 1 &= N_1^{(1)} + N_1^{(2)} \\
		M_e^- - e^{i\lambda x}\,e^{-2\delta\lambda} &= M_e^{(1)} + M_e^{(2)} \qquad &\text{and}& \qquad
		&N_e^- - e^{i\lambda x}\,e^{-2\delta\lambda} &= N_e^{(1)} + N_e^{(2)} \\
	\end{align*}
	satisfying 
	\begin{align*}
		\inn{x}^{1+\upsilon} \left|M_{(\dotarg)}^{(1)}(x)\right| \lesssim 1 
			\quad (\text{for } x \ll -1), \qquad 
		\inn{x}^{1+\upsilon} \left|N_{(\dotarg)}^{(1)}(x)\right| \lesssim 1
			\quad (\text{for } x \gg 1),
	\end{align*}
	and
	\begin{align*}
		\inn{\dotarg}^\tau M_{(\dotarg)}^{(2)}, 
			~\inn{\dotarg}^\tau N_{(\dotarg)}^{(2)} \in L^2(\mathbb R)
	\end{align*}
	for any $\upsilon \in \left(0,\frac{1}{2}\right)$ and $\tau \in [0,1)$, where we 
	use the notation  $a\lesssim b$ to indicate $a \leq C \, b$ for some constant $C>0$.
	\label{sym:lesssim}
	% \begin{subequations}\label{eq0:JostDEasymp}
	% 	\begin{align}
	% 		\left\{
	% 			\begin{aligned}
	% 				&\lim_{x\to -\infty}	M_1^+(x; \lambda, \delta) - 1
	% 					= \lim_{x\to \infty} N_1^+(x; \lambda, \delta) - 1
	% 					= 0 \\
	% 				&\lim_{x\to -\infty} \big( M_e^+(x; \lambda, \delta) - e^{i\lambda x}\big)
	% 					= \lim_{x\to \infty} \big( N_e^+(x; \lambda, \delta) - e^{i\lambda x}\big)
	% 					= 0
	% 			\end{aligned}
	% 		\right.
	% 		\qquad (\lambda \ne 0)
	% 	\end{align}
	% 	and
	% 	\begin{align}
	% 		\left\{
	% 			\begin{aligned}
	% 				&\lim_{x\to -\infty}
	% 						\inn{x}^{-1} \big(M_1^+(x; \lambda, \delta) - 1\big)
	% 						\\
	% 				&\qquad\qquad= \lim_{x\to \infty} \inn{x}^{-1} 
	% 						\big(N_1^+(x; \lambda, \delta) - 1\big)
	% 					= 0 \\
	% 				&\lim_{x\to -\infty}\inn{x}^{-1} 
	% 						\big( M_e^+(x; \lambda, \delta) - e^{i\lambda x}\big)
	% 						\\
	% 					&\qquad\qquad= \lim_{x\to \infty}\inn{x}^{-1} 
	% 						\big( N_e^+(x; \lambda, \delta) - e^{i\lambda x}\big)
	% 					= 0
	% 			\end{aligned}
	% 		\right.
	% 		\qquad (\lambda = 0)
	% 	\end{align}
	% \end{subequations}
\end{defn}

For a given 
$u(x)$, the corresponding output $r = \mathscr D u$ of direct scattering map is
given by 
$r(\lambda; \delta) = b(\lambda;\delta) / a(\lambda; \delta)$,\label{sym0:reflection} 
where
\begin{align*}
	b(\lambda)
		&:= 
			\frac{i}{1-2\delta\zeta(-\lambda)} 
			\int_{\mathbb R} e^{-i\lambda x} 
				u(x) \, M_1^+(x; \lambda,\delta) 
			\, \mathrm{d}x
			\\
	a(\lambda)
		&:=
			1 
			+ \frac{i}{1-2\delta \zeta(\lambda)}
				\int_{\mathbb R} 
					u(x) \, M_1^+(x; \lambda,\delta) 
				\, \mathrm{d}x.
\end{align*}
In this dissertation, we prove the following result:
\begin{thm}\label{thm0:MainResult}
	For sufficiently small $c_0 > 0$ the map
	\begin{align*}
		\mathscr D : B_X(0, c_0) &\to L^\infty(\mathbb R) \\
		                       u &\mapsto r
	\end{align*}
	is well-defined for all real $\lambda$, where 
	$X$ denotes the space $\inn{\dotarg}^{-4} L^2(\mathbb R)$, and 
	$B_X(0, c_0)$ \label{sym:ball} 
	is the ball in the 
	space $X$ about zero with radius $c_0$.\footnote{See Theorem \ref{thm4:Dwelldefined}
	in Section \ref{sec4:DM} of Chapter \ref{cptr04:DM}.}
	Further, $\mathscr D$\label{sym0:DSM} is 
	Lipschitz continuous as a map from $B_X(0, c_0)$ to 
	$L^\infty\big((-\infty, k]\cup[k, \infty)\big)$ for each fixed 
	$k>0$.\footnote{See Theorem \ref{thm4:DlipR} in Section \ref{sec4:DM} of Chapter 
	\ref{cptr04:DM}.}
\end{thm}

A crucial first step to showing that $\mathscr D$ is well-defined is showing that 
for each given $u \in B_X$ the corresponding Jost solutions both exist and are 
unique. To do so, one uses 
the (formal) Green's Functions $G_L$ and $G_R$, given by 
\begin{subequations}
	\label{eq0:GFs}
	\begin{align}
		\label{eq0:GFL}
		G_L(z; \lambda, \delta) 
			&= 
				\lim_{\varepsilon \searrow 0} 
				\frac{1}{2\pi} 
				\int\limits_{\mathbb R - i\varepsilon} 
					\frac{e^{iz\xi}}
						{\xi -  \zeta(\lambda) \left(1- e^{-2 \delta \xi}\right) } 
				\, \mathrm{d}\xi, 
				\qquad \left(z\in \overline{\mathcal{S}}_\delta \right)\\
		\intertext{and}
		\label{eq0:GFR}
		G_R(z; \lambda, \delta) 
			&= 
				\lim_{\varepsilon \searrow 0} 
				\frac{1}{2\pi} 
				\int\limits_{\mathbb R + i\varepsilon} 
					\frac{e^{iz\xi}}
						{\xi -  \zeta(\lambda) \left(1- e^{-2 \delta \xi}\right) } 
				\, \mathrm{d}\xi,
				\qquad \left(z\in \overline{\mathcal{S}}_\delta \right)
	\end{align}
\end{subequations}
to rewrite \eqref{eq0:SpecProb} with asymptotic conditions \eqref{eq0:JostDEasymp} 
as the integral equations
\begin{subequations}
	\label{eq0:JostIE}
	\begin{align}
		\label{eq0:JostIEleft}
		\begin{pmatrix}
			M_1^+(x; \lambda, \delta) \\
			M_e^+(x; \lambda, \delta)
		\end{pmatrix}
			&= 
				\begin{pmatrix}
					1 \\
					e^{i\lambda x} 
				\end{pmatrix}
				+ \int_{\mathbb R} G_L^+(x - x'; \lambda, \delta) 
					u(x')
					\begin{pmatrix}
						M_1^+(x'; \lambda, \delta) \\
						M_e^+(x'; \lambda, \delta) 
					\end{pmatrix}
					\, \mathrm{d}x' \\[0.3\baselineskip]
		\label{eq0:JostIEright}
		\begin{pmatrix}
			N_1^+(x; \lambda, \delta) \\
			N_e^+(x; \lambda, \delta)
		\end{pmatrix}
			&= 
				\begin{pmatrix}
					1 \\
					e^{i\lambda x} 
				\end{pmatrix}
				+ \int_{\mathbb R} G_R^+(x - x'; \lambda, \delta) 
					u(x')
					\begin{pmatrix}
						N_1^+(x'; \lambda, \lambda) \\
						N_e^+(x'; \lambda, \lambda) 
					\end{pmatrix}
					\, \mathrm{d}x',
	\end{align}
\end{subequations}
where we again use the ``$+$'' superscript to indicate the lower boundary
values of functions analytic in the complex strip $\mathcal S_\delta$.

Formally, assuming that the solutions to equations \eqref{eq0:JostIE} 
have analytic extensions to the strip $\mathcal S_\delta$ with upper
boundary values
$M_1^-$, $M_e^-$, $N_1^-$, and $N_e^-$, one can show through a simply 
heuristic computation that solutions to the integral equations \eqref{eq0:JostIE} 
should satisfy the spectral problem \eqref{eq0:SpecProb} with asymptotic 
conditions \eqref{eq0:JostDEasymp}. However, as discussed in Section 
\ref{sec1:RootsOfP},the integrand of $G_L$ and $G_R$ has exactly two poles 
along the real line as shown in \textemdash{}namely $\xi = 0$ and 
$\xi = \lambda$\textemdash{}and countably many poles in the complex plane. 
Worse, the Fourier symbol $p(\xi)$ of $G_L$, $G_R$ does not belong to any
standard symbol class due to its radically different asymptotic behavior 
as $\xi \to -\infty$, and $\xi \to +\infty$, respectively. As such,
it is hardly obvious that $G_L$ and $G_R$ are even remotely well 
defined as convolution operators. As such, before we can even begin to prove 
that the direct scattering map $\mathscr D$ is well-defined, we require a 
thorough understanding of the Green's functions $G_L$, $G_R$ as convolution 
operators\textemdash{}indeed, such is the focus of Chapters \ref{cptr01:GF} 
through \ref{cptr03:xContin} of this dissertation.

We begin our study of the Green's functions $G_L$, $G_R$ in Chapter 
\ref{cptr01:GF} by analyzing the properties of the lower boundary 
values $G_L^+$, $G_R^+$ as functions.
Using a combination of a contour shift
and several dyadic decompositions, we show
that the boundary values $G_L^+$ and $G_R^+$ can be represented as 
\begin{align}\label{eq0:GFrep}
	G_L^+(x; \lambda, \delta)
		&= 
			\begin{cases}
				K^+(x; \lambda, \delta), &x < 0 \\
				K^+(x; \lambda, \delta)
				+ i \alpha(\lambda; \delta) 
				+ i \beta(\lambda; \delta) \, e^{i \lambda \, x}, 
				&x >0 
			\end{cases}
\end{align}
where $G_L(z; \lambda, \delta) = \overline{G_R(-\re z +i\im z; \lambda, \delta)}$, 
\begin{align*}
	\alpha(\lambda; \delta) 
		= \frac{1}{1-2\delta\lambda(\lambda)},
	\qquad
	\beta(\lambda; \delta) 
		= \frac{1}{1 - 2\delta\lambda(-\lambda) e^{-2\delta\lambda}},
\end{align*}
\label{sym0:residues}and the function $K^+$\label{sym0:K} satisfies the properties
\begin{itemize}
	\item[(i)] $K^+(x) = \mathcal O \left( \frac{e^{-\pi|x|}}{x} \right)$ 
		for $|x| \geq 1$, and
	\item[(ii)] $|K^+(x)| \leq C + C \log\left(\frac{1}{|x|}\right)$ for $|x| < 1$.
\end{itemize}

Our study of the boundary values $G_L^+$, $G_R^+$ continues in Chapter 
\ref{cptr02:GFmapping} as we use \ref{eq0:GFrep} to understand the mapping
properties of $G_L^+$, $G_R^+$ as convolution operators. More specifically, 
we study the operators $T_{L, \lambda, u}$, $T_{R, \lambda, u}$ given by
\begin{align} \label{eq0:Tstar}
	T_{L, \lambda, u} f:= G_L^+(\dotarg; \lambda, \delta)*(uf) 
	\qquad \text{and} \qquad 
	T_{R, \lambda, u} f:= G_R^+(\dotarg; \lambda, \delta)*(uf) 
\end{align}
and show that the are bounded operators acting the space 
$\inn{\dotarg}L^\infty(\mathbb R)$ whose operator norms depend only 
on the $\nm{u}_X$ and not on $\lambda$. We further show in Chapter 
\ref{cptr02:GFmapping} that for every 
$f \in \inn{\dotarg} L^\infty(\mathbb R)$
and $u \in X$ the operators $T_{L, \lambda, u}$, $T_{R, \lambda, u}$
satisfy the asymptotic behavior
\begin{align*}
	\lim_{x\to-\infty} T_{L, \lambda, u} f(x) 
		= \lim_{x\to-\infty} T_{R, \lambda, u} f(x) 
		= 0
\end{align*}
when real $\lambda \ne 0$, and 
\begin{align*}
	\lim_{x\to-\infty} \inn{x}^{-1} T_{L, \lambda, u} f(x) 
		= \lim_{x\to-\infty} \inn{x}^{-1} T_{R, \lambda, u} f(x) 
		= 0
\end{align*}
for every $\lambda \in \mathbb R$. That $T_{L, \lambda, u}$, 
$T_{R, \lambda, u}$ satisfy the above limits is a property we use later to 
prove solutions to the integral equations \eqref{eq0:JostIE} satisfy the 
asymptotic conditions in \eqref{eq0:JostDEasymp}.

The focus for our final chapter on the the Green's functions, Chapter 
\ref{cptr03:xContin}, is analytically extending $G_L^+$, $G_R^+$ in 
the variable $x$ to $G_L$ and 
$G_R$ defined on the complex strip $\mathcal S_\delta$ and showing that $G_L$, $G_R$ 
have upper boundary values $G_L^-$, $G_R^-$. Analytically extending $G_L^+$, 
$G_R^+$ is important as it allows us to analytically extend the solutions 
$M_1^+$, $M_e^+$, $N_1^+$, $N_e^+$ to the integral equations \ref{eq0:JostIE},
which is a prerequisite to showing that solutions to the integral 
equations \ref{eq0:JostIE} are Jost solutions. While, extending $G_L^+$, $G_R^+$
(as convolution operators) to the open strip $\mathcal S_\delta$ is straight
forward, showing the existence of the upper boundaries $G_L^-$, $G_R^-$ is 
considerably more involved, as it involves working with a singular operator
that is reminiscent of the Hilbert transform. In fact, nearly the entirety of
Chapter \ref{cptr03:xContin} is devoted to proving the existence (in an $L^2$ 
sense) of $G_L^-$ and $G_R^-$.

Our analysis of the Green's functions in \ref{cptr01:GF} through 
\ref{cptr03:xContin} allows us to finally prove in Section \ref{sec4:equiv}
of Chapter \ref{cptr04:DM} the equivalence of Jost solutions and solutions to
the integral equations \eqref{eq0:JostIE}. The big pay-off in proving this 
equivalence is that it allows us to consider the Jost solutions as solutions
to Volterra type integral equations instead of an ordinary differential 
equation on a complex strip involving complex boundary values. Being able to do
so is invaluable as the theory of Volterra type integral equations is far better 
understood (at least by this author) than the theory of such complex ordinary 
differential equations. Indeed, it is precisely by treating the Jost solutions 
as solutions to integral equations \eqref{eq0:JostIE} that we are ultimately 
able in Chapter \ref{cptr04:DM} to prove that $\mathscr D$ is well-defined and, 
at least for real $\lambda$ values away from zero, $\mathscr D$ is also 
Lipschitz continuous.
\end{document}



% {\color{red}
% Using \eqref{eq0:GFrep}, we further show in Chapter \ref{cptr01:GF} that, when 
% taken as a convolution operator, $G_L^+$ and $G_R^+$ are bounded and continuous 
% operators on $L^1\cap L^p$ for $p \in (1, 2]$. This allows us 
% to show in Chapter \ref{cptr03:xContin} that taken as convolution 
% operators, $G_L$, $G_R$ are analytic in the open strip 
% $\mathcal S_\delta = \{ z\in \mathbb C \,:\, 0 < z < 2\delta \}$.

% Using techniques from harmonic analysis, we show in Chapter \ref{cptr03:xContin} the 
% existence of the upper boundary values $G_L^-$, $G_R^-$, and study their mapping 
% properties as convolution operators. Doing so is important as it allows us to 
% then prove the equivalence between solutions of the differential equation 
% \eqref{eq0:SpecProb} satisfying 
% asymptotic conditions \eqref{eq0:JostDEasymp} and solutions of the integral equations 
% \eqref{eq0:JostIE}. Proving this equivalence allows us to use the integral equations
% \eqref{eq0:JostIE} to prove the existence and uniqueness of solutions to the linear 
% spectral problem \eqref{eq0:SpecProb} governed by asymptotic conditions 
% \eqref{eq0:JostDEasymp}.
% Further, by studying the asymptotics of the solutions to \eqref{eq0:JostIE}, we 
% are then able to determine in Section \ref{sec3:IE} that the scattering is then given by the integral 
% formulas
% \reversemarginpar
% \marginnote{\color{red} Need to prove $a$, $b$ continuous in $u$. 
% Want to characterize $a$, $b$ and specify spaces for $a$, $b$.}
% \begin{align*} 
% 	a(\lambda)	
% 		&=	1 + i \alpha(\lambda) 
% 			\int_{-\infty}^x u(x') M_1^+(x',\lambda) \, \mathrm{d}x'	\\
% 	\intertext{and}
% 	b(\lambda)	
% 		&=	i \beta(\lambda) 
% 			\int_{-\infty}^x e^{-i\lambda x'} u(x') M_1^+(x',\lambda) \, \mathrm{d}x'.
% \end{align*}
% \sout{Since we are able to use the properties of $M_1^+$ to prove that the above integral
% equations are uniquely solvable}, this completes the Direct Map of the ILW's 
% Inverse Scattering Transform. 


% The first step in constructing the inverse scattering map involves setting up and 
% solving a Riemann-Hilbert problem (RHP) involving the scattering data. 
% However, in order to construct the correct
% RHP, we need to first understand the analytic (in $\lambda$) properties of the 
% scattering data, which means also understanding the analytic (in $\lambda$) 
% properties of the previously mentioned Jost solutions and Green's functions.  
% While this dissertation focuses primarily on the Direct Map, a discussion
% of the analytic properties of the scattering data is provided at the end of 
% this disseration in {\color{red}?Chapter/Section? ??.??} as a staging point for 
% future research into the ILW's Inverse Scattering Transform.}