%%===========================
%% Section 1.00: Introduction
%%===========================

\documentclass[../dissertation.tex]{subfiles}

\begin{document}
\setcounter{section}{-1}
\section{Introduction}\label{sec1:Intro}

Proving the direct scattering map $\mathscr D$ for the Inverse Scattering Transform
of the Intermediate Long Wave equation is both well-defined and Lipschitz 
continous hinges on our ability to reformulate the linear spectral problem 
\eqref{eq0:SpecProb} with prescribed asyptotic conditions \eqref{eq0:JostDEasymp}
as the integral equations \eqref{eq0:JostIE} and understand the behavior of 
the solutions to \eqref{eq0:JostIE}. Both require a deep understanding of
the properties of the Green's functions $G_L$ and $G_R$ defined in equation
\eqref{eq0:GFs} of Section \ref{sec0:DM}. Indeed, this is the first of three
chapters devoted solely to the study of $G_L$ and $G_R$.


The focus of this chapter is to study the properties 
of the lower boundary values $G_L^+$ and $G_R^+$ as functions on $\mathbb R$, 
where the symbol $\mathbb R$ denotes the of all real numbers.\label{sym:Reals}
In particular, we use a contour shift to derive the alternate formulas
\eqref{eq1:GFrep} and \eqref{eq1:GRrep} for $G_L^+$ and $G_R^+$ from 
Theorem \ref{thm1:GFRep} (Section \ref{sec1:GreensFunctions}), which
we use to study the asymptotic properties of $G_L^+$, 
$G_R^+$ (Section \ref{sec1:AsympK}) and the singularity both functions have 
at $x=0$ (Section \ref{sec1:AsympK}). We continue our study of the Green's
functions in Chapters \ref{cptr02:GFmapping} and \ref{cptr03:xContin} 
where we use our analyses from this chapter 
to first study mapping properties of $G_L^+$, $G_R^+$ as convolution operators 
(Chapter \ref{cptr02:GFmapping}), and then to prove that $G_L^+$, $G_R^+$
extend analytically in the variable $x$ to the complex strip $\mathcal S_\delta$
(Chapter \ref{cptr03:xContin}).

% with this 
% chapter's main results summarized below in Theorem's \ref{thm1:GFRep} and 
% \ref{thm1:krep}. This analysis is continuo

% The focus of this chapter is to provide a heuristic motivation for our definitions
% of $G_L$ and $G_R$ (Section \ref{sec1:Motivation}), and study the properties 
% of the lower boundary values $G_L^+$ and $G_R^+$ as functions on $\mathbb R$ (Sections)

% the Intermediate Long Wave equation 
% Inverse Scattering Transform



% In this chapter, we study the basic mapping properties of the lower boundary values $G_L^+$, $G_R^+$
% for the Green's functions $G_L$ and $G_R$ introduced in equation \eqref{eq0:GFs} of Section 
% \ref{sec0:DM}. In particular, we ultimately show in Section \ref{sec1:MappingProps} that
% $G_L^+$ and $G_R^+$ taken as convolutions operators are bounded linear operators from 
% $L^1(\mathbb R) \cap L^p(\mathbb R)$ to $L^\infty(\mathbb R)$, for $p \in (1, 2]$, which 
% satisfy
% \begin{align}\label{eq1:introMappingProp}
% 	\lim_{x\to-\infty} \big(G_L^+ * f\big) (x) 
% 		= \lim_{x\to+\infty} \big(G_R^+ * f\big) (x)
% 		= 0
% \end{align}
% for $f \in L^1(\mathbb R) \cap L^p(\mathbb R)$. 

% This result helps us to analyze the integral equations \eqref{eq0:JostIE} in Chpater 
% \ref{cptr04:DM} and prove
% that solutions to these equations satisfy the asymptotic conditions given in 
% \eqref{eq0:JostDEasymp}.

% Central to proving this result is the is the following representation theorem 
% (Theorem \ref{thm1:GFRep}) derived in Section \ref{sec1:GreensFunctions} 
% and Theorem \ref{thm1:krep}, which establishes some important bounds on $G_L^+$
% and $G_R^+$. The proof for Theorem \ref{thm1:krep} is given in Sections 
% \ref{sec1:AsympK} and \ref{sec1:KSingularity}.

We summarize the primary results of this chapter below in Theorems
\ref{thm1:GFRep} and \ref{thm1:krep}.

\begin{thm}[Green's Function Representation]\label{thm1:GFRep}
	The Green's functions given above in \eqref{eq0:GFs} can be 
	written as
	\begin{subequations}
		\label{eq1:GFrepLong}
		\begin{align}\label{eq1:GLrepLong}
			G_L^+(x; \lambda, \delta)
				=
					\begin{cases}
						K^+(x; \lambda, \delta) 
							+ i
							\big[ 
								\alpha(\lambda; \delta) 
								+ \beta(\lambda; \delta) e^{i\lambda x}
							\big] \chi_L(x)
							& \lambda \ne 0 \\
						K^+(x; \lambda, \delta) 
							+ i
							\left[ 
								\frac{2}{3} + i \frac{x}{\delta}
							\right] \chi_L(x)
							& \lambda = 0
					\end{cases}
		\end{align}
		and
		\begin{align}\label{eq1:GRrepLong}
			G_R^+(x; \lambda, \delta)
				=
					\begin{cases}
						K^+(x; \lambda, \delta) 
							- i
							\big[ 
								\alpha(\lambda; \delta) 
								+ \beta(\lambda; \delta) e^{i\lambda x}
							\big] \chi_R(x)
							& \lambda \ne 0 \\
						K^+(x; \lambda, \delta) 
							- i
							\left[ 
								\frac{2}{3} + i \frac{x}{\delta}
							\right] \chi_R(x)
							& \lambda = 0
					\end{cases}
		\end{align}
	\end{subequations}
	where $\chi_L := \chi_{(0, \infty)}$ and $\chi_R := \chi_{(-\infty, 0)}$ respectively
	denote the characteristic functions on the intervals $(0, \infty)$ and $(-\infty, 0)$, 
	\label{sym:chi}
	\begin{align*}
		\alpha(\lambda; \delta) 
			&:= \frac{1}{1-2\delta\zeta}  
			= \frac{1 - e^{2\delta\lambda}}
				{2\delta\lambda e^{2\delta\lambda} + 1 - e^{2\delta\lambda}},
			\\[1\baselineskip]
		\beta(\lambda; \delta) 
			&:= \frac{1}{1-2 \delta \zeta^*} 
			= \frac{1-e^{2\delta\lambda}}{1+2\delta\lambda-e^{2\delta\lambda}},
	\end{align*}
	\label{sym:alphabeta}
	are respectively determined by the residues of the integrand of $G_L^+$
	and $G_R^+$ at $\xi=0$ and $\xi=\lambda$, $\zeta^*$ is the non-linear reflection
	given by\label{sym:zetastar}
	\[
		\zeta^* := \zeta(-\lambda) = \zeta\big( - \lambda(\zeta) \big)
	\]
	and
	\begin{align*}
		K^+(x; \lambda, \delta) 
				:= \frac{e^{-\pi |x|}}{2\pi} 
					\int_{\mathbb R} e^{i x \xi} 
						\frac{1}
							{
								\xi - \zeta(\lambda) 
								\big( 
									1-e^{-2\xi \delta} 
								\big) 
								+ i \sign(x) \pi
							}
					\, \mathrm{d}\xi
	\end{align*}
	results from shifting the contour of integration for the integral in $G_L^+$ and
	$G_R^+$.
\end{thm}


\begin{thm}\label{thm1:krep}
	Suppose $\alpha$, $\beta$, and $K^+$ are as defined in Theorem \ref{thm1:GFRep}.
	Then the functions $\alpha$ and $\beta$ satisfy the following properties
	\begin{align}
		\lim_{\lambda\to0}|\alpha(\lambda; \delta)| 
			= \lim_{\lambda\to0}|\beta(\lambda; \delta)| 
			= \infty,\nonumber \\
		\intertext{and} 
		\label{eq1:PoleColapseResidue}
		\lim_{\lambda\to0} \big[\alpha(\lambda; \delta) + \beta(\lambda; \delta)e^{ix\lambda}\big]
			=  \frac{2}{3} + i \frac{x}{\delta}.
		% \lim_{\lambda \to -\infty} \alpha(\lambda) 
		% 	&= 1 
		% 		& \lim_{\lambda \to \infty} \alpha(\lambda) 
		% 			&= 0 \\
		% \lim_{\lambda \to -\infty} \beta(\lambda) 
		% 	&= 0 
		% 		& \lim_{\lambda \to \infty} \beta(\lambda) 
		% % 			&= 1 \\
		% \lim_{\lambda\nearrow0} \alpha(\lambda) 
		% 	&= \infty 
		% 		& \lim_{\lambda\searrow0} \alpha(\lambda) 
		% 			&= -\infty \\
		% \lim_{\lambda\nearrow0} \beta(\lambda) 
		% 	&= -\infty 
		% 		& \lim_{\lambda\searrow0} \beta(\lambda) 
		% 			&= \infty,
	\end{align}
	% and $\lim_{\lambda\to0} \alpha(\lambda) + \beta(\lambda) e^{ix\lambda} 
	% = \frac{2}{3} + i x$.
	Further, $K^+$ is uniformly bounded in $\lambda$ and
	\begin{align}
		K^+(x; \lambda, \delta) = 
			\begin{cases}
				C \log^+\left(\frac{1}{|x|}\right) + \mc O(1), & |x| < 1 \\
				\mc O\( \frac{e^{-\pi|x|}}{|x|} \), & |x| \geq 1
			\end{cases}
	\end{align}
	for some constant $C \in \mathbb C$, where $\mathbb C$\label{sym:Complex} 
	denotes the set of all complex numbers and $\log^+$ is the function defined
	by $\log^+(x):=\max\big\{ \log(x), \, 0 \big\}$.\label{sym:logplus}
\end{thm}

\begin{rmk}\label{rmk1:littlek}
	An important and immediate consequence of Theorem \ref{thm1:krep} and our
	work in Sections \ref{sec1:AsympK} and \ref{sec1:KSingularity} is that 
	$K^+$ can be written as
	\begin{align*}\label{eq1:littlek}
		K^+(x; \lambda, \delta) = e^{-\pi|x|} k(x; \lambda, \delta)
	\end{align*}
	where $k(\dotarg; \lambda, \delta) \in L^2(\mathbb R)$ and $k$ is uniformly 
	bounded in $\lambda$ for all real $\lambda$. Unless necessary to avoid 
	confusion, we commonly write $k(x; \lambda, \delta)$ as $k(x)$.
\end{rmk}

We begin this chapter by first motivating our choice of Green's functions
in Section \ref{sec1:RootsOfP}. 
Since we need to know the locations of the 
Green's functions' integrand singularities to be able to both justify 
the contours of integration for the Green's functions and
to justify the representation theorem above (Theorem \ref{thm1:GFRep}), 
locating these singularities is also a primary task for Section 
\ref{sec1:RootsOfP}.

To simplify notation, throughout the remainder of this dissertation, we use notation
\[
	p(\xi; \lambda, \delta) := \xi - \zeta(\lambda) \big( 1-e^{-2\xi \delta} \big),
	\label{sym:GFintegrand}
\]
where $e^{i\xi x} / p(\xi; \lambda, \delta)$ is the integrand for $G_L^+$ and $G_R^+$.
Since it is occasionally more useful to consider the Green's functions as parameterized 
by $\zeta$ rather than $\lambda$, we use the notation $p(\xi; \lambda, \delta)$ and
$p(\xi; \zeta, \delta)$ interchangeably. Further, we will not always need to consider
the af{}fects of the parameters $\lambda$ and $\delta$ in our subsequent analyses. In 
such cases, we often use the shorter notation $p(\xi)$ or $p(\xi;\lambda)$ \textit{en lieu} 
of the more cumbersome $p(\xi; \lambda, \delta)$ or $p(\xi; \zeta, \delta)$.

As we see in Section \ref{sec1:RootsOfP}, the only roots of $p(\xi; \lambda, \delta)$ 
in the complex strip 
\[
	\mathcal R_\delta := \{z \in \mathbb C ~:~ -\pi/\delta \leq \im z \leq \pi/\delta \}	
\]
are $\xi = 0$ and $\xi = \lambda$ (provided $\lambda \ne 0$). As such, we may
use analyticity to write $G_L^+$ and $G_R^+$ as \label{sym:GFbndry}
\begin{align*}
	G_L^+(x; \lambda, \delta)
		&:= 
			\frac{1}{2\pi} 
			\int_{\Gamma_L} 
				e^{i\xi x} \frac{1}{p(\xi; \lambda, \delta)} 
			\, \mathrm{d}\xi \\
	G_R^+(x; \lambda, \delta) 
		&:= 
			\frac{1}{2\pi} 
				\int_{\Gamma_R} 
					e^{i\xi x} \frac{1}{p(\xi; \lambda, \delta)} 
				\, \mathrm{d}\xi,
\end{align*}
where the symbol $\Gamma_L$ \label{sym:Gamma} is used to denote a contour from $-\infty$ to 
$\infty$ along the real axis which is deformed in small circular arcs around 
$\xi = 0$ and $\xi=\lambda$ so that the contour bypasses these two real roots 
of $p$ from below (Figure \ref{fig1:GammaL}), and $\Gamma_R$ denotes the 
corresponding contour which bypasses
the roots $\xi = 0$ and $\xi=\lambda$ from above (Figure \ref{fig1:GammaR}). 

\begin{figure}[H]
	\centering
	\def\outSpacing{1.3}
	\def\innSpacing{1.3}
	% \def\ep{.65}
	\def\ep{.3}
	\def\lambdaam{\innSpacing+\ep+\ep}
	\def\arrowsize{15mm}
	\def\dotsize{1pt}
	\begin{subfigure}[t]{0.49\textwidth}
		\centering
		\begin{tikzpicture}[
				directed/.style={
					decoration={markings, mark=at position .55 with \arrow{stealth}[arrowhead=\arrowsize]},
						postaction={decorate}
				},
				cont/.style={smooth, thick}
			]
			

			% Draw the contour
			\draw[cont, directed] ({-\outSpacing-\ep},0) -- ({-\ep}, 0);
			\draw[cont, directed] ({\ep}, 0) -- ({\lambdaam - \ep},0);
			\draw[cont, directed] 
				({\lambdaam + \ep},0) -- ({\lambdaam + \ep + \outSpacing},0);


			\draw[cont, domain=180:360, variable=\th]
				plot ({\ep*cos(\th)}, {\ep*sin(\th)});
			\draw[cont, domain=180:360, variable=\th]
				plot ({\ep*cos(\th)+\lambdaam}, {\ep*sin(\th)});

			% Draw zeros
			\tkzDefPoint(0,0){zero}
			\tkzDefPoint(\lambdaam,0){lambda}
			\tkzLabelPoint[above](zero){$\xi=0$}
			\tkzLabelPoint[above](lambda){$\xi=\lambda$}
			\tkzLabelPoint[below](zero){\phantom{$\xi=0$}}
			\tkzLabelPoint[below](lambda){\phantom{$\xi=\lambda$}}
			\foreach \n in {zero, lambda}
				\node at (\n)[circle, fill, inner sep=\dotsize]{};
		\end{tikzpicture}
		\caption{$\Gamma_L$ Contour}
		\label{fig1:GammaL}
	\end{subfigure}
	\begin{subfigure}[t]{0.49\textwidth}
		\centering
		\begin{tikzpicture}[
				directed/.style={
					decoration={markings, mark=at position .5 with \arrow{stealth}[arrowhead=\arrowsize]},
						postaction={decorate}
				},
				cont/.style={smooth, thick}
			]

			%% Draw the contour
			\draw[cont, directed] ({-\outSpacing-\ep},0) -- ({-\ep}, 0);
			\draw[cont, directed] ({\ep}, 0) -- ({\lambdaam - \ep},0);
			\draw[cont, directed] 
				({\lambdaam + \ep},0) -- ({\lambdaam + \ep + \outSpacing},0);

			\draw[cont, domain=180:360, variable=\th]
				plot ({\ep*cos(\th)}, {-\ep*sin(\th)});
			\draw[cont, domain=180:360, variable=\th]
				plot ({\ep*cos(\th)+\lambdaam}, {-\ep*sin(\th)});


			%% Draw zeros
			\tkzDefPoint(0,0){zero}
			\tkzDefPoint(\lambdaam,0){lambda}
			\tkzLabelPoint[above](zero){\phantom{$\xi=0$}}
			\tkzLabelPoint[above](lambda){\phantom{$\xi=\lambda$}}
			\tkzLabelPoint[below](zero){$\xi=0$}
			\tkzLabelPoint[below](lambda){$\xi=\lambda$}
			\foreach \n in {zero, lambda}
				\node at (\n)[circle, fill, inner sep=\dotsize]{};
		\end{tikzpicture}
		\caption{$\Gamma_R$ Contour}
		\label{fig1:GammaR}
	\end{subfigure}
	\caption{Contours of integration $\Gamma_L$ and $\Gamma_R$ for $G_L$ and $G_R$.}
	\label{fig1:GammaStarContours}
\end{figure}

\begin{rmk}\label{rmk1:StarNotation}
	Throughout this dissertation, we commonly use the symbol ``$\star$''
	as a place holder for both $L$ and $R$. For example, if we write ``$G_\star$ 
	($\star = L \text{, or } R$) are a bounded as a convolution operators,'' then 
	what we mean is that ``both $G_L$ and $G_R$ are bounded as convolution 
	operators.'' As a further example of how we use this notational convention, 
	please see the following two very important remarks.
\end{rmk}




\begin{rmk}\label{rmk1:ResidueLimits}
	When $\lambda = 0$, the function $\zeta(\lambda; \delta)$ is technically 
	undefined. However, since 
	$\lim_{\lambda\to0}\zeta(\lambda; \delta) = \frac{1}{2\delta}$, $\lambda = 0$
	is a removable singularity of $\zeta$. As such, we define 
	$\zeta(\lambda; \delta) := \frac{1}{2\delta}$. Further, the case $\lambda=0$ 
	also corresponds to the case when the two roots $\xi = 0$ and $\xi = \lambda$ 
	of the function $p$\textemdash{}which are simple when 
	$\lambda \ne 0$\textemdash{}coalesce to form a single double zero of $p$. Under 
	the caveat that we define $\zeta(0; \delta):= \frac{1}{2\delta}$, a direct computation 
	shows that the residue of $e^{ix\xi} / p(\xi; 0, \delta)$ at $\xi = 0$ is 
	$\frac{2}{3} + i \frac{x}{\delta}$\textemdash{}hence the piecewise (in $\lambda$)
	definition of $G_L^+$ and $G_R^+$ in \eqref{eq1:GFrepLong}.
	Thus, even though the residue sum 
	\[
		\mathpzc R_\star (x; \lambda, \delta) 
			:= i 
				\left[
					\alpha(\lambda; \delta) + \beta(\lambda; \delta) e^{i\lambda x}
				\right] \chi_\star
			\qquad (\star = L \text{, or } R)
	\]
	is not technically defined at 
	$\lambda = 0$, it nonetheless makes sense for us to agree on the convention 
	that
	\[
		\mathpzc R_\star(x; \lambda = 0; \delta)
			:= \left[ i \frac{2}{3} - \frac{x}{\delta} \right] \, \chi_\star.
		\qquad (\star = L \text{, or } R)
	\]
	Under this convention, \eqref{eq1:GFrepLong} can be written slightly more succinctly as
	\begin{subequations}
		\label{eq1:GFrep}
		\begin{align}
			\label{eq1:GLrep}
			G_L^+(x; \lambda, \delta)
				&=
					K^+(x; \lambda, \delta) 
					+ \mathpzc R_L(x; \lambda, \delta)\\
			\label{eq1:GRrep}
			G_R^+(x; \lambda, \delta) &= 
				K^+(x; \lambda, \delta) 
					- \mathpzc R_R(x; \lambda, \delta).
		\end{align}
	\end{subequations}
\end{rmk}




\begin{rmk}\label{rmk1:ContourAdjustment}
	Continuing to our discussion on the case of $\lambda = 0$ and the 
	coalescing of the Green's function integrand poles $\xi = 0$, $\xi = \lambda$,
	under this scenario, we take the contours $\Gamma_\star$ 
	($\star = L \text{, or } R$) such that they have only one circular deformation
	away from the real line which allows them to bypass the single (double) pole
	at $\xi = 0$. 
\end{rmk}



\end{document}