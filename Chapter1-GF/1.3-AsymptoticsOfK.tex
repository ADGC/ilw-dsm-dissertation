%%=================================
%% Section 1.04: Asymptotics of K^+
%%=================================
\documentclass[../dissertation.tex]{subfiles}
\def\rest{\text{rest}}

\begin{document}

\section{Asymptotics of $K^+$}\label{sec1:AsympK}
A cursory inspection of the integrand of $G_\star^+$ ($\star = L \text{, or }R$) 
indicates that $G_\star^+$ behavior
differs wildly for $x \in \mathbb R$ near zero, and $|x| > c$, where $c>0$ is any
fixed constant. While $1/p(\xi)$ decays roughly exponentially as $\xi \to -\infty$, 
the fact that $1/p(\xi)$ decays as $1/\xi$ for $\xi \gg 1$ means the oscillatory term 
$e^{ix\xi}$ in the integrand of $G_\star^+$ is imperative for the contour 
integral in $G_\star^+$ of $e^{ix\xi} / p(\xi)$ to even have a chance for convergence.
In this section we study the ``nice'' case of $|x|>1$ and show that not only does 
the integral in $G_\star^+$ converge in this scenario, it is rapidly decaying (as long 
as $x$ stays away from zero). In the Section \ref{sec1:KSingularity}, we study
the behavior of $G_\star^+$ for $x$ near zero and show $G_\star^+$ has at worst a log 
type singularity at $x=0$. Taken together, the results from this section 
combined with the results from Section \ref{sec1:KSingularity} constitute a 
proof of Theorem \ref{thm1:krep} from the introduction of this chapter. 

Key to the analysis in both this section and Section \ref{sec1:KSingularity}
are the representation formulas \eqref{eq1:GFrep} proven in Section 
\ref{sec1:GreensFunctions}. In particular, \eqref{eq1:GFrep} allows us
to reduce our analyses to a thorough study of $K^+$

To understand the properties of $K^+$, we study the convergence of the integral
\begin{align*}
	\int\limits_{\Sigma_{\sign(x)}} \frac{e^{ix\xi}}{p(\xi)} \, \mathrm{d}\xi,
\end{align*}
where $\Sigma_{\sign(x)} := \mathbb R + i \sign(x) \pi$.\label{sym1:SigRealLine}
Let $\Sigma\big(R,~\sign(x)\big)$\label{sym1:SigR} denote the contour 
$(-R, R) + i \sign(x) \pi$. Recall 
that $p(\xi)$ can be written as 
\[
	p(\xi) = \xi - \zeta(\lambda)\( 1 - e^{-2\xi} \),
\] 
which implies $p'(\xi) = 1 - 2 \zeta(\lambda) \, e^{-2\xi}$. In which case
\begin{align*}
	\int\limits_{\Sigma\big(R,~\sign(x)\big)} \frac{e^{ix\xi}}{p(\xi)} \, \mathrm{d}\xi
		&=  \left.\frac{e^{ix\xi}}{p(\xi)}\right|_{-R + i \sign(x)\pi}^{R + i \sign(x)\pi}
			- \frac{1}{ix} \int\limits_{\Sigma\big(R,~\sign(x)\big)} e^{ix\xi} \, 
			\frac{p'(\xi)}{\big(p(\xi)\big)^2} \, \mathrm{d}\xi.
\end{align*}
Now 
\begin{align*}
	p'(t \pm i \pi) 
		= 1 - 2 \zeta(\lambda) e^{-2t} e^{\mp 2 \pi i}
		= 1 - 2 \zeta(\lambda) e^{-2t} 
		= p'(t), \qquad t\in \RR.
\end{align*}
Further
\[
	\zeta(t \pm i \pi ) - \zeta(\lambda)
		= \frac{t}{1 - e^{-2t}} - \zeta(\lambda) \pm \frac{i\pi}{1 - e^{-2t}}
		= \zeta(t) - \zeta(\lambda) \pm \frac{i\pi}{1 - e^{-2t}}, \qquad t\in \RR,
\]
which implies
\begin{align} \label{eq1:shiftedP}
	p(t \pm i \pi)
		&= \big(\zeta(t \pm i \pi) - \zeta(\lambda) \big)\( 1- e^{-2t} \) \\
		&= \big(\zeta(t) - \zeta(\lambda)\big)\(1 - e^{-2t}\) \pm i\pi
		= p(t) \pm i \pi. \nonumber
\end{align}
Combining the two calculations above, we see that
\begin{align*}
	\frac{1}{ix} \int\limits_{\Sigma\big(R,~\sign(x)\big)} e^{ix\xi} \, 
			\frac{p'(\xi)}{\big(p(\xi)\big)^2} \, \mathrm{d}\xi
		= \frac{e^{-|x|\pi}}{ix} \int_{-R}^R e^{ixt} 
			\frac{p'(t)}
			{\big( p(t) + i \sign(x) \pi \big)^2} \, \mathrm{d}t.
\end{align*}	
Thus, 
\[
	\lim_{R\to\infty} \frac{1}{\left|p\big(\pm R + i \sign(x)\pi\big)\right|}=0
\]
and, formally,
\begin{align} \label{eq1:RemainderIntegralBnd}
	\left| \int\limits_{\Sigma_{\sign(x)}} \frac{e^{ix\xi}}{p(\xi)} \, \mathrm{d}\xi \right|
		\leq \frac{e^{-|x|\pi}}{|x|} \int_{\RR} 
			\frac{\left| p'(t) \right|}{p(t)^2 + \pi^2} \, \mathrm{d}t,
\end{align}
where we proceed to establish the convergence of the integral on the right-hand
side of \eqref{eq1:RemainderIntegralBnd}. Note that $p'(t) = 0$ only when
$t = \frac{1}{4} \ln\big( \zeta(\lambda) \big)$. Further, given 
$p''(t) = 4\zeta(\lambda)\,e^{-2t} > 0$ for all real $t$, 
\[
	|p'(t)| =
		\begin{cases}
			-p'(t), & t < t_0 \\
			p'(t), & t \geq t_0,
		\end{cases}
\]
where $t_0:= \frac{1}{4} \ln\big( \zeta(\lambda) \big)$. Given this fact, we now proceed
to evaluate the integral on the right-hand side of \eqref{eq1:RemainderIntegralBnd}
through $u$-substitution by setting $u = p(t)$. Observe that
\begin{align}
	\int \frac{p'(t)}{p(t)^2 + \pi^2} \, \mathrm{d}t
		= \int \frac{1}{u^2 + \pi^2}
		= \frac{1}{\pi} \arctan\( \frac{u}{\pi} \) + C
		= \frac{1}{\pi} \arctan\( \frac{p(t)}{\pi} \) + C.
\end{align}
So, for $R > |t_0|$, 
\[
	\int_{-R}^{t_0} \frac{|p'(t)|}{p(t)^2 + \pi^2}\,dt
		= - \int_{-R}^{t_0} \frac{p'(t)}{p(t)^2 + \pi^2}
		= -\frac{1}{\pi} \left[ \arctan\(\frac{p(t_0)}{\pi}\) 
			- \arctan\(\frac{p(-R)}{\pi}\)\right],
\]
and
\[
	\int_{t_0}^R \frac{|p'(t)|}{p(t)^2 + \pi^2}\,dt
		= \frac{1}{\pi} \left[\arctan\(\frac{p(R)}{\pi}\) 
			- \arctan\(\frac{p(t_0)}{\pi}\)\right].
\]
Now
\[
	\lim_{R\to \infty} p(\pm R) 
		= \lim_{R\to \infty} \left[ \pm R - \zeta(\lambda) \( 1 - e^{\mp 2R} \) \right]
		= \infty,
\]
which implies that
\begin{align*}
	\lim_{R\to \infty} \int_{-R}^R \frac{|p'(t)|}{p(t)^2 + \pi^2}\,dt
		&= \lim_{R\to\infty} \(\int_{-R}^{t_0} \frac{|p'(t)|}{p(t)^2 + \pi^2}\,dt
			+ \int_{t_0}^{R} \frac{|p'(t)|}{p(t)^2 + \pi^2}\,dt \)\\
		&= \frac{1}{\pi}\left[ \lim_{R\to\infty} \arctan\(\frac{p(R)}{\pi}\) 
				+\lim_{R\to\infty} \arctan\(\frac{p(-R)}{\pi}\) \right. \\
		&\qquad\quad- \left. 2 \arctan\(\frac{p(t_0)}{\pi}\) \right] \\
		&= 1 - \frac{2}{\pi} \arctan\( \frac{p(t_0)}{\pi} \)
\end{align*}

Since
\[
	p(t_0) 
		= \frac{1}{4} \ln\zeta(\lambda) - \zeta(\lambda) 
			+ \big(\zeta(\lambda) \big)^{\frac{1}{2}},
\]
$p(t_0) \to - \infty$ as $\lambda \to \pm \infty$, and 
\[
	\lim_{\lambda\to\pm\infty} \int_\RR \frac{|p'(t)|}{p(t)^2 + \pi^2}\,dt = 2.
\]
Putting everything together, we see that
\begin{align}\label{eq1:Kasymp}
	K^+(x; \lambda) = \mc O\left( \frac{e^{-|x| \pi}}{|x|} \right)
\end{align}
for $|x| \geq 1$.
% \begin{align}\label{eq1:FinalGffromula}
% 	G_L^+(x, \lambda, 1) =
% 		\begin{cases}
% 			i \frac{1 - e^{2\lambda}}{2\lambda\,e^{2\lambda} - e^{2\lambda}+1} 
% 				+ i \frac{1 - e^{2\lambda}}{2\lambda - e^{2\lambda}+1} e^{ix\lambda}
% 				+ \mc O\( \frac{e^{-|x| \pi}}{|x|} \), & x > 1 \\ 
% 			~\\
% 			\mc O\( \frac{e^{-|x| \pi}}{|x|} \), & x < -1,
% 		\end{cases}
% \end{align}
% where the $\mc O\( \frac{e^{-|x| \pi}}{|x|} \)$ terms are uniformly bounded in 
% $\lambda$.

% Finally, since
% \[
% \begin{array}{ccc}
% 	\ds \lim_{\lambda\to-\infty} \frac{1 - e^{2\lambda}}{2 \lambda e^{2\lambda} - e^{2\lambda}+1} = 1, 
% 	&\qquad&
% 	\ds \lim_{\lambda\to\infty} \frac{1 - e^{2\lambda}}{2 \lambda e^{2\lambda} - e^{2\lambda}+1} = 0, \\
% 	\ds \lim_{\lambda\to-\infty} \frac{1 - e^{2\lambda}}{2 \lambda - e^{2\lambda}+1} = 0, 
% 	&\qquad& 
% 	\ds \lim_{\lambda\to\infty} \frac{1 - e^{2\lambda}}{2 \lambda - e^{2\lambda}+1} = 1,
% \end{array}
% \]
% we see from \eqref{eq1:Gffraw} that $|G_L^+(x; \lambda)|$ is uniformly bounded in $\lambda$.

\end{document}
