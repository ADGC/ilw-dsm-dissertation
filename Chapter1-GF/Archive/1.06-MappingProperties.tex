%%==========================================
%% Section 1.06: Mapping Properties of G_L^+
%%==========================================
\documentclass[../dissertation.tex]{subfiles}
\def\rest{\text{rest}}

\begin{document}

\section{Mapping Properties of $G_L^+$ and $G_R^+$}\label{sec1:MappingProps}

%%
%% The following was taken directly (with a modification of the notation) 
%% from from Prof. Perry's section of the kpw1.tex file
%% 
With Theorems \ref{thm1:GFRep} and \ref{thm1:krep}, we now have the necessary
tools to turn our attention to proving the mapping property \eqref{eq1:introMappingProp}
claimed in this chapter's introduction. From our previous work in Section 
\ref{sec1:GreensFunctions}, we have
\begin{align}
	G_L^+(x; \lambda, \delta) 
		= 
			\begin{cases}
				K^+(x; \lambda, \delta), & x < 0 \\
				K^+(x; \lambda, \delta)
				i \alpha(\lambda; \delta)  
					+ i \beta(\lambda; \delta) e^{i \lambda \, x}, 
					& x > 0
			\end{cases}
\end{align}
where 
\begin{align*}
	K^+(x; \lambda, \delta) 
			:= \frac{e^{-\pi |x|}}{2\pi} 
				\int_{\mathbb R} e^{i x \xi} 
					\frac{1}{p(\xi; \lambda, \delta) + i \pi \sign(x)}
				\, \mathrm{d}\xi.
\end{align*}
Although the integral for $K^+$ is conditionally convergent, it avoids zeros of the symbol 
$p$, and may be understood through the $L^q$ theory of the Fourier transform, since the 
integrand belongs to $L^q$ for any $q>1$.  Moreover, if $K_h^+(x) = K^+(x+h)$, it follows 
from the Hausdorff-Young inequality and dominated convergence that 
$\lim_{h \to 0} \norm[L^{q'}]{K_h^+ - K^+} =0$ for any $q \in (1,2]$. As a consequence,
the convolution
\begin{align}
	\label{ILW.GL*f}
	G_L^+ * f(x,\lambda) 	
		&=	\int_{\mathbb R} G_L^+(x-x',\lambda) f(x') \, \mathrm{d}x' 	\\
		&=	i \alpha (\lambda) \int_{-\infty}^x f(x') \, \mathrm{d}x' 
			+ i \beta(\lambda) e^{i\lam x} \int_{-\infty}^x f(x') \, \mathrm{d}x'
			\nonumber \\
		&\quad 	- \int_{-\infty}^x K^+(x-x',\lambda) f	(x') \, \mathrm{d}x' \nonumber\\
		&\quad	- \int_x^\infty K^+(x-x',\lambda) f(x') \, \mathrm{d}x'\nonumber
\end{align}
defines a bounded continuous function for any $f \in L^1(\RR) \cap L^p(\RR)$ for any $p \in (1,2]$ with
\begin{equation}
	\label{ILW.GL.bd}
	\norm[L^\infty(\mathbb R)]{G_L^+*f} \lesssim \norm[L^1 \cap L^p]{f}.
\end{equation}
For $f \in C_0^\infty(\mathbb R)$, it is easy to see that
$\lim_{x \to -\infty} G_L^+ * f(x;\lambda) = 0$. It now follows from a density 
argument and \eqref{ILW.GL.bd} that if $p \in (1,2]$, then
\begin{equation}
	\label{ILW.GL.vanish}
	\lim_{x \to -\infty} \left(G_L^+*f\right)(x) = 0
\end{equation}
for any $f \in L^1(\RR) \cap L^p(\RR)$. A similar analysis shows that 
\begin{equation}
	\label{ILW.GR.vanish}
	\lim_{x \to +\infty} \left(G_R^+*f\right)(x) =0
\end{equation}
for any such $f$.


{\color{red}Move definition of space $X$ to here.}

\begin{prop}
	
\end{prop}


\end{document}