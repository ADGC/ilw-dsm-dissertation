%%========================================================
%% Section: Analytic Continuation of the Greens Functions
%%========================================================


\documentclass[GreensFunctions.tex]{subfiles}

\begin{document}
\section{Analytic Continuation of $m_+$}

Let $G(z, \z)$ denote the extension (in $x$) of $G_+$ to the complex strip
$0 < \im z < 2$. Note that $G(z)$ is bounded in this strip along each 
line of the form $\im z = k$ for each fixed $k$ with $0 < k < 2$,
as the integral
\[
	G(x + iy) 
		= \frac{1}{2 \pi} \int_{\R - i \ve} e^{ix \xi} 
				\frac{e^{-y\xi} \, \mathrm{d}\xi}{\xi - \z (1-e^{-2\xi})}
\]
is absolutely convergent (for $x \in \R$ and $0 < y < 2$). Further, 
using our previously derived formula for $G_+$, one can show that for
sufficiently nice $f$, the function
\[
	F_R(z) := \int_{|x'| < R} G(z - x') f(x') \, \mathrm{d}x'
\]
is analytic

% We consider $z = x + i y \in \C$ for $ 0 < y < 2$, and 
% define
% \[
% 	G_y(x) 
% 		:= \int_{\R} \frac{e^{-y \xi}}{p(\xi, \l) +\sign(x) \pi} 
% 			e^{ix \xi }  \, \mathrm{d}\xi.
% \]
% In analogy to our previous work, in this section we consider the convergence
% of $G_y(x)$---beginning with the case $y\ll 1$.

% %%===========================
% %% Subsection 01: Small \im z
% %%===========================
% \subsection{Small $\im z$} In this subsection we reproduce Allen's work 
% from his worklog in an effort to modify it to the current situation.
% Throughout this subsection, we assume $x > 0$, and note that analogous 
% results hold for the $x< 0$ case. With this assumption, we write $\z$
% to stand for $\z(\l)$ and consider
% three major cases: {\bf I.} small $\z$, {\bf II.} large $\z$, and {\bf III.} 
% complex $\z$. In all three cases, we break $G_y$ up into a dyadic decomposition
% \[
% 	G_y(x) = \sum_{p \in \Z} G_p(x),
% \]
% where
% \[
% 	G_p(x) 
% 		:= \int_{\R} \frac{e^{-y \xi} \chi(2^{-p} x \xi)}
% 							{p(\xi, \l) +\sign(x)\pi}
% 				e^{ix \xi }  \, \mathrm{d}\xi
% 		= \int_{\R} \frac{e^{-y \xi} \chi(2^{-p} x \xi)}
% 							{p(\xi, \l) + \pi}
% 				e^{ix \xi }  \, \mathrm{d}\xi,
% \]
% and $\chi$ is an even, smooth cut-off function with support near
% $|\xi| = 1$.

% \noi{\bf Case I.} As in Allen's worklog, we divide this case up
% into three subcases: values of $p$ for which {\bf (IA)} $2^{-p} x \xi \in \supp
% \chi$ and $\frac{1}{2} < \frac{|\xi|}{|\l|} < 2$, {\bf (IB)} 
% $x \lesssim 2^p \lesssim 1$, and {\bf (IC)} $p \gtrsim 1$. Note that
% in subcases {\bf (IB)} and {\bf (IC)}, we assume $p \in \Z$ is such that
% $|\xi| \leq \frac{1}{2}|\l|$ or $|\xi| \geq 2 |\l|$ whenever $2^{-p} x \xi
% \in \supp \chi$.

% \noi{\bf Subcase (IA)} As Allen notes there are at most five $p$ which 
% satisfy this subcase. Indeed, we may assume 
% $\frac{|\xi|}{|\l|} \approx \frac{2^p}{|\l| x}$---else $\chi(2^{-p} x \xi ) = 0$.
% Hence, 
% \[
% 	\frac{|\xi|}{|\l|} \approx \frac{2^p}{|\l| x}.
% \]
% Solving the inequalities $\frac{1}{2} < \frac{2^p}{|\l| x} < 2$ for $p$
% yields
% \[
% 	\frac{\log\big(|\l| x \big)}{\log(2)} - 1 
% 		< p 
% 		< \frac{\log\big(|\l| x \big)}{\log(2)} + 1.
% \]
% Accounting for the fact that $p$ is an integer and $0 < |\supp \chi| \ll 1$, 
% Allen's assertion follows.


\end{document}