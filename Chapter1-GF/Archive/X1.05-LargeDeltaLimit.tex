%%========================================================
%% Difference of BO and ILW Greens functions' singularity
%%========================================================

\documentclass[GreensFunctions.tex]{subfiles}

\def\BO{\text{BO}}
\def\fourier{\mathcal{F}}
\def\rest{\text{rest}}

\begin{document}
\section{Large $\d$ Limit}\label{sec:LargeDeltaLimit}

Let $\chi_+$ denote the characteristic function on $(0, \infty)$, and define
$p_\BO := (\xi - \l) \chi_+(\xi)$, and recall that
\[
	g_\pm(x, \l) = \fourier^{-1}\(\frac{\chi_+(\xi)}{p_\BO(\xi, \l \pm i0)}\),
\]
is the Greens Function for the Benjamin-Ono equation, where $\fourier$ denotes
the Fourier transform.
In this section, we argue $G_+ \to g_+$ 
as $\d \to \infty$. 

We begin by considering the limit $\z(\xi)$, $\d \to \infty$. Recall
\[
	\z(\xi) := \frac{\xi e^{\d \xi}}{e^{\d \xi} - e^{-\d\xi}}.
\]
Since $\lim_{\xi \to 0} \z(\xi) = \frac{1}{2\d}$, by l'H\^opital's rule, 
we have that
\begin{align*}
	\lim_{\d \to \infty} \z(\xi)
		&= \xi \lim_{\d \to \infty}	\frac{e^{\d \xi}}{e^{\d \xi} - e^{-\d\xi}} \\
		&= \xi \lim_{\d \to \infty}
			\begin{cases}
				\ds\frac{e^{\d}}{e^{\d} - e^{-\d}}   & \xi > 0 \\
				\ds\frac{1}{2 \d}, 			      & \xi = 0 \\
				\ds\frac{e^{-\d}}{e^{-\d} - e^{\d}}, & \xi <0
			\end{cases} \\
		&=
			\begin{cases}
				\xi, & \xi > 0 \\
				0,   & \xi \leq 0
			\end{cases},
\end{align*}
as 
\begin{align*}
	\lim_{\d \to \infty} \frac{e^{\d}}{e^{\d} - e^{-\d}} 
		= \lim_{\d \to \infty} \frac{1}{1 - e^{-2\d}} 
		= 1, \qquad \text{and} \qquad 
	\lim_{\d \to \infty} \frac{e^{-\d}}{e^{-\d} - e^{\d}}
		= \lim_{\d \to \infty} \frac{1}{1 - e^{2\d}} 
		=0.
\end{align*}
That is, $\z(\xi)$ converges pointwise  to the function $\xi \chi_+(\xi)$ as 
$\d \to \infty$. 
% Further, $\(1- e^{-2 \d \xi}\) \to -\infty$ as $\d \to \infty$
% whenever $\xi < 0$. 
Remembering from \eqref{eq:FactoredP} that $p$ can 
be factored as
\[
	p(\xi , \l) = \big( \z(\xi) - \z(\l) \big) (1 - e^{-2\d \xi}),
\]
we compute the pointwise limit
\begin{align}
	\lim_{\d \to \infty} \frac{1}{p(\xi, \l)} 
		&= 
		\begin{cases}
			\frac{1}{\xi - \l}, & \xi > 0 \\
			0, & \xi \leq 0
		\end{cases} \nonumber \\
		&= \frac{\chi_+(\xi)}{p_\BO(\xi, \l)}.
\end{align}

Note that for each fixed $\d > 0$ 
\[
	p(\xi, \l) 
		= \z(\l) e^{-2 \d \xi} + \xi - \z(\l)
		= O\( e^{-\xi} \),
\]
for $\xi< 0$ with $|\xi|$ sufficiently large. As such, it follows 
from the Dominated Convergence Theorem that 
\begin{align}
	\lim_{\d \to \infty} 
		\int_{-\infty}^{-\ve} \frac{e^{i x \xi}}{p(\xi, \l)} \, \mathrm{d}\xi
			= 0,
\end{align}
for every fixed $\ve > 0$.

Fix $\l >0$, and define the cut-off functions 
$\chi_\nu$, $\nu\in\{0, \, \l, \, \rest\}$
as in the beginning of Section \ref{sec:Gfformulas}. 
Since the only two zeros of the 
function $| p(\xi) \, p_\BO(\xi)|$ are $\xi = 0, \l$,
we claim that  there exists some
real number $M > 0$ so that 
\begin{align}\label{eq:MClaim}
	\big| p(\xi) \, p_\BO(\xi) \big| \geq M
\end{align}
for all $\xi \in \supp \chi_\rest \cap (0, \infty)$. Certainly, 
for each fixed $\d > 0$, there exists $M(\d) > 0$ so that
$|p(\xi)| \geq M(\d)$ for $\xi \in \supp \chi_\rest \cap (0, \infty)$. 
If $\limsup_{\d \to \infty} M(\d) = 0$, then, since $p$ converges pointwise
(in $\d$) to $p_\BO$ and both $p$ and $p_\BO$ diverge as $\xi \to \infty$
(regardless of the value of $\d$), this assumption would imply that
either $\l \pm 2\b$ or $2 \b$ are zeros of $p_\BO$---which clearly isn't
true. Consequently, given $p_\BO$ is independent of $\d$, we conclude
the existence of a real number $M>0$ satisfying  \eqref{eq:MClaim}.
Further, 
\begin{align*}
	\left|\frac{1}{p(\xi)} - \frac{1}{p_\BO(\xi)}\right|
		\leq \frac{1}{M} \( |\z(\l) - \l| + e^{-2 \ve \d}  \),
\end{align*}
for every $\xi \in \supp \chi_\rest \cap (0, \infty)$. Hence 
$1/p$ converges uniformly to $1/p_\BO$ on $\supp \chi_\rest \cap (0, \infty)$, 
which allows us to conclude that 
\begin{align}
	\lim_{\d\to \infty} 
		\int_0^\infty \frac{e^{i x \xi}}{p(\xi)} \, \chi_\rest(\xi) \, \mathrm{d}\xi
		= \int_0^\infty \frac{e^{i x \xi}}{p_\BO(\xi)} \, \chi_\rest(\xi) \, \mathrm{d}\xi.
\end{align}

Since $\l = \xi$ is an order one zero of $p$, there exists an
analytic function
$g(\xi):= g(\xi, \l, \d)$ so that
\[
	\frac{1}{p}(\xi) = \frac{a_{-1}}{\xi - \l} + g(\xi),
\]
where
\[
	a_{-1} 
		:= a_{-1} (\l , \d) 
		= \Res_{\xi = \l} \( \frac{1}{p} \)
		= \frac{e^{2 \d \l }-1}{e^{2 \d \l}-1-2 \d \l }
\]
is the minus first coefficient in the Laurent expansion of $1/p$.
It is easy to compute that $a_{-1} \to 1$ as $\d \to \infty$. Further, 
since $1/p$ converges pointwise to $1/(\xi - \l)$ (as $\d \to \infty$), 
the function $g$ must converge pointwise to zero. Therefore,
\begin{align*}
	\int_0^\infty e^{i x \xi} \, \chi_\l(\xi) 
		\(\frac{1}{p}(\xi) - \frac{1}{p_\BO}(\xi) \) \, \mathrm{d}\xi
	= \int_0^\infty e^{i x \xi} \, \chi_\l(\xi) 
		\(\frac{a_1 - 1}{\xi - \l} + g(\xi) \) \, \mathrm{d}\xi.
\end{align*}

We claim $\nm{g(\dotarg, \d)}_\infty \to 0$ as $\d \to \infty$. By 
way of contradiction, assume there exists some $\ve > 0$ so that
for each $N \in \N$ there exists $\d_N > 0$ satisfying 
$\nm{g(\dotarg, \d_N)}_\infty \geq \ve$. In which case, since
$\liminf_{N} \nm{g(\dotarg, \d_N)}_\infty \geq \ve$, the continuity 
of $g$ implies the existence of closed intervals $\{ I_N \}_{N \in \N}$
with $m:= \liminf_N |I_N| > 0$ so that $g(\dotarg, \d_N) \geq \ve/2$
on $I_N$. We may assume without loss of generality that
$|I_N| > m /2$. As such, $\{I_N\}$ must have a subsequence
with nontrivial intersection by Lemma \ref{lma:FiniteMeasureIntervalSequence}.
% ---otherwise, $\{I_N\}$ would have 
% a pairwise disjoint subsequence, which would violate the fact
% that $\supp \chi_\l$ has finite measure. 
So, by passing to 
a subsequence as necessary, there exists $x \in \cap_N I_N$, 
Hence, $\lim_{N \to \infty} g(x, \d_N) \geq \ve /2$, which
is a contradiction that validates the claim, as $g(\dotarg, \d_N)$ 
converges pointwise to zero. It therefore follows that
\begin{align}
	\lim_{\d \to \infty} \int_0^\infty e^{i x \xi} \, \chi_\l(\xi) 
		\(\frac{1}{p}(\xi) - \frac{1}{p_\BO}(\xi) \) \, \mathrm{d}\xi
	= 0,
\end{align}
as $\lim_{\d \to \infty}\left| a_{-1} - 1 \right| = 0$. 

% Now
% \begin{align*}
% 	\frac{1}{p(\xi)}
% 		&= \frac{1}{ \dfrac{\xi e^{\xi \d}}
% 			{e^{\xi\d}-e^{-\xi\d}}-\dfrac{\l e^{\l\d}}{e^{\l\d} - e^{-\l\d}}}
% 			\dotarg \frac{1}{1 - e^{-2\xi\d}} \\
% 		&= \frac{e^{\xi\d} e^{\l\d} -e^{\xi\d} e^{-\l\d} - e^{-\xi\d} e^{\l\d}+e^{-\xi\d}e^{-\l\d}}
% 			{(\xi - \l) e^{\xi\d} e^{\l\d} + \l e^{-\xi\d} e^{\l\d} - \xi e^{\xi\d} e^{-\l\d}} 
% 			\dotarg \frac{1}{1 - e^{-2\xi\d}},
% 		% &\sim \frac{1}{(\xi - \l) e^{-2 \xi \d}}
% \end{align*}
% which implies, 
% \begin{align}\label{eq:PRecipForLargeDelta}
% 	\frac{1}{p(\xi)} \sim \frac{1}{(\xi - \l)(1- e^{-2 \xi \d})},
% \end{align}
% for large $\d$. As such, if we set 
% $M:= \max_{\xi\in [0, \l/2]} \left| \frac{1}{\xi - \l} \right| = \frac{2}{\l}$, 
% then, since $\supp \chi_0 \cap [0, \infty) \su [0, \l/2]$ so that
% $M \chi_0 \geq |\chi_0/p|$ for all sufficiently large $\d$. The Dominated 
% Convergence Theorem therefore implies that
% \begin{align}
% 	\lim_{\d\to\infty} \int
% \end{align}


% Set $\wt M(\d) := \max_{\xi \in \supp \chi_\l} g(\xi)$. We claim
% $\supp_{\d>0} \wt M(\d) < \infty$. Indeed, 
% since $\wt M(\d)< \infty$, for each $\d > 0$, to verify this claim
% it is enough to show that $\wt M:= \limsup_{\d \to \infty} < \infty$. 
% By way of contradiction, suppose $\wt M = \infty$. Then, since
% $g(\dotarg, \d)$ is continuous for each $\d > 0$, the Extreme Value 
% and Supremum Approximation Theorems imply the existence of sequences
% $\{ \d_n \}_{n \in \N}$ and $\{x_n\}_{n \in \N} \su \supp \chi_\l$ 
% so that $g(x_n, \d_n) \to \infty$ as $n \to \infty$. To simplify
% notation, define $g_n (x):= g(x, \d_n)$. Since $\supp \chi_\l$ 
% is compact, by passing to a subsequence, we may assume $x_j \to x \in 
% \supp \chi_\l$ as $j \to \infty$. Let $C > 0$ be an arbitrary, large
% number and choose $N \in \N$ sufficiently large that $g_j(x_j) > 2C$
% for each $j > N$. Fix $j > N$, and note that since $g_j$ is continuous, 
% we may exploit the fact that $



\end{document}