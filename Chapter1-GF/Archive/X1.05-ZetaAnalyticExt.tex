%%===================================================
%% Section 5: Analytic Continuation of $G_L^+$ in $\zeta$
%%===================================================


\documentclass[GreensFunctions.tex]{subfiles}

\begin{document}
\section{Analytic Continuation of $G_L^+$ in $\zeta$}

To facilitate our later analysis of the Inverse Map, we wish to obtain a uniform upper 
bound for $K_{\zeta}(x)$ when $x\in \mathbb R$ and $\zeta$ is away from the 
positive half line. We summarize the ultimate results of this analysis in Theorem 
\ref{thm:1.05-main} below. In Lemmas \ref{lma:1.05-smallx} and \ref{lma:1.05-bigx} 
we respectively consider the cases $0< x < 1$ and $1 \geq x$.

\begin{thm}\label{thm:1.05-main}
	For $|x| < 1$ and $\zeta \in \mathbb C \backslash \mathbb R^+$, 
	the function $K_\zeta$ is bounded by
	\[
		\left| K_\zeta(x) \right| \leq C + C \big| \log|x| \big|,
	\]
	and for $|x| \geq 1$ and $\zeta \in \mathbb C \backslash \mathbb R^+$, 
	the function $K_\zeta$ is bounded by
	\[
		\left| K_\zeta(x) \right| \leq \frac{C}{x},
	\]
	where $C>0$ is uniform in $\zeta$.
\end{thm}


\begin{lma}\label{lma:1.05-smallx}
	For $0<x <1$, the integral $K(x)$ is bounded by
	\[
		|K(x)| \leq C + C |\log x|.
	\]
\end{lma}
\begin{proof}
	Be decompose the proof of Lemma \ref{lma:1.05-smallx} into 8 steps:

	%%=======
	%% Step 1
	%%=======
	{\em Step 1.} 
	We write $\zeta$ as $\zeta+i\varepsilon$, where $0<\zeta<\zeta_0$, 
	$0<\varepsilon<\varepsilon_0$. $\zeta_0$ and $\varepsilon_0$ are suitably chosen 
	small positive number. In this case, when we 
	shift the contour of integration for $G_L^+$ to $\mathbb R + i\pi$, one picks 
	up a residue of 
	$2\pi i \frac{(1-e^{-2\lambda})e^{i\lambda x }}{1-e^{-2\lambda} -2\lambda e^{-2\lambda}}$ 
	at a $\lambda$ in the shaded region $A$ of Figure 1 on page 566 in \cite{Kodama1982} for 
	which $\zeta(\lambda) = \zeta+i\varepsilon$. One then estimates the integral remaining
	after this contour shift
	\begin{align*}
		K_{\zeta}(x) 
			= 
				\int_{\mathbb R} 
					\frac{e^{ix\xi}}{\xi-\zeta(1-e^{-2\xi})+i[\pi-\varepsilon(1-e^{-2\xi})]}
				\,d\xi,	
	\end{align*}
	By choosing $\varepsilon_0$ sufficiently small, $\pi-\varepsilon(1-e^{-2\xi})>\frac\pi 2$. 
	Therefore the estimates in Section \ref{lma:1.04-smallz} are mostly still applicable, 
	with the exception that after integration by parts, the integral 
	\eqref{eq:1.04-smallz-9} is now
	\begin{align*}
		2^{-q} 
		\int_{\mathbb R} 
			\chi(2^{-q}x\xi)
			\left|
				\frac{\xi(1-2\zeta e^{-2\xi}-2i\varepsilon e^{-2\xi})}
					{[\xi-\zeta(1-e^{-2\xi})+i[\pi-\varepsilon(1-e^{-2\xi})]]^2}
			\right|
		\,d\xi.	
	\end{align*}
	The new term to be estimated is 
	\begin{align}
		&2^{-q} 
			\int_{\mathbb R} 
				\chi(2^{-q}x\xi)
				\left|
					\frac{\varepsilon \xi e^{-2\xi}}
						{[\xi-\zeta(1-e^{-2\xi})+i[\pi-\varepsilon(1-e^{-2\xi})]]^2}
				\right|
			\,d\xi 
			\notag\\
		&\qquad \le 
			C2^{-q} 
			\int_{\mathbb R} 
				\chi(2^{-q}x\xi)
				\frac{| \xi |\varepsilon e^{-2\xi}}
					{|\xi|^2+(\varepsilon e^{-2\xi})^2}
			\,d\xi.
			\label{eq: step 1 new term}
	\end{align}
	Let $\xi_0<-1$ be the large negative number such that $|\xi_0|= \varepsilon e^{-2\xi_0}$. 
	If $\xi<2\xi_0$, we claim that $|\xi|^2<C|\varepsilon e^{-2\xi}|$. In fact, for 
	$s = 2\xi_0-\xi>0$, 
	\begin{align*}
		\frac{\xi^2}{\varepsilon e^{-2\xi}} 
			= 
				\frac{(\xi_0-s)^2}{|\xi_0 e^{-2\xi_0}e ^{2s}|} 
			= 
				\left(
					|\xi_0| + 2s + \frac{s^2}{\xi_0}
				\right)
				e^{2\xi_0}e^{-2s}
			<C.
	\end{align*}
	On the other hand, if $\xi > \frac12\xi_0$, we claim that $\varepsilon e^{-2\xi}<C$. 
	In fact, for $s =\xi -  \frac12\xi_0>0$,
	\begin{align*}
		\varepsilon e^{-2\xi} = |\xi_0|e^{\xi_0} e^{-2s}<C.		
	\end{align*}
	Thus, \eqref{eq: step 1 new term} is bounded by
	\begin{align*}
	&C2^{-q}
		\int_{\xi<2\xi_0} 
			\frac{1}{(\varepsilon e^{-2\xi})^{1/2}}
		\,d\xi 
		+ C2^{-q}
		\int_{\xi>\frac12\xi_0} 
			\chi(2^{-q}x\xi) \frac{1}{|\xi|}
		\,d\xi
		\\
	&\qquad\quad+ 
		C2^{-q} 
		\int_{2\xi_0<\xi<\frac12\xi_0} 
			\frac{| \xi |\varepsilon e^{-2\xi}}{|\xi|^2+(\varepsilon e^{-2\xi})^2}
		\,d\xi
		\\
	&\qquad\le
		C2^{-q}\frac{e^{\xi_0}}{\sqrt{|\xi_0|}}
		+C2^{-p} 
		+C2^{-q}
		\int_{2\xi_0}^{\xi_0}
			\frac{|\xi_0|}{\varepsilon e^{-2\xi}}
		\,d\xi 
		+ C2^{-q}
		\int_{\xi_0}^{\frac12\xi_0}
			\frac{\varepsilon e^{-2\xi}}{|\xi_0|}
		\,d\xi
		\\
	&\qquad\le 
		C2^{-q}.
	\end{align*}

	%%=======
	%% Step 2
	%%=======
	{\em Step 2.} 
	We write $\zeta$ as $\zeta-i\varepsilon$, where $0<\zeta<\zeta_0$, 
	$0<\varepsilon <\varepsilon_0$. In this case, we shift the contour 
	of integration for $G_L^+$ to ${\mathbb R} + i\frac\pi 2$. By analyticity,
	the remaining after this contour is $K$. We then end up estimating
	\begin{align}
		K(x) 
			&= 
				\int_{\mathbb R} 
					\frac{e^{ix\xi}}
						{
							\xi-\zeta(1+e^{-2\xi})
							+i
							\left(
								\frac\pi2 + \varepsilon(1+e^{-2\xi})
							\right)
						}
				\,d\xi
				\\
			&= K_1(x) + K_2(x)\\
			&= 
				\int_{\mathbb R} 
					\frac{e^{ix\xi}\chi_0(\xi)}
						{
							\xi-\zeta(1+e^{-2\xi})
							+i
							\left(
								\frac\pi2 + \varepsilon(1+e^{-2\xi})
							\right)
						}
				\,d\xi 
				\label{eq: step 2 one} \\
			&\quad+
				\int_{\mathbb R} 
					\frac{e^{ix\xi}(1-\chi_0(\xi))}
						{
							\xi-\zeta(1+e^{-2\xi})
							+i
							\left(
								\frac\pi2 + \varepsilon(1+e^{-2\xi})
							\right)
						}
				\,d\xi.
	\end{align}
	\eqref{eq: step 2 one} is clearly bounded. It is also clear that on the support of 
	$1-\chi_0(\xi)$, one has $|\xi-\zeta(1+e^{-2\xi})|\ge \frac12|\xi|$. We use a 
	dyadic decomposition 
	\begin{align*}
		K_2(x) 
			= \sum_q K_{2q}(x) 
			= \sum_q 
				\int_{\mathbb R} 
					\frac{e^{ix\xi}(1-\chi_0(\xi))\chi(2^{-q}x\xi)}
						{
							\xi-\zeta(1+e^{-2\xi})
							+i
							\left(
								\frac\pi2 + \varepsilon(1+e^{-2\xi})
							\right)
						}
				\, \mathrm{d}\xi.
	\end{align*}
	We observe that $K_{2q}$ is nonzero only for $2^q\gtrsim x$. For 
	$x\lesssim 2^q\lesssim 1$,
	\begin{align*}
		|K_{2q}(x)|\le C\int_{\mathbb R} \frac{\chi(2^{-q}x\xi)}{|\xi|}~d\xi\le C.
	\end{align*}
	For $2^q\gtrsim 1$, integrate by part as before, and the only nontrivial term is
	\begin{align}
		&2^{-q}
		\int_{\mathbb R} 
			\big(1-\chi_0(\xi)\big)\chi(2^{-q}x\xi)
			\frac{|\xi(1+2\zeta e^{-2\xi} -2i\varepsilon e^{-2\xi})|}
				{
					\left|
						\xi-\zeta(1+e^{-2\xi})
						+i
						\left(
							\frac\pi2 + \varepsilon(1+e^{-2\xi})
						\right)
					\right|^2
				}
		\, \mathrm{d}\xi 
		\notag \\
		&\qquad\le 
			C2^{-q}
				\int_{\mathbb R} 
					\frac{\chi(2^{-q}x\xi)}{|\xi|}
				\,d\xi 
				+ C2^{-q}
				\int_{\xi\le 0}
					\frac{\chi(2^{-q}x\xi)|\xi\zeta e^{-2\xi}|}
						{\xi^2 + \left(\zeta e^{-2\xi}\right)^2}
				\,d\xi\notag\\
		&\qquad\qquad
			+ C2^{-q}
			\int_{\mathbb R}
				\frac{\chi(2^{-q}x\xi)|\xi\varepsilon e^{-2\xi}|}
					{\xi^2 + (\varepsilon e^{-2\xi})^2}
			\,d\xi \label{eq: step 2 two}\\
		&\qquad\le C2^{-q}.
			\notag
	\end{align}
	Here the last two terms in \eqref{eq: step 2 two} are treated in the same way as that 
	for \eqref{eq: step 1 new term}.


	%%=======
	%% Step 3
	%%=======
	{\em Step 3.} 
	Consider the case when $\zeta$ is $\zeta \pm i\varepsilon$. Here $\zeta>\zeta_1$, 
	$\varepsilon<\varepsilon_0$, where $\zeta_1$ is a suitably chosen large positive 
	number. In this case, we again shift the contour of integration to ${\mathbb R}+i\pi$ 
	and estimate
	\begin{align*}
		K(x) 
			= 
				\int_{\mathbb R} 
					\frac{e^{ix\xi}}
						{\xi-\zeta(1-e^{-2\xi})+i[\pi \mp \varepsilon(1-e^{-2\xi})]}
				\, \mathrm{d}\xi.	
	\end{align*}
	We may follow closely the estimates in Lemma \ref{lma:1.04-smallz}'s proof. In fact, 
	we similarly have the split to $K_1$ and $K_2$, where
	\begin{align}
		|K_1(x)| 
			&= 
				\left|
					\int_{\mathbb R} 
						\frac{e^{ix\xi}(1-\chi_{\xi_0+}(\xi))}
							{\xi-\zeta(1-e^{-2\xi})+i[\pi \mp \varepsilon(1-e^{-2\xi})]}
					\,d\xi
				\right|
				\\
			&\le 
				\int_{-1}^{\xi_0} 
					\frac{1}{|\pi\mp \varepsilon(1-e^{-2\xi})|}
				\,d\xi 
				+ \frac{1}{\zeta(\lambda_1)-\zeta(-1)}
				\int_{-\infty}^{-1} \frac{1}{e^{-2\xi}-1}~d\xi.\label{eq: step 3 one}
	\end{align}
	We have chosen $\varepsilon$ sufficiently small to make sure the denominator in 
	the first integral in \eqref{eq: step 3 one} is bounded way from zero. We now 
	estimate $K_2$ as in the proof of Lemma \ref{lma:1.04-smallz}. Notice that for 
	$\xi>\xi_0$, $\pi \mp \varepsilon(1-e^{-2\xi})>\frac\pi 2$. Thus most of the 
	estimates are similar. The only new term is
	\begin{align*}
		2^{-q} 
		\int_{\mathbb R} 
			\chi_{\xi_0+}(\xi)
			\frac{|\xi-\lambda|\varepsilon e^{-2\xi}}
				{
					\left|\xi-\lambda+i[\pi\mp\varepsilon(1-e^{-2\xi})]\right|^2
				}
				\chi(2^{-q}x(\xi-\lambda))
		\,d\xi
	\end{align*}
	which is now easily seen to be bounded by 
	\begin{align*}
		C2^{-q}
			\int_{\mathbb R} 
				\frac{\chi(2^{-q}x(\xi-\lambda))}{|\xi-\lambda|}
			\, \mathrm{d}\xi
		\le C2^{-q}
	\end{align*}
	because $\xi>\xi_0>0$.

	%%=======
	%% Step 4
	%%=======
	{\em Step 4.} 
	Write $\zeta$ as $\zeta\pm i\varepsilon$, where $\zeta_0\le \zeta\le \zeta_1$, and 
	$\varepsilon<\varepsilon_0$. This case can be handled by the remark at the end of 
	the proof for Lemma \ref{lma:1.04-smallz}.

	%%=======
	%% Step 5
	%%=======
	{\em Step 5.} 
	Write $\zeta$ as $-\zeta\pm i\varepsilon$, where $\zeta>0$ and 
	$\varepsilon<\varepsilon_0$. As before, in this case we also shift 
	the contour of integration to ${\mathbb R}+i\pi$, and estimate
	\begin{align*}
		K(x) 
			= 
				\int_{\mathbb R} 
					\frac{e^{ix\xi}}
						{\xi+\zeta(1-e^{-2\xi})+i[\pi\mp \varepsilon(1-e^{-2\xi})]}
				\,d\xi.	
	\end{align*}
	We decompose $K(x)$ as
	\begin{equation}
		K(x) 
			= 
				\sum_q K_q(x) 
			= 
				\sum_q
				\int_{\mathbb R} 
					\frac{e^{ix\xi}\chi(2^{-q}x\xi)}
						{\xi+\zeta(1-e^{-2\xi})+i[\pi\mp \varepsilon(1-e^{-2\xi})]}
				\,d\xi.
	\end{equation}
	Note that $|\xi|\approx 2^q/x$. If $2^q/x\lesssim 1$, $|\xi|\le 1$, and 
	\[
		|K_q(x)|\le C\int_{\mathbb R} \chi(2^{-q}x\xi)~d\xi\le C2^q/x.
	\]
	We have chosen $\varepsilon_0$ small enough to make sure 
	$\pi\mp \varepsilon(1-e^{-2\xi})$ is bounded away from zero when $|\xi|\lesssim 1$. 
	Next, observe that $\xi$ and $\zeta(1-e^{-2\xi})$ have the same sign and 
	$|\xi+\zeta(1-e^{-2\xi})|\ge |\xi|$. Thus if $x\lesssim 2^q \lesssim 1$, 
	\[
		|G_q(x)|\le \int_{\mathbb R} \frac{\chi(2^{-q}x\xi)}{|\xi|}~d\xi\le C.
	\]
	If $2^q\gtrsim 1$, integrate by parts and we end up estimating
	\begin{align}\label{eq: step 5 two}
		2^{-q}
		\int_{\mathbb R} 
			\frac{|\xi(1+2\zeta e^{-2\xi} \mp 2i\varepsilon e^{-2\xi})|\chi(2^{-q}x\xi)}
				{|\xi+\zeta(1-e^{-2\xi})+i[\pi\mp \varepsilon(1-e^{-2\xi})]|^2}
		\,d\xi.		
	\end{align}
	Let us work on the three terms on the numerator of the integrand separately. 
	Since $|\xi+\zeta(1-e^{-2\xi})|\ge |\xi|$, 
	\begin{align*}
		&2^{-q}
		\int_{\mathbb R} 
			\frac{|\xi|\chi(2^{-q}x\xi)}
				{|\xi+\zeta(1-e^{-2\xi})+i[\pi\mp \varepsilon(1-e^{-2\xi})]|^2}
			\,d\xi 
			\\
		&\qquad \le 2^{-q}
			\int_{\mathbb R} \frac{\chi(2^{-q}x\xi)}{|\xi|}~d\xi 
			\le C2^{-q}.
	\end{align*}
	Since $\xi$ and $\zeta(1-e^{-2\xi})$ have the same sign, we have for the second 
	term on the numerator
	\begin{align}\label{eq: step 5 one}
		2^{-q}
		\int_{\mathbb R} 
			\frac{|\xi \zeta e^{-2\xi} |\chi(2^{-q}x\xi)}{\xi^2+\zeta^2(1-e^{-2\xi})^2}
		\,d\xi.
	\end{align}
	If $\zeta<\zeta_0$ is sufficiently small, then \eqref{eq: step 5 one} is bounded by
	\begin{align*}
		2^{-q}
		\int_{\mathbb R} 
			\frac{|\xi \zeta e^{-2\xi} |\chi(2^{-q}x\xi)}{\xi^2+(\zeta e^{-2\xi})^2}
		\,d\xi,	
	\end{align*}
	which can be treated in the same way as that of \eqref{eq: step 1 new term}. Notice 
	that we have used the fact that $|1-e^{-2\xi}|\ge Ce^{-2\xi}$ when 
	$|\xi|\approx 2^q/x\gtrsim 1$. If $\zeta>\zeta_0$, then \eqref{eq: step 5 one} is 
	bounded by
	\begin{align}
		&2^{-q}
			\int_{\xi\le 0} 
				\frac{\xi\zeta e^{-2\xi}}{(\zeta e^{-2\xi})^2}
			\,d\xi 
			+ C 2^{-q}
			\int_{\xi>0} 
				\frac{\xi\zeta e^{-2\xi}}{\xi^2 + \zeta^2}
			\,d\xi
			\\
		&\qquad \le
			2^{-q} \frac1{\zeta_0}
			\int_{\xi\le 0}
				\xi e^{2\xi}
			\,d\xi 
			+ C2^{-q} \frac1{\zeta_0} \int_{\xi>0}\xi e^{-2\xi}\,d\xi
			\\
		&\qquad \le C2^{-q}.
	\end{align}
	The last term on the numerator in \eqref{eq: step 5 two} gives
	\begin{align*}
		&2^{-q}
			\int_{\mathbb R}
				\frac{|\xi|\varepsilon e^{-2\xi} \chi(2^{-q}x\xi)}
					{\xi^2 + [\pi\mp \varepsilon(1-e^{-2\xi})]^2}
			\,d\xi
			\\
		&\qquad \le 2^{-q}
			\int_{|\varepsilon(1-e^{-2\xi})|>2\pi} 
				\frac{|\xi|\varepsilon e^{-2\xi} \chi(2^{-q}x\xi)}
					{\xi^2 + (\varepsilon e^{-2\xi})^2}
				\,d\xi 
				\\
		&\qquad\qquad + C2^{-q}
			\int_{|\varepsilon(1-e^{-2\xi})|\le 2\pi} 
				\frac{\chi(2^{-q}x\xi)}{|\xi|}
			\,d\xi
			\\
		&\qquad\le C2^{-q}.
	\end{align*}

	%%=======
	%% Step 6
	%%=======
	{\em Step 6.} 
	Let $\zeta$ be $\zeta\pm i\varepsilon$, where $\zeta>\zeta_1$, 
	$\varepsilon\ge\varepsilon_0$. Here $\zeta_1$ is the large positive number, and 
	$\varepsilon_0$ is the small positive number chosen already in the previous steps. 
	We define the contour of integration $\Gamma$ as follows: it is almost the real line 
	${\mathbb R}$ except for a small neighborhood of $\xi=0$, where it traverses a 
	semicircle in the upper half plane, with radius so small that the image under 
	$\zeta(\xi) = \frac{\xi}{1-e^{-2\xi}}$ is within a strip of width 
	$\frac{\varepsilon_0}2$ around the real line. Let $\chi_0$ be a cutoff function 
	supported near the part of $\Gamma$ between $-1$ and 1. We then have 
	\begin{align*}
		K(x) 
			&= 
				\int_\Gamma 
					\frac{e^{ix\xi}}{\xi-(\zeta\pm i\varepsilon)(1-e^{-2\xi})}
				\,d\xi\\
			&= 
				G_1(x) + G_2(x)
				\\
			&= 
				\int_\Gamma 
					\frac{e^{ix\xi} \chi_0(\xi)}
						{\xi-(\zeta\pm i\varepsilon)(1-e^{-2\xi})}
				\,d\xi
				+
				\int_\Gamma 
					\frac{e^{ix\xi}(1-\chi_0(\xi))}
						{\xi-(\zeta\pm i\varepsilon)(1-e^{-2\xi})}
				\,d\xi,
	\end{align*}
	and
	\begin{align*}
		|K_1(x)|
			\le 
				\int_\Gamma 
					\frac{\chi_0(\xi)}
						{|(1-e^{-2\xi})(\zeta(\xi)-(\zeta\pm i\varepsilon))|}
					\,d\xi
			\le C.
	\end{align*}
	We treat $K_2$ similarly to the proof of Lemma \ref{lma:1.04-largez}, and end up 
	estimating
	\begin{align*}
		G_q(x) 
			= 
				\int_{\mathbb R} 
					\frac{e^{ix\xi}\chi_{\xi_0+}(\xi)}
						{\xi-(\zeta\pm i\varepsilon)(1-e^{-2\xi})}
					\chi(2^{-q}x(\xi-\lambda))
				\,d\xi.	
	\end{align*}
	Note that when $\xi\ge \xi_0$, $|\varepsilon(1-e^{-2\xi})|\ge C\varepsilon_0$. The 
	terms for $2^q\lesssim 1$ can be estimated similarly as in proof of Lemma 
	\ref{lma:1.04-largez}. When $2^q\gtrsim 1$, the integral we end up estimating is 
	\begin{align}
		2^{-q} 
		\int_{\mathbb R} 
			\chi_{\xi_0+}(\xi)
			\frac{|\xi-\lambda||1-2\zeta e^{-2\xi} \mp 2i\varepsilon e^{-2\xi}|}
				{|\xi-\lambda|^2+[\varepsilon(1-e^{-2\xi})]^2}
			\chi(2^{-q}x(\xi-\lambda))
		\,d\xi.
	\end{align}
	Compared with the estimates on \eqref{eq: Gp part 3}, the new term to be estimated is
	\begin{align*}
		2^{-q}
			\int_{\mathbb R} 
				\chi_{\xi_0+}(\xi) 
				\frac{|\xi-\lambda|\varepsilon e^{-2\xi}}
					{|\xi-\lambda|^2+\varepsilon^2}
			\,d\xi
		\le 2^{-q}
			\int_{\mathbb R} \chi_{\xi_0+}(\xi) e^{-2\xi}~d\xi 
		\le C2^{-q}.	
	\end{align*}

	%%=======
	%% Step 7
	%%=======
	{\em Step 7.}
	Let $\zeta$ be $\zeta\pm i\varepsilon$, where $\zeta<\zeta_0$, including negative 
	values, and $\varepsilon\ge\varepsilon_0$. We use the same integration contour 
	$\Gamma$ as in Step 6, and the same splitting into $G_1$ and $G_2$. The $\xi\le -1$ 
	part of $K_2$ is bounded by
	\begin{align*}
		\left|\int_{\xi\le -1}\frac{1}{\varepsilon_0|1-e^{-2\xi}|}\right|~d\xi
		\le C.	
	\end{align*}
	For the $\xi\ge 1$ part of $K_2$, we decompose into
	\begin{align*}
		G_q(x) 
			= 
				\int_{\xi\ge 0} 
					\frac{e^{ix\xi} (1-\chi_0(\xi))\chi(2^{-q}x\xi)}
						{\xi-\zeta(1-e^{-2\xi})\mp i\varepsilon(1-e^{-2\xi})}
				\,d\xi.
	\end{align*}
	We integrate by parts and find out the worse term to be
	\begin{align}
		&2^{-q}
			\int_{\xi\ge 1} 
				\frac{
					(1-\chi_0(\xi))\chi(2^{-q}x\xi)|\xi(1-2\zeta e^{-2\xi} 
					\mp 2i\varepsilon e^{-2\xi})|
				}{
					|\xi-\zeta(1-e^{-2\xi})\mp i\varepsilon(1-e^{-2\xi})|^2
				}
			\,d\xi
			\\
		&\quad\le
			C2^{-q}
			\int_{\xi\ge 1} 
				\frac{
					(1-\chi_0(\xi))\chi(2^{-q}x\xi)|\xi(1-2\zeta e^{-2\xi} 
					\mp 2i\varepsilon e^{-2\xi})|
				}{
					|\xi-\zeta|^2 +\varepsilon^2
				}
				\,d\xi. 
				\label{eq: step 7 one}
	\end{align}
	When $0<\zeta<\zeta_0$, \eqref{eq: step 7 one} is bounded by
	\begin{align*}
		C2^{-q}
		\int_{\mathbb R} 
			\frac{\chi(2^{-q}x\xi)|\xi|}
				{|\xi|^2+\varepsilon^2}
		\,d\xi 
		+ 
			C2^{-q}
			\int_{\xi\ge 1}
				\frac{\xi e^{-2\xi}}{\varepsilon_0}
			\,d\xi
		\le C2^{-q}.
	\end{align*}
	When $\zeta<0$, \eqref{eq: step 7 one} is bounded by
	\begin{align}
		C2^{-q}
		\int_{\xi\ge 1}
			\frac{\chi(2^{-q}x\xi)(\xi+\xi|\zeta| e^{-2\xi})}{\xi^2+\zeta^2}
		\,d\xi 
		+ C2^{-q}
			\int_{\xi\ge 1} 
				\frac{\varepsilon\xi e^{-2\xi}}{\xi^2+\varepsilon^2}
			\,d\xi
		\le C2^{-q}.
	\end{align}

	%%=======
	%% Step 8
	%%=======
	{\em Step 8.}
	Let $\zeta$ be $\zeta\pm i\varepsilon$, where $\zeta_0\le \zeta\le \zeta_1$, and 
	$\varepsilon\ge \varepsilon_0$. We use the same integration contour $\Gamma$ as in 
	Step 6. Essentially the remark at the end of the proof of Lemma \ref{lma:1.04-medz} 
	does the job. Notice that in the current case, we only need to focus on 
	$\xi> 2\zeta_1$, where $|i\varepsilon(1-e^{-2\xi})|\approx \varepsilon$. The extra 
	term compared with Lemma \ref{lma:1.04-medz} is
	\begin{align*}
		2^{-q}
		\int_{\xi\ge 2\lambda_1} 
			\frac{\xi\varepsilon e^{-2\xi}}{\xi^2+\varepsilon^2}
		\,d\xi
		\le C2^{-q}.	
	\end{align*}
\end{proof}



\begin{lma}\label{lma:1.05-bigx}
	For $x \geq 1$, $\left|K_\zeta(x)\right| \leq C/x$ where $C>0$ is uniform in $C$.
\end{lma}
\begin{proof}
	We now concern ourselves with the uniform estimates of $K_\zeta(x)$ for $x\ge 1$. 
	We basically go through the eight steps from the proof of Lemma \ref{lma:1.05-smallx}
	and use the same integration contours here as there, although in the current case, we 
	immediately integrate by parts.


	%%=======
	%% Step 1
	%%=======
	{\em Step 1.}
	\begin{align*}
		\left|K_\zeta(x)\right| 
			\le 
				\frac1x
				\int_{\mathbb R} 
					\frac{|1-2\zeta e^{-2\xi}-2i\varepsilon e^{-2\xi}|}
						{|\xi-\zeta(1-e^{-2\xi})+i[\pi -\varepsilon(1-e^{-2\xi})]|^2}
					\,d\xi.
	\end{align*}
	Let $\zeta=\zeta(\lambda)$. By the mean value theorem,
	\begin{align*}
		\frac{\xi}{1-e^{-2\xi}}-\frac{\lambda}{1-e^{-2\lambda}} 
			= \frac{1-(1+2\xi)e^{-2\xi}}{(1-e^{-2\xi})^2}\bigg|_{\xi=\theta}(\xi-\lambda)
	\end{align*}
	for some $\theta$ between $\xi$ and $\lambda$. Thus, if $\lambda-1<\xi<\lambda +1$,
	\begin{align*}
		|\xi-\zeta(1-e^{-2\xi})| 
			&= (e^{-2\xi}-1)|\zeta(\xi)-\zeta(\lambda)|\\
			&\approx e^{2(\theta-\xi)}\theta|\xi-\lambda|
			\approx \lambda|\xi-\lambda|.
	\end{align*}
	When $\xi>\lambda+1$, the estimates in the proof of Lemma \ref{lma:1.05-smallx} give
	\begin{align*}
		|\xi - \zeta(1-e^{-2\xi})|
			>\frac12|\xi|.
	\end{align*}
	When $\xi<\lambda-1$, 
	\begin{align*}
		\zeta e^{-2\xi}
			\approx |\zeta(1-e^{-2\xi})|>2|\xi|.
	\end{align*}
	Thus
	\begin{align*}
		|K_\zeta(x)|
			&\le \frac Cx 
				\int_{\xi<\lambda-1}
					\frac{\zeta e^{-2\xi}}{(\zeta e^{-2\xi})^2}
				\,d\xi 
				+ \frac Cx 
				\int_{\xi>\lambda+1}
					\frac{1+|\lambda| e^{-2(\xi-\lambda)}}{\xi^2+1}
				\,d\xi
				\\
			&\qquad+
				\frac Cx 
					\int_{\lambda-1<\xi<\lambda+1}
						\frac{|\lambda| e^{-2(\xi-\lambda)}}{\lambda^2(\xi-\lambda)^2+1}
					\,d\xi 
					+ \frac Cx 
					\int_{\mathbb R} 
						\frac{\varepsilon e^{-2\xi}}{1 + (\varepsilon e^{-2\xi})^2}
					\,d\xi
					\\
			&\le 
				\frac Cx
				\int_{\mathbb R} 
					\frac{\zeta e^{-2\xi}}{1+(\zeta e^{-2\xi})^2}
				\,d\xi 
				+ 
				\int_{\lambda+1<\xi<\frac\lambda 2}
					\frac{e^{-2(\xi-\lambda)}}{|\lambda|}
				\,d\xi+\frac C x \int_{\xi>\frac\lambda 2}\frac{1}{\xi^2+1}~d\xi 
				\\
			&\qquad+ 
				\frac{C}{x} 
				\int_{\mathbb R} 
					\frac{|\lambda |~d\xi}{\lambda^2\xi^2+1}
				\,d\xi 
				+ \frac Cx 
				\int_{\mathbb R} 
					\frac{\varepsilon e^{-2\xi}}{1 + (\varepsilon e^{-2\xi})^2}
				\,d\xi
				\\
			&\le \frac Cx.
	\end{align*}



	%%=======
	%% Step 2
	%%=======
	{\em Step 2.} 
	Notice that in this case $\xi$ and $-\zeta(1+e^{-2\xi})$ have the same sign when 
	$\xi<0$ and $\xi-\zeta(1+e^{-2\xi})>\frac12\xi$ when $\xi>1$. We have
	\begin{align*}
		\left|K_\zeta(x)\right|
			&\le 
				\frac Cx 
				\int_{\mathbb R} 
					\frac{|1+2\zeta e^{-2\xi}-2i\varepsilon e^{-2\xi}|}
						{
							|\xi-\zeta(1+e^{-2\xi})
							+i
							\left(
								\frac{\pi}{2}+\varepsilon(1+e^{-2\xi})
							\right)
							|^2
						}
				\,d\xi\\
			&\le  
				\frac Cx 
				\int_{\mathbb R} \frac{1}{1+\xi^2}~d\xi 
				+ \frac Cx 
				\int_{\mathbb R} 
					\frac{\zeta e^{-2\xi}}{1+(\zeta e^{-2\xi})^2}
				\,d\xi 
				+ \frac{C}{x}
				\int_{\mathbb R} 
					\frac{\varepsilon e^{-2\xi}}{1+(\varepsilon e^{-2\xi})^2}
				\,d\xi
				\notag\\
		&\le 
			\frac Cx.
	\end{align*}



	%%=======
	%% Step 3
	%%=======
	{\em Step 3.} 
	We have
	\begin{align*}
		\left|K_\zeta(x)\right|
			\le \frac Cx
			\int_{\mathbb R} 
				\frac{|1-2\zeta e^{-2\xi}\mp 2i\varepsilon e^{-2\xi}|}
					{|\xi-\zeta(1-e^{-2\xi})+i[\pi\mp\varepsilon (1-e^{-2\xi})]|^2}
			\,d\xi.
	\end{align*}
	When $\xi<-1$, 
	\begin{align*}
		|\xi-\zeta(1-e^{-2\xi})| 
			= |e^{-2\xi}-1||\zeta(\xi)-\zeta|\approx \zeta e^{-2\xi}.
	\end{align*}
	When $-1<\xi<\xi_0$ for some large fixed $\xi_0$,
	\begin{align*}
		|\xi-\zeta(1-e^{-2\xi})| 
			= |e^{-2\xi}-1||\zeta(\xi)-\zeta|\approx\zeta |\xi|,
	\end{align*}
	and by choosing $\varepsilon_0$ small,
	\begin{align}
		\label{eq: large x step 3 one}
		|\pi\mp\varepsilon (1-e^{-2\xi})|\approx 1.
	\end{align}
	When $\xi>\xi_0$, 
	\begin{align*}
		|\xi-\zeta(1-e^{-2\xi})|\approx |\xi-\lambda|,
	\end{align*}
	and \eqref{eq: large x step 3 one} continues to hold. Thus
	\begin{align}
		\left|K_\zeta(x)\right| 
			&\le 
				\frac Cx 
				\int_{\xi<-1}
					\frac{1+\zeta e^{-2\xi}}{\zeta^2 e^{-4\xi}}
				\,d\xi 
				+ \frac Cx 
				\int_{-1<\xi<\xi_0}
					\frac{\zeta}{\zeta^2\xi^2+1}
				\,d\xi 
				\notag\\
			&\qquad+ 
				\frac C x
				\int_{\xi>\xi_0}
					\frac{\lambda e^{-2\xi}}{|\xi-\lambda|^2+1}
				\,d\xi\\
			&\le 
				\frac{C}{x} 
				+\frac{C}{x} 
				\int_{\xi_0<\xi<\frac\lambda 2}
					\frac{e^{-2\xi}}{\lambda}
				\,d\xi 
				+ \frac Cx 
				\int_{\xi>\frac\lambda 2} 
					\frac{\lambda e^{-\lambda}}{|\xi-\lambda|^2+1}
				\, \mathrm{d}\xi 
				\label{eq: x large step 3 two}\\
			&\le 
				\frac Cx.
	\end{align}



	%%=======
	%% Step 4
	%%=======
	{\em Step 4.} 
	Easy.


	%%=======
	%% Step 5
	%%=======
	{\em Step 5.}
	We get
	\begin{align*}
		\left|K_\zeta(x)\right| 
			\le 
				\frac C x
				\int_{\mathbb R} 
					\frac{|1+2\zeta e^{-2\xi}\mp 2i\varepsilon e^{-2\xi}|}
						{|\xi+\zeta(1-e^{-2\xi})+i[\pi\mp\varepsilon(1-e^{-2\xi})]|^2}
				\,d\xi.
	\end{align*}
	If $0<\zeta<\zeta_0$ for some small $\zeta_0$, the first two terms on the numerator of 
	the integrand of $K_\zeta(x)$ can be estimated by
	\begin{align*}
		\frac Cx 
		+ \frac Cx 
		\int_{|\xi|\ge 1}\frac{\zeta e^{-2\xi}}{\xi^2+(\zeta e^{-2\xi})^2}~d\xi 
		\le 	
			\frac Cx.
	\end{align*}
	If $\zeta>\zeta_0$, then, as in Step 5 from the proof of Lemma \ref{lma:1.05-smallx},
	the first two terms on the numerator gives a
	\begin{align*}
		\frac Cx 
		+\frac Cx 
		\int_{\xi<-1}
			\frac{\zeta e^{-2\xi}}{(\zeta e^{-2\xi})^2}
		\,d\xi 
		+ \frac Cx 
		\int_{\xi>1}
			\frac{\zeta e^{-2\xi}}{\xi^2+\zeta^2}
		\,d\xi
		\le 
			\frac Cx.
	\end{align*}
	The last term on the numerator is bounded by
	\begin{align*}
		\frac Cx 
		\int_{\mathbb R} 
			\frac{\varepsilon e^{-2\xi}}{\xi^2 + [\pi\pm\varepsilon(1-e^{-2\xi})]^2}
		\,d\xi
		\le \frac Cx.
	\end{align*}



	%%=======
	%% Step 6
	%%=======
	{\em Step 6.} 
	\begin{align*}
		|K(x)|
			\le \frac Cx 
				\int_\Gamma 
					\frac{|1-(\zeta\pm i\varepsilon)2e^{-2\xi}|}
						{|\xi-(\zeta\pm i\varepsilon)(1-e^{-2\xi})|^2}
				\,d\xi.	
	\end{align*}
	Using the same $\chi_0$ cut off as in Step 6 of the proof of Lemma \ref{lma:1.05-smallx}, 
	we end up estimating
	\begin{align*}
		\int_{\mathbb R} 
			\frac{\chi_{\xi_0+}(\xi)|1-2\zeta e^{-2\xi}\mp 2i\varepsilon e^{-2\xi}|}
				{|\xi-\lambda|^2+[\varepsilon(1-e^{-2\xi})]^2}
		\,d\xi
	\end{align*}
	The integral involving the first two terms on the numerator can be estimated as for 
	\eqref{eq: x large step 3 two}. The last term is bounded by
	\begin{align*}
		\int_{\mathbb R} 
			\frac{\varepsilon e^{-2\xi}}{|\xi-\lambda|^2+\varepsilon^2}
		\, \mathrm{d}\xi 
		\le 
			\int_{\mathbb R} \frac{\varepsilon }{|\xi-\lambda|^2+\varepsilon^2} \,d\xi 
		= C.
	\end{align*}


	%%=======
	%% Step 7
	%%=======
	{\em Step 7.} 
	\begin{align*}
		|K(x)| 
			\le \frac{C}{x} 
				\int_\Gamma 
					\frac{|1-(\zeta\pm i\varepsilon)2e^{-2\xi}|}
						{|\xi-(\zeta\pm i\varepsilon)(1-e^{-2\xi})|^2}
				\, \mathrm{d}\xi.	
	\end{align*}
	Similar to Step 7 in the proof of Lemma \ref{lma:1.05-smallx}, we reduce the estimates to the region 
	$\xi\ge 1$, where 
	\begin{align*}
		|\xi-\zeta(1-e^{-2\xi}) |\approx \xi, ~\zeta e^{-2\xi}\le 1
	\end{align*}
	if $0<\zeta<\zeta_0$, and
	\begin{align*}
		|\xi-\zeta(1-e^{-2\xi}) |\approx \xi+|\zeta|
	\end{align*}
	if $\zeta<0$. Thus
	\begin{align*}
		|K(x)|
			\le \frac{C}{x} 
				\int_{\xi\ge 1}\frac{1+\varepsilon e^{-2\xi}}{\xi^2+\varepsilon^2 }\,d\xi 
			\le \frac{C}{x}
	\end{align*}
	if $0<\zeta<\zeta_0$, and 
	\begin{align}
		\left|K_\zeta(x)\right|
			\le \frac Cx 
				\int_{\xi\ge 1}
					\frac{1+\zeta e^{-2\xi} + \varepsilon e^{-2\xi}}
						{\xi^2+\zeta^2+\varepsilon^2 }
				\,d\xi 
			\le \frac Cx
	\end{align}
	if $\zeta<0$.

	%%=======
	%% Step 8
	%%=======
	{\em Step 8.} 
	Easy.
\end{proof}

\end{document}