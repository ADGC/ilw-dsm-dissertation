%%==========================
%% Section 2.01: Boundedness as a Convolution Operator
%%==========================
\documentclass[../dissertation]{subfiles}


\begin{document}

\section{Boundedness as a Convolution Operator}\label{sec2:bndness}

As a warm-up, we begin our study of the Green's functions mapping 
properties by studying in Proposition \ref{prop2:pold} the 
mapping properties of $G_L^+$, $G_R^+$ as convolution operators
under the constraint $\lambda \ne 0$. 

\begin{prop}\label{prop2:pold}
	For each fixed $\lambda \in \mathbb R \sm \{0\}$, $G_\star^+$ 
	($\star = L \text{, or } R$) are bounded as convolution operators
	from $L^1(\mathbb R) \cap L^p$ ($1< p \leq 2$) with 
	\[
		\| G_\star^+ * f \|_{L^\infty} 
			\lesssim_\lambda \|f \|_{L^1\cap L^p},
	\]
	\label{sym:lesssimdep}
	where the implied constant depends on $\lambda$ when 
	$|\lambda|< 1$. Further, 
	\begin{align}
		\lim_{x \to -\infty} \big(G_L^+ * f\big)(x) 
			= \lim_{x \to +\infty} \big(G_R^+ * f\big)(x) 
			= 0
	\end{align}
	whenever the spectral parameter $\lambda$ is both real and non-zero.
\end{prop}
\begin{proof}
	%% The following was modified from Prof. Perry's kpw1.tex file
	From our previous work in Section 
	\ref{sec1:GreensFunctions} proving Theorem \ref{thm1:GFRep}, we have
	\begin{align}
		G_L^+(x; \lambda, \delta) 
			= 
				K^+(x; \lambda) 
				+ 
					\big[
						i \alpha(\lambda) + i \beta(\lambda)\, e^{ix\lambda} 
					\big] \chi_+(x)
	\end{align}
	and
	\begin{align}
		G_R^+(x; \lambda, \delta) 
			= 
				K^+(x; \lambda) 
				- 
					\big[
						i \alpha(\lambda) + i \beta(\lambda)\, e^{ix\lambda} 
					\big] \chi_-(x)
	\end{align}
	where 
	\begin{align*}
		K^+(x; \lambda) 
				:= \frac{e^{-\pi |x|}}{2\pi} 
					\int_{\mathbb R} e^{i x \xi} 
						\frac{1}{p(\xi; \lambda) + i \pi \sign(x)}
					\, \mathrm{d}\xi.
	\end{align*}
	and $\chi_-:= \chi_{(-\infty, 0)}$, $\chi_+:= \chi_{(0, \infty)}$ denote 
	\label{sym2:chipm}the respective characteristic functions on the open 
	intervals $(-\infty, 0)$, $(0, \infty)$. 
	Although the integral for $K^+$ is conditionally convergent, it avoids 
	zeros of the symbol $p$, and may be understood through the $L^q$ theory 
	of the Fourier transform, since the integrand belongs to $L^q$ for any 
	$q>1$.  Moreover, it follows from the 
	Hausdorff-Young inequality and dominated convergence that 
	$\lim_{h \to 0} \norm[L^{q'}]{K(\dotarg+h)^+ - K^+} =0$ for any $q \in (1,2]$. 
	As a consequence, the convolutions
	\begin{align}
		\label{ILW.GL*f}
		G_L^+(\dotarg; \lambda) * f(x) 	
			&=	\int_{\mathbb R} G_L^+(x-x',\lambda) f(x') \, \mathrm{d}x' 	\\
			&=	
				\int_{-\infty}^x K^+(x-x',\lambda) f(x') \, \mathrm{d}x'
				+\int_x^{\infty} K^+(x-x',\lambda) f(x') \, \mathrm{d}x'
				\nonumber \\
			&\quad 	+ i \alpha (\lambda) \int_{-\infty}^x f(x') \, \mathrm{d}x' 
				+ i \beta(\lambda) e^{i\lam x} 
						\int_{-\infty}^x e^{-i\lambda x'} f(x') \, \mathrm{d}x', 
				\nonumber\\
		\intertext{and}
		\label{ILW.GR*f}
		G_R^+(\dotarg; \lambda) * f(x) 
			&= 
				\int_{-\infty}^x K^+(x-x',\lambda) f(x') \, \mathrm{d}x'
				+\int_x^{\infty} K^+(x-x',\lambda) f(x') \, \mathrm{d}x'
				\\
			&\quad- i \alpha (\lambda) \int_x^{\infty} f(x') \, \mathrm{d}x' 
				- i \beta(\lambda) e^{i\lam x} 
						\int_x^{\infty} e^{-i\lambda x'} f(x') \, \mathrm{d}x', 
				\nonumber
	\end{align}
	define bounded continuous functions for any 
	$f \in L^1(\mathbb R) \cap L^p(\mathbb R)$ for any $p \in (1,2]$ with
	\begin{equation}
		\label{ILW.GL.bd}
		\norm[L^\infty(\mathbb R)]{G_\star^+*f} \lesssim_\lambda \norm[L^1 \cap L^p]{f}.
	\end{equation}
	For $f \in C_0^\infty(\mathbb R)$, it is easy to see that
	\begin{align*}
		\lim_{x\to-\infty} \alpha (\lambda) \int_{-\infty}^x f(x') \, \mathrm{d}x' 
				+ \beta(\lambda) e^{i\lam x} 
						\int_{-\infty}^x e^{-i\lambda x'} f(x') \, \mathrm{d}x'
				= 0
	\end{align*}
	and 
	\begin{align*}
		\lim_{x\to\infty}
				\alpha (\lambda) \int_x^{\infty} f(x') \, \mathrm{d}x' 
				+  \beta(\lambda) e^{i\lam x} 
						\int_x^{\infty} e^{-i\lambda x'} f(x') \, \mathrm{d}x'
			= 0.
	\end{align*}
	Further, the Dominated Convergence Theorem implies 
	\begin{align*}
		\lim_{x\to \pm \infty} K^+*f(x) 
			= \lim_{x\to \pm \infty} \int_{\mathbb R} K^+(x-x') f(x') \, \mathrm{d}x
			= 0,
	\end{align*}
	as $|K^+(x') f(x-x')| \leq \|f\|_{L^\infty} |K^+(x)| \in L_x^1(\mathbb R)$,
	and $K^+(x) = \mathcal O\left(e^{-|x|}\right)$ for $|x| \geq 1$. 

	% $\lim_{x \to -\infty} G_L^+ * f(x;\lambda) = 0$. 


	It now follows from a density 
	argument and \eqref{ILW.GL.bd} that if $p \in (1,2]$, then
	\begin{equation}
		\label{ILW.GL.vanish}
		\lim_{x \to -\infty} \left(G_L^+*f\right)(x) 
			= \lim_{x \to +\infty} \left(G_R^+*f\right)(x) 
			= 0
	\end{equation}
	for any $f \in L^1(\mathbb R) \cap L^p(\mathbb R)$.
\end{proof}

We now turn our focus towards the operators $T_{\star, \lambda, u}$
($\star = L \text{, or } R$). In Proposition \ref{prop2:Tbnd} we prove that as 
an operator on the space $\inn{\dotarg} L^\infty(\mathbb R)$, the operators 
$T_{\star, \lambda, u}$ are uniformly bounded in $\lambda$, whose operator norms 
depend only on the norm of their corresponding potential $u$. As a reminder, the 
function space for potentials $u$ is 
$X:=L^{2, 4}(\mathbb R) = \inn{\dotarg}^{-4} L^2(\mathbb R)$. Beginning in
Proposition \ref{prop2:Tbnd}, we also introduce the notation 
$Y\toitself$\label{sym:toitself} to
denote a map from $Y$ into $Y$, and the notation 
$\|\dotarg\|_{Y\to Z}$\label{sym:opnorm} to
denote the implied operator norm for an operator which maps from $Y$ to $Z$.
Thus, the notation $\|\dotarg\|_{\inn{\dotarg}L^\infty\toitself}$ used in 
Equation \eqref{eq3:Tbndxbrac} of Proposition \ref{prop2:Tbnd} denotes the 
operator norm for an operator which maps from the space 
$\inn{\dotarg}L^\infty(\mathbb R)$ into itself.

\begin{prop}\label{prop2:Tbnd}
	Consider the operators $T_{\star,\lambda, u}$ 
	($\star = L \text{, or }R$)
	given by
	\[
		\big(T_{\star,\lambda, u} f\big)(x) 
			:= \int_{\mathbb R} G_\star^+(x-x'; \lambda) \, u(x') f(x') \, \mathrm{d}x'.
	\]
	For every $u \in X$, operators \label{sym:mapsto}
	$T_{\star, \lambda, u}: \inn{\dotarg} L^\infty(\mathbb R) 
	\to \inn{\dotarg} L^\infty(\mathbb R)$ are bounded uniformly in $\lambda\in\mathbb R$ with
	\begin{align}\label{eq3:Tbndxbrac}
		\|T_{\star, \lambda, u}\|_{\inn{\dotarg} L^\infty \toitself} 
			\lesssim \|u\|_X.
	\end{align}
\end{prop}
\begin{proof}
	To simplify notation, we write $T_\star$ instead of $T_{\star, \lambda, u}$
	throughout this proof. We begin by noting that
	\begin{align*}
		|T_\star f(x)| 
			&= 
				\left| 
					\int_{\mathbb R} 
						G_\star^+(x-x'; \lambda) \big(u(x')\inn{x'}\big)
						\big(\inn{x'}^{-1} f(x')\big)
					\, \mathrm{d}x'
				\right| \\
			&\leq 
				\|f\|_{\inn{\dotarg}L^\infty} 
				\int_{\mathbb R}
					\big|G_\star^+(x-x'; \lambda)\big|
					\inn{x'} |u(x')|
				\, \mathrm{d}x'.
	\end{align*}
	Hence
	\begin{align}
		\|T_\star\|_{\inn{\dotarg} L^\infty \toitself} 
			&\leq \sup_{x\in \mathbb R} \inn{x}^{-1}
				\int_{\mathbb R}
					\big|G_\star^+(x-x'; \lambda)\big|
					\inn{x'} |u(x')|
				\, \mathrm{d}x'.
	\end{align}
	
	Recall from Theorem \ref{thm1:GFRep} that
	\begin{subequations}\label{eq3:Grep}
		\begin{align}
			G_L^+(x; \lambda) 
				&= K^+(x; \lambda) + 
				\big[i \alpha(\lambda) + i \beta(\lambda) e^{ix \lambda}\big]
					 \chi_L(x) \\
			G_R^+(x; \lambda)
				&= K^+(x; \lambda) -
				\big[i \alpha(\lambda) + i \beta(\lambda) e^{ix \lambda}\big]
					 \chi_R(x)
		\end{align}
	\end{subequations}
	where we define $\chi_L:=\chi_{\mathbb R^+}$,  $\chi_R:= \chi_{\mathbb R^-}$,
	and $\chi_{\mathbb R^+}$, $\chi_{\mathbb R^-}$
	respectively denote the characteristic functions on the intervals
	$(0, \infty)$ and $(-\infty, 0)$.
	According to Theorem \ref{thm1:krep}
	\[
		\lim_{\lambda \to 0} 
			\alpha(\lambda) + \beta(\lambda) \, e^{ix\lambda}
			= \frac{2}{3} + i x,
	\]
	which implies that for sufficiently small $\varepsilon > 0$, 
	\begin{align}\label{eq2:polesumbnd}
		\big| \alpha(\lambda) + \beta(\lambda) \, e^{ix\lambda} \big|
			&\lesssim_{\varepsilon}
				\begin{cases}
					1, & |\lambda| \geq \varepsilon \\
					1 + |x|, & |\lambda| < \varepsilon
				\end{cases}
	\end{align}
	Further, since Theorem \ref{thm1:krep} also states that 
	\begin{align*}
		|K^+(x-x'; \lambda)| \lesssim 1 + \log_+\left(\frac{1}{|x-x'|}\right) 
	\end{align*}
	we see from \eqref{eq3:Grep} that 
	\begin{align}
		|G_\star^+ (x - x'; \lambda)|
			\lesssim 1 + |x-x'| + \log_+(1/|x-x'|)
	\end{align}
	where the implied constant is $\lambda$ independent. 

	Thus, the operator norm of $T_\star$
	is bounded by 
	\begin{align}\label{eq3:Tbnd}
		\sup_{x\in \mathbb R} \inn{x}^{-1}
				\int_{\mathbb R}
					\left[1 + |x-x'| + \log_+\left(\frac{1}{|x-x'|}\right)\right]
					\inn{x'} |u(x')|
				\, \mathrm{d}x'.
	\end{align}
	To estimate \eqref{eq3:Tbnd}, first note that 
	since $\inn{x} \geq 1$ for $x\in \mathbb R$, we have $\inn{x'}\inn{x}^{-1} 
	\leq \inn{x'}$. Further, 
	\[
		\inn{x}^{-1} |x - x'| 
			= \frac{|x-x'|}{\inn{x}\inn{x'}}\inn{x}^2 
			\lesssim \inn{x'}^2.
	\]
	and, lastly, $\inn{x}^{-1} \inn{x'} \lesssim 1$ whenever $|x-x'|\leq 1$. 
	As such, 
	\begin{align*}
		\|T_\star\|_{\inn{\dotarg} L^\infty \toitself}
			&\lesssim 
				\int_{\mathbb R} \inn{x'} |u(x')| \, \mathrm{d}x' 
				\\
			&\qquad+ \esssup_{x\in \mathbb R} 
				\int_{|x-x'|\leq 1} 
					\log\left(\frac{1}{|x-x'|}\right)
					|u(x')|
				\, \mathrm{d}x' 
				\\
		&\lesssim \|u\|_X,
	\end{align*}
	by Remark \ref{rmk2:PostX}.
\end{proof}

If we don't need to worry about non-zero $\lambda$, then we can 
improve Proposition 
\ref{prop2:Tbnd} slightly by proving that $T_{\star,\lambda,u}$ 
is actually a bounded operator on (unweighted) essentially bounded,
measurable functions. We do this next in Proposition \ref{prop2:Tbndl}.


\begin{prop}\label{prop2:Tbndl}
	For real $\lambda \ne 0$ and $u \in X$, the operators 
	$T_{\star,\lambda,u}$ ($\star = L\text{, or }R$) map from 
	$L^\infty(\mathbb R)$ to $L^\infty(\mathbb R)$
	with
	\begin{align}
		\|T_\star\|_{L^\infty \to L^\infty} 
			\lesssim_\lambda \|u\|_X,
	\end{align}
	where the implied constant depends on $\lambda$.
\end{prop}
\begin{proof}
	By repeating our work in the proof of Proposition 
	\ref{prop2:Tbnd}, we obtain the following estimate
	\begin{align} \label{eq2:TbndLinfty}
		\| T_\star f \|_{L^\infty}
			&\lesssim_{\lambda} \|\inn{x} u\|_{L^1} \|f\|_{\inn{x}L^\infty} \\
			&\quad + \left[ 
					\esssup_{x\in \mathbb R}  
					\left(
						\int_{|x-x'|\leq 1}
							\log\left(\frac{1}{|x-x'|}\right)
							\inn{x'}|u(x')| 
						\, \mathrm{d}y
					\right)
				\right], \nonumber
	\end{align}
	which holds for $\lambda \in (-\infty,0) \cup (0, \infty)$. Since
	$\left| \inn{x} \, u(x) \right| \leq \left| \inn{x}^2 \, u(x) \right|$,
	we see that $\|\inn{x} u\|_{L^1} \leq \|u\|_X$. Hence, Proposition 
	\ref{prop2:Tbndl} follows from \eqref{eq2:TbndLinfty}.
\end{proof}


\begin{prop}\label{prop2:Tasymp}
	The operators $T_{\star, \lambda, u}$ satisfy the asymptotic conditions 
	\begin{align}\label{eq2:Tasymp}
		\lim_{x\to-\infty} \inn{x} T_{L, \lambda, u} f(x) 
			= \lim_{x\to+\infty} \inn{x} T_{R, \lambda, u} f(x) 
			= 0
	\end{align}
	for every real $\lambda$, $u\in X$ and $f\in \inn{\dotarg}L^\infty(\mathbb R)$. 
	Alternatively stated, the limits
	\[
		\lim_{x\to-\infty} T_{L, \lambda, u} f(x), 
		\qquad \text{and} \qquad 
		\lim_{x\to+\infty} T_{R, \lambda, u} f(x) 
	\]
	converge to zero faster than $1/x$.
\end{prop}
\begin{proof}
	We prove $T_L$ satisfies asymptotic conditions \eqref{eq2:Tasymp} noting 
	that the corresponding proof for $T_R$ is similar. Since $G_L^+$ experiences
	linear growth for $|x|\gg 1$ only when $\lambda = 0$ and is otherwise bounded
	for large $|x|$, we consider the $\lambda = 0$ case first.

	Recall from Remark \ref{rmk1:littlek} in Section \ref{sec1:Intro} that
	the function $K^+(x; \lambda)$ can be written in the form
	\begin{align}
		K^+(x; \lambda) = e^{-\pi |x|}k(x; \lambda)
	\end{align}
	where $k(\dotarg; \lambda) \in L^2(\mathbb R)$ for all real $\lambda$, and 
	$k$ is uniformly bounded in $\lambda$. Consequently, 
	\begin{align*}
		T_{L,0,u} f(x)
			&= 
				\int_{-\infty}^x 
					\left(i \frac{2}{3} - (x-x')\right)
					u(x') f(x') \, \mathrm{d}x' \\
			&\quad+ \int_{\mathbb R} e^{-\pi|x-x'|} u(x') f(x')\, \mathrm{d}x',
	\end{align*}
	which implies
	\begin{align*}
		|T_{L,0,u} f(x)|
			\lesssim I_1 + I_2,
	\end{align*}
	where
	\begin{align*}
		I_1(x) &:= \int_{\mathbb R} \inn{x-x'} |u(x')|\inn{x'}\, \mathrm{d}x' \\
		I_2(x) &:= \int_{\mathbb R} \inn{x-x'}^{-2} |k(x-x')| |u(x')|\inn{x'}\, \mathrm{d}x',
	\end{align*}
	where we use the fact that $f\in \inn{\dotarg}L^\infty(\mathbb R)$ to estimate
	$f \lesssim \inn{\dotarg}$ and note that that $e^{-\pi|x|} \lesssim \inn{x}^{-N}$
	for all whole numbers $N$ (in $I_2$, we select $N=2$). 
	
	To bound $I_1$, we assume $x < 0$ and note that this implies $\inn{x'} \geq \inn{x}$
	and $\inn{x'}^{-1} \leq \inn{x}^{-1}$
	since the function $\inn{\dotarg}$ is strictly decreasing on the interval $(\infty,0)$.
	Further, it is straightforward to show that $\inn{x-x'}\leq\inn{x} + \inn{x'} 
	\leq 2\inn{x'}\lesssim \inn{x'}$. Hence
	\begin{align*}
		|I_1(x)| 
			\lesssim 
				\int_{-\infty}^x 
					\left(\inn{x'}^2 \inn{x'}^{-3}\right) 
					\left(\inn{x'}^3|u(x')|\right) 
				\, \mathrm{d}x'
			\lesssim
				\inn{x}^{-1} 
				\nm{\inn{\dotarg}^3 u}_{L^1}.
	\end{align*}
	Since the Dominated Convergence Theorem implies that the integral of
	$\inn{\dotarg}^3 |u| \, \chi_{(-\infty, x)}$ goes to zero as $x\to-\infty$, 
	we see that
	\begin{align*}
		\lim_{x\to-\infty} \inn{x} I_1(x) = 0.
	\end{align*}
	Similarly,
	\begin{align}\label{eq2:I2}
		|I_2(x)| 
			&\lesssim
				\inn{x}^{-2} 
				\int_{-\infty}^x 
					\frac{\inn{x}^2}{\inn{x-x'}^2 \inn{x'}^2}
					|k(x-x')| \inn{x'}^2 |u(x')| 
				\, \mathrm{d}x' 
				\\
			&\lesssim 
				\inn{x}^{-2} \nm{k}_{L^2} \nm{u}_{L^{2,2}},
				\nonumber
	\end{align}
	as 
	\[
		\frac{\inn{x}}{\inn{x-x'} \inn{x'}} \leq \frac{1}{\inn{x-x'}} \leq 1
	\] 
	when $x <0$ and $x' < x$. Estimate \eqref{eq2:I2} therefore implies
	\[
		\lim_{x\to -\infty} \inn{x} I_2(x)
			= \lim_{x\to -\infty} \inn{x} I_1(x)
			= 0,
	\]
	which in turn implies $T_{L,0,u} f(x)$ satifies the asymptotic
	condition \eqref{eq2:Tasymp}

	Since the sum of the residue terms in $G_L^+$ is bounded (in $x$)
	for all fixed $\lambda \ne 0$, a slight modification of the above 
	argument shows that $T_{L,\lambda, u}f(x)$ also satisfies 
	\eqref{eq2:Tasymp} when $\lambda \ne 0$. 
\end{proof}
% \begin{prop}\label{prop2:Tasymp}
% 	The limits 
% 	\begin{align}\label{eq2:Tasymp}
% 		\lim_{x\to-\infty} \inn{x}^{-1}T_{L, \lambda, u} f(x) 
% 			= \lim_{x\to+\infty} \inn{x}^{-1} T_{R, \lambda, u} f(x) 
% 			= 0
% 	\end{align}
% 	hold for every $\lambda \in \mathbb R$, $u \in X$ and 
% 	$f \in \inn{\dotarg} L^\infty(\mathbb R)$. If $\lambda \ne 0$, 
% 	then 
% 	\begin{align}\label{eq2:Tasymp0}
% 		\lim_{x\to-\infty} T_{L, \lambda, u} f(x) 
% 			= \lim_{x\to+\infty} T_{R, \lambda, u} f(x) 
% 			= 0
% 	\end{align}
% \end{prop}
% \begin{proof}
% 	Since the Dominated Convergence Theorem implies that
% 	\begin{align*}
% 		\lim_{|x| \to \infty} \big[K^+*(u\,f)\big](x) = 0
% 	\end{align*}
% 	for all real $\lambda$, the validity of this proposition depends on the 
% 	impact of the Green's function pole terms ($\mathpzc R_\star$) on the limits 
% 	\eqref{eq2:Tasymp} and \ref{eq2:Tasymp0}.
% 	Now, for $\lambda \ne 0$, 
% 	\begin{align*}
% 		\frac{1}{i} \mathpzc R_L * uf
% 			&= 
% 				\alpha(\lambda) \int_{-\infty}^x u(x')\,f(x')\,dx'
% 				+\beta(\lambda) e^{i\lambda x} 
% 					\int_{-\infty}^x e^{-i\lambda x'}u(x')\,f(x')\,dx' \\
% 		\intertext{and}
% 		\frac{1}{i} \mathpzc R_R * uf
% 			&= 
% 				\alpha(\lambda) \int_x^{\infty} u(x')\,f(x')\,dx'
% 				+\beta(\lambda) e^{i\lambda x} 
% 					\int_x^{\infty} e^{-i\lambda x'}u(x')\,f(x')\,dx',
% 	\end{align*}
% 	from which it is easy to see that \eqref{eq2:Tasymp0} follows from 
% 	the Dominated Convergence Theorem. On the other hand, for 
% 	$\lambda = 0$, we have 
% 	$\left|\mathpzc R_\star\right| \lesssim (1+|\dotarg|)\chi_\star$
% 	and
% 	\begin{align*}
% 		\inn{x}^{-1}
% 		\left|
% 			\mathpzc R_\star * u\,f
% 		\right|
% 			&\lesssim
% 				\inn{x}^{-1}
% 				\int_{\mathbb R} 
% 					\chi_\star(x-x')
% 					\big(1+ |x-x'|\big)
% 					u(x') \, f(x')
% 				\, \mathrm{d}x' \\
% 			&\leq
% 				\nm{f}_{\inn{\cdot}L^\infty}
% 				\int_{\mathbb R} 
% 					\chi_\star(x-x')
% 					\left(\frac{1+ |x-x'|}{\inn{x}\inn{x'}}\right)
% 					\big[\inn{x'}^2 u(x')\big]
% 				\, \mathrm{d}x' \\
% 			&\lesssim
% 				\nm{f}_{\inn{\cdot}L^\infty}
% 				\int_{\mathbb R} 
% 					\chi_\star(x-x')
% 					\big[\inn{x'}^2 u(x')\big]
% 				\, \mathrm{d}x',
% 	\end{align*}
% 	where we use the estimate $\frac{1+ |x-x'|}{\inn{x}\inn{x'}} \lesssim 1$ in the 
% 	above computation. \eqref{eq2:Tasymp} therefore follows from the Dominated
% 	Convergence Theorem.
% 	% Now,
% 	% \begin{align*}
% 	% 	&\big[ 
% 	% 			\alpha(\lambda) +\beta(\lambda) e^{i (\dotarg) \lambda}  
% 	% 	\big] \chi_{(0, \infty)} * (u\,f) \\
% 	% 	&\qquad=
% 	% 		\big[\alpha(\lambda) + \beta(\lambda) \, e^{i \lambda x} \big]
% 	% 		\int_{-\infty}^x e^{-i\lambda x'} \, u(x') \, f(x') \, \mathrm{d}x'.
% 	% \end{align*}
% 	% Since $\alpha(\lambda) + \beta(\lambda) \, e^{i \lambda x}$ grows 
% 	% linearly in $x$ for small $|\lambda|$, it suffices to show the integral
% 	% $\int_{-\infty}^x e^{-i\lambda x'} \, u(x') \, f(x') \, \mathrm{d}x'$ decays
% 	% like $1/x^s$, for $s>1$, as $x\to-\infty$. To that end, for $x < 0$
% 	% we find
% 	% \begin{align*}
% 	% 	\left| \int_{-\infty}^x e^{-i\lambda x'} \, u(x') \, f(x') \, \mathrm{d}x' \right|
% 	% 		&\leq \int_{-\infty}^x  |u(x') \, f(x')| \, \mathrm{d}x' \\
% 	% 		&\leq \|f\|_{\inn{\dotarg}L^\infty} 
% 	% 			\int_{-\infty}^x \inn{x'}^{-1} \big[ \inn{x'}^2 u(x') \big]\, \mathrm{d}x'
% 	% 			\\
% 	% 		&\lesssim_f \inn{x}^{-1} \int_{-\infty}^x \inn{x'}^2 u(x') \, \mathrm{d}x',
% 	% \end{align*}
% 	% where we used the fact that $\inn{x'}\geq \inn{x}$ whenever $x' < x < 0$ in 
% 	% the final line of the computation above. A similar computation also shows
% 	% that
% 	% \begin{align*}
% 	% 	\left| \int_x^{\infty} e^{-i\lambda x'} \, u(x') \, f(x') \, \mathrm{d}x' \right|
% 	% 		&\lesssim_f \inn{x}^{-1} \int_x^{\infty} \inn{x'}^2 u(x') \, \mathrm{d}x'
% 	% \end{align*}
% 	% for $x > 0$. Since
% 	% \[
% 	% 	\lim_{x\to-\infty} \int_{-\infty}^x \inn{x'}^2 u(x') \, \mathrm{d}x'
% 	% 		= \lim_{x\to+\infty} \int_x^{\infty} \inn{x'}^2 u(x') \, \mathrm{d}x'
% 	% 		= 0
% 	% \]
% 	% by the Dominated Convergence Theorem, the desired result therefore follows
% 	% from the $\lambda$-invarient estimate
% 	% $\left|\alpha(\lambda) + \beta(\lambda) e^{i\lambda x} \right| \inn{x}^{-1}
% 	% \lesssim 1$.
% \end{proof}

% \begin{prop}\label{prop2:Tasymp}
% 	The limits 
% 	\begin{align}\label{eq2:Tasymp}
% 		\lim_{x\to-\infty} \inn{x}^{-1} T_{L, \lambda, u} f(x) 
% 			= \lim_{x\to+\infty} \inn{x}^{-1} T_{R, \lambda, u} f(x) 
% 			= 0
% 	\end{align}
% 	hold for every $\lambda \in \mathbb R$, $u\in X$, and 
% 	$f \in \inn{\dotarg} L^\infty(\mathbb R)$. Further, 
% 	if $\lambda \ne 0$, then the stronger result 
% 	\begin{align}\label{eq2:Tasymplnonzero}
% 		\lim_{x\to-\infty} T_{L, \lambda, u} f(x) 
% 			= \lim_{x\to+\infty} T_{R, \lambda, u} f(x) 
% 			= 0
% 	\end{align}
% 	holds whenever $f \in L^\infty(\mathbb R)$ or 
% 	$f \in \inn{\dotarg} L^\infty(\mathbb R)$
% \end{prop}
% \begin{proof}
% 	Since the Dominated Convergence Theorem implies that
% 	\begin{align*}
% 		\lim_{|x| \to \infty} \big[K^+*(u\,f)\big](x) = 0,
% 	\end{align*}
% 	the validity of this proposition depends on the impact of the Green's function 
% 	pole terms on the limits \eqref{eq2:Tasymp} and \eqref{eq2:Tasymplnonzero}.
% 	In light of \eqref{eq2:polesumbnd}, we find
% 	\begin{align*}
% 		&\big[ 
% 				\alpha(\lambda) +\beta(\lambda) e^{i (\dotarg) \lambda}  
% 		\big] \chi_L * (u\,f) \\
% 			&\qquad=
% 				\int_{-\infty}^x
% 					\big[\alpha(\lambda) + \beta(\lambda)e^{ix\lambda}]
% 					e^{-i x' \lambda} \,
% 					u(x') \, f(x') \,
% 				dx' \\
% 			&\qquad\lesssim
% 				\inn{x}^{-1} 
% 				\int_{-\infty}^x 
% 					\Big(
% 						\frac{1+|x-x'|}{(1+|x|)(1+|x'|)} 
% 					\Big)
% 					\Big(
% 						(1+|x'|)^2 \, u(x')
% 					\Big)
% 					\Big(
% 						(1+|x'|)^{-1} \, f(x')
% 					\Big)
% 				\, \mathrm{d}x \\
% 			&\qquad\lesssim
% 				\inn{x}^{-1} \| f \|_{\inn{\dotarg} L^\infty} 
% 				\int_{-\infty}^x \inn{x'}^2 \,u(x') \, \mathrm{d}x,
% 	\end{align*}
% 	and, similarly, 
% 	\begin{align*}
% 		&\big[ 
% 				\alpha(\lambda) +\beta(\lambda) e^{i (\dotarg) \lambda}  
% 		\big] \chi_R * (u\,f) \\
% 			&\qquad\lesssim
% 				\inn{x}^{-1} \| f \|_{\inn{\dotarg} L^\infty} 
% 				\int_x^{\infty} \inn{x'}^2 \,u(x') \, \mathrm{d}x.
% 	\end{align*}
% 	Since $\inn{x'}^2 \,u(x') \, \chi_{(-\infty, x)}$ is bounded by an $L^1$ 
% 	function, the Dominated Convergence Theorem therefore implies 
% 	\eqref{eq2:Tasymp} holds for every $\lambda \in \mathbb R$, $u \in X$, 
% 	and $f \in \inn{\dotarg} L^\infty(\mathbb R)$. Limit \eqref{eq2:Tasymplnonzero}
% 	follows from \eqref{eq2:polesumbnd} and the Dominated Convergence Thorem.
% 	% From the proof of Proposition \ref{prop2:pold}, it is clear that 
% 	% \[
% 	% 	\lim_{x \to - \infty} T_L \, g(x) 
% 	% 		= \lim_{x \to + \infty} T_R \, g(x)
% 	% 		= 0
% 	% \]
% 	% whenever $g \in C_0^\infty(\mathbb R)$, as 
% 	% $g \in C_0^\infty(\mathbb R)$ and $u \in X$ implies $u \, g$ is compactly 
% 	% supported. Moreover, since $K^+*(ug)$ decays exponentially for $|x| \gg 1$
% 	% we also have that 
% 	% \[
% 	% 	\lim_{x \to - \infty} \inn{x}^{-1} T_L \, g(x) 
% 	% 		= \lim_{x \to + \infty} \inn{x}^{-1} T_R ,\ g(x)
% 	% 		= 0.
% 	% \]

% 	% For $f \in \inn{\dotarg} L^\infty(\mathbb R)$, by density let 
% 	% $f_n \in C_0^\infty(\mathbb R)$ be a sequence such that 
% 	% $\lim_{n\to \infty} \| f - f_n \|_{\inn{\dotarg} L^\infty} = 0$. To verify
% 	% $f$ satisfies \eqref{eq2:Tasymp}, let $\ve > 0$ be arbitrary, and use 
% 	% the boundedness of $T_\star: \inn{\dotarg}L^\infty(\mathbb R) \toitself$ 
% 	% to fix an integer $N > 0$ so that  
% 	% \[
% 	% 	\inn{x}^{-1} | T_{\star} f(x) - T_\star f_n(x) | < \varepsilon / 2
% 	% \]
% 	% for $x \in \mathbb R$ and every $n > N$. Take $n = N+1$ and
% 	% choose $\delta \gg 1$ sufficiently large that 
% 	% \[
% 	% 	\inn{x}^{-1} |T_L f_n(x)|, ~ \inn{y}^{-1} |T_R f_n(y)| 
% 	% 		< \varepsilon / 2.
% 	% \]
% 	% Then, 
% 	% \begin{align*}
% 	% 	\inn{x}^{-1} |T_L f(x) |
% 	% 		\leq \inn{x}^{-1} | T_L f(x) - T_L f_n(x) |
% 	% 				+ \inn{x}^{-1} | T_L f_n(x) |
% 	% 		< \varepsilon,
% 	% \end{align*}
% 	% and
% 	% \begin{align*}
% 	% 	\inn{y}^{-1} |T_R f(y) |
% 	% 		\leq \inn{y}^{-1} | T_R f(y) - T_R f_n(y) |
% 	% 				+ \inn{y}^{-1} | T_R f_n(y) |
% 	% 		< \varepsilon,
% 	% \end{align*}
% 	% for every $x < - \delta$ and $y > \delta$. Hence
% 	% \begin{align}
% 	% 	\lim_{x\to-\infty} \inn{x}^{-1} T_L f(x) 
% 	% 		= \lim_{x\to+\infty} \inn{x}^{-1} T_R f(x) 
% 	% 		= 0. 
% 	% \end{align}
% 	% {\color{red} Somehow use the fact that when $\lambda = 0$ the sum of the 
% 	% residues grows linearly to argue above limit must imply the desired result.}
% \end{proof}





\end{document}