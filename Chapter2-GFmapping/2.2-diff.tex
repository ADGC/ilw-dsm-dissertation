%%==========================================
%% Section 2.02: $\lambda$-DIFFERENTIABILITY
%%==========================================
\documentclass[../dissertation]{subfiles}


\begin{document}

\section{$\lambda$-Dif{}ferentiability}\label{sec2:diff}

We begin this section by proving a useful variant of Young's inequality
(Technical Lemma \ref{tlma2:1}) that we use in a number of proofs in this 
dissertation. We then prove that that the operators $T_{\star, \lambda, u}$ are
continuous in the spectral parameter $\lambda$ (Proposition 
\ref{prop2:Tlamcont}). The remainder of this section the focuses on proving the 
$\lambda$-dif{}ferentiability of $G_{\star}^+$. We direclty use the results 
from all four propositions in this section in proving that the Jost solution 
boundary $M_1^+$ is dif{}ferentiable in $\lambda$, and hence has a 
linearization in $\lambda$. The $\lambda$ linearization of $M_1^+$ is ultimately 
used in the proof that the direct scattering map $\mathscr D$ is Lipschitz 
continuous.

Since the proof of the $\lambda$-continuity of $T_{\star, \lambda, u}$ calls
Technical Lemma \ref{tlma2:1}, we start with the proof of that lemma.

\begin{tlma}\label{tlma2:1}
	For $f \in \inn{\dotarg}^s L^1(\mathbb R)$ and 
	$\inn{\dotarg}^s g \in L^\infty(\mathbb R)$ the inequality 
	\begin{align} \label{eq3:tlma}
		\| f * g\|_{\inn{\dotarg}^s L^\infty}
			&\leq \|f\|_{\inn{\dotarg}^s L^1} \|\inn{\dotarg}^s g\|_{L^\infty}
	\end{align}
	holds for $s\geq0$. Alternatively, if $\inn{\dotarg}^s f \in L^1(\mathbb R)$
	and $g \in \inn{\dotarg}^s L^\infty(\mathbb R)$ the estimate
	\begin{align}\label{eq3:tlma2}
		\| f * g\|_{\inn{\dotarg}^s L^\infty}
			&\leq \nm{\inn{\dotarg}^s f}_{L^1}  \nm{g}_{\inn{\dotarg}^s L^\infty}
	\end{align}
	holds instead for $s \geq 0$. 
\end{tlma}
\begin{proof}
	It is straightforward to show that
	\begin{align*}
		\frac{\inn{x'}}{\inn{x-x'}\inn{x}} \leq 1,
	\end{align*}
	for all $x, x' \in \mathbb R$, as
	\begin{align*}
		\left(\frac{\inn{x'}}{\inn{x-x'}\inn{x}}\right)^2
			= \frac{1+(x')^2}{1+(x-x')^2 + x^2 + x^2(x-x')^2}
			\leq \frac{1+(x')^2}{1+(x')^2}.
	\end{align*}
	As such, we find for $s> 0$ 
	\begin{align*}
		\| f * g\|_{\inn{\dotarg}^s L^\infty}
			&= \sup_{x\in\mathbb R} \inn{x}^{-s} 
				\int_{\mathbb R}
					\big| 
						f(x') \, g(x-x')
					\big|
				\, \mathrm{d}x'
				\\
			&= \sup_{x\in\mathbb R}
				\int_{\mathbb R}
					\Big| 
						\big[\inn{x'}^{-s} f(x')\big]
						\big[\inn{x-x'} g(x-x') \big]
					\Big|
						\left(
							\frac{\inn{x'}}{\inn{x-x'}\inn{x}}
						\right)^s
				\, \mathrm{d}x'
				\\
			&\leq 
				\left\|
						\big[\inn{\dotarg}^{-s} f \big]
						*
						\big[\inn{\dotarg}^{s} g \big]
				\right\|_{L^\infty} \\
			&\leq 
				\|f\|_{\inn{\dotarg}^s L^1} \|\inn{\dotarg}^s g\|_{L^\infty}
	\end{align*}
	by Minkowski's integral inequality \cite[Theorem 1.2.10]{Grafakos}. 
	If $s=0$, then \eqref{eq3:tlma} automatically holds by 
	\cite[Theorem 1.2.10]{Grafakos}. An analogous argument also verifies 
	\eqref{eq3:tlma2}.
\end{proof}


\begin{prop}\label{prop2:Tlamcont}
	For $u \in X \cap \inn{\dotarg}^{-2}L^\infty(\mathbb R)$, the operator 
	$T_{\star, \lambda, u} : \inn{\dotarg}L^\infty(\mathbb R) \to 
	\inn{\dotarg}L^\infty(\mathbb R)$ given by 
	\begin{align*}
		T_{\star, \lambda, u} : f\mapsto \big[G_\star^+(\cdot; \lambda)\big]*(u\,f)
	\end{align*}
	is continuous in the parameter $\lambda \in \mathbb R$ in the sense the limit 
	\[
		\lim_{h\to0} \| T_{\star, \lambda+h, u} - T_{\star, \lambda, u}
			 \|_{\inn{\dotarg}L^\infty\toitself}
			=0
	\]
	holds pointwise for each fixed $\lambda \in \mathbb R$.
\end{prop}
\begin{proof}
	To simplify notation, let $T_{\star, \lambda, u}$ be denoted by 
	$T_\lambda$. Then, we see from Technical Lemma \ref{tlma2:1}
	that
	\begin{align}\label{eq2:Tcont}
		\|(T_{\lambda+h} - T_\lambda)f\|_{\inn{\dotarg}L^\infty}
			&= 
				\left\| 
					\big[ 
						G_\star^+(\dotarg, \lambda+h) 
						- G_\star^+(\dotarg, \lambda)
					\big]
					*uf
				\right\|_{\inn{\dotarg}L^\infty}  \\
			&\leq 
				\left\| 
						G_\star^+(\dotarg, \lambda+h) 
						- G_\star^+(\dotarg, \lambda)
				\right\|_{\inn{\dotarg}L^\infty}
				\left\|
					\inn{\dotarg} u f
				\right\|_{L^\infty}
				\nonumber
	\end{align}
	for all $f \in \inn{\dotarg}L^\infty(\mathbb R)$ as 
	$u \in X \cap \inn{\dotarg}^{-2}L^\infty(\mathbb R)$ implies 
	$\inn{\dotarg} u\, f\in \inn{\dotarg}L^\infty(\mathbb R)$. 
	Noting that the argument in the proof of Proposition 
	\ref{prop2:ldervG} (which does not depend on this Proposition) implies
	\begin{align}\label{eq2:Glamcont}
		\lim_{h\to\infty} 
			\left\| 
					G_\star^+(\dotarg, \lambda+h) 
					- G_\star^+(\dotarg, \lambda)
			\right\|_{\inn{\dotarg}L^\infty}
			=0,
	\end{align}
	Proposition \ref{prop2:Tlamcont} follows from \eqref{eq2:Tcont}
	and \eqref{eq2:Glamcont}.
\end{proof}


\begin{rmk}\label{rmk2:notation}
	The notation in proofs of Propositions \ref{prop2:ldervG} through 
	\ref{prop2:ghconv} can get unnecessarily complicated. To avoid this, 
	in these proof we use the notation $G(x, \lambda)$ and $G(\lambda)$ 
	as stand-ins for $G_\star^+(x; \lambda)$. 
\end{rmk}

\begin{prop}\label{prop2:ldervG}
	For each fixed $x\ne 0$ and $\lambda \in \mathbb R$, the Green's function 
	boundary $G_\star^+$ ($\star = L \text{, or } R$) is dif{}ferentiable in 
	the spectral parameter $\lambda$, and 
	\begin{align*}
		\frac{\partial}{\partial \lambda} G_\star^+(x; \lambda)
			= \frac{1}{2\pi} 
				\int_{{\Gamma_\star}} e^{ix\xi} 
					\left(
						\frac{\partial}{\partial \lambda} \frac{1}{p(\xi; \lambda)}
					\right)
				\, \mathrm{d}\xi.
	\end{align*}
\end{prop}
\begin{rmk}
	Since
	\[
		\frac{\partial}{\partial \lambda} \frac{1}{p(\xi; \lambda)}
			= 
				\frac{\zeta'(\lambda)}
				{\Big(\xi - \zeta(\lambda)\big(1-e^{-2\xi}\big)\Big)^2},
	\]
	the function $\frac{\partial}{\partial \lambda} \frac{1}{p(\xi; \lambda)}$ 
	decays exponentially to zero as $\xi\to -\infty$, and, for $\xi > 0$, decays
	like $1/\xi^2$. As such, the integral
	\begin{align*}
		\int_{\Gamma_\star}
			\left(
				\frac{\partial}{\partial \lambda} \frac{1}{p(\xi; \lambda)}
			\right)
		\, \mathrm{d}\xi.
	\end{align*}
	converges absolutely.
\end{rmk}
\begin{proof}[Proof of Proposition \ref{prop2:ldervG}]
	In accordance with Remark \ref{rmk2:notation}, we write $G_\star^+(x; \lambda)$ 
	as either $G(x, \lambda)$ or as $G(\lambda)$. We further define $G_h$ to be the 
	dif{}ference quotient
	\[
		G_h:= \frac{G(\lambda + h)- G(\lambda)}{h}.
	\]
	and seek to prove that $\lim_{h\to0} G_h$ converges. For $|h|>0$ suf{}ficiently 
	small, we may assume by analyticity that $G(\lambda)$ and $G(\lambda+h)$ 
	share the same contour of integration $\Gamma$. If $\lambda = 0$, the
	contour $\Gamma$ runs along with a single small semi-circular detour below the 
	real axis to avoid passing through $\xi = 0$ and $\xi=h$. In which case
	\begin{align*}
		G_h(\lambda) = 
			\frac{1}{2\pi}\int_\Gamma e^{ix\xi}\,
				\frac{1}{h}\,
				\left[
					\frac{1}{p(\xi; \lambda +h)}
					-\frac{1}{p(\xi; \lambda)}
				\right]
			\, \mathrm{d}\xi
			= 
				\int_\Gamma e^{ix\xi} 
					\left(\frac{1}{p_\lambda(\xi)}\right)_{h}
				\, \mathrm{d}\xi,
	\end{align*}
	where we define
	\begin{align*}
		\left(\frac{1}{p_\lambda(\xi)}\right)_{h}
			:= \frac{1}{h}\,
				\left[
					\frac{1}{p(\xi; \lambda +h)}
					-\frac{1}{p(\xi; \lambda)}
				\right]
	\end{align*}
	as the dif{}ference quotient of $1/p$ with respect to $\lambda$.
	Since 
	\[
		\left(\frac{1}{p_\lambda(\xi)}\right)_{h}
			= \frac{1}{h}
				\frac{
					\big[
						\zeta(\lambda + h) - \zeta(h)
					\big]
					\big(1-e^{-2\xi}\big)
				}
				{p(\xi; \lambda+h)\,p(\xi;\lambda)}
	\]
	decays exponentially as $\xi \to -\infty$ and decays as 
	$1/\xi^2$ for positive $\xi$, 
	\[
		\left(\frac{1}{p_\lambda(\xi)}\right)_{h} \in L^1(\Gamma)
	\]
	for each fixed $h\ne0$. Further, using the continuity of $\zeta$ and the reverse 
	triangle inequality, we also have
	\[
		\left|\left(\frac{1}{p_\lambda(\xi)}\right)_{h}\right| 
			\lesssim_\lambda 
				\frac{
					\big|1-e^{-2\xi}\big|
				}{
					|\xi|\big| \xi - \zeta(\lambda) \big(1-e^{-2\xi}\big)\big|
				}
			=: \iota(\xi).
	\]
	Since $\iota$ is continuous in $\xi$ on $\Gamma$ and $\iota(\xi) 
	= \mathcal O(1/\xi^2)$ for large $|\xi|$, one application of the 
	Dominated Convergence Theorem completes this proof.
\end{proof}

\begin{prop}\label{prop2:lderivGspace}
	The partial derivative $\frac{\partial}{\partial \lambda} G_\star^+$
	($\star = L \text{, or } R$)
	of the Green's function boundary value $G_\star^+$ lies in $\inn{x}^s L_x^1(\mathbb R)$
	for $s> 3$ and all $\lambda \in \mathbb R$. 
	If $\lambda \in \mathbb R$ and $\lambda \ne 0$, then 
	$\frac{\partial}{\partial \lambda} G_\star^+ 
	\in \inn{x}^s L_x^1(\mathbb R)$ for $s > 2$.
\end{prop}
\begin{proof}
	Define
	\[
		g(\xi; \lambda) 
			:= e^{ix\xi}\, \frac{\partial}{\partial \lambda} \frac{1}{p},
	\]
	Through direct computation, one can show 
	\begin{align}\label{eq2:gzerores}
		\Res_{\xi=0} g
			= 
				\frac{
					2 e^{2\lambda}\big(e^{2\lambda}-2\lambda-1\big)
				}{
					\big(
						1 - e^{2\lambda}+ 2\lambda e^{2\lambda}
					\big)^2
				},
	\end{align}
	\begin{align}\label{eq3:glamres}
		\Res_{\xi=\lambda} g
			= 
				\frac{
					2 e^{2\lambda} - 2 - 4 \lambda e^{2\lambda}
					+ i 
					\big[
						x -
						2 x e^{2\lambda} + x e^{4\lambda} + 2x\lambda
						- 2 x \lambda e^{2\lambda}
					\big]
				}
				{\big(e^{2\lambda} - 2 \lambda -1\big)^2}
				\, e^{i x \lambda},
	\end{align}
	and 
	\begin{align}\label{eq2:gcollapsingpoles}
		\lim_{\lambda\to0} 
				\big[
					\Res_{\xi=0} g + \Res_{\xi=\lambda} g
				\big]
			= -\frac{1}{2} \, x^2 + i \, \frac{1}{3} \, x.
	\end{align}
	In particular, equations \eqref{eq2:gzerores}, \eqref{eq3:glamres},
	and \eqref{eq2:gcollapsingpoles} imply
	\begin{align}\label{eq3:gsummary}
		\begin{aligned}
			\big|\Res_{\xi=0} g \big| \lesssim_\lambda 1, \qquad
		\big|\Res_{\xi=\lambda} g \big| \lesssim_\lambda |x|, \\
		\lim_{\lambda\to0} 
				\big[
					\Res_{\xi=0} g + \Res_{\xi=\lambda} g
				\big]
			= \mathcal O(x^2)
		\end{aligned}
	\end{align}

	Further, on our work in the proof of Lemma \ref{prop2:ldervG}
	also implies that
	\begin{align}\label{eq3:pldervspace}
		\frac{\partial}{\partial \lambda} \frac{1}{p}
		 \in L_\xi^1(\mathbb R \pm i \pi).
	\end{align}
	As such, after applying the contour shift demonstrated in Figure
	\ref{fig1:GammaContour} to the contour of integration for
	$\frac{\partial}{\partial \lambda} G_L^+$, 
	Lemma \ref{prop2:lderivGspace} is an immediate 
	consequence of \eqref{eq3:gsummary} and \eqref{eq3:pldervspace}.
\end{proof}

\begin{prop}\label{prop2:ghconv}
	The dif{}ference quotient
	\[
		G_h(x; \lambda) := \frac{G_L^+(x; \lambda+h) - G_L^+(x; \lambda)}{h}
	\]
	converges to $\frac{\partial}{\partial \lambda} G_L^+$ in 
	$\inn{x}^s L_x^1(\mathbb R)$ for $s > 3$ and all real $\lambda$. 
	If $\lambda$ is real and non-zero, then 
	this convergence happens in $\inn{x}^s L_x^1(\mathbb R)$ for 
	$s > 2$.
\end{prop}
\begin{proof}
	Direct computation yields the following results
	\begin{align*}
		\Res_{\xi=0}
				\left[
					e^{ix\xi}
					\left(
						\frac{1}{p_\lambda(\xi)}
					\right)_h
				\right]
			&= 
				\frac{
					2 e^{2\lambda}
					\big(
						h \, e^{2(\lambda+h)} 
						+ \lambda 
						- e^{2h}(\lambda + h)
					\big)
				}{
					h 
					\big(
						1 + e^{2\lambda}(2\lambda - 1)
					\big)
					\big(
						1 + e^{2(\lambda+h)(2\lambda + 2h - 1)}
					\big)
				} \\
		\Res_{\xi=\lambda}
				\left[
					e^{ix\xi}
					\left(
						\frac{1}{p_\lambda(\xi)}
					\right)_h
				\right]
			&= - \frac{1}{h}\frac{e^{2\lambda}-1}{e^{2\lambda}-2\lambda-1}
				\, e^{ix\lambda} \\
		\Res_{\xi=\lambda+h}
				\left[
					e^{ix\xi}
					\left(
						\frac{1}{p_\lambda(\xi)}
					\right)_h
				\right]
			&= 
				 \frac{1}{h}
				 \frac{e^{2(\lambda+h)}-1}{e^{2(\lambda+h)}-2(\lambda+h)-1}
				\, e^{ix(\lambda+h)}
	\end{align*}
	As such, through further direct computation, we find the
	limits
	\begin{align*}
		\lim_{h\to0}
			\Res_{\xi=0}
				\left\{ 
				\left(
					e^{ix\xi}
					\frac{\partial}{\partial \lambda}
						\frac{1}{p_\lambda(\xi)}
				\right)
				-
				\Res_{\xi=0}
				\left[
					e^{ix\xi}
					\left(
						\frac{1}{p_\lambda(\xi)}
					\right)_h
				\right]
				\right\}
			=0,
	\end{align*}
	and
	\begin{align*}
		\lim_{h\to0}
			\Res_{\xi=\lambda}
				\left\{ 
				\left(
					e^{ix\xi}
					\frac{\partial}{\partial \lambda}
						\frac{1}{p_\lambda(\xi)}
				\right)
				-
				\sum_{k \in \{\lambda, \lambda+h\}}
				\Res_{\xi=k}
				\left[
					e^{ix\xi}
					\left(
						\frac{1}{p_\lambda(\xi)}
					\right)_h
				\right]
				\right\}
			=0
	\end{align*}
	hold pointwise for each $x \in \mathbb R$. Thus, since
	$\left(\frac{1}{p_\lambda(\xi)}\right)_h \in L_\xi^1(\mathbb R \pm i\pi)$
	implies by Fourier theory that $\int_{\mathbb R\pm i\pi} e^{ix\xi} 
	\left(\frac{1}{p_\lambda(\xi)}\right)_h \, \mathrm{d}\xi$ is continuous and bounded 
	(in $x \in \mathbb R$), we may complete this proof by doing a contour shift
	and then applying Dominated Convergence to 
	$\int_{\mathbb R} \inn{x}^{-s} \, G_h (x) \, \mathrm{d}x$.
	% \begin{align}
	% 	\Res_{\xi=\lambda}
	% 			\left[
	% 				e^{ix\xi}
	% 				\left(
	% 					\frac{1}{p_\lambda(\xi)}
	% 				\right)_h
	% 			\right]
	% 		= - \frac{1}{h}\frac{e^{2\lambda}-1}{e^{2\lambda}-2\lambda-1}
	% 			\, e^{ix\lambda}
	% \end{align}
	% \begin{align}
	% 	\Res_{\xi=\lambda+h}
	% 			\left[
	% 				e^{ix\xi}
	% 				\left(
	% 					\frac{1}{p_\lambda(\xi)}
	% 				\right)_h
	% 			\right]
	% 		= 
	% 			 \frac{1}{h}
	% 			 \frac{e^{2(\lambda+h)}-1}{e^{2(\lambda+h)}-2(\lambda+h)-1}
	% 			\, e^{ix(\lambda+h)}
	% \end{align}
	% \begin{align}
	% 	&\lim_{\lambda\to0}
	% 		\left\{
	% 			\sum_{k \in \{0, \,\lambda,\, \lambda+h\}}
	% 			\Res_{\xi=k}
	% 			\left[
	% 				e^{ix\xi}
	% 				\left(
	% 					\frac{1}{p_\lambda(\xi)}
	% 				\right)_h
	% 			\right]
	% 			% +\Res_{\xi=\lambda}
	% 			% \left[
	% 			% 	e^{ix\xi}
	% 			% 	\left(
	% 			% 		\frac{1}{p_\lambda(\xi)}
	% 			% 	\right)_h
	% 			% \right]
	% 			% +\Res_{\xi=\lambda+h}
	% 			% \left[
	% 			% 	e^{ix\xi}
	% 			% 	\left(
	% 			% 		\frac{1}{p_\lambda(\xi)}
	% 			% 	\right)_h
	% 			% \right]
	% 		\right\} \\
	% 	&\qquad =
	% 		\frac{1}{h}
	% 		\left[
	% 			\frac{e^{2h}-1}{e^{2h}-2h-1}\,e^{ihx}
	% 			-
	% 			\frac{2\big(e^{2h}(h+1)-1\big)}
	% 				{e^{2h}(6h-3)+3}
	% 			- \frac{1}{3}
	% 			- ix
	% 		\right]
	% 		\nonumber
	% \end{align}
	% \begin{align}
	% 	&\lim_{h\to0}
	% 		\lim_{\lambda\to0}
	% 			\left\{
	% 				\sum_{k \in \{0, \,\lambda,\, \lambda+h\}}
	% 				\Res_{\xi=k}
	% 				\left[
	% 					e^{ix\xi}
	% 					\left(
	% 						\frac{1}{p_\lambda(\xi)}
	% 					\right)_h
	% 				\right]
	% 			\right\} \\
	% 	&\qquad=
	% 		- \frac{1}{2} x^2 + i \frac{1}{3} x \nonumber
	% \end{align}
\end{proof}

\end{document}