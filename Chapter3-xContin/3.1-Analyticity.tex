%%==============================================
%% Section 3.01: Analytic Extension to the Strip
%%==============================================

\documentclass[../dissertation.tex]{subfiles}


\begin{document}
\section{Analytic Extension to the Strip}\label{sec3:Analyticity}

%%
%% The following was taken directly (with a modification of the notation) 
%% from from Prof. Perry's section of the kpw1.tex file
%% 
% We wish to show that $M_1^+$ is the lower boundary value of a function analytic in the 
% strip $\{ z \in \CC ~:~ 0 < \im z < 2\}$ which satisfies the spectral problem 
% \eqref{eq0:SpecProb}. For the moment we fix $\lambda$ and supress the $\lambda$ dependence of 
% $M_1$ and $G_L$ for brevity. As a candidate for the analytic continuation we consider the 
% function ($z=x+iy$)
% \begin{equation}
% 	\label{eq2:ILW.M.ac}
% 	M_1(z) = 1 + \int_{-\infty}^\infty G_L(x-x'+iy) u(x') M_1(x') \, \mathrm{d}x'.
% \end{equation}
% for $0<y<2$. 
A natural candidate for the analytic extension of $G_\star$ to the open strip 
$\mathcal S_1 = \{ z \in \mathbb C ~:~ 0 < \im z < 2\}$ is
\begin{equation}
	\label{eq3:ILW.GFz}
	G_\star(x+iy) 
		:= \frac{1}{2\pi} 
			\int_{\Gamma_\star} e^{ix\xi} 
				\frac{e^{-y\xi}}{\xi - \zeta\left(1-e^{-2\xi}\right)} 
			\, \mathrm{d}\xi,
		\qquad (\star = L \text{, or } R)
\end{equation}
which is a convergent integral for $0<y<2$ owing to the new factor $e^{-y\xi}$. 
A straightforward argument with the dominated convergence theorem shows that 
$G_\star(z)$ is continuous on the open strip $\mathcal S_1$. It then follows from 
Morera's theorem and \eqref{eq3:ILW.GFz} that $G_\star(z)$ is analytic in $z$.

Next we consider convolution of $G_\star(\dotarg+iy)$ with $L^1$ functions. Since the integral 
\eqref{eq3:ILW.GFz} is absolutely convergent for $y$ in compact subintervals of $(0,2)$, it 
follows that for any $f \in L^1(\RR)$, $G_\star(\dotarg+iy)* f$  obeys the uniform
in $\lambda$ bound 
\[
	\left\|G_\star(\dotarg+iy; \lambda)* f\right\|_{\inn{\dotarg}L^\infty} 
		\lesssim_{\, y} \left\|f\right\|_{L^1}
\] 
where the implied constant has the same uniformity. Another application of Morera's 
theorem shows that, for any $f \in C_0^\infty(\RR)$, the convolution $G_\star(\dotarg+iy)*f$ 
defines an analytic function of $z$ in $\mathcal S_1$. Finally let 
$f \in L^1(\mathbb R) \cap L^p(\mathbb R)$ for some 
$p \in (1,2]$, and let $\{ f_n \}_{n \in \NN}$ be a sequence from $C_0^\infty(\mathbb R)$ 
converging to $f$ in $L^1 \cap L^p$. Then $\left(G_\star(\dotarg+iy)*f_n\right)(x)$ converges 
uniformly to $\left(G_\star(\dotarg+iy)*f\right)(x)$ on compact subsets of $\mathcal S_1$, 
so $G_\star(\dotarg+iy)*f$ is also analytic in $\mathcal S_1$. 
% This shows that $M(z)$ as defined by \eqref{eq2:ILW.M.ac} defines an 
% analytic function on $S$. 

It remains to show that $G_\star$ as a convolution operator has an upper boundary 
value $G_\star^-$ and to obtain an effective formula for $G_\star^-$. Using the 
``boxcar'' contour in Figure \ref{fig1:GammaContour} again we can compute
\begin{subequations}
	\label{eq3:ILW.GFz.rep}
	\begin{align}
		\label{eq3:ILW.GLz.rep}
		G_L(x+ iy; \lambda)
			&= 
				K(x+iy; \lambda) 
				+ i \big[
						\alpha(\lambda) + \beta(\lambda) e^{i\lambda x}e^{-\lambda y} 
					\big] \chi_L(x) \\
			&= 
				K(x+iy; \lambda) + \mathpzc R_L(x+iy; \lambda) 
				\nonumber \\
		G_R(x+ iy; \lambda)
			&= 
				K(x+iy; \lambda) 
				- i \big[
						\alpha(\lambda) + \beta(\lambda) e^{i\lambda x}e^{-\lambda y} 
					\big] \chi_R(x) \\
			&= 
				K(x+iy; \lambda) - \mathpzc R_R(x+iy; \lambda) 
				\nonumber
	% G_L(x+iy; \lambda) =
	% 	\begin{cases}
	% 		-K(x+iy; \lambda),								
	% 			&	x<0 \\
	% 		i \left(
	% 			\alpha(\lambda) + \beta(\lambda) e^{i\lam x}e^{-y\lam} 
	% 		\right)
	% 		- K(x+iy; \lambda), 
	% 			& x>0	\\
	% 	\end{cases}
\end{align}
\end{subequations}
Here
\begin{equation}
	\label{eq3:ILW.K.z}
	K(x + iy; \lambda) 
				:= \frac{e^{-\pi|x|} \exp\big(- i \pi y \sign(x)  \big)}{2\pi} 
					\int_{\mathbb R} e^{i x \xi} 
						\frac{e^{-y\xi}}{p(\xi; \lambda) + i \pi \sign(x)}
					\, \mathrm{d}\xi
\end{equation}
is defined by a convergent integral for $y \in (0,2)$, so that $K(x+iy)$ is actually 
a bounded continuous function. We can now study boundary values for the convolution of 
$G_L(\dotarg+iy)$ with a function on the line as $y \nearrow 2$.  Note that
\[
	K(x + iy; \lambda)
		= \frac{1}{2 \pi} \int\limits_{\Sigma_{\sign(x)}} \frac{e^{i(x + iy) \xi}}{p(\xi)} \, \mathrm{d}\xi,
\]
where $\Sigma_{\sign(x)}$ is the contour defined at the beginning of Subsection \ref{sec1:AsympK},
as 
\[
	i(x + iy)\big(\xi + i \pi \sign(x) \big)
		= \big( i x \xi - y \xi\big) 
			+ \big( -\pi x \sign(x) - i \pi y \sign(x) \big).
\]

%%
%% End verbatum stealing from Prof. Perry's notes. 
%% Begin partial thievery of Prof. Perry's notes.
In our analysis of this limit, we first claim that for any $f \in L^1 \cap L^p$ 
($1 < p \leq 2$), we can rewrite the convolutions $G_\star*f$ as 
\begin{subequations}
	\label{eq3:ILW.(GFz*f)}
	\begin{align}
		\label{eq3:ILW.(GLz*f)}
		\big(G_L(\dotarg + iy)*f\big)(x) 
			&= i\alpha(\lambda) \int_{-\infty}^x f(x') \, \mathrm{d}x' \\
		    &\quad + i\beta(\lambda) \, e^{i\lam x}e^{-\lam y} 
		    		\int_{-\infty}^x e^{-i\lam x'} f(x')\, \mathrm{d}x' 
		    	\nonumber \\
		    &\quad +\int_{\mathbb R} \mathfrak C(x-x',y) f(x') \, \mathrm{d}x' 
				\nonumber \\
			&\quad + \frac{e^{-i \pi y}}{2 \pi i} 
				\int_{-\infty}^x e^{-\pi|x-x'|} \frac{1}{(x-x') - i(2-y)} f(x') \, \mathrm{d}x'
				\nonumber  \\
			&\quad + \frac{e^{i \pi y}}{2 \pi i} 
				\int_x^{\infty} e^{-\pi|x-x'|} \frac{1}{(x-x') - i(2-y)} f(x') \, \mathrm{d}x'
				\nonumber \\[0.5\baselineskip]
		\label{eq3:ILW.(GLz*f)}
		\big(G_R(\dotarg + iy)*f\big)(x) 
			&= i\alpha(\lambda) \int_{-\infty}^x f(x') \, \mathrm{d}x' \\
		    &\quad + i\beta(\lambda) \, e^{i\lam x}e^{-\lam y} 
		    		\int_{-\infty}^x e^{-i\lam x'} f(x')\, \mathrm{d}x' 
		    	\nonumber \\
		    &\quad +\int_{\mathbb R} \mathfrak C(x-x',y) f(x') \, \mathrm{d}x' 
				\nonumber \\
			&\quad + \frac{e^{-i \pi y}}{2 \pi i} 
				\int_{-\infty}^x e^{-\pi|x-x'|} \frac{1}{(x-x') - i(2-y)} f(x') \, \mathrm{d}x'
				\nonumber  \\
			&\quad + \frac{e^{i \pi y}}{2 \pi i} 
				\int_x^{\infty} e^{-\pi|x-x'|} \frac{1}{(x-x') - i(2-y)} f(x') \, \mathrm{d}x'
				\nonumber
	\end{align}
\end{subequations}
where $\mathfrak C$ is as defined in equations \eqref{eq3:GFcont} and \eqref{eq0:smallR} above.
Indeed, one can see directly from \eqref{eq3:ILW.GFz.rep} 
\begin{align*}
	\big(G_L(\dotarg + iy)*f\big)(x) 
		&= \big(K(\dotarg + iy) * f\big)(x) + \mathpzc R_L(\dotarg + iy) * f \\
	\big(G_R(\dotarg + iy)*f\big)(x) 
		&= \big(K(\dotarg + iy) * f\big)(x) + \mathpzc R_R(\dotarg + iy) * f.
\end{align*}
Further, since 
\begin{align*}
	\big(K(\dotarg + iy) * f\big)(x)
		&= 
			\frac{e^{-i \pi y}}{2 \pi i} 
				\int_{-\infty}^x 
					\left(
						e^{-\pi |x - x'|} 
						\int_{\mathbb R}
							e^{ix\xi} \frac{e^{-y\xi}}{p(\xi;\lambda) + i\pi}
						\, \mathrm{d}\xi
					\right)
				f(x')
				\, \mathrm{d}x' \\
		&\quad+
			\frac{e^{i \pi y}}{2 \pi i} 
				\int_x^{\infty}
					\left(
						e^{-\pi |x - x'|} 
						\int_{\mathbb R}
							e^{ix\xi} \frac{e^{-y\xi}}{p(\xi;\lambda) - i\pi}
						\, \mathrm{d}\xi
					\right)
				f(x')
				\, \mathrm{d}x' \\
\end{align*}
the identity
\begin{align}\label{eq3:RanId}
	\frac{e^{-y \xi}}{p(\xi , \lambda) \pm i \pi} 
		= \frac{1}{\zeta(\lambda)} e^{(2-y)\xi} \chi_{\mathbb R^-}(\xi) 
			+ \rho \big(\xi, y, \pm 1; \lambda\big),
\end{align}
where $\chi_{\mathbb R^-}$ denotes the characteristic function of $(-\infty, 0)$, along with the
Fourier identity
\[
	\int_{\mathbb R} 
		e^{i x \xi} \frac{1}{\zeta(\lambda)} e^{(2-y)\xi} \chi_{\mathbb R^-}(\xi)
	\, \mathrm{d}\xi
	= 
		\frac{1}{ix + (2-y)}
\]
imply equations \eqref{eq3:ILW.(GFz*f)} hold. 


To verify identity \eqref{eq3:RanId} note that 
\begin{align*}
	\zeta 
		&= \big(-\zeta + \zeta \, e^{-2\xi} + \zeta) e^{2\xi} \\
		&= 
			\big[
				\xi - \zeta\big(1-e^{-2\xi}\big) 
				\pm i \pi - \xi \mp i \pi 
			\big]
			e^{2\xi},
\end{align*}
which implies 
\begin{align*}
	&\frac{1}{\zeta} e^{(2-y)\xi} 
			+ \frac{1}{\zeta}
				\frac{
					\big(\zeta - \xi \mp i\pi\big)e^{(2-y)\xi}
				}
				{\xi-\zeta(1-e^{2\xi}) \pm i \pi}  \\
	&\qquad= 
		\frac{1}{\zeta} 
		\Big\{
			\big[
				\xi - \zeta\big(1-e^{-2\xi}\big) 
				\pm i \pi - \xi \mp i \pi 
			\big]
			e^{2\xi}
		\Big\}
		\frac{e^{-y\xi}}{p(\xi; \lambda) \pm i \pi} \\
	&\qquad= \frac{e^{-y\xi}}{p(\xi; \lambda) \pm i \pi},
\end{align*}
as claimed.


By introducing the notation\label{sym:almostExpCauchy}
\begin{align*}
	\left( \mathcal E_\varepsilon f \right)(x) 
		&:=  \frac{e^{-i \pi (2-\varepsilon)}}{2 \pi i} 
			\int_{-\infty}^x 
				\frac{e^{-\pi|x-x'|}}{(x-x') - i\varepsilon} f(x') 
			\, \mathrm{d}x' \\
		&\quad + \frac{e^{i \pi (2-\varepsilon)}}{2 \pi i} 
			\int_x^{\infty} 
				\frac{e^{-\pi|x-x'|} }{(x-x') - i\varepsilon} f(x') 
			\, \mathrm{d}x',
\end{align*}
equation \eqref{eq3:ILW.(GFz*f)} can be rewritten more simply as 
\begin{subequations}
	\label{eq3:ILW.(Gz*f).simp}
	\begin{align}
		\label{eq3:ILW.(GLz*f).simp}
		\big(G_L(\dotarg + iy)*f\big)(x) 
			&= \big[\mathfrak C(\dotarg, y) + \mathpzc R_L(\dotarg+iy)\big] * f(x) 
				+ \mathcal E_{(2-y)} f(x)
			\\
		\label{eq3:ILW.(GRz*f).simp}
		\big(G_R(\dotarg + iy)*f\big)(x) 
			&= \big[\mathfrak C(\dotarg, y) - \mathpzc R_R(\dotarg+iy) \big] * f(x) 
				+ \mathcal E_{(2-y)} f(x)
	\end{align}
\end{subequations}
Since the integrands of the residue terms
\begin{align*}
	\mathpzc R_L(\dotarg+iy)*f(x)
		&= i\alpha \int_{-\infty}^x f(x') \, \mathrm{d}x' 
		       + i\beta(\lambda) \, e^{\lambda(ix - y)}
		    		\int_{-\infty}^x e^{-i\lam x'} f(x')\, \mathrm{d}x' \\
	\mathpzc R_R(\dotarg+iy)*f(x)
		&=  i\alpha \int_{-\infty}^x f(x') \, \mathrm{d}x' 
		       + i\beta(\lambda) \, e^{\lambda(ix - y)}
		    		\int_x^{\infty} e^{-i\lam x'} f(x')\, \mathrm{d}x' 
\end{align*}
in \eqref{eq3:ILW.(Gz*f).simp} do not involve $y$, they are certainly well behaved 
under the limit $y \nearrow 2$. Further, the purpose of decomposing the convolution
operator $G_L$ as shown in \eqref{eq3:ILW.(Gz*f).simp} is to isolate the singularity that 
results under the limit $y \nearrow 2$. Indeed, a cursory inspection of the
convolution operator $\mathfrak C$ leads one to believe that it is well behaved under 
this limit, which 
is something we discuss further in Section \ref{sec3:ContPart}.

The $\mathcal E_{(2-y)} f(x)$ terms in \eqref{eq3:ILW.(Gz*f).simp} captures the singular 
portion of $K$ under the $y\nearrow 2$ limit. Understanding the behavior of these terms 
under this limit involves much more delicate analysis and is the subject of Sections 
\ref{sec3:CauchyTrans} and \ref{sec3:BndE}. 

\begin{rmk}
	That $\mathpzc R_\star$ grows linearly for $\lambda = 0$ is the single reason
	why we can only guarantee the limit \eqref{eq3:lim} converges as an $L^p$ limit 
	for functions with sufficient decay (\textit{i.e.} $L^{p,1}$) and not for all
	$L^p$ functions when $\lambda = 0$.
\end{rmk}

\end{document}