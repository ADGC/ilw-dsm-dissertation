%%==================================
%% Section 3.2: The Continuous Limit
%%==================================

\documentclass[../dissertation.tex]{subfiles}


\begin{document}
\section{The Continuous Limit}\label{sec3:ContPart}


With a little work, it is straightforward to observe that $\rho$ decays exponentially 
as $|\xi| \to \infty$. So, since $\rho$ is also bounded, the exponential decay of $\rho$
means that $\rho$ is certainly in $L^1(\mathbb R)$ and thus 
$\wc \rho \in L^\infty(\mathbb R)$. While the singularity of $\rho$ at $\xi = 0$ means 
that it is very unlikely $\rho$ is $L^1$, the exponential factor $e^{-\pi|\dotarg|}$ 
in $\mathfrak C(\dotarg, y)$, where 
$2\pi \, \mathfrak C(x, y) = e^{\pi|x|} e^{-i\sign(x)\, \pi\, y} \, \rho(x)$ does make 
$\mathfrak C$ an 
$L^1$ function for all $y \in [0, 2]$. As such, proving the $L^p$ convergence
of $\mathfrak C(\dotarg, y) * f$ to $\mathfrak C(\dotarg, 2) * f$ as $y\nearrow 2$ boils 
down to using 
just two tools: \cite[Theorem 1.2.10]{Grafakos}\footnote{\label{note1}See the appendix
titled \hyperref[app:HA]{``Harmonic Analysis Results.''}}
% \ref{app:HA}} 
and the Dominated Convergence
Theorem. Proving the pointwise (\textit{a.e.}) convergence of this limit comprises
using uniform (in $y$) estimates on $\mathfrak C$ and a density argument. 
We do all of this in the proof of the following result.

\begin{thm}\label{thm2:GoodStuff}
	For $f \in L^p(\mathbb R)$ with $1<p<\infty$, the limit 
	\begin{align}\label{eq2:GoodLim}
		\lim_{y\nearrow2} \big(\mathfrak C(\dotarg, y)*f)(x) 
			= 	\big(\mathfrak C(\dotarg, 2)*f)(x) 
	\end{align}
	holds both as a pointwise \textit{a.e.} limit and as an $L^p$ limit. 
\end{thm}
\begin{proof}
	Recall that
	\begin{align*}
		\mathfrak C(x,y)
			&:= \frac{1}{2\pi} e^{-\pi|x|} \, e^{- \sign(x) \, i \pi y }
				\int_{\mathbb R} e^{ix \xi} \rho\big(\xi, y, \sign(x)\big) \, \mathrm{d}\xi \\
			&= \frac{1}{2\pi} e^{-\pi|x|} \, e^{- \sign(x) \, i \pi y }
				\rho\big(x, y , \sign(x)\big)
	\end{align*}
	and
	\begin{align*}
		\rho\big(\xi, y, \sign(x); \lambda\big)
			:= 	
				\begin{cases}
					\dfrac{e^{-y\xi}}{p(\xi; \lambda) + \sign(x) i\pi}, & \xi > 0\\
					\\
					\dfrac{1}{\lambda} \dfrac{\big(\lambda- \xi - \sign(x) \, i\pi \big)e^{(2-y)\xi} }
							{p(\xi; \lambda) + \sign(x) \, i\pi},
						&	\xi < 0.
				\end{cases}
	\end{align*}
	As such, for each $x \in \mathbb R$,
	\begin{align}\label{eq2:GoodLp}
		&2 \pi |\mathfrak C(x, y) - \mathfrak C(x, 2)| \\
			&\qquad\leq e^{-\pi|x|} \Big|e^{-\sign(x) \, i \pi y} - e^{-\sign(x) \, 2 i \pi} \Big|
				\int_{\mathbb R} \big| \rho\big(\xi, y, \sign(x)\big)\big|\, \mathrm{d}\xi \nonumber \\
			&\qquad\quad + e^{\pi|x|} e^{-\sign(x) \, 2 i \pi} 
				\int_{\mathbb R}
					\Big| 
						\rho\big(\xi, y, \sign(x)\big) - \rho\big(\xi, 2, \sign(x)\big)
					\Big|
				\, \mathrm{d}\xi \nonumber 
	\end{align}
	Now 
	\begin{align*}
		\big|\rho\big(\xi, y, \sign(x); \lambda\big)\big|
			&=
				\begin{cases}
					\big(p(\xi)^2 + \pi^2\big)^{-\frac{1}{2}} \, e^{-y\xi}, & \xi > 0 \\
					\frac{1}{\lambda} 
						\left( 
							\frac{(\lambda-\xi)^2 + \pi^2}{p(\xi)^2 + \pi^2} 
						\right)^{\frac{1}{2}} \,
						e^{(2-y)\xi},
						& \xi < 0
				\end{cases}
	\end{align*}
	Hence $\frac{\partial}{\partial y}|r| < 0$ for $\xi > 0$ and 
	$\frac{\partial}{\partial y}|r| > 0$ for $\xi < 0$, which implies
	\begin{align*}
		&\big|\rho\big(\xi, y, \sign(x); \lambda\big)\big| \\
			&\qquad\qquad \leq \big|\rho\big(\xi, y=1, \sign(x); \lambda\big)\big| \chi_{\mathbb R^+}
			+ \big|\rho\big(\xi, y=2, \sign(x); \lambda\big)\big| \chi_{\mathbb R^-}
				\in L^1(\mathbb R),
	\end{align*}
	as $\big|\rho\big(\xi, y=1, \sign(x); \lambda\big)\big| \chi_{\mathbb R^+}$ decays according to
	$O\big(e^{-\xi}\big)$ as $\xi \to +\infty$ and 
	$\big|\rho\big(\xi, y=2, \sign(x); \lambda\big)\big| \chi_{\mathbb R^-}$ decays according to 
	$O\big(e^{-2\xi}\big)$ as $\xi \to -\infty$.
	Thus,
	\[
		\int_{\mathbb R} \big| \rho\big(\xi, y, \sign(x)\big)\big|\, \mathrm{d}\xi \leq C
	\]
	where $C>0$ is some constant independent of $y$, and DCT $\implies$
	\[
		\int_{\mathbb R}
			\Big| 
				\rho\big(\xi, y, \sign(x)\big) - \rho\big(\xi, 2, \sign(x)\big)
			\Big|
		\, \mathrm{d}\xi 
		\to 0
	\]
	as $y\nearrow 2$, which, by \eqref{eq2:GoodLp}, means 
	$|\mathfrak C(x,y) - \mathfrak C(x,2)| \to 0$ 
	as $y\nearrow2$ for each fixed $x \in \mathbb R$. 
	Moreover, since $e^{-\pi|x|} \rho(x)$ is dominated by an $L^1$ function that 
	is independent of $y$, the Dominated Convergence Theorem further implies
	\[
		\|\mathfrak C(x,y) - \mathfrak C(x,2)\|_{L^1} \to 0
	\]
	as $y\nearrow 2$. Thus, by 
	\cite[Theorem 1.2.10]{Grafakos}
	\begin{align}
		\left\|\big(\mathfrak C(\dotarg, y) - \mathfrak C(\dotarg, 2)\big)*f\right\|_{L^p}
			\leq \|\mathfrak C(\dotarg, y) - \mathfrak C(\dotarg, 2)\|_{L^1} \|f\|_{L^p}
	\end{align}
	which implies the limit 
	\eqref{eq2:GoodLim} holds as an $L^p$ limit.

	For pointwise limit, first take $h\in L^1(\mathbb R) \cap L^\infty(\mathbb R)$. Then
	\begin{align*}
		\big| \big( \mathfrak C(\dotarg, y) - \mathfrak C(\dotarg, 2) \big) * h \big| 
			\leq \|h\|_{L^\infty} \|\mathfrak C(\dotarg, y) - \mathfrak C(\dotarg, 2)\|_{L^1} \to 0,
	\end{align*}
	by our previous work. 

	We use a density argument to finish this proof. For $f \in L^p(\mathbb R)$ with 
	$p\geq 1$, we may approximate $f$ 
	by a bounded function $g \in L^p(\mathbb R) \cap L^\infty(\mathbb R)$ so 
	that 
	\[
		\| f - g \|_{L^p} < r^p,
	\]
	where $0 < r \ll 1$ is some arbitrarily small number. By Chebyshev's
	inequality and \cite[Theorem 1.2.10]{Grafakos}, 
	\begin{align}\label{eq:2.02-Cheby}
		&m\Big( 
			\big\{ 
				x \in \mathbb R 
				\,:\, 
				\big(\mathfrak C(\dotarg, y) - \mathfrak C(\dotarg, 2)\big)*(f-g)(x) > r  
			\big\}  
		\Big) \\
		&\qquad\qquad\leq 
			\frac{1}{r^p} 
			\left\|\big(\mathfrak C(\dotarg, y)- \mathfrak C(\dotarg, 2)\big)*(f-g) \right\|_{L^p}
			\nonumber\\
		&\qquad\qquad\leq 
			\left\|\mathfrak C(\dotarg, y)- \mathfrak C(\dotarg, 2) \right\|_{L^1},
			\nonumber
	\end{align}
	where $m(\dotarg)$\label{sym:lebesguemeasure} denotes the Lebesgue measure on $\mathbb R$.
	Given
	\begin{align*}
		&\big|\big(\mathfrak C(\dotarg, y)- \mathfrak C(\dotarg, 2)\big)*f(x) \big| \\
			&\qquad\qquad\leq 
				\big|\big(\mathfrak C(\dotarg, y)- \mathfrak C(\dotarg, 2)\big)*(f-g)(x)\big|
				+\big|\big(\mathfrak C(\dotarg, y)- \mathfrak C(\dotarg, 2)\big)*g(x) \big|,
	\end{align*}	
	pointwise \textit{a.e.} convergence follows, as \eqref{eq:2.02-Cheby} implies
	$\big(\mathfrak C(\dotarg, y)- \mathfrak C(\dotarg, 2)\big)*(f-g)(x)$ converges pointwise
	\textit{a.e.} to zero.
\end{proof}


\end{document}