%%======================================================
%% Section 3.03: Exponentially Weighted Cauchy Transform
%%======================================================

\documentclass[../dissertation.tex]{subfiles}


\begin{document}
\section{Exponential Cauchy Transform}\label{sec3:CauchyTrans}

As mentioned at the end of Section \ref{sec3:Analyticity}, the crux of proving 
Theorem \ref{thm3:main_result} involves analyzing the behavior of the 
$\mathcal E_{(2-y)} f(x)$ terms in \eqref{eq3:ILW.(Gz*f).simp}, which we 
write here as
\begin{align*}
	\left( \mathcal E_\varepsilon f \right)(x) 
		&:=  \frac{e^{-i \pi (2-\varepsilon)}}{2 \pi i} 
			\int_{-\infty}^x 
				\frac{e^{-\pi|x-x'|}}{(x-x') - i\varepsilon} f(x') 
			\, \mathrm{d}x' \\
		&\quad + \frac{e^{i \pi (2-\varepsilon)}}{2 \pi i} 
			\int_x^{\infty} 
				\frac{e^{-\pi|x-x'|} }{(x-x') - i\varepsilon} f(x') 
			\, \mathrm{d}x',
\end{align*}
for $\varepsilon = 2-y$. The exponential coef{}ficient in front of each integral 
in the above definition of $\mathcal E_{\varepsilon}f(x)$ add undesirable 
complexity to analyzing $\mathcal E_{\varepsilon}f(x)$.\label{sym:expCauchy} As 
such, we introduce the exponentially weighted Cauchy Transform $E_\varepsilon$ 
defined by
\begin{align*}
	\left( E_\varepsilon f\right)(x)
 		&:= \frac{1}{2\pi i} \int_{\mathbb R} 
 				\frac{e^{-\pi|x-x'|}}{(x - x') - i \ve} f(x') 
 			\,dx' \\
 		&= \frac{1}{2\pi i} \int_{\mathbb R} 
 				e^{-\pi|x-x'|} \frac{(x-x'+i\varepsilon)}{(x - x')^2 - \ve^2} f(x') 
 			\,dx'
\end{align*}
In Lemma \ref{lma2:ImElim}, we establish that $\mathcal E_\varepsilon$ and $E_\varepsilon$
share the same pointwise \textit{i.e.} and $L^p$ limits (as $\varepsilon \searrow 0$).
In the remainder of this section we therefore focus our attention on the exponentially 
weighted Cauchy transform $E_\varepsilon$. In particular, we prove the following theorem:
\begin{thm}\label{thm2:CTLimit}
	For $f \in L^p(\mathbb R)$ with $1<p<\infty$, the limit 
	\[
		\lim_{\varepsilon\searrow 0} E_\varepsilon f(x) = Ef(x) - \frac{1}{2} f(x)
	\]
	holds both as a pointwise \textit{a.e.} limit and as an $L^p$ limit. 
\end{thm}
In the process of proving Theorem \ref{thm2:CTLimit}, we assume that the exponentially
weighted Hilbert transform 
$E$ is a bounded linear operator on $L^p(\mathbb R)$\textemdash{}a fact that we
later prove in Section \ref{sec3:BndE}. The boundedness of $E$ in conjunction with 
Theorem \ref{sec3:CauchyTrans}, Theorem \ref{thm2:GoodStuff} and equation 
\eqref{eq3:GLminus} verify the existence and boundedness of the $L^p$ 
operators $G_\star^-$.

\begin{lma}\label{lma2:ImElim}
	For $f \in L^p(\mathbb R)$, the limit
	\begin{align}\label{eq2:ImElimlma1}
		\big(E_\varepsilon - \mathcal E_\varepsilon\big) f(x) \to 0
	\end{align}
	holds pointwise $a.e.$ and in $L^p$ for $p \geq 1$.
\end{lma}
\begin{proof}
	We begin by proving the $L^p$ convergence. With that goal in mind, we 
	rewrite $\left(E_\varepsilon - \mc E_\varepsilon\right)(f)(x)$ as
	\begin{align}\label{eq:Ediff}
		2 \pi i \big(E_\varepsilon - \mc E_\varepsilon\big)\big(f(x)\big)
			&= \left( 1 - e^{-i \pi (2-\varepsilon)}\right) 
				\int_{-\infty}^x 
					\frac{e^{-\pi|x-x'|} }{(x-x') - i\varepsilon} f(x') 
				\, \mathrm{d}x' \\
			&\quad+
				\left( 1 - e^{i \pi (2-\varepsilon)} \right) 
				\int_{x}^\infty 
					\frac{e^{-\pi|x-x'|} }{(x-x') - i\varepsilon} f(x') 
				\, \mathrm{d}x' \nonumber \\
			&= \left(  \left( 1 - e^{-i \pi (2-\varepsilon)}\right) 
				\frac{e^{-\pi|\dotarg|}}{(\dotarg) - i\varepsilon} \chi_{\mathbb R^+} \right)  * f 
				\nonumber \\ 
			&\quad+ \left(  \left( 1 - e^{i \pi (2-\varepsilon)}\right) 
				\frac{e^{-\pi|\dotarg|}}{(\dotarg) - i\varepsilon} \chi_{\mathbb R^-} \right)  * f,
				\nonumber
	\end{align}
	where $\chi_{R^\pm}$ respectively denote the characteristic functions for the 
	negative and positive real half-lines, and use Theorem 1.2.10 from \cite{Grafakos},
	which states that the $L^p$ operator norm of a convolution operator is 
	less than or equal to the $L^1$ norm of its kernel. Indeed,
	\begin{align*}
		\left\| 
			\frac{e^{-\pi|\dotarg|}}{(\dotarg) - i\varepsilon} \chi_{\mathbb R^-} 
		\right\|_{L^1(\mathbb R)}
			&\leq \int_{-\infty}^{-1} e^{-\pi |x|} \, \mathrm{d}x 
				+ \int_{-1}^0 \frac{1}{\sqrt{x^2 + \varepsilon^2}} \, \mathrm{d}x \\
			&\leq \frac{1-e^{-\pi}}{\pi} 
				+ \log\left[ x + \sqrt{x^2 + \varepsilon^2} \right]_{-1}^0 \\
			&= \frac{1-e^{-\pi}}{\pi} 
				+ \log(\varepsilon) - \log\left(\sqrt{1+\varepsilon^2}-1\right),
	\end{align*}
	from which it follows that 
	\begin{align}\label{eq:LeftL1ELim}
		\left\| \left( 1 - e^{i \pi (2-\varepsilon)}\right) 
			\frac{e^{-\pi|\dotarg|}}{(\dotarg) - i\varepsilon} \chi_{\mathbb R^-} 
		\right\|_{L^1(\mathbb R)}
			\to 0, 
			\qquad \text{as } \varepsilon \to 0. 
	\end{align}
	A similar argument shows that 
	\begin{align}\label{eq:RightL1ELim}
		\left\| \left( 1 - e^{-i \pi (2-\varepsilon)}\right) 
			\frac{e^{-\pi|\dotarg|}}{(\dotarg) - i\varepsilon} \chi_{\mathbb R^+} 
		\right\|_{L^1(\mathbb R)}
			\to 0, 
			\qquad \text{as } \varepsilon \to 0. 
	\end{align}
	Thus, it follows from \cite[Theorem 1.2.10]{Grafakos} and 
	Equations \eqref{eq:LeftL1ELim} and \eqref{eq:RightL1ELim},
	\eqref{eq:Ediff} that
	\begin{align*}
		\left\|\big(E_\varepsilon-\mc E_\varepsilon\big) f \right\|_{L^p(\mathbb R)}
			\leq \frac{1}{2\pi} 
					\left\|E_\varepsilon-\mc E_\varepsilon\right\|_{L^1(\mathbb R)}
					\|f\|_{L^p(\mathbb R)}
			\to 0
	\end{align*}
	as $\varepsilon \to 0$, where $E_\varepsilon-\mc E_\varepsilon$ on the right-hand
	side of the above inequality is used to denote the kernel of the convolution
	operator $E_\varepsilon-\mc E_\varepsilon$.


	To prove pointwise $a.e.$ convergence, we prove the result for bounded 
	functions and use Chebyshev's inequality to extend by density. Let 
	$h \in L^p(\mathbb R) \cap L^\infty(\mathbb R)$, for $p \geq 1$ Using Euler's 
	formula, 
	it is easy to see that
	\begin{align*}
		2 \pi i \big( E_\varepsilon - \mathcal E_\varepsilon \big) h(x)
			&= \big[ 1 -  \cos\big(\pi(2-\ve)\big) \big] 
					\int_{\mathbb R} \frac{e^{-\pi|x-x'|}}{(x - x') - i \ve} h(x') \,dx' \\
			&\quad +i \sin\big(\pi(2-\varepsilon)\big) 
				\int_{-\infty}^x 
					\frac{e^{-\pi|x-x'|} }{(x-x') - i\varepsilon} h(x') 
				\, \mathrm{d}x' \\
			&\quad - i \sin\big(\pi(2-\varepsilon)\big)
				\int_x^{\infty} 
					\frac{e^{-\pi|x-x'|} }{(x-x') - i\varepsilon} h(x') 
				\, \mathrm{d}x'.
	\end{align*}
	Now, 
	\begin{align*}
		&\left| \sin\big(\pi(2-\varepsilon)\big) 
					\int_{-\infty}^x 
						\frac{e^{-\pi|x-x'|} }{(x-x') - i\varepsilon} h(x') 
					\, \mathrm{d}x' \right| \\
			&\qquad= \left| \sin\big(\pi(2-\varepsilon)\big) 
					\left(\int_\infty^{x-1} + \int_{x-1}^x \right)
						\frac{e^{-\pi|x-x'|} }{(x-x') - i\varepsilon} h(x') 
					\, \mathrm{d}x' \right| \\
			&\qquad\leq \left| \sin\big(\pi(2-\varepsilon)\big) \right|
				\left(
					C + C \int_{x-1}^x \frac{dx'}{\sqrt{(x-x')^2 + \varepsilon^2}}
				\right) \\
			&\qquad\leq \left| \sin\big(\pi(2-\varepsilon)\big) \right|
				\left(
					C + C \log\left| \varepsilon \right|
					+ C \log\left| -1 + \sqrt{1^2 + \varepsilon^2}  \right|
				\right)
	\end{align*}
	and
	\begin{align*}
		&\left| \sin\big(\pi(2-\varepsilon)\big) 
					\int^\infty_x 
						\frac{e^{-\pi|x-x'|} }{(x-x') - i\varepsilon} h(x') 
					\, \mathrm{d}x' \right| \\
			&\qquad\leq \left| \sin\big(\pi(2-\varepsilon)\big) \right|
				\left(
					C + C \log\left| 1 + \sqrt{1^2 + \varepsilon^2}  \right|
					+ C \log\left| \varepsilon \right|
				\right),
	\end{align*}
	where the constants above depend only on the $L^\infty$ norm of $h$. It 
	therefore follows from an application of L'H\^opital's rule that
	\begin{align*}
		\lim_{\ve \to 0} 
		\left| \big[ 1 -  \cos\big(\pi(2-\ve)\big) \big] 
			\int_{\mathbb R} 
				\frac{e^{-\pi|x-x'|}}{(x - x') - i \ve} h(x') 
			\,dx'
		\right| 
			&= 0 \\
		\lim_{\ve \to 0} 
		\left| 
			\sin\big(\pi(2-\varepsilon)\big) 
			\int_{-\infty}^x 
				\frac{e^{-\pi|x-x'|} }{(x-x') - i\varepsilon} h(x') 
			\, \mathrm{d}x' 
		\right| 
			&= 0\\
		\lim_{\ve \to 0}\left| \sin\big(\pi(2-\varepsilon)\big) 
					\int^\infty_x 
						\frac{e^{-\pi|x-x'|} }{(x-x') - i\varepsilon} h(x') 
					\, \mathrm{d}x' \right|
			&=0,
	\end{align*}
	which implies
	\[
		\lim_{\varepsilon\to0} 
				\big( E_\varepsilon - \mc E_\varepsilon) h(x)
			= 0,
	\]
	for every $h \in  L^p(\mathbb R) \cap L^\infty(\mathbb R)$ ($p\geq 1$)
	and $x \in \mathbb R$. The result therefore follows from a density
	argument analogous to the one given at the end of the proof for
	Theorem \ref{thm2:GoodStuff}.
	% Lastly, for $f \in L^p(\mathbb R)$ with $p\geq 1$, we may approximate $f$ 
	% by a bounded function $g \in L^p(\mathbb R) \cap L^\infty(\mathbb R)$ so 
	% that 
	% \[
	% 	\| f - g \|_{L^p} < r^p,
	% \]
	% where $0 < r \ll 1$ is some arbitrarily small number. By Chebyshev's
	% inequality and \cite[Theorem 1.2.10]{Grafakos}, 
	% \begin{align}\label{eq:2.02-Cheby}
	% 	m\Big( 
	% 		\big\{ 
	% 			x \in \mathbb R 
	% 			\,:\, 
	% 			\big(E_\varepsilon-\mc E_\varepsilon\big)(f-g)(x) > r  
	% 		\big\}  
	% 	\Big)
	% 	&\leq 
	% 		\frac{1}{r^p} 
	% 		\left\|\big(E_\varepsilon-\mc E_\varepsilon\big)(f-g) \right\|_{L^p}
	% 		\\
	% 	&\leq 
	% 		\left\|E_\varepsilon-\mc E_\varepsilon \right\|_{L^1},
	% 		\nonumber
	% \end{align}
	% where $m(\dotarg)$ denotes the Lebesgue measure on $\mathbb R$.
	% Given
	% \begin{align*}
	% 	\big|\big(E_\varepsilon-\mc E_\varepsilon\big)(f)(x) \big|
	% 		\leq 
	% 			\big|\big(E_\varepsilon-\mc E_\varepsilon\big)(f-g)(x)\big|
	% 			+\big|\big(E_\varepsilon-\mc E_\varepsilon\big)(g)(x) \big|,
	% \end{align*}	
	% pointwise $a.e.$ convergence follows, as \eqref{eq:2.02-Cheby} implies
	% $\big(E_\varepsilon-\mc E_\varepsilon\big)(f-g)(x)$ converges pointwise
	% $a.e.$ to zero. 
\end{proof}


The first piece to proving Theorem \ref{sec3:CauchyTrans} is establishing that it
holds for suf{}ficiently ``nice'' functions. We do this next in Lemma 
\ref{lma:ptwise-Lim}.

\begin{lma}
	\label{lma:ptwise-Lim} 
	Let $f \in \mathscr S(\mathbb R)$ and use $E$ to denote
	the operator given by
	\[
		Ef(x) =\frac{1}{2\pi i} \pv \int_{\mathbb R} \frac{f(x') e^{-\pi|x - x'|}}{x-x'}\,dx'.
	\]
	Then, for each $x \in \mathbb R$ the following pointwise limit holds
	\[
		\lim_{\varepsilon\to 0} E_\varepsilon f(x) = Ef(x) - \frac{1}{2} f(x).
	\]
% \[
%     \lim_{\varepsilon\to 0} \frac{1}{2\pi i} 
%             \int_{\mathbb R} \frac{e^{-\pi|x - x'|}}{x-x'-i\varepsilon} f(x') \,dx'
%         = \frac{1}{2\pi i} \pv \int_{\mathbb R} \frac{e^{-\pi|x - x'|}}{x-x'} f(x') \,dx'
%             + \frac{1}{2} f(x).
% \] 
\end{lma}
\begin{proof}
    We follow Terence Tao's proof of the Plemelj formulae \cite{Tao}. 
    As in Tao's proof, we use translation invariance to take $x = 0$ and reduce the proof 
    to showing
    \begin{align*} %\label{eq:ptwLma-1}
        \lim_{\varepsilon\to 0} \frac{1}{2\pi i} 
            \int_{\mathbb R} \frac{f(x') e^{-\pi|x'|}}{-x'-i\varepsilon}\,dx'    
            + \frac{1}{2} f(0) 
            - \frac{1}{2\pi i} \int_{|x'|>\varepsilon} \frac{f(x') e^{-\pi|x'|}}{-x'}\,dx' = 0
    \end{align*}
    By multiplying by $2 \pi i$ and introducing the change of variables $x' = \varepsilon w$, we
    further reduce the proof to showing
    \begin{align} \label{eq:ptwLma-1}
        \lim_{\varepsilon\to 0} \int_{\mathbb R} f(\varepsilon w) e^{-\pi\varepsilon|w|}
            \left( \frac{1}{-w-i}  - \frac{\chi_{|w| > 1}}{-w}\right)  \,dw
            - \pi i f(0) = 0,
    \end{align}
    where $\chi_{|w|>1}$ denotes the characteristic function on the set $|w|>1$.
    As Tao notes, direct computation yields
    \[
        \int_{\mathbb R} \left( \frac{1}{-w-i}  - \frac{\chi_{|w| > 1}}{-w}\right)  \,dw
            = \pi i.
    \]
    Thus, \eqref{eq:ptwLma-1} can be rewritten as
    \begin{align}\label{eq:ptwLma-2}
        \lim_{\varepsilon\to 0} 
        	\int_{\mathbb R} 
        		\left(
        			f(\varepsilon w) e^{-\pi \varepsilon|w|} + f(0)
        		\right)
            	\left( 
            		\frac{1}{-w-i}  - \frac{\chi_{|w| > 1}}{-w}
            	\right)  
            \,dw
             = 0.
    \end{align}
    Since \eqref{eq:ptwLma-2} holds by Dominated Convergence, the result follows.
\end{proof}

\begin{rmk}
	$E$ is a bounded operator on $L^p(\mathbb R)$ for $1 < p < \infty$\textemdash{}which 
	is something 
	we show in subsection \ref{sec3:BndE} after discussing the pointwise and $L^p$ 
	convergence of $E_\varepsilon$. 
\end{rmk}

Taking our lead from Lemma \ref{lma:ptwise-Lim}, we decompose $E_\varepsilon$ as  
\begin{align} \label{eq:EepsRealIm}
	(E_\varepsilon f)(x) 
		= (\mathpzc E_\varepsilon * f)(x) 
			- \frac{1}{2}(\mathpzc P_\varepsilon*f)(x),
\end{align}
where 
\[  \label{sym:badCauchy}
	\mathpzc E_\varepsilon(y) 
		:= \frac{1}{2\pi i} \frac{y}{y^2 + \varepsilon^2} e^{-\pi|y|}, 
	\qquad \text{and} \qquad
	\mathpzc P_\varepsilon(y) 
		:= \frac{1}{\pi} \frac{\varepsilon}{y^2 + \varepsilon^2} e^{-\pi|y|}.
\]
We define the truncated exponentially weighted Hilbert transform 
$E^{(\varepsilon)}$\label{sym:truncExpHil} by 
\[
	\left( E^{(\varepsilon)} f \right)(x) 
		:= \frac{1}{2\pi i} \int_{|y|\geq \varepsilon} \frac{e^{-\pi|y|}}{y} f(x - y) \, \mathrm{d}y,
\]
and note that by definition
\[
	(Ef)(x) 
		:= \lim_{\varepsilon \searrow 0} \left( E^{(\varepsilon)} f \right)(x)
\]
for $f \in \mathscr S(\mathbb R)$. 

\begin{rmk}
	Before continuing, it is worth taking a brief respite to consider our strategy for the 
	analysis which follows. In \eqref{eq:EepsRealIm}, we break $E_\varepsilon$ into its
	(ef{}fective) real and imaginary parts $\mathpzc E_{\varepsilon}$ and 
	$\mathpzc P_\varepsilon$
	(which should both be thought of as convolution operators), respectively, and show 
	they respectively converge both $a.e.$ and in $L^p$ to the exponentially waited 
	Hilbert transform and the identity operator. Doing the former involves first showing 
	that the convolution operator $\mathpzc E_{\varepsilon}$ and the truncated 
	exponentially weighted Hilbert transform $E^{(\varepsilon)}$ (when applied to an $L^p$ 
	function) share the same pointwise $a.e.$ and $L^p$ limits (see Theorem 
	\ref{thm:TruncEepsEquivLim}. We then show that pointwise $a.e.$ and $L^p$ 
	limits $E^{(\varepsilon)}f \to Ef$ are actually well defined for general $L^p$ 
	functions $f$.
\end{rmk}

Since $\mathpzc P_\varepsilon$ is essentially the Poisson kernel with an exponential weight, it 
is the easier of the two operators to understand. As such, we first turn our attention 
to understanding its pointwise $a.e.$ and $L^p$ limits.

\begin{thm}\label{thm:2.02-poisson}
	The limit $\mathpzc P_\varepsilon * f \to f$ ($\varepsilon \searrow 0$) converges 
	pointwise $a.e.$ and in $L^p$ for $f \in L^p(\mathbb R)$, $p \geq 1$. 
\end{thm}
\begin{proof}
	Let $P_\varepsilon(y) = \frac{1}{\pi} \frac{\varepsilon}{y^2 + \varepsilon^2}$
	denote the Poisson kernel. Since the family $P_\varepsilon$ is an approximate
	identity, to prove Theorem \ref{thm:2.02-poisson} it suf{}fices by 
	\cite[Theorem 1.2.19]{Grafakos} to consider the pointwise $a.e.$ and $L^p$ 
	limits of $\big(P_\varepsilon - \mathpzc P_\varepsilon \big) * f$.]

	Since, 
	\[
		P_\varepsilon - \mathpzc P_\varepsilon 
			= \frac{1}{\pi}\frac{\varepsilon}{y^2+\varepsilon^2}
				\left(1-e^{-\pi|y|}\right)
	\]
	is radially symmetric and non-negative, to compute the $L^1$ norm of 
	$P_\varepsilon - \mathpzc P_\varepsilon$, it suf{}fices to integrate 
	$P_\varepsilon - \mathpzc P_\varepsilon$ on the half-line $(0, \infty)$. 
	Using integration by parts, we find
	\begin{align*}
		\int_0^\infty (P_\varepsilon - \mathpzc P_\varepsilon) \, \mathrm{d}y
			&= \left.
					\frac{1}{\pi}
					\big(1-e^{-\pi}\big)\arctan\left(\frac{y}{\varepsilon}\right)
				\right|_0^\infty
				- \int_0^\infty \arctan\left(\frac{y}{\varepsilon}\right) e^{-\pi y}\,dy.
	\end{align*}
	Hence, by Dominated Convergence, 
	\begin{align*}
		\lim_{\varepsilon\to0}
			 	\int_0^\infty (P_\varepsilon - \mathpzc P_\varepsilon) \, \mathrm{d}y
			 &= 0, 
	\end{align*}
	which implies $\left\| P_\varepsilon - \mathpzc P_\varepsilon \right\|_{L^1}
	\to 0$ as $\varepsilon \searrow 0$. As such, an argument analogous to the one 
	in the proof of Lemma \ref{lma2:ImElim} allows us to complete this proof. 
\end{proof}


Following the method employed in the proofs of 
\cite[Theorem 4.1.12, Theorem 5.1.5]{Grafakos} to study the Hilbert 
transform, we show that $E^{(\varepsilon)}$ converges $a.e.$ and in $L^p$ to
$E$ and $\mathpzc E_\varepsilon * f - E^{(\varepsilon)}(f) \to 0$ $a.e.$ and in $L^p$.
In doing so, we use \cite[Theorem 1.2.21]{Grafakos} and \cite[Corollary 2.1.19]{Grafakos}, 
whose statements can be found in the \hyperref[app:HA]{Harmonic Analysis Results} 
appendix.

\begin{thm} \label{thm:TruncEepsEquivLim}
	Let $1 < p < \infty$. For any $f \in L^p(\mathbb R)$, we have 
	\[
		\mathpzc E_\varepsilon * f - E^{(\varepsilon)}(f) \to 0
	\]
	in $L^p$ and almost everywhere as $\varepsilon \to 0$. 
\end{thm}
\begin{proof}
	Let
	\[
		Q_\varepsilon(y) = e^{-\pi|y|} \, \varepsilon^{-1} \psi(y/\varepsilon),
	\]
	where 
	\[
		\psi(t) =
			\begin{cases}
				\dfrac{t}{t^2 + 1} - \dfrac{1}{t}, & |t| \geq 1 \\ \\
				\dfrac{t}{t^2 + 1},                & |t| < 1 1.   
			\end{cases}
	\]
	As noted in \cite{Grafakos}, the function $\psi$ has integral zero with 
	a radially decreasing majorant $\Psi$ given by 
	\[
		\Psi(t) =
			\begin{cases}
				\dfrac{t}{t^2 + 1}, & |t| \geq 1 \\ \\
				1,                  & |t| < 1.   
			\end{cases}	
	\]
	Moreover, since the calculation
	\begin{align*}
		\varepsilon^{-1} \psi\left(\frac{y}{\varepsilon}\right)
			= \varepsilon^{-1}
				\left(
					\frac{\frac{y}{\varepsilon}}
						{\left(\frac{y}{\varepsilon}\right)^2+1^2}
					- \varepsilon\frac{\chi_{|y|\geq \varepsilon}}{y}
				\right)
			= \frac{y}{y^2+\varepsilon^2} - \frac{\chi_{|y|\geq \varepsilon}}{y}
	\end{align*}
	implies 
	\[
		\mathpzc E_\varepsilon * f - E^{(\varepsilon)}(f) 
			= \frac{1}{2\pi i}\,Q_\varepsilon * f,
	\]
	the result follows from Theorem \cite[Theorem 1.2.21]{Grafakos} and Corollary 
	\cite[Corollary 2.1.19]{Grafakos} (with $c = 0$).
\end{proof}

To prove the $L^p$ and pointwise $a.e.$ convergence of the convolution operator 
$\mathpzc E_\varepsilon$ to the exponentially weighted Hilbert transform $E$, 
it remains to show that $E^{(\varepsilon)}$ converges to $E$. We do so 
in Theorem \ref{thm:truncEtoE}. However, we must first establish the Cotlar
type inequality in Lemma \ref{lma:Cotlar} for the maximal operator
$E^*:= \left( E^{(\varepsilon)} \right)^*$\label{sym:maxCauchyT} associated with 
the with exponential Cauchy transform $\left\{ E^{(\varepsilon)}\right\}$, as we 
use this result in the proof
of Theorem \ref{thm:truncEtoE}.

% that $E$ 
% is a bounded operator on $L^p$ (Theorem \ref{thm:EbndOp} in Subsection 
% \ref{sec3:BndE}) and verify the Cotlar
% type inequality in Lemma \ref{lma:Cotlar} for the maximal operator
% $E^*:= \left( E^{(\varepsilon)} \right)^*$ associated with the with operator
% family $\left\{ E^{(\varepsilon)}\right\}$, as we use both results in the proof
% of Theorem \ref{thm:truncEtoE}.





\begin{lma}[Cotlar Type Inequality]
	\label{lma:Cotlar}
	Let $E^*$ given by
	\[
		E^*f(x)
				:= \left( E^{(\varepsilon)} \right)^* f(x)
				:= \sup_{\varepsilon>0} \left\{\left|E^{(\varepsilon)} f(x)\right|\right\}.
	\]
	denote the maximal operator associated with the operator family 
	$\left\{ E^{(\varepsilon)}\right\}$. If $f \in \mathscr S(\mathbb R)$, then 
	\begin{align} \label{eq:Cotlar}
		E^* f(x) \leq M E f(x) + C M f(x),
	\end{align}
	where $C$ is independent of $f$ and $M$\label{sym:hardy} denotes the 
	Hardy-Littlewood maximal operator
	defined by 
	\[
        M f(x) 
        	= \sup_{r > 0} 
        		\left\{ 
        			\frac{1}{B(0, r)} \int_{B(0, r)} |f(x - x')| \, \mathrm{d}x' 
        		\right\}.
    \]	
\end{lma}


% \begin{lma}[Cotlar Type Inequality]
% 	\label{lma:Cotlar}
% 	Let $E^*$ given by
% 	\[
% 		E^*f(x)
% 				:= \( E_{\varepsilon} \)^* f(x)
% 				:= \sup_{\varepsilon>0} \left\{\left|E_{\varepsilon} f(x)\right|\right\}.
% 	\]
% 	denote the maximal operator associated with the operator family $\{ E_\varepsilon\}$.
% 	If $f \in \mathscr S(\mathbb R)$, then 
% 	\[
% 		E^* f(x) \leq M E f(x) + C M f(x),
% 	\]
% 	where $C$ is independent of $f$ and $M$ denotes the Hardy-Littlewood maximal operator
% 	defined by 
% 	\[
%         M f(x) 
%         	= \sup_{r > 0} 
%         		\left\{ 
%         			\frac{1}{B(0, r)} \int_{B(0, r)} |f(x - x')| \, \mathrm{d}x' 
%         		\right\}.
%     \]	
% \end{lma}

In our proof of Lemma \ref{lma:Cotlar} we the standard analysis result shown in Proposition \ref{prop:Cotlar}.

\begin{prop}\label{prop:Cotlar}
    Suppose $f \in L^1_{\text{loc}}(\mathbb R^n)$. If $\phi : \mathbb R^n \to \mathbb R$ is nonnegative, 
    radial, radially decreasing, and integrable, then
    \begin{align}\label{eq:2.02-cotprop}
    	\sup_{\varepsilon > 0} \left\{ \left| \phi_\varepsilon * f(x) \right| \right\} 
            \leq \| \phi \|_1 M f(x), 
    \end{align}
    where $\phi_\varepsilon(x) := \frac{1}{\varepsilon} \phi\left( \frac{x}{\varepsilon} \right)$.
\end{prop}
\begin{proof}[Proof of Proposition \ref{prop:Cotlar}]
	We first show \eqref{eq:2.02-cotprop} holds for simple functions, then extend 
	\eqref{eq:2.02-cotprop} to arbitrary functions $\phi$ satisfying the the hypotheses 
	of Proposition \ref{prop:Cotlar} by density using the Monotone Convergence Theorem.
\end{proof}

We are now ready to prove Lemma \ref{lma:Cotlar}:

\begin{proof}[Proof of Lemma \ref{lma:Cotlar}]
    It suf{}fices to show 
    \[
        |E^{(\varepsilon)} f(x)| \leq M E f(x) + C M f(x)
    \]
    for every $\varepsilon > 0$. Choose $\phi \in \mathscr S(\mathbb R)$ satisfying the 
    hypotheses of Proposition \ref{prop:Cotlar} so that $\| \phi\|_1 = 1$ and 
    $\text{supp} \, \phi = \left(-\frac{1}{2}, \frac{1}{2} \right)$. We decompose 
    $\chi_{\{|\dotarg| > \varepsilon\}} \, \frac{e^{-\pi |\dotarg|}}{(\dotarg)}$ (the integrand 
    of the truncated exponentially weighted Hilbert transform) as
    \begin{align}\label{eq:1-Cotlar}
    	\chi_{\{|\dotarg| > \varepsilon\}} \, \frac{e^{-\pi |\dotarg|}}{(\dotarg)}
    		= E \phi_\varepsilon +
            	\left( 
            		 \frac{e^{-\pi |\dotarg|}}{(\dotarg)} \, \chi_{\{|\dotarg| > \varepsilon\}}
            			- E \phi_\varepsilon 
        	   	\right).
    \end{align}
    Thus, by taking the convolution of both sides of \eqref{eq:1-Cotlar} with $f$, we find
    \begin{align}\label{eq:2-Cotlar}
        |E^{(\varepsilon)} f(x)|
            \leq \frac{1}{2\pi} | (E\phi_\varepsilon)*f(x)| 
                + \frac{1}{2\pi} 
                	\left| 
                		\frac{e^{-\pi |\dotarg|}}{(\dotarg)} \, \chi_{\{|\dotarg| > \varepsilon\}}
                			- E\phi_\varepsilon 
                	\right|*|f|(x).
    \end{align}
    Note that by using two applications of Fubini's Theorem in conjunction with two 
    applications of Dominated Convergence, one can show
    $(E\phi_\varepsilon)*f(x) = \phi_\varepsilon * (Ef)(x)$. As such, we therefore 
    see from Proposition \ref{prop:Cotlar} that first term on the right-hand side 
    of \eqref{eq:2-Cotlar} satisfies
    \[
        \frac{1}{2 \pi} | (E\phi_\varepsilon)*f(x)| \leq M E f(x).
    \]

    As we consider the second term in the right-hand side of \eqref{eq:2-Cotlar}, we initially 
    take $\varepsilon = 1$ and examine the case $|\dotarg| <1$ and $|\dotarg| \geq 1$ separately. 
    Assume $|w| \geq 1$ and observe that
    \begin{align*}
        \left| 
        	 \frac{e^{-\pi | w |}}{ w }  - E\phi(w)
        \right|
            &= 
            	\left| 
            		\frac{e^{-\pi | w |}}{ w }
             		   - \int_{-\frac{1}{2}}^{\frac{1}{2}} \phi(x')\frac{e^{-\pi|w-x'|}}{w-x'} \, \mathrm{d}x' 
             	\right| \\
            &\leq 
            	\int_{-\frac{1}{2}}^{\frac{1}{2}} \phi(x') 
	                \left| 
	                	\frac{e^{-\pi | w |}}{ w }
	                		- \frac{e^{-\pi|w-x'|}}{w-x'} 
	                \right| 
                \, \mathrm{d}x'.
    \end{align*}
    Since $x' \in \mathop{\text{supp}} \phi$ only when $|x'| \leq \frac{1}{2}$, we have
    \begin{align*}
        \left| e^{-\pi|w|} (w - x') - e^{-\pi|w-x'|}(w) \right|
            \leq 2 |w| g(w) + e^{-\pi |w|} |x'|,
    \end{align*}
    where
    \[
        g(w)
            = \begin{cases}
                e^{-\pi \left|w + \frac{1}{2}\right|}, & w \geq 0 \\
                e^{-\pi \left|w - \frac{1}{2}\right|}, & w < 0.
            \end{cases}
    \]
    Hence
    \[
        \left| \frac{e^{-\pi|w|}}{w} - \frac{e^{-\pi|w-x'|}}{w-x'} \right|
            \leq \frac{2|w|\, g(w) + e^{-\pi |w|}|x'|}{|w| \, |w - x'|}.
    \]
    Further,
    \[
        \frac{|w|}{|w - x'|}
            \leq \frac{|w|}{\left|w - \frac{1}{2}\text{sign}(w)\right|}
            \leq 2,
    \]
    which implies 
    \[
        \frac{1}{|w||w-x'|}
            \leq \frac{2}{w^2},
    \]
    and
    \[
        \left| \frac{e^{-\pi|w|}}{w} - \frac{e^{-\pi|w-x'|}}{w-x'} \right|
            \leq 
            	\left( 
            		\frac{4|w| g(w)}{w^2} + \frac{2 e^{-\pi|w|} |x'|}{w^2} 
            	\right).
    \]
    As such,
    \[
        \left| \frac{e^{-\pi|w|}}{w - i \varepsilon} - E\phi_\varepsilon(w) \right|
        	\leq 
        		\int_{-\frac{1}{2}}^{\frac{1}{2}} 
        			\phi(x') 
        			\left(
        				\frac{4|w|g(w)}{w^2} + \frac{2 e^{-\pi|w|} |x'|}{w^2}
        			\right)
        		\, \mathrm{d}x'
            \leq \frac{4|w|g(u) + e^{-\pi|w|}}{w^2}
            \leq \frac{C}{w^2},
    \]
    for $|w| \geq 1$.

    On the other hand, for $|w| < 1$, 
    \begin{align*}
        \left| \frac{e^{-\pi|w|}}{w} \, \chi_{|w| > 1} - E \phi(w) \right|
        	&= \big| - E \phi(w) \big| \\
            &=
                \left| 
                    \pv \int_{-\frac{1}{2}}^{\frac{1}{2}} 
                        \frac{\phi(w-x')}{x'} e^{-\pi|x'|} 
                  \, \mathrm{d}x'
                 \right| \\
            &= 
                \left| 
                    \pv \int_{-\frac{1}{2}}^{\frac{1}{2}} 
                        \frac{\phi(w-x') -\phi(w)}{x'} e^{-\pi|x'|} 
                    \, \mathrm{d}x' 
                \right| \\
            &\leq C \| \phi' \|_{L^\infty} \\
            &=C,
    \end{align*}
    as
    \[
        \pv \int_{-\frac{1}{2}}^{\frac{1}{2}} 
            \frac{\phi(w)}{x'} e^{-\pi|x'|}
        \, \mathrm{d}x
        = 0.
    \]
    Combining the two cases, it follows from Proposition \ref{prop:Cotlar}
    \begin{align}\label{eq:3-Cotlar}
        \left| \frac{e^{-\pi|\dotarg|}}{(\dotarg)} - E \phi \right|*|f|(x)
            \leq C Mf(x).
    \end{align}
    
    Finally, to verify that \eqref{eq:3-Cotlar} holds for arbitrary $\varepsilon$ 
    $(0 < \varepsilon < 1)$, we use a dilation argument. Define $h$ and $g$ by 
    \[
        h(w, \varepsilon) := \left| \frac{e^{-\pi |w|}}{w} \, \chi_{|w|>\varepsilon} 
    	   	- E \phi_\varepsilon(w) \right|,
        \qquad \text{and} \qquad
        g(w) 
        	:=  
        		\left| \frac{e^{-\pi |w|}}{w}\right|  \, \chi_{|w|>1}  
        			+ \left| E \phi(w) \right|.
    \]
    Then 
    \[
        g_\varepsilon(w) 
          = \frac{1}{\varepsilon} g\left(\frac{w}{\varepsilon}\right)
          = \left| \frac{e^{-\pi |w/\varepsilon|}}{w} \right| \, \chi_{|w|>\varepsilon} 
          	+ \left| E \phi_\varepsilon(w) \right|.
    \]
    For $f \in \mathscr S(\mathbb R)$ and $f^\varepsilon(x):= f(\varepsilon x)$,
    \begin{align*}
        g*f^\varepsilon\left(\varepsilon^{-1} x \right)
            &= \int_{\mathbb R} 
            		f(\varepsilon y) \, g\left( \varepsilon^{-1} x - y \right) 
            	\, \mathrm{d}y \\
            &= \int_{\mathbb R} 
            		f(y) \varepsilon^{-1} 
            			g\left( \varepsilon^{-1} x - \varepsilon^{-1} y \right) 
            	\, \mathrm{d}y \\
            &= g_\varepsilon * f(x), \\
        \intertext{and}
        M f^\varepsilon(x)
            &= \sup \frac{1}{|B(0, r)|} 
            	\int_{B(0, r)} |f(\varepsilon x - \varepsilon y)| \, \mathrm{d}y \\
            &= \sup \frac{1}{|B(0, \varepsilon r)|} 
            	\int_{B(0, \varepsilon r)} |f(\varepsilon x - \varepsilon y)| \, \mathrm{d}y \\
            &= M f(\varepsilon x).
    \end{align*}
    Hence, by \eqref{eq:3-Cotlar}
    \begin{align*}
        h(\dotarg, \varepsilon) * |f|(x) 
            \leq g_\varepsilon * |f|(x) 
            = g*|f^{\varepsilon}|(\varepsilon^{-1} x) 
            \leq C M f^{\varepsilon}(\varepsilon^{-1} x)
            =C M f(x),
    \end{align*}
    from which the result follows.
\end{proof}

% \begin{proof}[Proof of Lemma \ref{lma:Cotlar}]
%     It suf{}fices to show 
%     \[
%         |E_{\varepsilon} f(x)| \leq M E f(x) + C M f(x)
%     \]
%     for every $\varepsilon > 0$. Choose $\phi \in \mathscr S(\mathbb R)$ satisfying the hypotheses of 
%     Proposition \ref{prop:Cotlar} so that $\| \phi\|_1 = 1$ and 
%     $\text{supp} \, \phi = \left(-\frac{1}{2}, \frac{1}{2} \right)$. We decompose 
%     $\frac{e^{-\pi|\dotarg|}}{(\dotarg) - i \varepsilon}$ as 
%     \begin{align}\label{eq:1-Cotlar}
%         \frac{e^{-\pi|\dotarg|}}{(\dotarg) - i \varepsilon}
%             = E \phi_\varepsilon + 
%             	\left( 
%             		\frac{e^{-\pi|\dotarg|}}{(\dotarg) - i \varepsilon} 
%             			- E \phi_\varepsilon 
%         	   	\right).
%     \end{align}
%     Thus, by taking the convolution of both sides of \eqref{eq:1-Cotlar} with $f$, we find
%     \begin{align}\label{eq:2-Cotlar}
%         |E_{\varepsilon} f(x)|
%             \leq \frac{1}{2\pi} | (E\phi_\varepsilon)*f(x)| 
%                 + \frac{1}{2\pi} 
%                 	\left| 
%                 		\frac{e^{-\pi|\dotarg|}}{(\dotarg) - i \varepsilon} - E\phi_\varepsilon 
%                 	\right|*|f|(x).
%     \end{align}
%     Given $(E\phi_\varepsilon)*f(x) = \phi_\varepsilon * (Ef)(x)$, we see from Proposition \ref{prop:Cotlar}
%     that first term on the right-hand side of \eqref{eq:2-Cotlar} satisfies
%     \[
%         \frac{1}{2 \pi} | (E\phi_\varepsilon)*f(x)| \leq M E f(x).
%     \]

%     As we consider the second term in the right-hand side of \eqref{eq:2-Cotlar}, we initially 
%     take $\varepsilon = 1$ and examine the case $|\dotarg| <1$ and $|\dotarg| \geq 1$ separately. 
%     Assume $|w| \geq 1$ and observe that
%     \begin{align*}
%         \left| \frac{e^{-\pi|w|}}{w - i} - E\phi(w) \right|
%             &= \left| \frac{e^{-\pi|w|}}{w - i} 
%                 - \int_{-\frac{1}{2}}^{\frac{1}{2}} \phi(x')\frac{e^{-\pi|w-x'|}}{w-x'} \, \mathrm{d}x' \right| \\
%             &\leq \int_{-\frac{1}{2}}^{\frac{1}{2}} \phi(x') 
%                 \left| \frac{e^{-\pi|w|}}{w - i} - \frac{e^{-\pi|w-x'|}}{w-x'} \right| \, \mathrm{d}x'.
%     \end{align*}
%     Since $x' \in \mathop{\text{supp}} \phi$ only when $|x'| \leq \frac{1}{2}$, we have
%     \begin{align*}
%         \left| e^{-\pi|w|} (w - x') - e^{-\pi|w-x'|}(w - i) \right|
%             \leq 2 |w| g(w) + e^{-\pi |w|} |x'| + g(w),
%     \end{align*}
%     where
%     \[
%         g(w)
%             = \begin{cases}
%                 e^{-\pi \left|w + \frac{1}{2}\right|}, & w \geq 0 \\
%                 e^{-\pi \left|w - \frac{1}{2}\right|}, & w < 0.
%             \end{cases}
%     \]
%     Hence
%     \[
%         \left| \frac{e^{-\pi|w|}}{w - i} - \frac{e^{-\pi|w-x'|}}{w-x'} \right|
%             \leq \frac{\big(2|u|+1\big) g(w) + e^{-\pi |w|}|x'|}{|w - i| \, |w - x'|}.
%     \]
%     Further,
%     \[
%         \frac{|w - i|}{|w - x'|}
%             \leq \frac{|w| + 1}{\left|w - \frac{1}{2}\text{sign}(w)\right|}
%             \leq 4,
%     \]
%     which implies 
%     \[
%         \frac{1}{|w - i||w-x'|}
%             \leq \frac{2}{|w - i|^2}
%             = \frac{4}{w^2 + 1},
%     \]
%     and
%     \[
%         \left| \frac{e^{-\pi|w|}}{w - i} - \frac{e^{-\pi|w-x'|}}{w-x'} \right|
%             \leq 4 \left( \frac{(2|w| + 1)g(w)}{w^2 + 1} + \frac{e^{-\pi|w|} |x'|}{w^2+1} \right).
%     \]
%     As such,
%     \[
%         \left| \frac{e^{-\pi|w|}}{w - i \varepsilon} - E\phi_\varepsilon(w) \right|
%             \leq \frac{4(2|w|+ 1) g(u) + e^{-\pi|w|}}{w^2+1}
%             \leq \frac{C}{w^2+1},
%     \]
%     for $|w| \geq 1$.

%     On the other hand, for $|w| < 1$, 
%     \begin{align*}
%         \left| \frac{e^{-\pi|w|}}{w - i} - E \phi(w) \right|
%             &\leq \left| \frac{e^{-\pi|w|}}{w - i \varepsilon} \right|
%                 + 
%                     \left| 
%                         \pv \int_{-\frac{1}{2}}^{\frac{1}{2}} 
%                             \frac{\phi(w-x')}{x'} e^{-\pi|x'|} 
%                       \, \mathrm{d}x'
%                      \right| \\
%             &= \left| \frac{e^{-\pi|w|}}{w - i \varepsilon} \right|
%                 + 
%                     \left| 
%                         \pv \int_{-\frac{1}{2}}^{\frac{1}{2}} 
%                             \frac{\phi(w-x') -\phi(w)}{x'} e^{-\pi|x'|} 
%                         \, \mathrm{d}x' 
%                     \right| \\
%             &\leq C + C \| \phi' \|_{L^\infty} \\
%             &=C,
%     \end{align*}
%     as
%     \[
%         \pv \int_{-\frac{1}{2}}^{\frac{1}{2}} 
%             \frac{\phi(w)}{x'} e^{-\pi|x'|}
%         \, \mathrm{d}x
%         = 0
%     \]
%     Combining the two cases, it follows from Proposition \ref{prop:Cotlar}
%     \begin{align}\label{eq:3-Cotlar}
%         \left| \frac{e^{-\pi|\dotarg|}}{(\dotarg) - i} - E \phi \right|*|f|(x)
%             \leq \frac{C}{(\dotarg)^2+1}*|f|(x) \leq C Mf(x).
%     \end{align}
    
%     Finally, to verify that \eqref{eq:3-Cotlar} holds for arbitrary $\varepsilon$ 
%     $(0 < \varepsilon < 1)$, we use a dilation argument. Define $h$ and $g$ by 
%     \[
%         h(w, \varepsilon) := \left| \frac{e^{-\pi |w|}}{w - i \varepsilon} 
%     	   	- E \phi_\varepsilon(w) \right|,
%         \qquad \text{and} \qquad
%         g(w) :=  \left| \frac{e^{-\pi |w|}}{w - i}\right| + \left| E \phi(w) \right|.
%     \]
%     Then 
%     \[
%         g_\varepsilon(w) 
%           = \frac{1}{\varepsilon} g\left(\frac{w}{\varepsilon}\right)
%           = \left| \frac{e^{-\pi |w/\varepsilon|}}{w - i \varepsilon}\right| 
%           	+ \left|\phi_\varepsilon * W(u) \right|.
%     \]
%     For $f \in \mathscr S(\mathbb R)$ and $f^\varepsilon(x):= f(\varepsilon x)$,
%     \begin{align*}
%         g*f^\varepsilon\left(\varepsilon^{-1} x \right)
%             &= \int_{\mathbb R} 
%             		f(\varepsilon y) \, g\left( \varepsilon^{-1} x - y \right) 
%             	\, \mathrm{d}y \\
%             &= \int_{\mathbb R} 
%             		f(y) \varepsilon^{-1} 
%             			g\left( \varepsilon^{-1} x - \varepsilon^{-1} y \right) 
%             	\, \mathrm{d}y \\
%             &= g_\varepsilon * f(x), \\
%         \intertext{and}
%         M f^\varepsilon(x)
%             &= \sup \frac{1}{|B(0, r)|} 
%             	\int_{B(0, r)} |f(\varepsilon x - \varepsilon y)| \, \mathrm{d}y \\
%             &= \sup \frac{1}{|B(0, \varepsilon r)|} 
%             	\int_{B(0, \varepsilon r)} |f(\varepsilon x - \varepsilon y)| \, \mathrm{d}y \\
%             &= M f(\varepsilon x).
%     \end{align*}
%     Hence, by \eqref{eq:3-Cotlar}
%     \begin{align*}
%         h(\dotarg, \varepsilon) * |f|(x) 
%             \leq g_\varepsilon * |f|(x) 
%             = g*f^{\varepsilon}(\varepsilon^{-1} x) 
%             \leq C M f^{\varepsilon}(\varepsilon^{-1} x)
%             =C M f(x),
%     \end{align*}
%     from which the result follows.
% \end{proof}


One final tool we need to prove Theorem \ref{thm:truncEtoE} is Theorem 2.1.14 from 
\cite{Grafakos}, which we present without proof below in Theorem \ref{thm:Graf2.1.14}

\begin{thm}[Theorem 2.1.14 from \cite{Grafakos}]\label{thm:Graf2.1.14}
	Let $(X, \mu)$ and $(Y, \nu)$ be measure spaces and let $0 < p, q < \infty$.
	Suppose for every $\varepsilon > 0$, $T_\varepsilon$ is a linear operator defined 
	on $L^P(X, \nu)$ with values in the set  of measurable functions on $Y$, and $D$ 
	is a dense subspace of $L^p(X)$. Define a sublinear operator
	\[
		T^*(f)(x) := \sup_{\varepsilon>0} \left| T_\varepsilon(f)(x) \right|.
	\]
	Suppose that for some $B>0$ and all $f \in L^p(X)$ we have
	\[
		\| T^*(f) \|_{L^{q, \infty}} \leq B \|f\|_{L^p}
	\]
	and that for all $f \in D$
	\begin{align}\label{eq:Graf2.1.14-1}
		\lim_{\varepsilon\to0} T_\varepsilon(f) = T(f)
	\end{align}
	exists and is finite $\nu$-$a.e.$ (and defines a linear operator on $D$).
	Then for all functions $f$ in $L^p(X)$ the limit \eqref{eq:Graf2.1.14-1}
	exists and is finite $\nu$-$a.e.$, and defines a linear operator $T$
	on $L^p(X)$ (uniquely extending $T$ defined on $D$) that satisfies
	\[
		\| T(f) \|_{L^{q, \infty}} \leq B \|f\|_{L^p}.
	\]
\end{thm}

\begin{thm}\label{thm:truncEtoE}
	For all $p$ ($1 < p < \infty$) there exists a constant $C_p$ depending
	only on $p$ such that
	\begin{align} \label{eq:E*bnd}
		\left\| E^* f \right\|_{L^p} \leq C_p \|f\|_{L^p}, 
		\qquad \forall f \in L^p(\mathbb R).
	\end{align}
	Moreover, for all $f \in L^p(\mathbb R)$, $E^{(\varepsilon)} f$ converges to 
	$Ef$ pointwise $a.e.$ and in $L^p$.
\end{thm}
\begin{proof}
	Inequality \eqref{eq:E*bnd} is an immediate consequence of Lemma \ref{lma:Cotlar},
	Theorem \ref{thm:EbndOp} from Subsection \ref{sec3:BndE}, of the fact that 
	convergence holds for Schwartz class 
	functions, and of Theorem \ref{thm:Graf2.1.14}. The $L^p$ convergences follows 
	from the almost every pointwise convergence and the dominated convergence 
	theorem combined with the Cotlar inequality \eqref{eq:Cotlar}.
\end{proof}

\end{document}