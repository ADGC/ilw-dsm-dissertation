%%=========================================================================
%% Section 2.2: Boundedness of the Exponentially Weighted Hilbert Transform
%%=========================================================================

\documentclass[../dissertation.tex]{subfiles}

\begin{document}
\section{Boundedness of the Exponentially Weighted Hilbert Transform}\label{sec3:BndE}

Proving that the Exponentially weighted Hilbert transform $E$ is bounded on 
$L^p(\mathbb R)$ for $1< p < \infty$ is a multistep process in which we first compute
the Fourier multiplier of $E$ (Lemma \ref{lma:FMultE}), use the Fourier multiplier
of $E$ to show that $E$ is strong type $(2,2)$ (Theorem \ref{thm:Estrong}), prove $E$ is 
weak type $(1, 1)$ (Theorem \ref{thm:Eweak}), and then use these results to ultimately
prove $E$ is strong type $(p, p)$ for $p>1$ (Theorem \ref{thm:EbndOp}).


\begin{lma}\label{lma:FMultE}
	The Exponentially weighted Hilbert transform $E$ has Fourier multiplier
	$$m_E(\xi) = \frac{1}{\pi} \arctan(\xi/\pi).$$
\end{lma}
\begin{proof}
	In order to compute the Fourier multiplier of $E$, it suffices to compute the Fourier
	transform of the function $x^{-1} e^{-\pi|x|}$, which is $2\pi i$ times the 
	convolution kernel of $E$. Letting $\theta$ denote the Heaviside function, we 
	write $x^{-1} e^{-\pi|x|}$ as 
	\begin{align}\label{eq:Ekern}
		\frac{e^{\pi |x|}}{x} 
			=
			\frac{e^{\pi \, x}}{x} \theta(-x) + \frac{e^{-\pi \, x}}{x} \theta(x),
	\end{align}
	and compute the Fourier transform of two terms on the right-hand side of 
	\eqref{eq:Ekern} separately. By further splitting $\frac{e^{\pi \, x}}{x} \theta(-x)$
	into even and odd parts, we obtain
	\begin{align}
		\mc F\left( \frac{e^{\pi (\dotarg)}}{(\dotarg)} \theta(-\dotarg) \right)
			&= \mc F^{(c)} \left( 
					\frac{e^{\pi (\dotarg)}\theta(-\dotarg)-e^{-\pi(\dotarg)}\theta(\dotarg)}
						{2(\dotarg)}  
				\right)(\xi)
				\\
			&\quad + i \, \mc F^{(s)} \left(
					\frac{e^{\pi(\dotarg)}\theta(-\dotarg)+e^{-\pi(\dotarg)}\theta(\dotarg)}
						{2(\dotarg)} 
				\right)(\xi),
				\nonumber
	\end{align}
	where $\mc F^{(c)}$ and $\mc F^{(s)}$ respectively denote the Fourier cosine and
	Fourier sine transformations. Direct computation yields 
	\begin{align*}
		\mc F^{(c)} \left( 
			\frac{e^{\pi (\dotarg)}\theta(-\dotarg)-e^{-\pi(\dotarg)}\theta(\dotarg)}
				{2(\dotarg)}  
		\right)(\xi)
			&= \frac{1}{\pi}
				\int_0^\infty 
					\frac{e^{\pi \,x}\theta(-x)-e^{-\pi\,x}\theta(x)}{2x}
				\, \cos(\xi x) \, \mathrm{d}x 
				\\
			&= \frac{1}{2} \big(
					\log(\xi^2 + \pi^2) + 2 \gamma
				\big)
				\nonumber
	\end{align*}
	and 
	\begin{align*}
		\mc F^{(s)} \left( 
			\frac{e^{\pi (\dotarg)}\theta(-\dotarg)+e^{-\pi(\dotarg)}\theta(\dotarg)}
				{2(\dotarg)}  
		\right)(\xi)
			&= \frac{1}{2\pi} 
				\int_0^\infty 
					\frac{e^{\pi \,x}\theta(-x)+e^{-\pi\,x}\theta(x)}{2x}
				\, \sin(\xi x) \, \mathrm{d}x 
				\\
			&= \frac{1}{2} \big(
					\arctan(\xi/\pi)
				\big),
				\nonumber
	\end{align*}
	where $\gamma$ denotes the Euler-Mascheroni constant, which implies
	\begin{align}
		\mc F \left( 
			 \frac{e^{\pi (\dotarg)}}{(\dotarg)} \theta(-\dotarg)
		\right)(\xi)
			&=
				\gamma
				+ \frac{1}{2} \log(\xi^2 + \pi^2) 
				+ i \, \arctan(\xi/\pi).
	\end{align}
	A similar computation also shows 
	\begin{align}
		\mc F \left( 
			 \frac{e^{-\pi (\dotarg)}}{(\dotarg)} \theta(\dotarg)
		\right)(\xi)
			&= 
				- \gamma
				- \frac{1}{2} \log(\xi^2 + \pi^2) 
				+ i \, \arctan(\xi/\pi),
	\end{align}
	from which the result follows.
\end{proof}

\begin{thm}\label{thm:Estrong}
	$E$ is strong type $(2, 2)$.
\end{thm}
\begin{proof}
	This result is an immediate consequence of Lemma \ref{lma:FMultE}, 
	Plancherel's Theorem, and the density of $\mathscr S(\mathbb R)$ 
	in $L^2(\mathbb R)$.
\end{proof}


\begin{thm}\label{thm:Eweak}
	$E$ is weak $(1, 1)$.
\end{thm}
\begin{proof}
    Fix $\Lambda > 0$, let $f$ be Schwartz class, and assume without loss of 
    generality that $f \in \mathscr S(\mathbb R)$ is real-valued and nonnegative 
    (otherwise, we can decompose $f$ in the appropriate pieces).
    Let $\{ I_j \}$ be the sequence sequence of dyadic intervals in the C-Z decomposition of 
    $f$ at height $\Lambda$. For $\Omega := \bigcup_j I_j$, define 
    \begin{align}\label{eq:C-Zdecomp}
        g(x) :=  
            \begin{cases}
                \displaystyle f(x), & x \notin \Omega \\ \\
                \displaystyle \frac{1}{|I_j|} \int_{I_j} f, & x \in I_j
            \end{cases},
        \qquad \text{and} \qquad
        b(x) := \sum_j b_j(x),
    \end{align}
    where 
    \[
        b_j(x) = \left( f(x) - \dfrac{1}{|I_j|} \int_{I_j} f \right) \chi_{I_j}(x).
    \]
    Note that $f = g+b$ and
    \begin{align} \label{eq:GoodPartBnd}
        g(x) \leq 2 \Lambda \quad \forall x \in \mathbb R.
    \end{align}
    To show that $Eg$ and $Eb$ are well defined, it suffices
    to bound $d_{Eg}(\Lambda)$ and $d_{Eb}(\Lambda)$ in terms of only $\Lambda$ and 
    $\|f\|_1$, where
    the notation 
    \[
        d_h (\Lambda) := \big| \big\{ x \in \mathbb R \, : \, |f(x)| > \Lambda \big\} \big|
    \]
    is used to denote the {\em distributional function} of a function $h$. Using
    Chebyshev's inequality, the $L^2$  boundedness of $E$ found in Theorem \ref{thm:Eweak}
    and equations \eqref{eq:GoodPartBnd} and \eqref{eq:C-Zdecomp}, we find
    \begin{align}\label{eq:dEgBound}
        d_{Eg}(\Lambda)
            &\leq \frac{1}{\Lambda^2} \int_{\mathbb R} E g(x)^2 \, \mathrm{d}x \\
            &\leq \frac{C}{\Lambda^2} \int_{\mathbb R} g(x)^2 \, \mathrm{d}x \nonumber\\
            &\leq \frac{C}{\Lambda^2} \int_{\mathbb R} g(x) \, \mathrm{d}x \nonumber\\
            &= \frac{C}{\Lambda^2} \left( \int_{\Omega} g 
                + \int_{\mathbb{R} \backslash \Omega} g  \right) \nonumber\\
            &\leq \frac{C}{\Lambda} \| f \|_1.  \nonumber
    \end{align}


    On the other hand, for $Eb$, let $\Omega^* = \bigcup_j 2 I_j$. Using the 
    Calder\'on-Zygmund Covering Lemma and Chebyshev's inequality we find
    \begin{align} \label{eq:dEb}
        d_{Eb}(\Lambda) 
            \leq |\Omega^*| + |\{ x \notin \Omega^* \, : \, |Eb(x)| > \Lambda \}|
            \leq \frac{2}{\Lambda} \|f\|_1 + \frac{1}{\Lambda} 
                \int_{\mathbb R \backslash \Omega^*} |Eb(x)| \, \mathrm{d}x.
    \end{align}
    So, to show that $Eb$ is well defined, we need to bound the integral on the 
    right hand side of the above inequality by $\|f\|_1$. To that end, note that
    if $x \notin \Omega^*$, then for each $j$, $x \notin 2 I_j$ and
    \begin{align*}
        E b_j(x) 
            = \frac{1}{2\pi} 
            	p.v. \int_{\mathbb R} 
            		\frac{e^{-\pi|x - x'|}}{x - x'} b_j(x') 
            	\, \mathrm{d}x'
            = \frac{1}{2\pi} 
            	\int_{I_j} 
            		\frac{e^{-\pi|x - x'|}}{x - x'} b_j(x') 
            	\, \mathrm{d}x 
            < \infty,
    \end{align*}
    as $\text{supp} \, b_j \subseteq I_j$. Since $E$ is a tempered 
    (Lemma \ref{lma:FMultE}) and
    $f \in \mathscr S(\mathbb R)$ means that $Ef \in L^2(\mathbb R)$ and, hence
    $\sum_j E b_j$ converges to $Eb$ in the $L^2$ norm, it follows that
    \[
        |Eb(x)| \leq \sum_j |Eb_j(x)| \quad a.e.
    \]
    As such, proving $Eb$ is well defined reduces to showing
    \begin{align}
        \int_{\mathbb R \backslash \Omega^*} \sum_{j} |E b_j(x)| \, \mathrm{d}x 
            \leq C \| f \|_1.
    \end{align}

    If we let $c_j$ denote the center of $I_j$, then, for $x \notin \Omega^*$, 
    since $b_j$ has zero average
    \begin{align*}
        \left|\int_{I_j} \frac{e^{-\pi|x - x'|}}{x - x'} b_j(x') \, \mathrm{d}x'\right|
            &= \left|\int_{I_j} e^{-\pi|x - x'|}
                \left(\frac{b_j(x')}{x - x'} - \frac{b_j(x')}{x - c_j}  \right)\, \mathrm{d}x'\right| \\
            &\leq \int_{I_j} e^{-\pi|x - x'|}
                \left|\frac{b_j(x')(x' - c_j)}{(x - x')(x-c_j)}\right| \, \mathrm{d}x' \\
            &\leq \int_{I_j} |b_j(x')| \frac{|I_j|}{(x - c_j)^2} \, \mathrm{d}x'
    \end{align*}
    as $|x - x'| \geq |x - c_j|/2$ and $|x' - c_j| \leq |I_j|/2$. Moreover, 
    \begin{align} \label{eq:IjIntBnd}
        \int_{\mathbb R \backslash \Omega^*} \frac{|I_j|}{(x - c_j)^2} \, \mathrm{d}x'
            \leq \int_{\mathbb R \backslash I_j} \frac{|I_j|}{(x - c_j)^2} \, \mathrm{d}x'
            \leq 4,
    \end{align}
    and so, by Fubini's Theorem,
    \begin{align} \label{eq:sumEbj}
        \sum_j \int_{\mathbb R \backslash \Omega^*} \left| E b_j(x) \right| \, \mathrm{d}x
            &\leq \frac{1}{2\pi} \sum_j\int_{\mathbb R \backslash \Omega^*} 
                \left| \int_{I_j} e^{-\pi|x-x'|} \frac{b_j(x')}{x - x'} \, \mathrm{d}x' \right| \, \mathrm{d}x 
                \\
            &\leq \frac{1}{2\pi} \sum_j \int_{\mathbb R \backslash \Omega^*}  \int_{I_j}
                |b_j(x')| \frac{|I_j|}{(x - c_j)^2} \, \mathrm{d}x' \, \mathrm{d}x 
                \nonumber \\
            &\leq \frac{2}{\pi} \sum_j \int_{I_j} |b_j(x')| \, \mathrm{d}x' 
                \nonumber \\
            &\leq \frac{4}{\pi} \| f\|_1. \nonumber
    \end{align}
    Putting everything together, we find
    \begin{align} \label{eq:E11bnd}
        d_{Ef}(\Lambda) 
            \leq d_{Eg}\left(\frac{\Lambda}{2}\right) + d_{Eb}\left(\frac{\Lambda}{2}\right)
            \leq \frac{C}{\Lambda} \| f\|_1,
    \end{align}
    where $C>0$ is independent of $\Lambda$ and $f$. Since $f$ is Schwartz class, we can 
    therefore extend the inequality \eqref{eq:E11bnd} to $L^1$ {\em via} density to
    conclude that $E$ is weak (1,1).
\end{proof}


\begin{thm}\label{thm:EbndOp}
	The operator $E$ is strong type $(p, p)$ for $p>1$.
\end{thm}
\begin{proof}
	Theorems \ref{thm:Estrong} and \ref{thm:Eweak} in conjunction with the 
	Marcinkiewicz Interpolation Theorem \cite[Theorem 1.3.2]{Grafakos} immediately imply that $E$ is strong
	type $(p,p)$ for $p\in (1, 2]$. As such, it remains only to show that
	$E$ is strong type $(p, p)$ for $p > 2$. 

    Let $E'$ denote the adjoint of $E$. By density, we need only consider 
    $f \in \mathscr S(\mathbb R)$. Fix $p >2$ and let $q$ 
    denote its H\"older Conjugate. Note that the map
    \[
        g \mapsto \int_{\mathbb R} E f \bar{g} =: \left\langle Ef, g \right\rangle 
    \]
    is a linear functional on $L^q$ with norm $\| E f \|_p$. As such, 
    we see by H\"older's inequality that
    \begin{align*}
        \| E f \|_p
            &= \sup_{\|g\|_q = 1} |\langle Ef, g \rangle| \\
            &= \sup_{\|g\|_q = 1} |\langle f, E'g \rangle| \\
            &\leq  \|f\|_p \|E\|_q \\
            &\leq  C \|f\|_p,
    \end{align*}
    as $p>2$ implies $1 < q < 2$, which means $E$ is strong type $(q, q)$, 
    and Theorem 5 from Chapter VII.3 of \cite{yosida} implies 
    $\|E\|_q = \|E'\|_q$.
\end{proof}

\end{document}