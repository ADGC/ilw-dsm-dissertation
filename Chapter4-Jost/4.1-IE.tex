%%========================================================
%% Chapter 4.1: Existence and Continuity of Jost Solutions
%%========================================================

\documentclass[../dissertation.tex]{subfiles}

\begin{document}
\section{Existence and Continuity of Jost Solutions}\label{sec4:Exist}



\begin{prop}[Existence and Uniqueness of Jost Solutions]\label{prop4:exist}
	There is a $c_0 > 0$ so that for real valued measurable functions $u \in X$
	with $\| u \|_X < c_0$ and any $\lambda \in \mathbb R$, integral equations 
	\eqref{eq4:JostIE} are uniquely solvable in $\inn{\dotarg} L^\infty(\mathbb R)$.
	If $\lambda \ne 0$ these solutions are essentially bounded 
	(\textit{i.e.} $L^\infty$).
\end{prop}
\begin{proof}
	Consider the operators $T_{\star,\lambda, u}$ ($\star = L \text{, or } R$)
	given by
	\[
		\big(T_{\star,\lambda, u} f\big)(x) 
			:= \int_{\mathbb R} G_\star^+(x-x'; \lambda) \, u(x') f(x') \, \mathrm{d}x'.
	\]
	Unless necessary to avoid confusion, we write $T_\star$ instead of 
	$T_{\star,\lambda,u}$. The integral equations \eqref{eq4:JostIE} can be 
	written in terms of $T_\star$ as
	\begin{subequations}
		\label{eq4:JostIET}
		\begin{align}
			\label{eq4:JostIETM}
			\begin{pmatrix}
				1 \\ e^{ix\lambda}
			\end{pmatrix}
			&= 
				(I - T_L) 
				\begin{pmatrix}
					M_1^+ \\
					M_e^+
				\end{pmatrix} \\
			\label{eq4:JostIETN}
			\begin{pmatrix}
				1 \\ e^{ix\lambda}
			\end{pmatrix}
			&= 
				(I - T_R) 
				\begin{pmatrix}
					N_1^+ \\
					N_e^+
				\end{pmatrix}
		\end{align}
	\end{subequations}

	As such, our task is to invert the operators $(I - T_\star)$ on the 
	appropriate spaces, which we do \text{via} Neumann series. Indeed, 
	Proposition \ref{prop2:Tbnd} implies that we may choose $c_0>0$ independently 
	of $\lambda \in \mathbb R$ so that 
	\begin{align}\label{eq4:Tbnd}
		\| T_\star \|_{\inn{\dotarg} L^\infty \toitself} < \frac{1}{2}
	\end{align}
	whenever $\| u\|_X < c_0$. Since the Jost solutions are the same as the 
	solutions to \eqref{eq4:JostIET}, estimate \eqref{eq4:Tbnd} allows us to
	conclude that the Jost solutions exist and are unique for 
	$u \in B_X(0, c_0) := \{ u\in X ~:~ \|u\|_X < c_0 \}$.

	Further, an analogous argument involving Proposition \ref{prop2:Tbndl} 
	also allows us to conclude that the Jost solutions are essentially 
	bounded for each fixed real $\lambda \ne 0$.
\end{proof}

\begin{lma}[Continuity in $u$]\label{lma4:cont}
	Let $c_0 > 0$ be the same as in Proposition \ref{prop4:exist}. Denote by 
	$M_1(x; \lambda, u), \ldots, N_e(x; \lambda, u)$ the Jost solutions
	$M_1, \ldots, N_e$ corresponding to the potential $u$. Then the maps
	\begin{align*}
		u &\mapsto M_1^+(\dotarg; \lambda, u ), &
		u &\mapsto M_e^+(\dotarg; \lambda, u ), \\ 
		u &\mapsto N_1^+(\dotarg; \lambda, u ), & 
		u &\mapsto N_e^+(\dotarg; \lambda, u ),
	\end{align*}
	from the open ball $B_X(0, c_0)$ into $\inn{\dotarg} L^\infty(\mathbb R)$ are 
	Lipschitz continuous with Lipschitz constant uniform in 
	$\lambda \in \mathbb R$.
\end{lma}
\begin{proof}
	We prove the Lipschitz continuity for the map $u \mapsto M_1^+$ and note 
	that analogous arguments are sufficient to prove continuity for the 
	remaining maps.

	By the second resolvent formula,
	\begin{align}\label{eq3:resolv}
		M_1^+(x; \lambda, u_1) - M_1^+(x; \lambda, u_2)
			&= \left[ 
					(1-T_{L, \lambda, u_1})^{-1} -  (1-T_{L,\lambda, u_2})^{-1}
				\right] 1 \\
			&= (1-T_{L, \lambda, u_1})^{-1} G_L^+(\dotarg; \lambda) * 
				\big[(u_1 - u_2) \, M_1(\dotarg; \lambda, u_2) \big]
				\nonumber
	\end{align}
	Thus, we see from \eqref{eq3:Tbndxbrac} that
	\begin{align}
		&\big\|
			M_1^+(x; \lambda, u_1) - M_1^+(x; \lambda, u_2)
		\big\|_{\inn{\dotarg} L^\infty} 
				\nonumber\\
		&\qquad\leq 
			\big\|
				\left(1 - T_{L,\lambda,u_1}\right)^{-1}
			\big\|_{\inn{\dotarg} L^\infty\toitself}
			\left\|T_{L, \lambda,u_1-u_2}\right\|_{\inn{\dotarg} L^\infty\toitself}
			\|M_1(\dotarg; \lambda, u_2) \| \nonumber\\
		&\qquad\lesssim \|u_1 - u_2\|_X,
	\end{align}
	as 
	\begin{align*}
		\big\|
			\left(1 - T_{L,\lambda,u_1}\right)^{-1}
		\big\|_{\inn{\dotarg} L^\infty\toitself}
			\leq \sum_{n\geq 0} \frac{1}{2^n} = 2
	\end{align*}
	by Neumann series expansion. For $\lambda \ne 0$, the estimate
	\begin{align*}
		\big\|M_1^+(x; \lambda, u_1) - M_1^+(x; \lambda, u_2)\big\|_{L^\infty} 
			\lesssim \|u_1 - u_2\|_X
	\end{align*}
	also follows from \eqref{eq3:resolv}.
\end{proof}


\begin{prop}\label{prop4:uMcont}
	Let $c_0 > 0$ be the same as in Proposition \ref{prop4:exist}. Then
	the maps
	\begin{align*}
		u &\mapsto u\, M_1^+(\dotarg; \lambda, u ), &
		u &\mapsto u\, M_e^+(\dotarg; \lambda, u ), \\ 
		u &\mapsto u\, N_1^+(\dotarg; \lambda, u ), & 
		u &\mapsto u\, N_e^+(\dotarg; \lambda, u ),
	\end{align*}
	from the open ball $B_X(0, c_0)$ into $L^{1,1}(\mathbb R)$ are 
	Lipschitz continuous with Lipschitz constant uniform in 
	$\lambda \in \mathbb R$.
\end{prop}
\begin{proof}
	As usual, we prove Proposition \ref{prop4:uMcont} for the map $u\mapsto u\, M_1^+$
	and note similar arguments prove that the remaining maps are Lipschitz.

	First note that $u\, M_1^+ \in L^{1,1}(\mathbb R)$ as 
	\[
		\|u\, M_1^+\|_{L^{1,1}}
			= \big\| 
					\left(\inn{\dotarg}^{-1} M_1^+\right) (\inn{\dotarg}^2 u) 
				\big\|_{L^1} 
			\leq \|M_1^+\|_{\inn{\dotarg}L^\infty} \|u\|_X.
	\]
	Choose any $u_1$, $u_2 \in B_X(0, c_0)$.
	Since the map 
	$B_X(0, c_0) \ni u \mapsto M_1^+ \in \inn{\dotarg} L^\infty(\mathbb R)$ 
	is Lipschitz (uniformly in $\lambda$) by Lemma \ref{lma4:cont}, 
	\[
		\big\| M_1^+( \dotarg; \lambda, u_1) \big\|_{\inn{\dotarg}L^\infty}
			= 
				\big\| 
					M_1^+( \dotarg; \lambda, u_1) - 0 
				\big\|_{\inn{\dotarg}L^\infty}
			\lesssim \|u_1 - 0\|_{X}
			< c_0
	\]
	where the implied constant is the Lipschitz constant for the map 
	$u \mapsto M_1^+$. We then have
	\begin{align*}
		&\big\| 
			u_1\, M_1(\cdot; \lambda, u_1) - u_2\, M_1(\cdot; \lambda, u_2)
		\big\|_{L^{1,1}} \\
		&\qquad= 
			\int_{\mathbb R} 
				\inn{x}
				\big| 
					u_1(x) \, M_1^+( x; \lambda, u_1) 
						- u_2(x) \, M_1^+( x; \lambda, u_1) 
				\big| 
			\, \mathrm{d}x \\
		&\qquad\quad+
			\int_{\mathbb R} 
				\inn{x}
				\big|
					u_2(x) \, M_1^+( x; \lambda, u_1)  
						- u_2(x) \,  M_1^+( x; \lambda, u_2) 
				\big|
			\, \mathrm{d}x \\
		&\qquad\leq \int_{\mathbb R} 
				\left(\inn{x}^{-1}\big|M_1^+( x; \lambda, u_1)\big|\right)
				\big( \inn{x}^2
				\big| 
					u_1(x) - u_2(x)
				\big| 
				\big)
			\, \mathrm{d}x \\
		&\qquad\quad+
			\int_{\mathbb R} 
				\left(
					\inn{x}^{-1}
					\big|
						 M_1^+( x; \lambda, u_1)  
							-  M_1^+( x; \lambda, u_2) 
					\big|
				\right)
				\big( \inn{x}^2 u_2(x) \big)
			\, \mathrm{d}x \\
		&\qquad\leq 
			\|M_1^+( \dotarg; \lambda, u_1) \|_{\inn{\dotarg} L^\infty}
			\left\|(u_1 - u_2)\inn{\dotarg}^2\right\|_{L^1} \\
		&\qquad\quad
			+ \left\|u_2\inn{\dotarg}^2\right\|_{L^1}
				\|
					M_1^+( \dotarg; \lambda, u_1)
					- M_1^+( \dotarg; \lambda, u_2)
				\|_{\inn{\dotarg} L^\infty} \\
		&\qquad\lesssim  \| u_1 - u_2 \|_X,
	\end{align*}
	where the implied constant is independent of $\lambda$ and $u$.
\end{proof}

\begin{rmk}
	Note that since $L^{1,1}(\mathbb R) \su L^1(\mathbb R)$, Proposition 
	\ref{prop4:uMcont} also implies that the maps involved are also Lipschitz 
	continuous, when considered as maps from $B_X(0, c_0)$ into $L^1(\mathbb R)$.
\end{rmk}

\begin{lma}\label{lma4:Mlamcont}
	For fixed $x\in \mathbb R$, the Jost solution boundary value $M_1^+$ is continuous 
	in $\lambda$.
\end{lma}
\begin{proof}
	Define
	\[
		M_h:= M_1^+(\dotarg; \lambda +h) - M_1^+(\dotarg; \lambda),
 	\]
 	and denote by $T_\lambda$ the convolution operator given by 
 	\[
 		T_\lambda f := G_L^+(\dotarg; \lambda) * (u\, f).
 	\]
 	To show $M_1^+$ is continuous in $\lambda$, it suffices to show 
 	$\|M_h\|_{\inn{x}L_x^\infty} \to 0$ as $h \to 0$.
 	Recalling that $M(\lambda) = 1+ T_\lambda M(\lambda)$, we see that 
 	\begin{align*}
 		M_h 
 			&= T_{\lambda+h}M(\lambda+h) - T_{\lambda} M(\lambda) \\
 			&= \big(T_{\lambda+h} - T_\lambda\big) M(\lambda+h) 
 				- T_\lambda M_h
 	\end{align*}
 	which implies that
 	\begin{align}
 		M_h 
 			= 
 				\big(1-T_{\lambda}\big)^{-1} 
 				\big(T_{\lambda+h} - T_\lambda\big) M(\lambda+h),
 	\end{align}
 	as $I - T_\lambda$ is invertible on $\inn{\dotarg}L^\infty(\mathbb R)$.
 	Since
 	\[
 		\left\|
 			\big(T_{\lambda+h} - T_\lambda\big)M(\lambda+h)
 		\right\|_{\inn{x}L_x^\infty}
 			\leq \|T_{\lambda+h} - T_\lambda\|_{\inn{x}L_x^\infty\toitself}
 					\|M(\lambda+h)\|_{\inn{x}L_x^\infty},
 	\]
 	the $\lambda$-continuity of $T_\lambda$ follows from the continuity of 
 	$\big(I - T_\lambda\big)^{-1}$ and Lemma \ref{prop2:Tlamcont}.
\end{proof}

\end{document}