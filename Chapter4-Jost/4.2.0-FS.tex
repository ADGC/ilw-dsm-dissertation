%%================================
%% Section 4.1.0: Functional Setup (Temporary Title)
%%================================

\documentclass[../dissertation.tex]{subfiles}

\begin{document}
% \section{Functional Setup (Temporary Title)}\label{sec3:FS}

In order to prove the equivalence of the Jost solutions and solutions to the 
integral equations \eqref{eq4:JostIE}, we need to define what doing so actually
means. To that end, we respectively define explicitly what a Jost solution is 
(Definition \ref{dfn4:DEsoln}) or what it means for a function to solve the 
linear spectral problem reformulated as integral equations equations (Definition 
\ref{dfn4:IEsoln}).

\begin{rmk}\label{rmk3:wlog}
	In this chapter we prove that the Jost solution $M_1$ 
	solves \eqref{eq0:JostIE} and \textit{vise versa} in the sense
	of Definitions \ref{dfn4:DEsoln} and \ref{dfn4:IEsoln} and note
	that the proofs of the analogous results for $M_e$, $N_1$ and $N_e$
	are similar. Definitions \ref{dfn4:DEsoln} and \ref{dfn4:IEsoln}
	are written accordingly, and, throughout this section, we write $M$ 
	\text{en lieu} of $M_1$. 
\end{rmk}

Unless stated otherwise, in the remainder of this section we take 
$u \in B_X(0, c_0)$ where $c_0$ is chosen according to Proposition 
\ref{prop4:exist} to ensure that the integral equations \eqref{eq4:JostIE}
are uniquely solvable. For 
$M^+ \in L^\infty(\mathbb R)$,
$u \in X$ implies $uM \in L^2(\mathbb R)$ and the solution map $u\mapsto M$
for \eqref{eq0:JostIE} is continuous from $L^2$ to $L^\infty$.

\begin{defn}[Analytic Weak Jost Solution]\label{dfn4:DEsoln}
	Fix $\lambda \in \mathbb R$. We say that a function $M$ analytic on the strip
	$$\mathcal S_1 = \{ z \in \mathbb C ~:~ 0 < \im z < 2 \}$$ with respective
	lower and upper boundary values $M^+$, $M^-$ solves the linear spectral problem
	\begin{align} \label{eq4:LinSpecProb}
		L_1(M)(x) 
			:= \frac{1}{i} \frac{\partial M^+}{\partial x}(x)
				- \zeta \big(M^+(x) - M^-(x)\big)
			= u(x) M^+(x)
	\end{align}
	with $M^+(x) \to 1$ as $x\to -\infty$ if 
	\begin{itemize}
		\item[\namedlabel{itm:asymp}{(i)}] $M$ satisfies the following asymptotic conditions:
			\begin{itemize}
				\item[(a)] \textbf{Lower boundary value asymptotic condition:}\\
					$\lim_{x\to-\infty} \inn{x} \left(M^+(x) -1\right)=0$
				\item[(b)] \textbf{Upper boundary value asymptotic condition:}\\
					There exist $M_1$, $M_2$ so that 
					\[
						M^-(x) -1 = M_1(x) + M_2(x), 
					\]
					where 
					\[
						\inn{x}^{1+\upsilon} | M_1(x) | \lesssim 1
					\]
					as $x\to -\infty$ and 
					\[
						\inn{\dotarg}^\tau M_2 \in L^2(\mathbb R)
					\]
					for any $\upsilon \in \left(0, \frac{1}{2}\right)$ and $\tau \in [0,1)$.
			\end{itemize} 
		\item[\namedlabel{itm:space}{(ii)}] $M^+ \in \inn{\dotarg} L^\infty(\mathbb R)$. 
			If $\lambda \ne 0$, then $M^+ \in L^\infty(\mathbb R)$.
		\item[\namedlabel{itm:cont}{(iii)}] $M$ is continuous in $0 \leq \im z \leq 2-\varepsilon$
			for any $0 < \varepsilon < 2$.
		\item[\namedlabel{itm:decomp}{(iv)}] There is a decomposition $M(z) = M_c(z) + M_s(z)$ for 
			$0 < \im  z < 2$ so that 
			\begin{itemize}
				\item[(a)] $M_c$ extends to a continuous function on the closure
					$\ol{\mathcal S}_1$ of $\mathcal S_1$ with 
					\[
						\mathop{\lim_{x\to-\infty}}_{x\in \mathbb R} M_c(x+2i) = 1	
					\]
				\item[(b)] The estimates
					\[
						\|M_s(\dotarg+iy)\|_{L^\infty} \leq (2-y)^{-1/2}, \qquad
						\sup_{0\leq y < 2} \|M_s(\dotarg+iy)\|_{L^2} < \infty
					\]
					hold. Moreover, $M_s$ has an $L^2$ boundary value 
					$M_s^-(x):= \lim_{\varepsilon \searrow 0} M_s\big(\dotarg+i(2-\varepsilon)\big)$
					on $\im z = 2$ with $M_s(x+iy) \to M_s^-(x)$ for almost every
					$x$.
			\end{itemize}
		\item[\namedlabel{itm:jost}{(v)}] Defining $M^-(x) = M_c(x+2i) + M_s(x+2i)$, the differential equation
			\eqref{eq4:LinSpecProb} holds in the weak sense, testing against 
			$\phi \in C_0^\infty(\mathbb R)$.
	\end{itemize}
\end{defn}



\begin{defn}[Associated Integral Equation Solution]\label{dfn4:IEsoln}
	Fix $\lambda \in \mathbb R$ and $u\in X$. A function 
	$M^+(x; \lambda, u) \in \inn{\dotarg} L_x^\infty(\mathbb R)$ 
	solves the integral form of the linear spectral problem if the identity
	\begin{align}\label{eq4:MPluseq}
		M^+(x) = 1 + G_L^+*(uM^+)(x)
	\end{align}
	holds for almost every $x\in \mathbb R$.
\end{defn}

As discussed following the statement of Theorem \ref{thm3:main_result}, 
In breaking up $M$ into $M_c$ and $M_s$ in Definition \ref{dfn4:DEsoln}, we 
are decomposing $M$ into a piece which has a continuous upper boundary value and a
piece which exists only in an $L^2$ sense. The inspiration for this decomposition
stems from the similar way we can decompose $G_\star$ ($\star = L \text{, or } R$)
hinted at in Theorem \ref{thm3:main_result}.
% given its particular relevance to 
% our work in this section, we restate the discussion Theorem \ref{thm3:main_result} 
% precedes as the following proposition

% \begin{prop}\label{prop3:Gdecomp}
% 	Let $u \in X$ and $M^+ \in \inn{\dotarg} L^\infty(\mathbb R)$ be a function which 
% 	satisfies the integral equation $M^+ = 1 + G_L^+*(uM^+)$. Then $M^+$ has an 
% 	analytic extension $M$ to the complex strip $\mathcal S_1$. Further, $M$ 
% 	has a decomposition $M(z) = 
% \end{prop}
 
% In breaking up $M$ into $M_c$ and $M_r$ in Definition \ref{dfn4:DEsoln}, we 
% are decomposing $M$ into a piece which has a continuous upper boundary value and a
% piece which exists only in an $L^2$ sense. The inspiration for this decomposition
% stems from the similar decomposition for $G_L$ hinted at in Chapter \ref{cptr03:xContin}
% and explicitly stated below in Proposition \ref{prop3:Gdecomp}.

% \begin{prop}\label{prop3:Gdecomp}
% 	Let $p \in (1, \infty)$, $\zeta \in (0, 1/2)\cup(1/2, \infty)$, and suppose 
% 	that $f \in L^p(\mathbb R)$. Then the limit
% 	\[
% 		\lim_{\varepsilon \searrow 0} G_L\big(\dotarg + i(2-\varepsilon)\big) * f(x)
% 			= T_c^- f(x) + T_s^- f(x)	
% 	\]
% 	holds pointwise \textit{a.e.}, where
% 	\begin{align*}
% 		T_c^- f(x) 
% 			&=	i \, \alpha(\lambda) \int_{-\infty}^x f(x')\, \mathrm{d}x' 
% 				+ i\, \beta(\lambda) \,e^{i\lambda x} \, e^{-2\lambda} 
% 					\int_{-\infty}^x e^{i\lambda x'} f(x')\, \mathrm{d}x' 
% 				- R(\dotarg, 2)*f(x)
% 	\end{align*}
% 	and
% 	\begin{align*}
% 		T_s^- f(x)
% 			&= - Ef(x) + \frac{1}{2} f(x).
% 	\end{align*}
% \end{prop}

In Section \ref{subsec4:IEtoDE}, we use the following property of functions analytic
on the complex strip $\mathcal S_1$. 
\begin{prop}\label{prop3:BndryRelProp}
	Suppose $F$ is analytic in the open strip $\mathcal S_1$ and that 
	$|F(x+iy)| \lesssim (2-y)^{-1/2}$ for $y \in [0,2)$. Suppose further that 
	$F=F_1+F_2$ where
	\begin{itemize}
		\item[(i)] $F_1$ is bounded and continuous on the closure $\ol{\mathcal S}_1$,
			and
		\item[(ii)] for any $\varepsilon \in (0, 2)$, $F_2$ is bounded and continuous
			on $\mathbb R \times [0, 2-\varepsilon)$, the estimate
			\[
				\sup_{0< y< 2} \|F_2(\dotarg + iy)\|_{L^2} < \infty
			\]
			holds, and there is a function $F_2(\dotarg + 2i)$ so that 
			$F(\dotarg + iy) \to F_2(\dotarg + 2i)$ in $L^2(\mathbb R)$
			as $y\nearrow 2$. 
	\end{itemize}
	Denote by $F^+$ the boundary value $F_1(x+ i0) + F_2( x + i0)$ and by 
	$F^-$ the boundary value $F_1\big(x + (2i-0i)\big) + F_2\big(x + (2i-0i)\big)$, 
	where $2i-0i$ is the implied limit $(2-\varepsilon)i$ as $\varepsilon \searrow0$.
	Then, as distributions in $\mathcal D'(\mathbb R)$,
	\[
		\big(\mathcal F F^+\big)(\xi) = e^{2\xi} \big(\mathcal F F^-\big)(\xi).
	\]
\end{prop}
\begin{proof}
	Let $\gamma$ denote the contour shown in red in Figure \ref{fig3:BndryRelCont}, where
	$R>0$ and $0< \varepsilon < 1$. 
	\begin{figure}[H]
		\centering
		\begin{tikzpicture}
		[
			scale=1.4,
			cont/.style={
				thick,
				red, 
				decoration={markings, mark=at position .55 with \arrow{stealth}[arrowhead=5mm]},
				postaction={decorate}
			}
		]
			\def\R{3}
			\def\cwidth{2}
			\def\eps{0.5}
			\def\xbuffer{0.5}
			\def\ybuffer{0.5}
			\def\xMin{-\R-\xbuffer}
			\def\xMax{\R+\xbuffer}
			\def\yMax{\cwidth+\ybuffer}
			\def\yMin{{0-\ybuffer}}
			\def\tickLength{0.1}
			%% Axes
			\draw[->, very thick] (\xMin, 0) -- (\xMax, 0) node[right] {$\re z$};
			\draw[->, very thick] (0, \yMin) -- (0, \yMax) node[above] {$\im z$};

			\draw[thick, dashed] (\xMin,\cwidth) -- (\xMax, \cwidth) node[right] {$\im z$};

			%% Draw Contour
			\coordinate (LB) at (-\R, \eps);
			\coordinate (RB) at (\R, \eps);
			\coordinate (RT) at (\R, {\cwidth - \eps});
			\coordinate (LT) at (-\R, {\cwidth - \eps});

			\draw[cont] (LB) -- (RB);
			\draw[cont] (RB) -- (RT);
			\draw[cont] (RT) -- (LT);
			\draw[cont] (LT) -- (LB);
			

			% \node[left] at (LB) {$\im z = \varepsilon$};
			% \node[left] at (LT) {$\im z = 2 - \varepsilon$};


			%% Draw Re axis tick marks
			\draw[very thick] (-\R, \tickLength) -- (-\R, -\tickLength) 
				node[below] {$-R$};
			\draw[very thick] (\R, \tickLength) -- (\R, -\tickLength) 
				node[below] {$R$};

			%% Draw Im axis tick marks
			\draw[very thick] (\tickLength, {\eps}) -- (-\tickLength, \eps) 
				node[below left] {$\im z = \varepsilon$};
			\draw[very thick] (\tickLength, {\cwidth - \eps}) -- (-\tickLength, {\cwidth - \eps})
				node[above left] {$\im z = 2 - \varepsilon$};

		\end{tikzpicture}
		\caption{Contour of integration for proof of Proposition \ref{prop3:BndryRelProp}}
		\label{fig3:BndryRelCont}
	\end{figure}
	Since $F(z)$ is analytic in the interior 
	of the strip $\mathcal S_1$, the integral of $e^{-i\xi x} F(z)$ around the contour 
	$\gamma$ is zero for all appropriate $R$ and $\varepsilon$. Thus, taking 
	$\varepsilon\searrow 0$ for a fixed $R>0$ yields
	\begin{align}\label{eq3:BndryPropIs}
		I_1 - I_3 = I_4 - I_2,
	\end{align}
	where 
	\begin{align*}
		I_1 &:= \int_{-R}^R e^{-i \xi x} F^+(x) \, \mathrm{d}x \\
		I_2 &:= i \int_0^2 e^{-i\xi R} e^{-t\xi} F(R+it) \, \mathrm{d}t \\
		I_3 &:= e^{2\xi} \int_{-R}^R e^{-i\xi x} F^-(x) \, \mathrm{d}x \\
		I_4 &:= i \int_0^2 e^{i\xi R} e^{-t\xi} F(-R+it) \, \mathrm{d}t
	\end{align*}
	correspond to the four straight segments of $\gamma$. We claim $I_2$ and $I_4$
	converge to zero as distributions as $R \to \infty$. 
	Indeed, for $\phi(\xi) \in C_0^\infty(\mathbb R)$,
	\begin{align*}
		\int_{\mathbb R} \phi(\xi) I_2(\xi) \, \mathrm{d}\xi
			&= i \int_0^2 \left(\int_{\mathbb R} \phi(\xi) e^{-i\xi R} e^{-t\xi} \, \mathrm{d}\xi\right) F(R+it) \, \mathrm{d}t
	\end{align*}
	Since $\phi$ is compactly supported, using integration by parts on the interior integral above yields
	\[
		\int_{\mathbb R} \phi(\xi) e^{-i\xi R} e^{-t\xi} \, \mathrm{d}\xi
			= \frac{1}{iR} 
				\int_{\mathbb R} 
					e^{i\xi R} 
					\big(
						\phi'(\xi) \, e^{-t\xi} - t \,\phi(\xi)\, e^{-t\xi}
					\big)
				\,d\xi,
	\]
	which implies 
	\begin{align}\label{eq3:BndryPropInnerIntBnd}
		\left| \int_{\mathbb R} \phi(\xi) I_2(\xi) \, \mathrm{d}\xi \right|
			&\leq \frac{1}{R}
				\int_{\mathbb R} 
					e^{-t\xi}
					\big(
						|\phi'(\xi)| + 2|\phi(\xi)
					\big)
				\,d\xi	
	\end{align}
	Using the hypothesis $|F(x+iy)| \lesssim (2-y)^{-1/2}$ in conjunction with estimate
	\eqref{eq3:BndryPropInnerIntBnd} allows us to conclude that the integral of $I_2$ with 
	$\phi$ is bounded (up to a positive constant) by the integral
	\begin{align}
		\frac{1}{R} \int_0^2 (2-R)^{-1/2} \int_{\mathbb R} e^{i\xi R} 
					\big(
						\phi'(\xi) \, e^{-t\xi} - t \,\phi(\xi)\, e^{-t\xi}
					\big)
				\,d\xi,
	\end{align}
	which vanishes in the limit $R\to \infty$ due to the compact support of $\phi$.
	The same argument with $R$ replaced by $-R$ also shows $I_4$ also vanishes 
	(as a distribution) in the $R\to\infty$ limit. It therefore follows
	from \eqref{eq3:BndryPropIs} that 
	\begin{align*}
		\int_{\mathbb R} 
			\phi(\xi) \, 
			\Big[ 
				\big(\mathcal F F^+\big)(\xi) - e^{2\xi}\big(\mathcal F F^-\big)(\xi) 
			\Big]
			\, \mathrm{d}\xi
		&= \lim_{R\to\infty} \int_{\mathbb R} (I_1-I_3) \, \phi(\xi) \, \mathrm{d}\xi
		=0
	\end{align*}
	as claimed.
\end{proof}


\end{document}