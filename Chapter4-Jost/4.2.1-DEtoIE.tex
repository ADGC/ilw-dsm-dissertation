
%%==============================
%% Section 4.01.01: DE $\implies$ IE (Temporary Title)
%%==============================


\documentclass[../dissertation.tex]{subfiles}

\begin{document}
\subsection{Jost Solutions Solve the Integral Equations}\label{subsec4:DEtoIE}

In this subsection we prove that Jost solutions also solve the 
corresponding integral equation, as respectively 
defined in Definitions \ref{dfn4:DEsoln} and \ref{dfn4:IEsoln}. We do 
so in two phases: first for $\lambda \ne 0$ (Lemma \ref{lma4:DEtoIE1}), and then 
for $\lambda = 0$ (Lemma \ref{lma4:DEtoIE2}). 

\begin{lma}\label{lma4:DEtoIE1}
	A Jost solution $M$ satisfying Definition \ref{dfn4:DEsoln} also solves the 
	associated integral equation \ref{eq4:MPluseq} in the sense of Definition 
	\ref{dfn4:IEsoln} whenever $\lambda \in \mathbb R \sm \{0\}$.
\end{lma}
\begin{proof}
	Proposition \ref{prop3:BndryRelProp} implies 
	$\wh{M^-} = e^{-2\xi} \wh{M^+}$. Moreover, $\wh{M^+}$ is a tempered 
	distribution by the hypothesis of Definition \ref{dfn4:DEsoln} which means
	that $e^{2\xi} \wh{M^-}$ is also tempered. Taking the distribution Fourier
	transform of both sides of \eqref{eq4:LinSpecProb} we consequently find
	\begin{align}\label{eq3:DEtoIEsymbol}
		\wh{uM^+} 
			&= \xi \wh{M^+} - \zeta\left( \wh{M^+} - \wh{M^-} \right) \\
			&= \left(\xi - \zeta(1-e^{-2\xi})\right) \wh{M^+} \nonumber \\
			&= p(\xi) \wh{M^+} \nonumber 
	\end{align}
	in the sense of distributions in $\mathcal D'(\mathbb R)$. To avoid the 
	zeros of the symbol $p$, we rewrite \eqref{eq3:DEtoIEsymbol} as 
	\begin{align}\label{eq3:peps}
		p(\xi - i\varepsilon) \wh{M^+} 
			= [p(\xi - i\varepsilon) - p(\xi)] \wh{M^+} + \wh{uM^+}
	\end{align}
	for $0< \varepsilon \ll 1$ and introduce the approximate Green's function 
	\begin{align}\label{eq4:approxGF}
		G_L^\varepsilon(x; \lambda) 
			= \frac{1}{2\pi} 
				\int_{\mathbb R} 
					e^{ix\xi} \frac{1}{p(\xi - i \varepsilon)}
				\,d\xi.
	\end{align}
	Using the contour shift 
	$\mathbb R-i\varepsilon \mapsto \mathbb R + i \sign(x) \pi$
	in our work from Section \ref{sec1:GreensFunctions} to prove
	\eqref{eq1:GLrep} shows that
	\begin{align}\label{eq3:GLeps}
		G_L^\varepsilon(x; \lambda)
			=
				\begin{cases}
					K^+(x; \lambda) 
						+ i
						\big[ 
							\alpha(\lambda) 
							+ \beta(\lambda) e^{i\lambda x}
						\big] e^{-\varepsilon x} \, \chi_L(x)
						& \lambda \ne 0 \\
					K^+(x; \lambda) 
						+ i
						\left[ 
							\frac{2}{3} + i x
						\right] e^{-\varepsilon x} \, \chi_L(x)
						& \lambda = 0
				\end{cases}
	\end{align}
	where $K^+$ is as defined in Theorem \ref{thm1:GFRep}
	\[
		K^+(x) 
			= \frac{e^{-\pi|x|}}{2\pi}
				\int_{\mathbb R}
					e^{ix\xi}
					\frac{1}{p(\xi)+i \sign(x) \pi}
				\, \mathrm{d}\xi.
	\]
	An immediate consequence of \eqref{eq1:GLrep} and \eqref{eq3:GLeps} is that
	\begin{align*}
		G_L^+(x; \lambda) - G_L^\varepsilon(x; \lambda)
			&= i 
				\Big[ 
					\alpha(\lambda) 
					+ \beta(\lambda) e^{ix\lambda}
				\Big]
				\big(1-e^{-\varepsilon x}\big)
				\chi_{(0, \infty)}.
	\end{align*}
	Hence, we see from \eqref{eq2:polesumbnd} that
	\begin{align}\label{eq3:GLminusGeps}
		\big| \left(G_L^+ - G_L^\varepsilon\right)*f(x) \big|
			\lesssim \int_{-\infty}^x 
					\left[ 1 -e^{-\varepsilon(x - x')}  \right] |f(x')| 
				\, \mathrm{d}x'.
	\end{align}
	for real $\lambda\ne0$. 

	The distribution identity $\wh 1 = 2\pi\delta_0$ (where $\delta_0$ denotes a 
	Dirac delta-function centered at $\xi=0$) allows us ``subtract 1'' from both
	sides of \eqref{eq3:peps} to obtain
	\begin{align*}
		&p(\xi -i \varepsilon)(\wh{M^+-1})(\xi) + 2\pi \, p(\xi-i\varepsilon)\,\delta_0 \\
		&\qquad\qquad= [p(\xi-i\varepsilon) - p(\xi)]
						(\wh{M^+-1})(\xi) 
						+ 2\pi\,p(\xi-i\varepsilon)\,\delta_0 + \wh{uM^+}
	\end{align*}
	Dividing both sides by $p(\xi-i\xi)$\textemdash{}since it has no zeros 
	for $\xi\in \mathbb R$\textemdash{}we have
	\begin{align}\label{eq3:Mfouriered}
		(\wh{M^+ -1})(\xi) 
			= \frac{p(\xi-i\varepsilon) - p(\xi)}{p(\xi - i\varepsilon)}(\wh{M^+-1})(\xi)
				+ \frac{1}{p(\xi - i\varepsilon)} \wh{uM^+}.
	\end{align}
	We therefore see from \eqref{eq3:Mfouriered} that in order to verify that $M^+$ satisfies
	definition \ref{dfn4:IEsoln}, it suffices to prove that the following two limits
	\begin{subequations}
		\label{eq3:suff}
		\begin{align}
			\label{eq3:suff1}
			\lim_{\varepsilon \searrow 0} \mathcal F^{-1}
				 	\left[\frac{1}{p(\xi - i\varepsilon)} \wh{uM^+}\right](x)
				 &= G_L^+*(uM)(x)
		\end{align}
		and
		\begin{align}
			\label{eq3:suff2}
				\lim_{\varepsilon \searrow 0} \mathcal F^{-1}
						\left[ 
							\frac{p(\xi-i\varepsilon) - p(\xi)}{p(\xi-i\varepsilon)}
							(\wh{M^+-1})(\xi) 
						\right]
						(x)
					&= 0
		\end{align}
	\end{subequations}
	hold for \textit{a.e.} $x$.

	Since $uM^+ \in L^1(\mathbb R)$, \eqref{eq3:suff1} follows from estimate 
	\eqref{eq3:GLminusGeps} and the Dominated Convergence Theorem. 

	To verify \eqref{eq3:suff2}, first note that 
	\begin{align*}
		\frac{p(\xi - i\varepsilon) - p(\xi)}{p(\xi - i \varepsilon)}
			= \frac{-i \varepsilon + \zeta e^{-2\xi}\left(1-e^{2i\varepsilon}\right)}
					{p(\xi-i\varepsilon)},
	\end{align*}
	which means it suffices to prove 
	\begin{subequations}
		\label{eq3:secondlims}
		\begin{align} \label{eq3:secondlims1}
			\lim_{\varepsilon \searrow 0} 
				\varepsilon \mathcal F^{-1}
					\left[ 
						\frac{1}{p(\xi-i\varepsilon)} (\wh{M^+-1})(\xi) 
					\right](x) 
			= 0
		\end{align}
		and
		\begin{align} \label{eq3:secondlims2}
			\lim_{\varepsilon \searrow 0} 
				(1-e^{2i\varepsilon}) \mathcal F^{-1}
					\left[ 
						\frac{e^{-2\xi}}{p(\xi-i\varepsilon)} (\wh{M^+-1})(\xi) 
					\right](x) 
			= 0.
		\end{align}
	\end{subequations}



	From the definition of $G_L^\varepsilon$ and \eqref{eq3:GLeps} we see that
	\begin{align*}
		\mathcal F^{-1}
				\left[ 
					\frac{1}{p(\xi-i\varepsilon)} (\wh{M^+-1})(\xi) 
				\right](x) 
			&= G_L^\varepsilon * (M^+-1)
	\end{align*}
	and
	\begin{align*}
		G_L^\varepsilon * (M^+-1)(x)
			&= i \alpha(\lambda) \, 
				\int_{-\infty}^x 
					e^{-\varepsilon(x - x')} \big(M^+(x')-1\big) 
				\, \mathrm{d}x' \\
			&\quad + i \beta(\lambda) \, 
					\int_{-\infty}^x 
						e^{i\lambda (x-x')} e^{-\varepsilon(x-x')}\big(M^+(x')-1\big) 
					\, \mathrm{d}x' \\
			&\quad + \left( 
						\int_{-\infty}^x K^+(x- x') + \int_{x}^\infty K^+(x - x') 
					\right)
					\big(M^+(x')-1\big) \, \mathrm{d}x'
	\end{align*}
	Thus, since $e^{i\lambda (x-x')}$ is a unitary phase (\textit{i.e.} has complex 
	modulus 1), in order to verify \eqref{eq3:secondlims1}, we need to show that
	\begin{subequations}
		\label{eq3:thirdlims}
		\begin{align}
			\label{eq3:thirdlims1}
			\lim_{\varepsilon\searrow 0} \varepsilon 
				\int_{-\infty}^x 
					e^{-\varepsilon(x-x')} |M^+(x') - 1| 
				\, \mathrm{d}x' = 0
		\end{align}
		and
		\begin{align}
			\label{eq3:thirdlims2}
			\lim_{\varepsilon\searrow 0} \varepsilon 
					\int_{\mathbb R} K^+(x- x') \big(M^+(x')-1\big) \, \mathrm{d}x'
				= 0.
		\end{align}
	\end{subequations}
	To prove \eqref{eq3:thirdlims1}, we choose an arbitrary $\varepsilon'>0$, 
	split the integral $\int_{-\infty}^x$ into $\int_{-\infty}^{x-L} + \int_{x-L}^x$, 
	and use the fact that $M^+(x) \to 0$ as $x \to -\infty$ to choose $L>0$ 
	sufficiently large that $|M^+(x')  - 1| < \varepsilon'/2$ for $x' < x-L$. Then,
	since 
	\[
		\int_{-\infty}^{x-L} e^{-\varepsilon (x-x')} \,dx'
			= \int_{-\infty}^{-L} e^{\varepsilon t} \, \mathrm{d}t
			< \int_{-\infty}^0 e^{\varepsilon t} \, \mathrm{d}t 
			= \frac{1}{\varepsilon},
	\]
	where $t = x'-x$, we have
	\begin{align} \label{eq3:DEtoIEthirdlim1v1}
		\varepsilon \int_{-\infty}^{x-L} e^{-\varepsilon(x-x')} |M(x') - 1| \,dx'
			< \frac{\varepsilon'}{2} \, \varepsilon \, \int_{-\infty}^{x-L} e^{-\varepsilon(x-x')} \, \mathrm{d}x'
			< \frac{\varepsilon'}{2}.
	\end{align}
	Now, since $M^+$ is continuous, $M^+-1$ is bounded by $\varepsilon'/2$ on $(\infty, x-L)$,
	and $[x-L, x]$ is compact, there exists a constant $C_x > 0$ depending only on $x$
	so that $\sup_{x'\leq x-L}|M^+(x')-1| \leq C_x$. 
	Set $\delta':= \frac{\varepsilon'}{2 L C_x}$.
	For all $\varepsilon < \delta'$, we have
	\begin{align} \label{eq3:DEtoIEthirdlim1v2}
		\varepsilon \int_{x-L}^{x} e^{-\varepsilon(x-x')} |M(x') - 1| \,dx'
			\leq \varepsilon \, C_x \int_{-L}^0 e^{\varepsilon x'} \, \mathrm{d}x'
			< \frac{\varepsilon'}{2},
	\end{align}
	as $e^{\varepsilon x'} \leq 1$ for $x' \leq 0$ implies 
	$\int_{-L}^0 e^{\varepsilon x'} \, \mathrm{d}x'\leq L$. Limit \eqref{eq3:thirdlims1} follows
	from \eqref{eq3:DEtoIEthirdlim1v1} and \eqref{eq3:DEtoIEthirdlim1v2}.

	Since we proved that $K^+\in \mathcal S(\mathbb R)$ in Section \ref{sec1:AsympK}
	and therefore in $L^1(\mathbb R)$, limit \eqref{eq3:secondlims2} is an 
	immediate consequence of Definition \ref{dfn4:DEsoln}(ii) which states
	that $M^+ \in L^\infty(\mathbb R)$.

	Lastly, to complete the proof that $M^+$ satisfies Definition \ref{dfn4:IEsoln}, 
	we now verify limit \eqref{eq3:secondlims2}. To do so, observe that 
	it suffices by the Taylor expansion of $1-e^{2i\varepsilon}$ 
	to verify the (slightly) simpler limit
	\begin{align}\label{eq3:simpler}
	 	\lim_{\varepsilon \searrow 0} 
				\varepsilon \mathcal F^{-1}
					\left[ 
						\frac{e^{-2\xi}}{p(\xi-i\varepsilon)} (\wh{M^+-1})(\xi) 
					\right](x)
			= 0,
	\end{align} 

	We use Proposition 
	\ref{prop3:BndryRelProp} and Definition \ref{dfn4:DEsoln}(v) to rewrite
	\eqref{eq3:simpler} as 
	\begin{align}
		% &\lim_{\varepsilon \searrow 0} 
		% 		\varepsilon \, \mathcal F^{-1}
		% 			\left[ 
		% 				\frac{e^{-2\xi}}{p(\xi-i\varepsilon)} (\wh{M^+-1})(\xi) 
		% 			\right](x) \\
		% &\qquad\qquad= \lim_{\varepsilon \searrow 0} 
		&\lim_{\varepsilon \searrow 0} 
				\varepsilon \, \mathcal F^{-1}
					\left[ 
						\frac{1}{p(\xi-i\varepsilon)} (\wh{M^--1})(\xi) 
					\right](x)
					\\
		&\qquad\qquad= \lim_{\varepsilon \searrow 0} 
				\varepsilon \, \mathcal F^{-1}
					\left[ 
						\frac{1}{p(\xi-i\varepsilon)} (\wh{M_c^--1})(\xi) 
					\right](x) 
				\nonumber \\
		&\qquad\qquad\quad+ \lim_{\varepsilon \searrow 0} 
				\varepsilon \, \mathcal F^{-1}
					\left[ 
						\frac{1}{p(\xi-i\varepsilon)} (\wh{M_s^-})(\xi) 
					\right](x)
				\nonumber
	\end{align}
	An analogous argument to the one employed to verify \eqref{eq3:secondlims1} 
	shows
	\[
		\lim_{\varepsilon \searrow 0} 
				\varepsilon \, \mathcal F^{-1}
					\left[ 
						\frac{1}{p(\xi-i\varepsilon)} (\wh{M_c^--1})(\xi) 
					\right](x) 
			= 0.
	\]
	%%========================================
	%% Begin thievery from Prof. Perry's notes
	%% Rewrite this section later.
	%%========================================
	To analyze the second right-hand term, we again appeal to the representation 
	\eqref{eq3:GLeps}. The ``pole terms'' in \eqref{eq3:GLeps} give
	two terms which can be estimated by
	\[
		\varepsilon \int_{-\infty}^x e^{-\varepsilon(x-x')} |M_s^-(x')| \, \mathrm{d}x'
	\]
	which is $\mathcal O\left(\varepsilon^{1/2}\right)$ by the Schwartz inequality. 
	To control the integrals involving $K^\pm$, we again use the $L^2$ 
	bound on $M_s^-$ to show that the integrals 
	\[
		\int_{\mathbb R} |K_\pm(x-x')| |M_s^-(x')| \, \mathrm{d}x
	\]
	converge, and hence the corresponding terms are 
	$\mathcal O\left(\varepsilon\right)$.
\end{proof}

We now finish our proof that Jost solutions solve the associated integral 
equation by considering the case where $\lambda = 0$.

\begin{lma}\label{lma4:DEtoIE2}
	Let $\lambda =0$ and suppose $M$ is a Jost solution in accordance 
	with Definition \ref{dfn4:DEsoln}. Then $M$ is a solution for 
	\ref{eq4:MPluseq} as specified in Definition \ref{dfn4:IEsoln}.
\end{lma}
\begin{proof}
	As in the proof of Lemma \ref{lma4:DEtoIE1}, we begin with the distribution 
	identity $p(\xi;\lambda) \widehat{M^+} = \widehat{u\,M^+}$, which may be rewritten as
	\[ 
		p(\xi,\lambda) \widehat{M^+-1} = \widehat{u\,M^+},
	\]
	since $p(0;\lambda)=0$. Mimicking our proof of Lemma \ref{lma4:DEtoIE1}, we write
	\[ 
		p(\xi-i\varepsilon) \widehat{M^+-1} 
			= \left[ p(\xi-i\varepsilon) - p(\xi) \right] \widehat{M^+-1} 
				+ \widehat{u\,M^+},
	\]
	where here and in what follows we write $p(\xi)$ for $p(\xi; \lambda = 0)$ 
	since $\lambda = 0$ is fixed throughout. Dividing we get
	\begin{equation}
		\label{Jost.half.pre-int}
		\widehat{M^+-1} 
			= \frac{p(\xi-i\varepsilon) - p(\xi)}{p(\xi-i\varepsilon)} 
					\widehat{M^+-1} 
				+ \frac{\widehat{u\,M^+}}{p(\xi-i\varepsilon)}.
	\end{equation}
	We wish to show that, on taking inverse Fourier transforms and taking 
	$\varepsilon \searrow 0$, we obtain
	\[
		M^+(x)-1 = G_L*(u\,M^+).
	\]
	Recalling the definition the approximate Green's function
	\[
		G_L^\varepsilon(x) 
			= \frac{1}{2\pi} \int \frac{e^{ix\xi}}{p(\xi-i\varepsilon)} \, \mathrm{d}\xi
	\]
	from Equation \ref{eq4:approxGF} in the proof of Lemma \ref{lma4:DEtoIE1}, 
	we note that
	\begin{equation}
		\label{GL.eps}
		 G_L^\varepsilon(x) 
		 	= 
		 		i \left(\frac{2}{3} + i x \right)
		 		e^{-\varepsilon x} \chi_L(x) 
		 		+ e^{-\pi|x|} k(x),
	\end{equation}
	where $k(x)$ is as defined in Remark \ref{rmk1:littlek}.

	Thus the inverse Fourier transform of the second right-hand term in 
	\eqref{Jost.half.pre-int} is given by
	\begin{align*}
	\mathcal F^{-1} \left( \frac{\widehat{uM}}{p(\xi-i\varepsilon)} \right)(x)
		&=
			i
			\int_{-\infty}^x 
				\left( \frac{2}{3} + i (x-x') \right) 
				e^{-\varepsilon(x-x')} u(x') M(x') \, \mathrm{d}x' \\
		&\quad + \int_{\mathbb R} e^{-\pi|x-x'|} k(x-x')  u(x') M(x') \, \mathrm{d}x'.
	\end{align*}
	It follows by dominated convergence that this expression approaches
	$G_L*(uM)(x)$ pointwise as $\varepsilon \searrow 0$ as 
	$u\in L^{2,4}(\mathbb R)$ implies 
	$u \in L^{1,1}(\mathbb R) \cap L^2(\mathbb R)$. 

	It remains to show that the first term vanishes pointwise as 
	$\varepsilon \searrow 0$. We write
	\begin{align*}
		\mathcal F^{-1} 
				\left( 
					\frac{p(\xi-i\varepsilon) - p(\xi)}
						{p(\xi-i\varepsilon)} \widehat{M^+-1} 
				\right)
			&=	i\varepsilon G_L^\varepsilon * (M^+ -1) \\
			&\quad- 
				\frac{1}{2} \left(e^{2i\varepsilon}-1 \right) 
				G_L^\varepsilon*(M^- - 1)
	\end{align*}
	where we used $\widehat{M^-} = e^{-2\xi} \widehat{M^+}$. The goal is to use
	the asymptotic behavior of $M^+$ and $M^-$ as $x \to -\infty$ to show that 
	these terms vanish as $\varepsilon \searrow 0$. Due to the linear growth 
	of the Green's function we need a more stringent rate of decay for 
	$M^+-1$ and $M^- - 1$ as $x \to -\infty$ to control convolution with the pole 
	term in $G_L^\varepsilon$.  

	First we consider 
	\begin{align*}
		i\varepsilon \, G_L^\varepsilon*(M^+ - 1) (x) 
			&=	
				-\varepsilon 
				\int_{-\infty}^x 
					\left( \frac{2}{3} + i(x-x') \right) 
					e^{-\varepsilon x} \left(M^+(x') - 1\right) 
				\, \mathrm{d}x' \\
			&\quad 	
				+ i\varepsilon \int e^{-\pi|x-x'|} k(x-x') \big(M^+(x')-1\big) \, \mathrm{d}x'
	\end{align*}
	We use equation \eqref{GL.eps} and asymptotic condition (a) from 
	property \ref{itm:asymp} of Definition \ref{dfn4:DEsoln}.
	The second right-hand integral is bounded by $\varepsilon$ times
	\[
		\int_{\mathbb R} \inn{x-x'}^{-2} |k(x-x')| \inn{x'} \, \mathrm{d}x' 
			\lesssim
				\inn{x} 
				\int_{\mathbb R} 
					\inn{x-x'}^{-1} |k(x-x')| \,dx'
			\lesssim \inn{x} \nm{k}_{L^2}
	\]
	and so goes to zero pointwise as $\varepsilon \searrow 0$. 
	Let $H(x) = \inn{x}\big(M^+(x)-1\big)$. The first right-hand integral is 
	bounded by 
	\begin{align*}
	\varepsilon 
		\int_{-\infty}^x 
			\inn{x-x'} e^{-\varepsilon (x-x')} \inn{x'}^{-1}   
			\left|H(x') \right| 
		\, \mathrm{d}x'  
		&\lesssim 
			\varepsilon 
			\int_{-\infty}^x 
				e^{-\varepsilon (x-x')} \left| H(x') \right| 
			\, \mathrm{d}x'
			\\
		&=	\int_0^{\infty} e^{-\Xi} \left| H(x-\Xi/\varepsilon) \right| \, \mathrm{d}\Xi
	\end{align*}
	which goes to $0$ as $\varepsilon \searrow 0$ by dominated convergence since
	$\lim_{x \to -\infty} H(x) = 0$, where we used the substitution 
	$\Xi = \varepsilon(x-x')$ in the above integral.

	We seek to carry out an analogous estimate for the term involving $M^-$.
	We will use equation \eqref{GL.eps} and asymptotic condition (b) from 
	property \ref{itm:asymp} of Definition \ref{dfn4:DEsoln}. Since 
	$e^{2i\varepsilon}-1$ is of order $\varepsilon$, it suffices to show that
	$\varepsilon \left| G_L^\varepsilon*(M^- -1) (x) \right| = o(1)$
	as $\varepsilon \searrow 0$, where we use the ``little oh'' notation
	$f = o(g)$ to indicate that $\lim{y\to a} \frac{f(y)}{g(y)} = 0$ (in this 
	case $y=\varepsilon$ and $a = 0$).
	We have
	\begin{align*}
		\varepsilon \left| (G_L^\varepsilon * (M_1))(x) \right|
			&\lesssim	
				\varepsilon 
				\int_{-\infty}^x 
					\inn{x-x'} e^{-\varepsilon x'} \inn{x'}^{-1-\upsilon} 
				\, \mathrm{d}x'
				\\
			&\lesssim  
				\int_{-\infty}^x 
					\varepsilon e^{-\varepsilon(x-x')} \inn{x'}^{-\upsilon} 
				\, \mathrm{d}x'	
				\\
			&=	
				\int_0{\infty}
					e^{-\Xi} \inn{x-\Xi/\varepsilon}^{-\upsilon} 
				\, \mathrm{d}\Xi
	\end{align*}
	which goes to zero as $\varepsilon \searrow 0$ by dominated convergence.
	We leave the second term, involving $k(x-x')$, as an exercise to the 
	reader.

	Finally
	\begin{align*}
		\varepsilon \left| (G_L^\varepsilon (M_2))(x) \right|
			&\lesssim 
				\varepsilon 
				\int_{-\infty}^x 
					e^{-\varepsilon (x-x')} \inn{x-x'} 
					\inn{x'}^{-1-\upsilon} g(x') 
				\, \mathrm{d}x',
		\end{align*}
	where $g \in L^2$. By the Cauchy-Schwarz inequality and the 
	fact that $\inn{x-x'} \inn{x'}^{-1}$ is bounded for $x' < x < 0$ we again 
	get an $\varepsilon^{\frac{1}{2}}$ estimate which suffices for the purpose. 

	We conclude that, for $u \in L^{1,2+\upsilon} \cap L^{2,2} \supset X$, 
	the Jost solution $M$ the satisfying asymptotic conditions \ref{itm:asymp}
	from Definition \ref{dfn4:DEsoln} solves the corresponding integral equation.
\end{proof}


\end{document}