%%===================================================================
%% Subsection 4.01.02: Integral Equation Solutions are Jost Solutions
%%===================================================================

\documentclass[../dissertation.tex]{subfiles}

\begin{document}
\subsection{Integral Equation Solutions are Jost Solutions}\label{subsec4:IEtoDE}

We continue in this subsection with our discussion of the equivalence of 
the linear spectral problem with prescribed asymptotics its corresponding
integral equation reformulation by turning our attention to showing 
that the analytic continuation $M$ to the strip $\mathcal S_1$ of a 
function $M^+$ satisfying 
Definition \ref{dfn4:IEsoln} satisfies Definition \ref{dfn4:DEsoln}.

From Proposition \ref{prop2:Tasymp} and our work in Chapter \ref{cptr03:xContin},
we already know that any function $M^+$ satisfying Definition \ref{dfn4:IEsoln}
has an analytic continuation to the complex strip $\mathcal S_1$ which
satisfies part (a) of property \ref{itm:asymp}, and property \ref{itm:space} 
and property \ref{itm:cont} of Definition \ref{dfn4:DEsoln}. As such, our task in this 
section is to first show that 
the boundary value $M^-$ satisfies property \ref{itm:asymp}, which
we do in Lemma \ref{lma4:upperasymp}. We then show in Lemma \ref{lma4:IEtoDE4}
$M$ has decomposition satisfying property \ref{itm:decomp}, and prove 
in \ref{lma4:IEtoDE5} that $M$ satisfies property \ref{itm:jost}\textemdash{}that is,
$M$ weakly solves \eqref{eq4:LinSpecProb}.


\begin{lma}\label{lma4:upperasymp}
	Suppose then $M^+$ satisfies Definition \ref{dfn4:IEsoln} and denote by $M$
	the analytic continuation of $M^+$ to the complex strip $\mathcal S_1$. The 
	upper boundary value $M^-$ of $M$ satisfies asymptotic condition (b) of 
	property \ref{itm:asymp} from Definition \ref{dfn4:DEsoln}. That is, there 
	exist $M_1$, $M_2$ so that 
	\[
		M^-(x) -1 = M_1(x) + M_2(x), 
	\]
	where 
	\[
		\inn{x}^{1+\upsilon} | M_1(x) | \lesssim 1
	\]
	as $x\to -\infty$ and 
	\[
		\inn{\dotarg}^\tau M_2 \in L^2(\mathbb R)
	\]
	for any $\upsilon \in (0,1]$ and $\tau \in [0,1)$.
\end{lma}
\begin{proof}
	Recall from Equation \eqref{eq3:GLminus} of Theorem \ref{eq3:GstarMinus} in 
	Section \ref{sec3:Intro} that 
	\begin{align*}
		M^-(x) - 1
			&= G_L(\dotarg; \lambda)^- * \big[u\,M^+\big](x) \\
			&= 
				\bigg[
					\mathfrak C(\dotarg, 2) + \mathpzc R_L(\dotarg+i2; \lambda)
				\bigg] 
				* 
				\big[u\,M^+\big](x)
				+
				\left[E - \frac{1}{2}\right]\big[u\,M^+\big](x)
	\end{align*}
	Define
	\begin{align*}
		I_1(x;\lambda) &:= \mathfrak C(\dotarg, 2) * \big[u\,M^+\big](x) \\
		I_2(x;\lambda) &:= \mathpzc R_L(\dotarg+i2; \lambda) * \big[u\,M^+\big](x) \\
		I_3(x;\lambda) &:= \left[E - \frac{1}{2}\right]\big[u\,M^+\big](x),
	\end{align*}
	and set
	\begin{align*}
		M_1(x;\lambda) :=  I_1(x;\lambda) + I_2(x;\lambda),
			\qquad\text{and}\qquad
		M_2(x;\lambda) := \left[E - \frac{1}{2}\right]\big[u\,M^+\big](x).
	\end{align*}
	
	A result of Fefferman-Stein from the theory of Calder\'on-Zygmund operators
	with Muckenhaupt $A_p$ weights holds that Calder\'on-Zygmund operators are 
	bounded between $L^p(\omega \,dx)$ if $1 < p < \infty $ and $\omega \in A_p$.
	Since $|\dotarg|^\iota$ is an $A_2$-weight in $\mathbb R^1$ for any 
	$\iota\in [0, 1)$ and the exponentially weighted Hilbert trasnform is a 
	Calder\'on-Zygmund operator, we have that
	\[
		\int_{\mathbb R} (1+|x|)^{2\iota} \left|(Ef)(x)\right|^2 \, \mathrm{d}x
			\lesssim_{\iota} 
				\int_{\mathbb R}(1+|x|)^{2\iota} |f(x)|^2 \, \mathrm{d}x
	\]
	for any $\iota \in [0, 1)$. Hence, since $u\in L^{2,4}(\mathbb R)$ implies 
	$u\,M^+ \in L^{2,\tau}(\mathbb R)$ for $\tau \in [0, 3]$, we conclude
	that $E(uM^+) \in L^{2, \tau}(\mathbb R)$ and thus $M_2 \in L^{2,\tau}(\mathbb R)$
	for $\tau \in [0, 1)$.

	Since 
	\[
		\mathfrak C(x, 2)
			= \frac{e^{-\pi|x|}}{2\pi}	\
				\int_{\mathbb R} e^{ix\xi} \rho\big(\xi, 2, \sign(x)\big) \, \mathrm{d}x,
	\]
	where $\rho$ is absolutely integrable in $\xi$, it is easy to see that 
	$\mathfrak C(\dotarg, 2)$ is bounded by a Schwartz class function. As such, 
	a simple application of Dominated Convergence implies 
	$\inn{x}^{1+\upsilon} I_1(x; \lambda) \to 0$ as $x\to -\infty$ for each 
	$\lambda \in \mathbb R$. 

	For $I_2$, we note that $\mathpzc R_L$ grows linearly only when $\lambda=0$
	and is otherwise bounded. As such, we verify that 
	$\inn{x}^{1+\upsilon} I_2(x; \lambda =0)$ is bounded for $\upsilon \in (0,1]$
	and $x < 0$ and note that the corresponding result for $\lambda \ne 0$ follows.
	Now
	\begin{align*}
		I_2(x; \lambda=0) 
			= 
				\int_{-\infty}^x 
					\left(i\frac{2}{3} - (x - x')\right) u(x')\,M^+(x')
				\,dx.
	\end{align*}
	Assuming $x < 0$, then $\inn{x'} \geq \inn{x}$, $\inn{x}^{-1} \geq \inn{x'}^{-1}$, and
	\[
		\inn{x-x'} \leq \inn{x} + \inn{x'} \lesssim \inn{x'}
	\]
	whenever $x' \leq x$.
	% \lsidenote{This is the tightest bound on $I_2$ I've been able to find. 
	% Not sure if it is enough to get $\inn{x}^{1+\upsilon}I_2$ bounded.}
	Consequently
	\begin{align*}
		|I_2(x; \lambda = 0)|
			&\lesssim \int_{-\infty}^x \inn{x - x'} |u(x')| \left|M^+(x')\right| \, \mathrm{d}x' 
				\\
			&\lesssim 
				\inn{x}^{-1-\upsilon} 
				\int_{-\infty}^x 
					\inn{x'}^{2+\upsilon} |u(x')| \inn{x'} 
					\left(\inn{x'}^{-1} \left|M^+(x')\right| \right)
				\, \mathrm{d}x'
				\\
			&\lesssim 
				\inn{x}^{-1-\upsilon} 
				\nm{M^+}_{\inn{\dotarg}L^\infty} \
				\nm{u}_{\inn{\dotarg}^{3+\upsilon}L^1},
	\end{align*}
	where $\nm{u}_{\inn{\dotarg}^{3+\upsilon}L^1}<\infty$ as 
	\begin{align*}
		\nm{u}_{\inn{\dotarg}^{3+\upsilon}L^1}
			= 
				\left| 
					\int_{\mathbb R} 
						\inn{x}^{-1+\upsilon}\inn{x}^{4} u(x) 
					\, \mathrm{d}x
				\right|
			\leq 
				\nm{\inn{\dotarg}^{-1+\upsilon}}_{L^2} 
				\nm{\inn{\dotarg}^{4} u}_{L^2},
	\end{align*}
	and $-1+\upsilon < -\frac{1}{2}$ implies 
	$\inn{\dotarg}^{-1+\upsilon} \in L^2(\mathbb R)$. It therefore follows that 
	$\inn{x}^{1+\upsilon} |M_1(x)| \lesssim 1$ for $x < 0$. 
\end{proof}


\begin{lma}\label{lma4:IEtoDE4}
	Suppose then $M^+$ satisfies Definition \ref{dfn4:IEsoln}. Then 
	its analytic continuation $M$ to the complex strip $\mathcal S_1$
	satisfies property \ref{itm:decomp} of Definition \ref{dfn4:DEsoln}.
	That is, there is a decomposition $M(z) = M_c(z) + M_s(z)$ for 
	$0 < \im  z < 2$ so that 
	\begin{itemize}
		\item[(a)] $M_c$ extends to a continuous function on the closure
			$\ol{\mathcal S}_1$ of $\mathcal S_1$ with 
			\[
				\mathop{\lim_{x\to-\infty}}_{x\in \mathbb R} M_c(x+2i) = 1	
			\]
		\item[(b)] The estimates
			\[
				\|M_s(\dotarg+iy)\|_{L^\infty} \leq (2-y)^{-1/2}, \qquad
				\sup_{0\leq y < 2} \|M_s(\dotarg+iy)\|_{L^2} < \infty
			\]
			hold. Moreover, $M_s$ has an $L^2$ boundary value 
			$M_s^-(x):= \lim_{\varepsilon \searrow 0} M_s\big(\dotarg+i(2-\varepsilon)\big)$
			on $\im z = 2$ with $M_s(x+iy) \to M_s^-(x)$ for almost every
			$x$.
	\end{itemize}
\end{lma}
\begin{proof}
	Recalling the discussion following Theorem \ref{thm3:main_result}, 
	we note that 
	\begin{align*}
		M(z) = M_c(z) + M_s(z),
	\end{align*}
	where 
	\begin{align*}
		M_c(x+iy) 
			:= 1 
				+ 
				\big[
					\mathfrak C(\dotarg, y) + \mathpzc R_L(\dotarg+iy; \lambda)
				\big]
				*
				\big[u\,M^+\big](x)
	\end{align*}
	and
	\begin{align*}
		M_s(x+iy)
			:= \mathcal E_{2-y} \big[u\,M^+\big](x).
	\end{align*}
	That $M_c(x+i2)$ is continous for $x \in \mathbb R$ follows from our work in 
	Sections \ref{sec3:Analyticity} and \ref{sec3:ContPart}. Further,
	since $\mathfrak C(x, 2)$ is bounded by Schwartz class function, it is easy 
	to show through direct computation that 
	$\mathfrak C(\dotarg, 2) * [u\,M^+](x) \to 0$ as $|x| \to \infty$.
	By following the proof of Proposition \ref{prop2:Tasymp}, one can 
	also easily show that $\mathpzc R_L(\dotarg + i2; \lambda) * [u\,M^+] \to 0$
	as $x \to -\infty$ for every real $\lambda$, which implies $M_c(x+i2) \to 1$ as 
	$x \to - \infty$. Lastly, the convergence of $M_r(x+iy)$ to $M_r^-(x)$ and the 
	estimates on $\|M_s(\dotarg+iy)\|_{L^\infty}$ and 
	$\sup_{0\leq y < 2} \|M_s(\dotarg+iy)\|_{L^2}$ are all consequences of our 
	analyses in Sections \ref{sec3:CauchyTrans} and \ref{sec3:BndE}.
\end{proof}


\begin{lma}\label{lma4:IEtoDE5}
	The analytic continuation $M$ of 
	a solution $M^+$ of integral equation form of the linear spectral problem 
	as defined by Definition \ref{dfn4:IEsoln} also satisfies property \ref{itm:jost}
	of Definition \ref{dfn4:DEsoln} and is therefore a Jost solution.
\end{lma}
\begin{proof}
	%%====================================================================
	%% Begin direct thievery (with modification) from Prof. Perry's notes.
	%% Rewrite this section later.
	%%====================================================================
	To prove that the solution $M$ of \eqref{dfn4:IEsoln} solves 
	\eqref{eq4:LinSpecProb} in the sense of Definition \ref{dfn4:DEsoln} we 
	first consider the case $u \in C_0^\infty(\mathbb R)$. It is not difficult to 
	see that if $u \in C_0^\infty(\mathbb R)$, $M$ is also $C^\infty$. By Laurent 
	Schwartz's formulation of the Paley-Wiener Theorem \cite{Schwartz1952}, the 
	function $\widehat{u\,M^+}$ is entire and rapidly decaying in $\xi$ for 
	$\im \xi$ bounded. Thus we may compute 
	\[ 
		M^+(x) - 1 
			= \int_{\Gamma_L} \frac{e^{i\xi x}}{p(\xi)} \wh{u\,M^+}(\xi) \, \mathrm{d}\xi
	\]
	where the right-hand side makes sense owing to the analyticity of $\wh{uM}$. 
	We also have
	\[
		M(x+iy) - 1 
			= 
				\int_{\Gamma_L} 
					\frac{e^{i\xi x} e^{-y\xi}}{p(\xi)} \wh{u\,M^+}(\xi) 
				\, \mathrm{d}\xi
	\]
	from which it follows that
	\[ 
		\frac{1}{i} \frac{\partial M^+}{\partial x}(x)  - \zeta(M^+(x) - M^-(x))
			=	\int_{\Gamma_L} e^{i\xi x} \wh{u\,M^+}(\xi) \, \mathrm{d}\xi 
			= u(x) M^+(x)
	\]
	for each $x$, where we used analyticity of $\widehat{u\,M^+}$ to deform the 
	contour from $\Gamma_L$ to $\mathbb R$. 

	Now let $u \in X$ and let $\{ u_n \}$ be a sequence from $C_0^\infty(\mathbb R)$ 
	with $u_n \to u$ in $X$. Let $M_n$ be the corresponding solution of 
	\eqref{eq4:MPluseq} for $u=u_n$. For any $\varphi \in C_0^\infty(\mathbb R)$ we have
	\begin{align}\label{eq4:Mnweak}
		\frac{1}{i} \inn{\varphi', \, M^+_n} - \zeta \inn{\varphi,\,M_n^+ - M_n^-}  
			- \inn{\varphi, \, u_n M^+_n} = 0.
	\end{align}
	As $u_n \to u$ in $X$ it follows that $M_n \to M$ in $L^\infty$, hence 
	$u_n M_n \to uM$ in $L^{1,2}$. To show that $M$ is a weak solution, 
	that is, $M$ satisfies \eqref{eq4:Mnweak}, it
	suffices to show that for any $\varphi \in C_0^\infty(\mathbb R)$,  the 
	differences
	\[
		\inn{\varphi', \, M^+-M^+_n}, \quad
		\inn{\varphi, \, (M^\pm - M_n^\pm)}, \quad
		\inn{\varphi, \, u\,M^+ - u_n\,M_n^+}
	\]
	all converge to zero as $n \to \infty$. 
	Note that 
	\begin{align*}
		\left|\inn{\varphi', \, M^+-M^+_n}\right|
			&= \left|\int_{\mathbb R} \phi'\big(M^+ - M_n^+\big) \, \mathrm{d}x \right| \\
			&\leq \nm{M^+ - M_n^+}_{\inn{\dotarg}L^\infty} \nm{\inn{\dotarg}\phi'}_{L^1}.
	\end{align*}
	Since $\phi' \in C_0^\infty$ implies $\nm{\inn{\dotarg}\phi'}_{L^1}$ and the map 
	$B_X(0, c_0) \ni u \mapsto M\in\inn{\dotarg}L^\infty(\mathbb R)$ is Lipschitz
	by Lemma \ref{lma4:cont}, it follows that
	\[
		\lim_{n\to\infty} \inn{\varphi', \, M^+-M^+_n} = 0.
	\]
	A similar argument also shows that 
	\[
		\lim_{n\to\infty} \inn{\varphi, \, (M^+ - M_n^+)}= 0
	\]
	The Lipschitz continuity of the map 
	$B_X(0, c_0) \ni u \mapsto u\,M \in L^1(\mathbb R)$ (Lemma \ref{prop4:uMcont}) 
	in conjunction with the fact that $\phi$ is bounded again implies
	\[
		\lim_{n\to\infty} \inn{\varphi, \, u\,M^+ - u_n\,M_n^+} = 0.
	\]

	To prove $\inn{\varphi, \, (M^- - M_n^-)} \to 0$ as $n \to \infty$, we 
	recall that

	\[
		\big[M^-(x;\lambda)- M_n^-(x;\lambda)\big]
			= \left[M^-_c(x;\lambda) - \left( M_n^- \right)_c(x;\lambda)\right] 
				+ \left[M^-_s(x;\lambda) - \left( M_n^- \right)_s(x;\lambda)\right]
	\]
	where
	\begin{align}
		 M^-_c(x;\lambda) - \left( M_n^- \right)_c(x;\lambda)
			= 
				\big[ 
					\mathfrak C(\dotarg, 2) + \mathpzc R_L(\dotarg+i2, \lambda)
				\big] 
				*
				\big[u\,M^+ - u_n\,M_n^+](x)
	\end{align}
	and
	\begin{align}\label{ILW.Mnr}
		 &M^-_s(x;\lambda) - \left( M_n^- \right)_s(x;\lambda) \\
			&\qquad= 
				E\big[u\,M^+(\dotarg;\lambda)- u_n\,M_n^+(\dotarg;\lambda)\big](x)
				- \frac{1}{2} \big[u\,M^+(x;\lambda)- u_n\,M_n^+(x;\lambda)\big]
	\end{align}
	Hence, using the fact that $\phi \in C_0^\infty(\mathbb R)$ implies 
	$\inn{\dotarg} \phi \in L^1$ and estimate \eqref{eq3:tlma2} from 
	Technical Lemma \ref{tlma2:1} we find for all $\lambda \in \mathbb R$
	\begin{align*}
		\inn{\phi, \, M^-_c - \left( M_n^- \right)_c }
			&\leq \nm{\inn{\dotarg} \phi }_{L^1} 
				\nm{M^-_c - \left( M_n^- \right)_c}_{\inn{\dotarg} L^\infty} 
				\\
			&\leq \nm{\inn{\dotarg} \phi }_{L^1}
				\nm{M^+ - M_n^+}_{L^{1,1}} 
				\nm{\mathfrak C(\dotarg, 2) + \mathpzc R_L(\dotarg + i2)}_{\inn{\dotarg}L^\infty}
	\end{align*}
	Again appealing to Lemma \ref{prop4:uMcont} we conclude,  
	\begin{align*}
		\lim_{n\to\infty} \inn{\phi, \, M^-_c - \left( M_n^- \right)_c } = 0.
	\end{align*}
	Lastly, to prove
	\begin{align*}
		\lim_{n\to\infty} \inn{\phi, \, M^-_s - \left( M_n^- \right)_s}
			= 0,
	\end{align*}
	we can use the $\inn{\dotarg} L^\infty$ convergence of $M_n^+$ to $M^+$, the $X$-convergence 
	of $u_n$ to $u$, and dominated convergence to show that the right-hand side of 
	\eqref{ILW.Mnr} goes to zero in $L^2$ as $n \to \infty$, and, by passing to a 
	subsequence, goes to zero almost everywhere. 
\end{proof}




\label{lastpagePenultimateSection}



\end{document}