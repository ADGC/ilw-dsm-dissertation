%%========================================
%% Section 4.3: The Direct Scattering Map
%%========================================


\documentclass[../dissertation.tex]{subfiles}

\begin{document}
\section{The Direct Scattering Map}\label{sec4:DM}

We now have the tools we need to prove that the direct scattering map is well-defined
as a map from $B_X(0, c_0)$ to $L^\infty(\mathbb R)$ and Lipschitz continuous 
as a map from $B_X(0, c_0)$ to $L^\infty\big(\mathbb R \sm (-k,k)\mathbb)$ 
(for any $k > 0$). As a warm-up exercise, we first verify in
Proposition \ref{prop4:SD} the validity of the scattering equations first 
presented (without proof) in \cite{Kodama1982}, which are key to the 
construction of the inverse scattering map for the ILW. Following Proposition 
\ref{prop4:SD}, we turn our attention to proving that the direct scattering map 
is well-defined (Lemma \ref{lma4:rBnd} and Theorem \ref{thm4:Dwelldefined}), and
wrap up this dissertation with results on the Lipschitz continuity of the direct
scattering map (Theorem \ref{prop4:exist} and Corollary \ref{cor4:Lip}).


\begin{prop}[Scattering Equations]\label{prop4:SD}
	Suppose that $u$ satisfies the hypotheses of Proposition \ref{prop4:exist}.
	Let
	\begin{subequations}
		\label{eq4:SD}
		\begin{align}
			% \label{eq4:SDa}
			a(\lambda)
				&:= 1 + i \alpha(\lambda) \,
					\int_{\mathbb R} u(x) \, M_1^+(x; \lambda, u) \, \mathrm{d}x \\
			% \label{eq4:SDb}
			b(\lambda)
				&= i \beta(\lambda) \, 
					\int_{\mathbb R} e^{-ix\lambda} \, u(x) \, M_1^+(x; \lambda, u) \, \mathrm{d}x
		\end{align}
	\end{subequations}
	\vspace{-\baselineskip}
	\begin{subequations}
		\label{eq4:SDbrev}
		\begin{align}
			% \label{eq4:SDa}
			\breve{a}(\lambda)
				&:= 1 + \alpha(\lambda)
					\int_{\mathbb R} u(x) \, N_1(x; \lambda, u) \, \mathrm{d}x \\
			% \label{eq4:SDb}
			\breve{b}(\lambda)
				&= i \beta(\lambda) 
					\int_{\mathbb R} e^{-ix\lambda} \, u(x) \, N_1(x; \lambda, u) \, \mathrm{d}x
		\end{align}
	\end{subequations}
	For $\lambda \in \mathbb R \sm \{0\}$, 
	\begin{align}
		M_1 (x; \lambda)
			&= a(\lambda) \, N_1(x; \lambda) + b(\lambda) \, N_e(x; \lambda) 
			\label{eq3:M1N} \\
		N_1 (x; \lambda)
			&= \breve{a}(\lambda) \, M_1(x; \lambda) 
				+ \breve{b}(\lambda) \, M_e(x; \lambda)
			\label{eq3:N1M}
	\end{align}
\end{prop}
\begin{proof}
	An immediate consequence of formulas \eqref{eq1:GFrep} is the jump 
	relation\footnote{This jump relation can also be proven by taking a contour 
	around the real axis. For details, please see 
	\hyperref[app:JumpRelation]{Appendix 3: Jump Relation}.}
	\begin{align}\label{eq4:GFjump}
		G_L^+ - G_R^+ = i \alpha(\lambda) + i \beta(\lambda) e^{i \lambda x}.
	\end{align}
	Under the jump relation \eqref{eq4:GFjump}, the integral equation 
	\eqref{eq4:JostIEleft} for $M_1^+$ becomes
	\begin{align}\label{eq4:altM1}
		M_1^+(x; \lambda) 
			&= 1 + \int_{\mathbb R} G_L^+(x-x'; \lambda) u(x') M_1^+(x')\,dx' \\
			&= 1 
				+ \int_{\mathbb R} 
					\big(
						i \alpha(\lambda) 
						+ i \beta(\lambda) \, e^{i \lambda (x-x')}
					\big)
					u(x') M_1^+(x') 
				\, \mathrm{d}x \nonumber \\
			&= \left(
					1 
					+ i \alpha(\lambda) 
						\int_{\mathbb R} u(x')M_1^+(x'; \lambda)\,dx 
				\right)  
				\nonumber \\
			&\qquad+ i \beta(\lambda) \, e^{i\lambda x} 
					\int_{\mathbb R} e^{-i\lambda x'} (x')M_1^+(x'; \lambda)\,dx 
				\nonumber \\
			&\qquad+ \int_{\mathbb R} G_R^+(x-x'; \lambda) u(x') M_1^+(x')\,dx'
				\nonumber \\
			&= a(\lambda) + b(\lambda) \,e^{i\lambda x} 
				+ G_R^+(\dotarg; \lambda) * \big[uM_1^+(\dotarg; \lambda)\big].
				\nonumber
	\end{align}
	Recalling from Proposition \ref{prop2:Tasymp} that 
	$\lim_{x\to\infty} G_R^+*uM_1^+(x) = 0$, we see that $M_1^+$ satisfies the 
	asymptotic condition
	\begin{align} \label{eq4:asympM}
		\lim_{x\to+\infty} 
			\big|M_1^+(x; \lambda) - a(\lambda) - b(\lambda)e^{i\lambda x}\big|
			= 0.
	\end{align}

	The simple computation
	\begin{align*}
		G_R^+ * \big[ u(a\,N_1 + b\, N_e)\big]
			&= a \, G_R^+ *(u\, N_1) + b \, G_R^+ *(u\, N_e) \\
			&= a(N_1 - 1) + b(N_e - e^{i\lambda x}) \\
			&= a\, N_1 + b \, N_e - a - b\, e^{i\lambda x}
	\end{align*}
	shows that $a\, N_1 + b \, N_e$ is a solution to \eqref{eq4:altM1}. Further, 
	since $a\, N_1 + b \, N_e$ also satisfies \eqref{eq4:asympM} as
	$\lim_{x\to+\infty} |N_1-1| = \lim_{x\to+\infty} |N_e - e^{ix\lambda}| = 0$,
	equation \eqref{eq3:M1N} follows from the uniqueness of Jost solutions. 
	Equation \eqref{eq3:N1M} is verified analogously.
\end{proof}

While perhaps not immediately apparent, the significance of the following lemma, 
Lemma \ref{lma4:rBnd}, is that it allows us to conclude 
that the reflection coefficient $r(\lambda) = b(\lambda)/a(\lambda)$ is bounded 
in $\lambda.$ That is, 
$r \in L_\lambda^\infty(\mathbb R)$, which is the final piece we need to prove 
that the ILW direct scattering map \label{sym:DM}
$\mathscr D: B_X(0, c_0) \ni u \mapsto r \in L_\lambda^\infty(\mathbb R)$ is 
well-defined. 

%%==============
%% Allen's Lemma 
%%==============
\begin{lma}\label{lma4:rBnd}
	For $\lambda \ne 0$, the functions $a$ and $b$ defined in Proposition \ref{prop4:SD} satisfy
	the equation 
	\begin{align}\label{eq4:rbnd}
		\big| a(\lambda) \big|^2
			= 1 + \frac{2\zeta(-\lambda) - 1}{1 - 2 \zeta(\lambda)} 
				\big| b(\lambda) \big|^2.
	\end{align}
\end{lma}
The following proof is taken from the unpublished notes of Professor Allen Wu.
% {\color{red} I thank Prof. Allen Wu from the University of Oklahoma for the following clever 
% proof of Lemma \ref{lma4:rBnd}.}
\begin{proof}
	In this proof, we use the identity 
	\begin{align*}
		\inn{G_L^+(\dotarg; \lambda)*f, \, g}
			&= \inn{f, \, G_L^+(\dotarg; \lambda)*g} \\
			&\quad+ i\alpha(\lambda) \inn{f,1}\inn{1,g}
				+ i\beta(\lambda) \inn{f, \, e^{i(\dotarg)\lambda}}
					\inn{e^{i(\dotarg)\lambda}, \, g}
	\end{align*}
	which follows from Proposition \ref{prop1:DiaConj} identity (ii) and
	the jump relation \eqref{eq4:GFjump} as 
	\begin{align*}
		\ol{G_L^+(x; \lambda)}
			= G_R^+(-x; \lambda)
			= G_L^+(-x; \lambda) - i \alpha(\lambda) - i \beta(\lambda) \, e^{-i\lambda x}.
	\end{align*}

	Using $M_1^+ = 1 + G_L^+ (\dotarg; \lambda) * u M_1$ we compute
	\begin{align*}
		\inn{M_1^+, \, uM_1^+}
			&= \inn{1 + G_L^+ (\dotarg; \lambda) * u M_1^+, \, uM_1^+} \\
			&= \inn{1, \, uM_1^+} + \inn{uM_1^+, \, G_L^+ (\dotarg; \lambda) * u M_1^+} \\
			&\quad+ i \alpha(\lambda) |\inn{uM_1^+,\,1}|^2 
				+ i \beta(\lambda) |\inn{uM_1^+,\,e^{i(\dotarg)\lambda}}|^2 \\
			&=  \inn{1, \, uM_1^+} + \inn{uM_1^+, \, M_1^+ - 1} \\
			&\quad+ i \alpha(\lambda) |\inn{uM_1^+,\,1}|^2 
				+ i \beta(\lambda) |\inn{uM_1^+,\,e^{i(\dotarg)\lambda}}|^2
	\end{align*}
	Since $u$ is real, we have $\inn{M_1^+, \, u M_1^+} = \inn{uM_1^+, \, M_1^+}$ and
	\begin{align*}
		0 = 
			\ol{\inn{uM_1^+, \, 1}} - \inn{uM_1^+, \, 1}
			+ i \alpha(\lambda) \left|\inn{uM_1^+, \, 1} \right|^2
			+ i \beta(\lambda) \left|\inn{uM_1^+, \, e^{i(\dotarg)\lambda}} \right|^2.
	\end{align*}
	Identity \ref{eq4:rbnd} then follows, as 
	$\inn{uM_1^+, \,1} = \frac{1}{i} \frac{a-1}{\alpha(\lambda)}$, 
	$\inn{uM_1^+, \, e^{i(\dotarg)\lambda}} = \frac{1}{i} \frac{b}{\beta(\lambda)}$, 
	$\alpha(\lambda) = \frac{1}{1-2\zeta(\lambda)}$, and $\beta(\lambda) = \frac{1}{1-2\zeta(-\lambda)}$.
\end{proof}

While Lemma \ref{lma4:rBnd} holds only for $\lambda \ne 0$, we can nonetheless use 
Lemma \ref{lma4:rBnd} to show that the $r$ remains bounded near $\lambda = 0$ and is 
therefore at least essentially bounded and hence in $L_\lambda^\infty$. 

% In proving that $r$ remains essentially bounded in a neighborhood of $\lambda = 0$,
% we use the property that $M_1^+$ has a linear expansion centered at $\lambda =0$, 
% as stated in Lemma \ref{lma4:Mtexpansion}.




\begin{thm}\label{thm4:Dwelldefined}
	Let $r(\lambda) := b(\lambda) / a(\lambda)$. The direct scattering map
	$\mathscr D$ given by 
	\begin{align*}
		\begin{array}{rcl}
			\mathscr D: B_X(0, c_0) &\to& L_\lambda^\infty(\mathbb R)  \\
			            u           &\mapsto& r
		\end{array}
	\end{align*}
	is well-defined.
\end{thm}
\begin{proof}
	Since $M_1^+$ and $N_1^+$ exist and are unique for each $u \in B_X(0, c_0)$, the 
	map $u \mapsto r$ is well-defined as a function. Moreover, it is easy to check 
	that $\frac{1-2\zeta(\lambda)}{2\zeta(-\lambda)-1}$ is both positive and uniformly
	bounded in $\lambda$ for all real $\lambda \ne 0$, which means that \eqref{eq4:rbnd} and Lemma 
	\ref{lma4:rBnd} implies both that $|a(\lambda)| \geq 1$ for all 
	$\lambda \ne 0$ and, as a consequence
	\[
		\left| \frac{b(\lambda)}{a(\lambda)} \right|^2
			= 
				\frac{1-2\zeta(\lambda)}{2\zeta(-\lambda)-1} 
				\left[
					1 - \left|\frac{1}{a(\lambda)}\right|^2 
				\right].
	\]
	Since $|a(\lambda)| \to \infty$ as $\lambda \to 0$ and a simple computation shows that
	\[
		\lim_{\lambda \to 0} \frac{1-2\zeta(\lambda)}{2\zeta(-\lambda)-1} 
			= 1,
	\]
	$r = b/a$ is at least essentially bounded near $\lambda = 0$. Moreover, 
	since
	\begin{align}
		\label{eq4:rNegLamLimit}
		\lim_{\lambda\to-\infty}\frac{1-2\zeta(\lambda)}{2\zeta(-\lambda)-1}
			= 0,
	\end{align}
	we need only prove $r(\lambda)$ stays bounded for large \textit{positive} 
	$\lambda$. Indeed, a straight forward computation shows
	\begin{align}\label{eq4:reslimits}
		\lim_{\lambda\to+\infty} |\beta(\lambda)| = 1,
		\qquad \text{and} \qquad
		\lim_{\lambda\to+\infty} \alpha(\lambda) = 0.
	\end{align}
	Recalling that
	\begin{align*}
		a(\lambda)
				&:= 1 + i \alpha(\lambda) \,
					\int_{\mathbb R} u(x) \, M_1^+(x; \lambda, u) \, \mathrm{d}x \\
			b(\lambda)
				&= i \beta(\lambda) \, 
					\int_{\mathbb R} 
						e^{-ix\lambda} \, u(x) \, M_1^+(x; \lambda, u) 
					\, \mathrm{d}x,
	\end{align*}
	we see that
	\[
		\left|\frac{b(\lambda)}{a(\lambda)}\right|
			\lesssim \nm{u\,M_1^+(\dotarg; \lambda)}_{L^1}
			\leq 
				\nm{M_1^+(\dotarg; \lambda)}_{\inn{\dotarg}L^\infty} 
				\nm{u}_{L^{1,1}}
	\]
	for $\lambda \gg 1$. Given $M_1^+ = \left(1 - T_{L, \lambda, u} \right)^{-1} 1$
	and the operator $T_{L, \lambda, u}$ is bounded uniformly in 
	$\lambda$\textemdash{}in fact, 
	$\nm{T_{L, \lambda, u}}_{\inn{\dotarg}L^\infty\toitself}
	<\frac{1}{2}$\textemdash{}we conclude by Neumann series that 
	$\nm{M_1^+(\dotarg; \lambda)}_{\inn{\dotarg}L^\infty}$ is also uniformly 
	bounded in $\lambda$. The result therefore follows.
\end{proof}

While, we do not yet have a proof that the ILW direct
scattering map is Lipschitz as a map from $B_X(0, c_0)$ into 
$L_\lambda^\infty$, we are able to prove that it is Lipschitz in 
more restrictive regimes.  


%%===================
%% Lipschitz Theorems
%%===================
\begin{thm}\label{thm4:DlipR}
	For $c_0>0$ from Proposition \ref{prop4:exist} and for all fixed 
	$k > 0$, the ILW direct scattering map 
	\[
		\mathscr D: B_X(0, c_0) \ni u \mapsto r \in
			L_\lambda^\infty\big((-\infty,-k]\cup[k,\infty)\big)
	\] 
	is Lipschitz continuous with Lipschitz constant depending on $k$. 
\end{thm}
\begin{proof}
	Let $u_1, u_2 \in B_X(0, c_0)$ be arbitrary and respectively denote by 
	$r_1 = b_1 / a_1$, $r_2 = b_2 / a_2$ the corresponding ILW direct scattering
	map $\mathscr D$ outputs. Since $|a_1(\lambda)| \geq 1$ for all $\lambda \in \mathbb R$
	by Lemma \ref{lma4:rBnd}, we find 
	\begin{align}\label{eq4:DlipPnT}
		\left| \frac{b_1}{a_1} - \frac{b_2}{a_2} \right|
			&\leq \left|\frac{b_1}{a_1} - \frac{b_2}{a_1}\right|
				+ \left|\frac{b_2}{a_1} - \frac{b_2}{a_2}\right| \\
			&= \frac{1}{|a_1|} |b_1 - b_2| 
				+ \frac{1}{|a_1|} 
					\left|\frac{b_2}{a_2}\right| 
					\left| a_1 - a_2 \right|.
					\nonumber
	\end{align}
	Proposition \ref{prop4:uMcont} implies the map $u \mapsto uM_1^+$ is Lipschitz as a map
	into $L^1_x$. As such, 
	\begin{align}\label{eq4:DlipB}
		\frac{1}{|a_1|} |b_1 - b_2| 
			&= \frac{|\beta(\lambda)|}{|a_1(\lambda)|}
				\left|
					\int_{\mathbb R} 
						e^{ix\lambda} 
						\big(
							u_1(x) \, M_1^+( x; \lambda, u_1) 
								- u_2(x) \, M_1^+( x; \lambda, u_2) 
						\big)
					\, \mathrm{d}x
				\right| \\
			&\leq 
				\frac{|\beta(\lambda)|}{|a_1(\lambda)|} 
				\nm{
					u_1\, M_1^+(\dotarg; \lambda, u_1) 
					- u_2\, M_1^+(\dotarg; \lambda, u_2)
				}_{L^1} 
				\nonumber \\
			&\lesssim 
				\frac{|\beta(\lambda)|}{|a_1(\lambda)|}
				\nm{u_1 - u_2}_X,
				\nonumber 
	\end{align}
	where the implied constant is uniform in $\lambda$. Similarly,
	\begin{align}\label{eq4:DlipA}
		|a_1 - a_2|
			& \leq 
				|\alpha(\lambda)|
				\nm{
					u_1 \, M_1^+(\dotarg; \lambda, u_1) 
					- u_2 \, M_1^+(\dotarg; \lambda, u_2)
				}_{L^1} 
				\\
			&\lesssim |\alpha(\lambda)| \nm{u_1 - u_2}_X,
				\nonumber
	\end{align}
	where the implied constant is again uniform in $\lambda$.
	Now, the proof of Theorem \ref{thm4:Dwelldefined} implies 
	that the term
	\[
		\frac{1}{|a_1|} \left| \frac{b_2}{a_2} \right| |a_1 - a_2|
	\]
	is bounded for $|\lambda| > 0$. Through direct computation,
	it is straightforward to show 
	\begin{align*}
		\lim_{\lambda \to -\infty} |\alpha(\lambda)| 
			= \lim_{\lambda \to +\infty} |\beta(\lambda)| 
			= 1, 
			\\
		\lim_{\lambda \to +\infty} |\alpha(\lambda)| 
			= \lim_{\lambda \to -\infty} |\beta(\lambda)| 
			= 0,  
	\end{align*}
	which implies $|\alpha(\lambda)|$ and $|\beta(\lambda)|$ are bounded for 
	$|\lambda| \gg 1$. Further, since $\alpha$ and $\beta$ have exactly one 
	singularity, namely $\lambda = 0$, we conclude by estimates 
	\eqref{eq4:DlipPnT} through \eqref{eq4:DlipA} that
	\begin{align}\label{eq4:DlipFinal}
		\nm{r_1 - r_2}_{L_\lambda^\infty}
			\lesssim_k \|u_1 - u_2\|_X
	\end{align}
	as $|a_1| \geq 1$ for $\lambda \ne 0$, where the implied constant depends 
	on $k$ but is otherwise independent of $\lambda$.
\end{proof}

\begin{rmk}\label{rmk4:Lip}
	The difficulty in extending Theorem \ref{thm4:DlipR} to all values of real 
	$\lambda$ is due to the possibility of the implied constant in 
	\eqref{eq4:DlipFinal} ``blowing-up'' as $k\to 0$\textemdash{}especially when 
	either $\int_{\mathbb R} u_1 \, M_1^+(x; \lambda = 0, u_1) \, \mathrm{d}x$ or 
	$\int_{\mathbb R} u_2 \, M_1^+(x; \lambda = 0, u_2) \, \mathrm{d}x$ are zero. To see 
	why this is so, note that 
	\[
		\frac{\alpha(\lambda)}{a(\lambda)}
			= \frac{\alpha(\lambda)}
				{
					1 
						+ \alpha(\lambda) 
							\int_{\mathbb R} 
								u \, M_1^+(x; \lambda) 
							\, \mathrm{d}x
				}
			=\frac{1}
				{
					\frac{1}{\alpha(\lambda)} 
						+ \int_{\mathbb R} u \, M_1^+(x; \lambda) \, \mathrm{d}x,
				}
	\]
	and
	\[
		\frac{\beta(\lambda)}{a(\lambda)}
			= \frac{\beta(\lambda)}
				{
					1 
						+ \alpha(\lambda) 
							\int_{\mathbb R} 
								u \, M_1^+(x; \lambda) 
							\, \mathrm{d}x
				}
			=\frac{1}
				{
					\frac{1}{\beta(\lambda)} 
						+ \frac{\alpha(\lambda)}{\beta(\lambda)}
							\int_{\mathbb R} 
								u \, M_1^+(x; \lambda) 
							\, \mathrm{d}x.
				}
	\]
	which means 
	\[
		\frac{\alpha(\lambda)}{a(\lambda)},~\frac{\beta(\lambda)}{a(\lambda)}
			\sim \mathcal O
				\left(
					\frac{1}
						{
							\int_{\mathbb R} 
								u \, M_1^+(x; \lambda) 
							\, \mathrm{d}x
						}
				\right)
	\]
	for $|\lambda| \ll 1$, as 
	$\lim_{\lambda\to0} \alpha(\lambda) / \beta(\lambda) = 1$.
	Thus, if either $\int_{\mathbb R} u_1 \, M_1^+(x; \lambda = 0, u_1) \, \mathrm{d}x$ or 
	$\int_{\mathbb R} u_2 \, M_1^+(x; \lambda = 0, u_2) \, \mathrm{d}x$ are zero, then the 
	approach in the proof of Theorem \ref{thm4:DlipR} fails miserably for 
	$|\lambda|$ that is not controlled below.
\end{rmk}

In light of Remark \ref{rmk4:Lip}, we obtain the following easy ``extension''
of Theorem \ref{thm4:DlipR}, which emphasizes the challenge in actually extending
Theorem \ref{thm4:DlipR} to all real values of $\lambda$. 

\begin{cor}\label{cor4:Lip}
	For every $u \in B_X(0, c_0)$ with the property that
	\[
		\int_{\mathbb R} u\,M_1^+(x; \lambda = 0, u) \, \mathrm{d}x \ne 0,
	\]
	there is a neighborhood $\mathcal N(u)$ in $B_X(0, c_0)$ about $u$ 
	for which the map $\mathscr D : \mathcal N(u) \mapsto 
	L_\lambda^\infty(\mathbb R)$ is Lipschitz continuous. 
\end{cor}
\begin{proof}
	Fix $\varepsilon > 0$ so that 
	$\left| \int_{\mathbb R} u\,M_1^+(x; \lambda = 0, u) \, \mathrm{d}x \right| > 2\varepsilon$
	Using the Lipschitz continuity of the map 
	$B_X(0, c_0) \ni w(x) \to  w(x)\, M_1^+(x; \lambda, w)
	\in L_x^1(\mathbb R)$, we may choose $\mathcal N_\varepsilon(u)$ so that 
	every $w$ in $\mathcal N(u)$ satisfies 
	\[
		\left| \int_{\mathbb R} w(x)\, M_1^+(x; \lambda=0, w)\,dx \right|
			\geq \varepsilon.
	\]
	Then, Corollary \ref{cor4:Lip} follows from the proof of Theorem 
	\ref{thm4:DlipR}, Remark \ref{rmk4:Lip}, and the Dominated Convergence
	Theorem.
\end{proof}
% \begin{proof}
% 	We prove this result using the following steps:
% 	\begin{itemize}
% 		\item[\textbf{Step 1:}] Prove 
% 			$B_X(0, c_0) \ni u \mapsto u M_1^+(\dotarg; \lambda, u) \in 
% 			L^1(\mathbb R)$ is Lipschitz with Lipchitz constant uniform in 
% 			$\lambda \in \mathbb R$.
% 		\item[\textbf{Step 2:}] Prove result for $|\lambda| \gg 0$.
% 		\item[\textbf{Step 3:}] For some fixed $k>0$, and $|\lambda|\ll 1$
% 			prove result for
% 			case at least one of 
% 			$\left|\int u_1\,M_1^+(\dotarg; \lambda, u_1)\right|$, 
% 			$\left|\int u_2\,M_1^+(\dotarg; \lambda, u_2)\right|$,
% 			is bounded below by $k$
% 		\item[\textbf{Step 4:}] Prove result for case 
% 			$\left|\int u_1\,M_1^+(\dotarg; \lambda, u_1)\right|=0$
% 			$\left( \text{or}~ 
% 				\left|\int u_2\,M_1^+(\dotarg; \lambda, u_2)\right|=0
% 			\right)$
% 			and $u$ Schwartz class.
% 		\item[\textbf{Step 5:}] Use density argument to generalize result from 
% 			\textbf{Step 3} to arbitrary $u \in B_X(0, c_0)$.
% 			{\color{red} (Not sure if this step is possible)}
% 		\item[\textbf{Step 6:}] Wrap up
% 	\end{itemize}

% 	\noi\textbf{Steps 1 \& 2:} Let $u_1, u_2 \in B_X(0, c_0)$ be 
% 	arbitrary, and denote by $r_1 = b_1 / a_1$, $r_2 = b_1 / a_1$
% 	the corresponding images under the direct scattering map.
% 	Since 
% 	\[
% 		\frac{2\zeta(-\lambda) - 1}{1- 2\zeta(\lambda)} \geq 0
% 	\]
% 	for all $\lambda \in \mathbb R$, one consequence of \eqref{eq4:rbnd}
% 	is that $|a(\lambda)| \geq 1$. As such,
% 	\begin{align*}
% 		\left| \frac{b_1}{a_1} - \frac{b_2}{a_2} \right|
% 			&\leq \left|\frac{b_1}{a_1} - \frac{b_2}{a_1}\right|
% 				+ \left|\frac{b_2}{a_1} - \frac{b_2}{a_2}\right| \\
% 			&= \frac{1}{|a_1|} |b_1 - b_2| 
% 				+ \frac{1}{|a_1|} 
% 					\left|\frac{b_2}{a_2}\right| 
% 					\left| a_1 - a_2 \right|
% 	\end{align*}
% 	where
% 	\begin{align*}
% 		\frac{1}{|a_1|} |b_1 - b_2| 
% 			&= \frac{|\beta(\lambda)|}{|a_1(\lambda)|}
% 				\left|
% 					\int_{\mathbb R} 
% 						e^{ix\lambda} 
% 						\big(
% 							u_1(x) \, M_1^+( x; \lambda, u_1) 
% 								- u_2(x) \, M_1^+( x; \lambda, u_2) 
% 						\big)
% 					\, \mathrm{d}x
% 				\right|,
% 	\end{align*}
% 	and 
% 	\begin{align*}
% 		&\big| 
% 			u_1 \, M_1^+( \dotarg; \lambda, u_1) 
% 				- u_2 \, M_1^+( \dotarg; \lambda, u_2) 
% 		\big| \\
% 			&\qquad\leq \big| M_1^+( \dotarg; \lambda, u_1) \big| |u_1 - u_2|
% 				+ |u_2| 
% 					\big| 
% 						M_1^+( \dotarg; \lambda, u_1) - M_1^+( \dotarg; \lambda, u_2) 
% 					\big|
% 	\end{align*}

% 	Since the map $B_X(0, c_0) \ni u \mapsto M_1^+ \in \inn{\dotarg} L^\infty$ 
% 	is Lipschitz
% 	(uniformly in $\lambda$) by Proposition \ref{prop3:Lip}, 
% 	\[
% 		\big\| M_1^+( \dotarg; \lambda, u_1) \big\|_{\inn{x}L^\infty(\mathbb R)}
% 			=\big\| M_1^+( \dotarg; \lambda, u_1) - 0 \big\|_{\inn{x}L^\infty(\mathbb R)}
% 			\lesssim \|u_1 - 0\|_{X}
% 			< c_0
% 	\]
% 	where the implied constant is the Lipschitz constant for the map 
% 	$u \mapsto M_1^+$. We then have
% 	\begin{align*}
% 		&\left|
% 			\int_{\mathbb R} 
% 				e^{ix\lambda} 
% 				\big(
% 					u_1(x) \, M_1^+( \dotarg; \lambda, u_1) 
% 						- u_2(x) \, M_1^+( \dotarg; \lambda, u_2) 
% 				\big)
% 			\, \mathrm{d}x
% 		\right| \\
% 		&\qquad\leq 
% 			\int_{\mathbb R} 
% 				\big| 
% 					u_1(x) \, M_1^+( x; \lambda, u_1) 
% 						- u_2(x) \, M_1^+( x; \lambda, u_1) 
% 				\big| 
% 			\, \mathrm{d}x \\
% 		&\qquad\quad+
% 			\int_{\mathbb R} 
% 				\big|
% 					u_2(x) \, M_1^+( x; \lambda, u_1)  
% 						- u_2(x) \,  M_1^+( x; \lambda, u_2) 
% 				\big|
% 			\, \mathrm{d}x \\
% 		&\qquad\leq \int_{\mathbb R} 
% 				\left(\inn{x}^{-1}\big|M_1^+( x; \lambda, u_1)\big|\right)
% 				\big( \inn{x}^2
% 				\big| 
% 					u_1(x) - u_2(x)
% 				\big| 
% 				\big)
% 			\, \mathrm{d}x \\
% 		&\qquad\quad+
% 			\int_{\mathbb R} 
% 				\left(
% 					\inn{x}^{-1}
% 					\big|
% 						 M_1^+( x; \lambda, u_1)  
% 							-  M_1^+( x; \lambda, u_2) 
% 					\big|
% 				\right)
% 				\big( \inn{x}^2 u_2(x) \big)
% 			\, \mathrm{d}x \\
% 		&\qquad\leq 
% 			\|M_1^+( \dotarg; \lambda, u_1) \|_{\inn{x} L^\infty}
% 			\left\|(u_1 - u_2)\inn{\dotarg}^2\right\|_{L^1} \\
% 		&\qquad\quad
% 			+ \left\|u_2\inn{\dotarg}^2\right\|_{L^1}
% 				\|
% 					M_1^+( \dotarg; \lambda, u_1)
% 					- M_1^+( \dotarg; \lambda, u_2)
% 				\|_{\inn{x} L^\infty} \\
% 		&\qquad\lesssim  \| u_1 - u_2 \|_X,
% 	\end{align*}
% 	where the implied constant is independent of $\zeta$ and $u$.
% 	{\color{red} Prove $B_X(0, c_0) \ni u \mapsto uM_1^+ \in L^1$ is Lipschitz 
% 	as separate prop (or lemma) to simplify above calculation.}
% 	Since $|a(\lambda; u)| \geq 1$, the above calculation shows that the term
% 	$\frac{1}{|a_1|}|b_1 - b_2|$ is controlled for $\lambda$ away from zero.
% 	Since $\frac{|\beta(\lambda)|}{|a_1(\lambda)|} \sim 
% 	\frac{1}{ \int u_1 \, M_1^+( \dotarg; \, u_1) }$ for $|\lambda|$ small,
% 	for small $|\lambda|$ we may assume without loss of generality 
% 	(as per \textbf{Step 3}) that $\left|\int u_1 M(\dotarg; u_1) \right| \geq k$
% 	allowing us to control $\frac{1}{|a_1|}\|b_1 - b_2\|_{L^\infty}$ 
% 	uniformly in $\lambda$ (for the case of \textbf{Step 3}) by 
% 	$\frac{1}{k} \| u_1 - u_2 \|_X$.

% 	Equation \eqref{eq4:rbnd} implies that 
% 	\[
% 		\left| \frac{b_2(\lambda)}{a_2(\lambda)} \right|
% 			= \sqrt{ 
% 					\frac{1-2\zeta(\lambda)}{2 \zeta(-\lambda) - 1} 
% 					\left(
% 						1 - \frac{1}{|a_2 (\lambda)|^2}
% 					\right)
% 				}	
% 	\]
% 	We make the following observations
% 	\begin{align*}
% 		\frac{1-2\zeta(\lambda)}{2 \zeta(-\lambda) - 1} \geq 0 
% 			\qquad \forall \lambda \in \mathbb R \setminus \{0\}
% 	\end{align*}
% 	\begin{align*}
% 		\lim_{\lambda \to 0}	\frac{1-2\zeta(\lambda)}{2 \zeta(-\lambda) - 1} = 1
% 	\end{align*}
% 	\begin{align*}
% 		\lim_{\lambda \to -\infty} \frac{1-2\zeta(\lambda)}{2 \zeta(-\lambda) - 1} = 0
% 	\end{align*}
% 	\begin{align*}
% 		\lim_{\lambda \to +\infty} \frac{1-2\zeta(\lambda)}{2 \zeta(-\lambda) - 1} = \infty
% 	\end{align*}
% 	\begin{align*}
% 		\lim_{\lambda \to +\infty} \alpha(\lambda)
% 			 \sqrt{ \frac{1-2\zeta(\lambda)}{2 \zeta(-\lambda) - 1}} = 0
% 	\end{align*}
% 	\begin{align*}
% 		\lim_{\lambda \to 0} \frac{|\alpha(\lambda)|}{|a_1(\lambda)|}
% 			= \frac{1}{| \int u_1 M_1^+(u_1) |}
% 	\end{align*}
% 	Since the map $B_X(0, c_0)\ni u \to uM \in L^1(\mathbb R)$,
% 	is Lipschitz with Lipschitz constant uniform in $\lambda$, 
% 	\[
% 		|a_1 - a_2|
% 			\leq |\alpha(\lambda)| 
% 				\int_{\mathbb R} 
% 					\big| 
% 						u_1 M_1^+(x; \lambda, u_1) - u_2 M_1^+(x; \lambda, u_2)
% 					\big|
% 				\, \mathrm{d}x
% 			\lesssim |\alpha(\lambda)| \| u_1 - u_2 \|_X.
% 	\]
% 	The above observations allow us to control the 
% 	$\frac{1}{|a_1|}\frac{|b_2|}{|a_2|} |a_1 - a_2|$ term by 
% 	$\frac{1}{\int u_1 M(\dotarg; u_1)} \| u_1 - u_2 \|_X$ (for small $|\lambda|$)
% 	or $\| u_1 - u_2 \|_X$ for $|\lambda|$ away from zero.

% 	\textbf{Step 4:}
% \end{proof}

The following lemma, Lemma \ref{lma4:Mtexpansion}, was developed as part of an, as 
yet, unsuccessful bid to 
extend Theorem \ref{thm4:DlipR} to all real $r$. We include this lemma here in the 
hopes that it may eventually be useful in completing the proof that the ILW 
direct scattering map is Lipschitz continuous. 

\begin{lma}\label{lma4:Mtexpansion}
	For $u \in X\cap \inn{x}^{-5} L_x^\infty(\mathbb R)$, the Jost solution 
	boundary value
	$M_1^+$ has the following $\inn{x}^4L_x^\infty(\mathbb R)$
	linear approximation in $\lambda$ centered at $\lambda = 0$
	\begin{align}
		M_1^+(x; \lambda, u) 
			= M^{(0)}(x; u) + \lambda \, M^{(1)}(x; u) + o(\lambda),
	\end{align}
	where $M^{(0)}(x; u) = M_1^+(x; \lambda = 0, u)$ and
	$M^{(1)}$ is $(\partial M_1^+ / \partial \lambda)(x; 0)$
\end{lma}
\begin{proof}
	It suffices to prove that $M_1^+$ has an $\inn{x}^4L^\infty(\mathbb R)$ 
	derivative in $\lambda$ at $\lambda = 0$. To do so, we 
	define $M_h$ as the difference quotient
	\[
		M_h(\lambda; x) :=
			\frac{M_1^+(x; \lambda +h) - M_1^+(x; \lambda)}{h}.
	\]
	In order to simplify notation, throughout the rest of this proof,
	we denote $G_L^+$ by $G$, $M_1^+$ by $M$, and suppress $x$ dependency.
	That is, $G(\lambda):= G_L^+(x; \lambda)$ and $M(\lambda):=M_1^+(x; \lambda)$.
	Please note that while not explicitly indicated by the notation in this
	proof, all convolutions are with respect to the variable $x$.

	By linearity of convolution operators 
	% \begin{align*}
	% 	M_h(\lambda; x)
	% 		&= 
	% 			\frac{
	% 				G_L^+(\dotarg; \lambda+h)
	% 					* \big[u\,M_1^+(\dotarg, \lambda+h)\big](x)
	% 				-G_L^+(\dotarg; \lambda)
	% 					* \big[u\,M_1^+(\dotarg, \lambda)\big](x)
	% 			}
	% 			{h} \\[.1\baselineskip]
	% 		&= 
	% 			\frac{1}{h}
	% 			\Big\{
	% 				\big[
	% 					G_L^+(\dotarg; \lambda+h) - G_L^+(\dotarg; \lambda)
	% 				\big]
	% 					* \big[u\,M_1^+(\dotarg, \lambda+h)\big](x) \\
	% 		&\qquad
	% 				- G_L^+(\dotarg; \lambda) 
	% 					* 
	% 						\big[
	% 							u
	% 							\big(
	% 								M_1^+(\dotarg; \lambda+h)
	% 								-M_1^+(\dotarg; \lambda)
	% 							\big)
	% 						\big](x)
	% 			\Big\} \\
	% 		&= 
	% 			\left(
	% 				\frac{G_L^+(\dotarg; \lambda+h)-G_L^+(\dotarg; \lambda+h)}{h}
	% 			\right)
	% 				* \big[u\,M_1^+(\dotarg, \lambda+h)\big](x)
	% 			- G_L^+(\dotarg; \lambda) 
	% 				* \big[ u \, M_h(\lambda; \dotarg) \big](x)
	% \end{align*}
	\begin{align*}
		M_h(\lambda; x)
			&= 
				\frac{
					G(\lambda+h)* \big[u\,M(\lambda+h)\big]
					-G(\lambda) * \big[u\,M(\lambda)\big]
				}
				{h} \\[.1\baselineskip]
			&= 
				\frac{
					\big[
						G(\lambda+h) - G(\lambda)
					\big]
						* \big[u\,M(\lambda+h)\big]
			% &\qquad
					- G(\lambda) 
						* 
							\big\{
								u
								\big[
									M(\lambda+h)
									-M(\lambda)
								\big]
							\big\}
				}
				{h} \\
			&= 
				\left(
					\frac{G(\lambda+h)-G(\lambda)}{h}
				\right)
					* \big[u\,M(\lambda+h)\big]
				- G(\lambda) 
					* \big[ u \, M_h(\lambda) \big]
	\end{align*}
	Define $G_h$ to be the difference quotient 
	$G_h = \frac{1}{h}\big[G(\lambda+h) - G(\lambda)\big]$ and let 
	$T_\lambda$ denote the operator given by $T_\lambda f = G(\lambda)*(u\,f)$.
	Since, as we see in the proof of Proposition \ref{prop4:exist}, $I + T_\lambda$
	is invertible, the following formula for $M_h$ follows from the above 
	computation:
	\begin{align}\label{eq3:Mh}
		M_h(\lambda) := (I + T_\lambda)^{-1} \big(G_h * u\, M(\lambda+h)\big).
	\end{align}
	Given the continuity of $(I + T_\lambda)^{-1}$, equation \eqref{eq3:Mh}
	implies that $M_1^+$ is differentiable in $\lambda$ ($\lambda \in \mathbb R$)
	if and only if the limit 
	\[
		\lim_{h\to0} G_h * u\, M(\lambda+h)
	\]
	holds pointwise for each $x \in \mathbb R$. 

	Since a natural candidate for the limit of $G_h * u\, M(\lambda+h)$
	as $h\to 0$ is $\left(\frac{\partial}{\partial \lambda}G\right) * (uM)$,
	note that
	\begin{align}
		&G_h*\big(uM(\lambda+h)\big) 
				- \left(\frac{\partial}{\partial \lambda}G\right) 
					* \big(uM(\lambda)\big) \\
			&\qquad=
				\left(G_h - \frac{\partial}{\partial \lambda}G\right)
						* \big(uM(\lambda +h)\big)
					+ \left(\frac{\partial}{\partial \lambda}G\right)
						* u \big(M(\lambda+h) - M(\lambda)\big).
				\nonumber
	\end{align}
	
	By Technical Lemma \ref{tlma2:1}, 
	\begin{align*}
		\left\|
			\left(G_h - \frac{\partial}{\partial \lambda}G\right)
			* \big( uM(\lambda+h)\big)
		\right\|_{\inn{x}^4L_x^\infty}
			&\leq 
				\left\|
					G_h - \frac{\partial}{\partial \lambda}G
				\right\|_{\inn{x}^4 L_x^1}
				\left\|
					\inn{\dotarg}^4
					u\,M(\lambda+h)
				\right\|_{L_x^\infty}.
	\end{align*}
	% Recalling the inequality
	% \begin{align*}
	% 	\frac{\inn{x'}}{\inn{x-x'}\inn{x}} \leq 1,
	% \end{align*}
	% we find
	% \begin{align*}
	% 	&\left\|
	% 		\left(G_h - \frac{\partial}{\partial \lambda}G\right)
	% 		* \big( uM(\lambda+h)\big)
	% 	\right\|_{\inn{x}^4L_x^\infty} \\
	% 		&\qquad= 
	% 			\sup_{x\in \mathbb R} \inn{x}^{-4} 
	% 			\int_{\mathbb R}
	% 				\left|
	% 					\left[G_h - \frac{\partial}{\partial \lambda}G\right](x')
	% 					\big[u\,M(\lambda + h)\big](x-x')
	% 				\right|
	% 			\, \mathrm{d}x' \nonumber \\
	% 		&\qquad=
	% 			\sup_{x\in \mathbb R}
	% 			\int_{\mathbb R}
	% 				\left|
	% 					\left\{
	% 						\inn{x'}^{-4}
	% 						\left[G_h - \frac{\partial}{\partial \lambda}G\right](x')
	% 					\right\}
	% 					\left(
	% 						\frac{\inn{x'}}{\inn{x}\inn{x-x'}}
	% 					\right)^4
	% 				\right|
	% 			\nonumber \\
	% 		&\qquad\qquad\times
	% 			\left|
	% 			\big[ \inn{\dotarg}^4 u M(\lambda +h) \big](x-x')
	% 			\right|
	% 			\, \mathrm{d}x' \nonumber \\
	% 		&\qquad\leq
	% 			\left\|
	% 				\left[
	% 					\inn{\dotarg}^{-4} 
	% 					\left(G_h - \frac{\partial}{\partial\lambda}G\right)
	% 				\right] 
	% 				*
	% 				\left[
	% 					\inn{\dotarg}^4
	% 					u\, M(\lambda + h)
	% 				\right]
	% 			\right\|_{L^\infty}
	% 			\nonumber \\
	% 		&\qquad\leq
	% 			\left\|
	% 				G_h - \frac{\partial}{\partial \lambda}G
	% 			\right\|_{\inn{x}^4 L_x^1}
	% 			\left\|
	% 				\inn{\dotarg}^4
	% 				u\,M(\lambda+h)
	% 			\right\|_{L_x^\infty} \nonumber
	% \end{align*}
	Hence
	\begin{align} \label{eq3:??}
		\lim_{h\to\infty} 
				\left\|
					\left(G_h - \frac{\partial}{\partial \lambda}G\right)
					* \big( uM(\lambda+h)\big)
				\right\|_{\inn{x}^4 L_x^\infty} 
			= 0,
	\end{align}
	as 
	\begin{align*}
		\left\|
			\inn{\dotarg}^4
			u\,M(\lambda+h)
		\right\|_{L_x^\infty}
		=
			\esssup_{x\in\mathbb R} \big(\inn{x}^{-1} M\big) \big(\inn{x}^5 u\big)
		\leq \|M\|_{\inn{x}L_x^\infty} \|u\|_{\inn{x}^{-5}L_x^\infty}.
	\end{align*}

	Similarly, 
	\begin{align*}
		&
			\left\|
				\left(\frac{\partial}{\partial \lambda} G\right)
				*
				u\big(M(\lambda+h) - M(\lambda)\big)
			\right\|_{\inn{x}^4 L_x^\infty} \\
		&\qquad\leq
			\left\|
				\frac{\partial}{\partial \lambda} G
			\right\|_{\inn{x}^4 L^1}
			\left\|	
				\inn{\dotarg}^4 u 
				\big[
					M(\lambda + h) - M(\lambda)
				\big]
			\right\|_{L_x^\infty} 
			\nonumber \\
		&\qquad\leq
			\left\|
				\frac{\partial}{\partial \lambda} G
			\right\|_{\inn{x}^4 L^1}
			\|u\|_{\inn{x}^{-5} L_x^\infty}
			\|M(\lambda+h) - M(\lambda)\|_{\inn{x}L_x^\infty}
			\nonumber
	\end{align*}
	Hence, by Lemma \ref{lma4:Mlamcont}, we also find
	\begin{align}
		\lim_{h\to\infty}
			\left\|
				\left(\frac{\partial}{\partial \lambda} G\right)
				*
				u\big(M(\lambda+h) - M(\lambda)\big)
			\right\|_{\inn{x}^4 L_x^\infty} 
			=0.
	\end{align}
	{\color{red} finish this by explaining what these limits imply 
	for pointwise (in $x$) limits.}
\end{proof}




% {\color{red} What follows are informal notes to myself. Will turn this into 
% proper prose later}.


% First, note that as $\lambda \to 0$ ($\zeta \to 1/2$), the two first order poles
% $1/p$ collapse to a single second order pole. As this happens, sum of the residues 
% (from $\xi =0$ and $\xi = \lambda$) become $\mathcal O(x)$. As such, in order to 
% obtain uniform (in $\lambda$) estimates needed proving both existence / uniqueness 
% of Jost solutions and Lipschitz continuity of the map $u \mapsto b/r$, we use assume 
% $u \in X$, where
% \[
% 	\|u\|_X := \| \inn{\dotarg}^2 \, u(\dotarg) \|_{L^1}
% 					+ \esssup_{x\in \mathbb R} 
% 						\left[
% 							\int_{|x-x'|\leq 1}
% 								\log\left(\frac{1}{|x-x'|}\right)
% 								\inn{x'} | u(x') | 
% 							\, \mathrm{d}x'
% 						\right]
% \]
% and
% \[
% 	\|f\|_{L^{1,2}}:= \left(\int_{\mathbb R} \inn{x}^{sp} |f(x)|^p \right)^{1/p},
% \]
% and
% \[
% 	\inn{x}:= \big(1 + |x|^2\big)^{1/2}.
% \]

% In what follows, we first need to show  that the map $u\mapsto r = b/a$ is well
% defined, which involves first showing that the Jost IE admit unique solutions 
% in $\inn{x} L^\infty(\mathbb R)$ for $u$ with $\|u\|_X < c_0$ for some small 
% (and fixed) constant $c_0>0$, and then showing that 
% $r \in \inn{x}L^\infty(\mathbb R)$ whenever 
% $u \in B_X(0, c_0):= \left\{ f \in X ~:~ \|f\|_X < c_0 \right\}$.

% After showing that $B_X(0, c_0) \ni u \mapsto r \in \inn{x}L^\infty(\mathbb R)$
% is well defined, we need to show that it is also Lipschitz. This is accomplished
% in the following stages:
% \begin{itemize}
% 	\item[A. ] Prove Lipschitz for $|\lambda| \geq k$, where $k >0$ is some fixed 
% 		constant
% 		\begin{itemize}
% 			\item[1. ] Prove $B_X(0, c_0) \ni u \mapsto M_1^+(\dotarg; \zeta, u) \in 
% 				\inn{x}L^\infty(\mathbb R)$ is Lipschitz with Lipschitz constant uniform
% 				in $\zeta$ ({\color{red} this proof is in the ilw.pdf})
% 			\item[2. ] Prove $B_X(0, c_0) \ni u \mapsto u M_1^+(\dotarg; \zeta, u) 
% 				\in L^1(\mathbb R)$ is also Lipschitz with Lipschitz constant uniform in 
% 				$\zeta$
% 			\item[3. ] Prove
% 			\item[4. ] Prove $\frac{|\beta(\zeta)|}{|a(\zeta)|}$ is bounded in $\zeta$ near
% 				$\zeta = \frac{1}{2}$ (\textit{i.e.} $\lambda = 0$)
% 			\item[5. ] {\color{red} complete this list $\ldots$}
% 		\end{itemize}
% 	\item[B. ] Prove Lipschitz for $0 < |\lambda| < k$, where $k >0$ is some 
% 		fixed constant
% 		\begin{itemize}
% 			\item[6. ] {\color{red} complete this list $\ldots$}
% 		\end{itemize}
% \end{itemize}

% For reference, $M_1^+ = 1+ G*(uM_1^+)$ and $a$, $b$ are defined as follows:

% \begin{align*}
%  	a(\lambda) 
%  		&= 1 +  i \alpha(\lambda) 
%  			\int_{\mathbb R}
%  				 u(x) M_1^+(x; \lambda) 
%  			\, \mathrm{d}x \\
%  	b(\lambda) 
%  		&= i \beta(\lambda) 
%  			\int_{\mathbb R}
%  				e^{i \lambda x} u(x) M_1^+(x; \lambda) 
%  			\, \mathrm{d}x,
% \end{align*}
% where
% \begin{align*}
% 	\alpha(\lambda)
% 		= \frac{1}{1 - 2 \zeta(\lambda)}
% 		= \frac{1 - e^{2\lambda}}{2\lambda e^{2\lambda} + 1 - e^{2\lambda}},
% \end{align*}
% and
% \begin{align*}
% 	\beta(\lambda)
% 		= \frac{1}{1-2 \zeta^*}
% 		= \frac{1-e^{2\lambda}}{1+2\lambda-e^{2\lambda}}
% \end{align*}


% Assuming that the well-definedness of the map $u\mapsto r$ is established 
% (in fact, this is actually done in the ILW paper), then the 
% steps for proving that $u\mapsto r$ is Lipschitz is based on the following two
% ``put and takes''
% \begin{align*}
% 	\left| \frac{b_1}{a_1} - \frac{b_2}{a_2} \right|
% 		&\leq \left|\frac{b_1}{a_1} - \frac{b_2}{a_1}\right|
% 			+ \left|\frac{b_2}{a_1} - \frac{b_2}{a_2}\right| \\
% 		&= \frac{1}{|a_1|} |b_1 - b_2| 
% 			+ \frac{1}{|a_1|} 
% 				\left|\frac{b_2}{a_2}\right| 
% 				\left| a_1 - a_2 \right|
% \end{align*}
% where
% \begin{align*}
% 	\frac{1}{|a_1|} |b_1 - b_2| 
% 		&= \frac{|\beta(\lambda)|}{|a_1(\lambda)|}
% 			\left|
% 				\int_{\mathbb R} 
% 					e^{ix\lambda} 
% 					\big(
% 						u_1(x) \, M_1^+( x; \lambda, u_1) 
% 							- u_2(x) \, M_1^+( x; \lambda, u_2) 
% 					\big)
% 				\, \mathrm{d}x
% 			\right|
% \end{align*}
% and 
% \begin{align*}
% 	&\big| 
% 		u_1 \, M_1^+( \dotarg; \lambda, u_1) 
% 			- u_2 \, M_1^+( \dotarg; \lambda, u_2) 
% 	\big| \\
% 		&\qquad\leq \big| M_1^+( \dotarg; \lambda, u_1) \big| |u_1 - u_2|
% 			+ |u_2| 
% 				\big| 
% 					M_1^+( \dotarg; \lambda, u_1) - M_1^+( \dotarg; \lambda, u_2) 
% 				\big|
% \end{align*}

% Assuming that we've accomplished task (1) above, then 
% \[
% 	\big\| M_1^+( \dotarg; \lambda, u_1) \big\|_{L^\infty(\mathbb R)}
% 		\lesssim c_0,
% \]
% where the implied constant is the Lipschitz constant for the map 
% $u \mapsto M_1^+$. We then have
% \begin{align*}
% 	&\left|
% 		\int_{\mathbb R} 
% 			e^{ix\lambda} 
% 			\big(
% 				u_1(x) \, M_1^+( \dotarg; \lambda, u_1) 
% 					- u_2(x) \, M_1^+( \dotarg; \lambda, u_2) 
% 			\big)
% 		\, \mathrm{d}x
% 	\right| \\
% 	&\qquad\leq 
% 		\int_{\mathbb R} 
% 			\big| 
% 				u_1(x) \, M_1^+( x; \lambda, u_1) 
% 					- u_2(x) \, M_1^+( x; \lambda, u_1) 
% 			\big| 
% 		\, \mathrm{d}x \\
% 	&\qquad\quad+
% 		\int_{\mathbb R} 
% 			\big|
% 				u_2(x) \, M_1^+( x; \lambda, u_1)  
% 					- u_2(x) \,  M_1^+( x; \lambda, u_2) 
% 			\big|
% 		\, \mathrm{d}x \\
% 	&\qquad\lesssim c_0 \, \| u_1 - u_2\|_X 
% 		+ \| u_2 \|_X \, 
% 			\big\|
% 				M_1^+( \dotarg; \lambda, u_1) - \, M_1^+( \dotarg; \lambda, u_2) 
% 			\big\|_{L^\infty} \\
% 	&\qquad\lesssim c_0 \, \| u_1 - u_2\|_X 
% 		+ c_0 \, \, \| u_1 - u_2\|_X \\
% 	&\qquad\lesssim  \| u_1 - u_2 \|_X,
% \end{align*}
% where the implied constant is independent of $\zeta$ and $u$. Note that the
% second line is justified by the fact that
% \[
% 	\int_{\mathbb R} |u_2| \, \mathrm{d}x
% 		= \int_{\mathbb R} \inn{x} |u_2| \inn{x}^{-1} \, \mathrm{d}x
% 		\leq \| \inn{\dotarg} u_2 \|_{L^2} 
% 			\left\| \inn{\dotarg}^{-1} \right\|_{L^2}
% 		\lesssim \| u_2 \|_X.
% \]
% In essence, the above calculation shows that the map
% $B_X(x, c_0) \ni u \mapsto u M_1^+(\dotarg; u) \in L^1$ is Lipschitz.
% Since $|a(\lambda; u)| \geq 1$, the above calculation shows that 
% the term $\frac{1}{|a_1|}|b_1 - b_2|$ is controlled for $\lambda$ 
% away from zero. However, since $\frac{|\beta(\lambda)|}{|a_1(\lambda)|} \sim 
% \frac{1}{ \int u_1 \, M_1^+( \dotarg; \, u_1) }$ for $|\lambda|$ small, 
% we will need to be a little 
% cleverer. We will return to the small $|\lambda|$ singularity issue later. 

% Recall from the 3 June 2019 entry of Allen's worklog (page 27) that 
% \[
% 	|a(\lambda)|^2 
% 		= 1 
% 			+ \frac{2 \zeta(-\lambda) - 1}{1-2\zeta(\lambda)}
% 				|b(\lambda)|^2
% \]
% which implies
% \[
% 	\left| \frac{b_2(\lambda)}{a_2(\lambda)} \right|
% 		= \sqrt{ 
% 				\frac{1-2\zeta(\lambda)}{2 \zeta(-\lambda) - 1} 
% 				\left(
% 					1 - \frac{1}{|a_2 (\lambda)|^2}
% 				\right)
% 			}	
% \]
% We make the following observations
% \begin{align*}
% 	\frac{1-2\zeta(\lambda)}{2 \zeta(-\lambda) - 1} \geq 0 
% 		\qquad \forall \lambda \in \mathbb R \setminus \{0\}
% \end{align*}
% \begin{align*}
% 	\lim_{\lambda \to 0}	\frac{1-2\zeta(\lambda)}{2 \zeta(-\lambda) - 1} = 1
% \end{align*}
% \begin{align*}
% 	\lim_{\lambda \to -\infty} \frac{1-2\zeta(\lambda)}{2 \zeta(-\lambda) - 1} = 0
% \end{align*}
% \begin{align*}
% 	\lim_{\lambda \to +\infty} \frac{1-2\zeta(\lambda)}{2 \zeta(-\lambda) - 1} = \infty
% \end{align*}
% \begin{align*}
% 	\lim_{\lambda \to +\infty} \alpha(\lambda)
% 		 \sqrt{ \frac{1-2\zeta(\lambda)}{2 \zeta(-\lambda) - 1}} = 0
% \end{align*}
% \begin{align*}
% 	\lim_{\lambda \to 0} \frac{|\alpha(\lambda)|}{|a_1(\lambda)|}
% 		= \frac{1}{| \int u_1 M_1^+(u_1) |}
% \end{align*}

% {\color{red}
% 	\begin{rmk}
% 		Obviously, if $\int u_1 M_1^+(u_1) = \int u_2 M_1^+(u_2) = 0$, 
% 		this approach won't work. However, if only one of the two 
% 		integrals equals zero, then the approach can still be addapted 
% 		as appropriate.
% 	\end{rmk}
% }
% Since the map $B_X(0, c_0) \to L^1(\mathbb R)$, $u \mapsto uM$ 
% is Lipschitz, 
% \[
% 	|a_1 - a_2|
% 		\leq |\alpha(\lambda)| 
% 			\int_{\mathbb R} 
% 				\big| 
% 					u_1 M_1^+(x; \lambda, u_1) - u_2 M_1^+(x; \lambda, u_2)
% 				\big|
% 			\, \mathrm{d}x
% 		\lesssim |\alpha(\lambda)| \| u_1 - u_2 \|_X.
% \]
% For $\lambda$ away from zero, the above observations alow us to control the 
% $\frac{1}{|a_1|}\frac{|b_2|}{|a_2|} |a_1 - a_2|$ term by $\| u_1 - u_2 \|_X$ times 
% a constant that depends only on $k$, where we assume $|\lambda| \geq k$.


% {\color{red}
% 	\begin{rmk}
% 		Given that this approach is essentially giving us that $\|r_1 - r_2\|$ is
% 		bounded (modulo a constant) by $1/(\int u_1 M_1^+(u_1)) \|u_1 - u_2\|$, our
% 		next approach to this problem should be to fix some constant $k_1$ and consider 
% 		the case where modulus of at least one of the integrals $\int u_1 M_1^+(u_1)$
% 		or $\int u_2 M_1^+(u_2))$ is bounded below by $k_1$ separately from the case where
% 		both such integrals are bounded above by $k_1$. 
% 	\end{rmk}
% }

% % While technically the direct scattering map takes $u$ to the refection coefficient $r = b/a$, 
% % the purpose of these notes, we will think about the direct scattering map as $u \mapsto b$,
% % where
% % \begin{align*}
% %  	a(\lambda) 
% %  		&= 1 +  i \alpha(\lambda) 
% %  			\int_{-\infty}^x 
% %  				e^{-i \lambda x'} u(x) M_1^+(x'; \lambda) 
% %  			\, \mathrm{d}x' \\
% %  	b(\lambda) 
% %  		&= i \beta(\lambda) 
% %  			\int_{-\infty}^x 
% %  				e^{-i \lambda x'} u(x) M_1^+(x'; \lambda) 
% %  			\, \mathrm{d}x'.
% % \end{align*}
% % Under this perspective, to show Lipschitz continuity, we want to show
% % that 
% % \[
% % 	\| b_1 - b_2 \|_{\L^\infty} \lesssim \| u_1 - u_2 \|.
% % \]
% % To do so, first note that there are two maps at play\textemdash{}namely, 
% % $u\mapsto ^+(\dotarg; \lambda, u)$ and $M_1^+(\dotarg; \lambda, u) \mapsto b(\lambda)$. 
% % In particular, 
% % \begin{align*}
% % 	\| b_1 - b_2 \|		
% % 		&\leq \nm{
% % 				\beta(\lambda)
% % 				\int_{\mathbb R} 
% % 					e^{ix \lambda} 
% % 					\big(u_1(x) - u_2(x)\big) M_1^+(x; \lambda, u_1)
% % 				\, \mathrm{d}x
% % 			} \\
% % 		&\qquad + \nm{
% % 				\beta(\lambda)
% % 				\int_{\mathbb R} 
% % 					e^{ix \lambda} u_2(x)
% % 					\big(M_1^+(x; \lambda, u_1) - M_1^+(x; \lambda, u_2)\big) 
% % 				\, \mathrm{d}x
% % 			}
% % \end{align*}
% % \begin{rmk}
% % 	{\color{red}
% % 	While $\beta(\lambda)$ blows up (in both directions) as $\lambda$ approaches 
% % 	zero,
% % 	\[
% % 		\lim_{\lambda\to 0} \beta(\lambda) e^{i x \lambda} = 0
% % 	\]
% % 	We should be able to use this to show that the integrals 
% % 	$\int \beta(\lambda) e^{ix\lambda} \ldots \, \mathrm{d}x$ are well defined and 
% % 	bounded even as $\lambda \to 0$. 
% % 	}
% % \end{rmk}

% % The first step is to show that the map $u \mapsto r$ is well-defined. That means showing
% % both the existence and uniqueness of $M_1^+(\dotarg; \lambda, u)$.


\end{document}