%%========================================
%% Section 4.3: The Direct Scattering Map
%%========================================


\documentclass[../dissertation.tex]{subfiles}

\begin{document}
\section{The Direct Scattering Map}\label{sec4:DM}

We now have the tools we need to prove that the direct scattering map is well-defined
as a map from $B_X(0, c_0)$ to $L^\infty(\mathbb R)$ and Lipschitz continuous 
as a map from $B_X(0, c_0)$ to $L^\infty\big(\mathbb R \sm (-k,k)\mathbb)$ 
(for any $k > 0$). As a warm-up exercise, we first verify in
Proposition \ref{prop4:SD} the validity of the scattering equations first 
presented (without proof) in \cite{Kodama1982}, which are key to the 
construction of the inverse scattering map for the ILW. Following Proposition 
\ref{prop4:SD}, we turn our attention to proving that the direct scattering map 
is well-defined (Lemma \ref{lma4:rBnd} and Theorem \ref{thm4:Dwelldefined}), and
wrap up this dissertation with results on the Lipschitz continuity of the direct
scattering map (Theorem \ref{prop4:exist} and Corollary \ref{cor4:Lip}).


\begin{prop}[Scattering Equations]\label{prop4:SD}
	Suppose that $u$ satisfies the hypotheses of Proposition \ref{prop4:exist}.
	Let
	\begin{subequations}
		\label{eq4:SD}
		\begin{align}
			% \label{eq4:SDa}
			a(\lambda)
				&:= 1 + i \alpha(\lambda) \,
					\int_{\mathbb R} u(x) \, M_1^+(x; \lambda, u) \, \mathrm{d}x \\
			% \label{eq4:SDb}
			b(\lambda)
				&= i \beta(\lambda) \, 
					\int_{\mathbb R} e^{-ix\lambda} \, u(x) \, M_1^+(x; \lambda, u) \, \mathrm{d}x
		\end{align}
	\end{subequations}
	\vspace{-\baselineskip}
	\begin{subequations}
		\label{eq4:SDbrev}
		\begin{align}
			% \label{eq4:SDa}
			\breve{a}(\lambda)
				&:= 1 + \alpha(\lambda)
					\int_{\mathbb R} u(x) \, N_1(x; \lambda, u) \, \mathrm{d}x \\
			% \label{eq4:SDb}
			\breve{b}(\lambda)
				&= i \beta(\lambda) 
					\int_{\mathbb R} e^{-ix\lambda} \, u(x) \, N_1(x; \lambda, u) \, \mathrm{d}x
		\end{align}
	\end{subequations}
	For $\lambda \in \mathbb R \sm \{0\}$, 
	\begin{align}
		M_1 (x; \lambda)
			&= a(\lambda) \, N_1(x; \lambda) + b(\lambda) \, N_e(x; \lambda) 
			\label{eq3:M1N} \\
		N_1 (x; \lambda)
			&= \breve{a}(\lambda) \, M_1(x; \lambda) 
				+ \breve{b}(\lambda) \, M_e(x; \lambda)
			\label{eq3:N1M}
	\end{align}
\end{prop}
\begin{proof}
	An immediate consequence of formulas \eqref{eq1:GFrep} is the jump 
	relation\footnote{This jump relation can also be proven by taking a contour 
	around the real axis. For details, please see 
	\hyperref[app:JumpRelation]{Appendix 3: Jump Relation}.}
	\begin{align}\label{eq4:GFjump}
		G_L^+ - G_R^+ = i \alpha(\lambda) + i \beta(\lambda) e^{i \lambda x}.
	\end{align}
	Under the jump relation \eqref{eq4:GFjump}, the integral equation 
	\eqref{eq4:JostIEleft} for $M_1^+$ becomes
	\begin{align}\label{eq4:altM1}
		M_1^+(x; \lambda) 
			&= 1 + \int_{\mathbb R} G_L^+(x-x'; \lambda) u(x') M_1^+(x')\,dx' \\
			&= 1 
				+ \int_{\mathbb R} 
					\big(
						i \alpha(\lambda) 
						+ i \beta(\lambda) \, e^{i \lambda (x-x')}
					\big)
					u(x') M_1^+(x') 
				\, \mathrm{d}x \nonumber \\
			&= \left(
					1 
					+ i \alpha(\lambda) 
						\int_{\mathbb R} u(x')M_1^+(x'; \lambda)\,dx 
				\right)  
				\nonumber \\
			&\qquad+ i \beta(\lambda) \, e^{i\lambda x} 
					\int_{\mathbb R} e^{-i\lambda x'} (x')M_1^+(x'; \lambda)\,dx 
				\nonumber \\
			&\qquad+ \int_{\mathbb R} G_R^+(x-x'; \lambda) u(x') M_1^+(x')\,dx'
				\nonumber \\
			&= a(\lambda) + b(\lambda) \,e^{i\lambda x} 
				+ G_R^+(\dotarg; \lambda) * \big[uM_1^+(\dotarg; \lambda)\big].
				\nonumber
	\end{align}
	Recalling from Proposition \ref{prop2:Tasymp} that 
	$\lim_{x\to\infty} G_R^+*uM_1^+(x) = 0$, we see that $M_1^+$ satisfies the 
	asymptotic condition
	\begin{align} \label{eq4:asympM}
		\lim_{x\to+\infty} 
			\big|M_1^+(x; \lambda) - a(\lambda) - b(\lambda)e^{i\lambda x}\big|
			= 0.
	\end{align}

	The simple computation
	\begin{align*}
		G_R^+ * \big[ u(a\,N_1 + b\, N_e)\big]
			&= a \, G_R^+ *(u\, N_1) + b \, G_R^+ *(u\, N_e) \\
			&= a(N_1 - 1) + b(N_e - e^{i\lambda x}) \\
			&= a\, N_1 + b \, N_e - a - b\, e^{i\lambda x}
	\end{align*}
	shows that $a\, N_1 + b \, N_e$ is a solution to \eqref{eq4:altM1}. Further, 
	since $a\, N_1 + b \, N_e$ also satisfies \eqref{eq4:asympM} as
	$\lim_{x\to+\infty} |N_1-1| = \lim_{x\to+\infty} |N_e - e^{ix\lambda}| = 0$,
	equation \eqref{eq3:M1N} follows from the uniqueness of Jost solutions. 
	Equation \eqref{eq3:N1M} is verified analogously.
\end{proof}

While perhaps not immediately apparent, the significance of the following lemma, 
Lemma \ref{lma4:rBnd}, is that it allows us to conclude 
that the reflection coefficient $r(\lambda) = b(\lambda)/a(\lambda)$ is bounded 
in $\lambda.$ That is, 
$r \in L_\lambda^\infty(\mathbb R)$, which is the final piece we need to prove 
that the ILW direct scattering map \label{sym:DM}
$\mathscr D: B_X(0, c_0) \ni u \mapsto r \in L_\lambda^\infty(\mathbb R)$ is 
well-defined.\label{sym4:dsm}

%%==============
%% Allen's Lemma 
%%==============
\begin{lma}\label{lma4:rBnd}
	For $\lambda \ne 0$, the functions $a$ and $b$ defined in Proposition \ref{prop4:SD} satisfy
	the equation 
	\begin{align}\label{eq4:rbnd}
		\big| a(\lambda) \big|^2
			= 1 + \frac{2\zeta(-\lambda) - 1}{1 - 2 \zeta(\lambda)} 
				\big| b(\lambda) \big|^2.
	\end{align}
\end{lma}
The following proof is taken from the unpublished notes of Professor Allen Wu.
% {\color{red} I thank Prof. Allen Wu from the University of Oklahoma for the following clever 
% proof of Lemma \ref{lma4:rBnd}.}
\begin{proof}
	In this proof, we use the identity 
	\begin{align*}
		\inn{G_L^+(\dotarg; \lambda)*f, \, g}
			&= \inn{f, \, G_L^+(\dotarg; \lambda)*g} \\
			&\quad+ i\alpha(\lambda) \inn{f,1}\inn{1,g}
				+ i\beta(\lambda) \inn{f, \, e^{i(\dotarg)\lambda}}
					\inn{e^{i(\dotarg)\lambda}, \, g}
	\end{align*}
	which follows from Proposition \ref{prop1:DiaConj} identity (ii) and
	the jump relation \eqref{eq4:GFjump} as 
	\begin{align*}
		\ol{G_L^+(x; \lambda)}
			= G_R^+(-x; \lambda)
			= G_L^+(-x; \lambda) - i \alpha(\lambda) - i \beta(\lambda) \, e^{-i\lambda x}.
	\end{align*}

	Using $M_1^+ = 1 + G_L^+ (\dotarg; \lambda) * u M_1$ we compute
	\begin{align*}
		\inn{M_1^+, \, uM_1^+}
			&= \inn{1 + G_L^+ (\dotarg; \lambda) * u M_1^+, \, uM_1^+} \\
			&= \inn{1, \, uM_1^+} + \inn{uM_1^+, \, G_L^+ (\dotarg; \lambda) * u M_1^+} \\
			&\quad+ i \alpha(\lambda) |\inn{uM_1^+,\,1}|^2 
				+ i \beta(\lambda) |\inn{uM_1^+,\,e^{i(\dotarg)\lambda}}|^2 \\
			&=  \inn{1, \, uM_1^+} + \inn{uM_1^+, \, M_1^+ - 1} \\
			&\quad+ i \alpha(\lambda) |\inn{uM_1^+,\,1}|^2 
				+ i \beta(\lambda) |\inn{uM_1^+,\,e^{i(\dotarg)\lambda}}|^2
	\end{align*}
	Since $u$ is real, we have $\inn{M_1^+, \, u M_1^+} = \inn{uM_1^+, \, M_1^+}$ and
	\begin{align*}
		0 = 
			\ol{\inn{uM_1^+, \, 1}} - \inn{uM_1^+, \, 1}
			+ i \alpha(\lambda) \left|\inn{uM_1^+, \, 1} \right|^2
			+ i \beta(\lambda) \left|\inn{uM_1^+, \, e^{i(\dotarg)\lambda}} \right|^2.
	\end{align*}
	Identity \ref{eq4:rbnd} then follows, as 
	$\inn{uM_1^+, \,1} = \frac{1}{i} \frac{a-1}{\alpha(\lambda)}$, 
	$\inn{uM_1^+, \, e^{i(\dotarg)\lambda}} = \frac{1}{i} \frac{b}{\beta(\lambda)}$, 
	$\alpha(\lambda) = \frac{1}{1-2\zeta(\lambda)}$, and $\beta(\lambda) = \frac{1}{1-2\zeta(-\lambda)}$.
\end{proof}

While Lemma \ref{lma4:rBnd} holds only for $\lambda \ne 0$, we can nonetheless use 
Lemma \ref{lma4:rBnd} to show that the $r$ remains bounded near $\lambda = 0$ and is 
therefore at least essentially bounded and hence in $L_\lambda^\infty$. 




\begin{thm}\label{thm4:Dwelldefined}
	Let $r(\lambda) := b(\lambda) / a(\lambda)$. The direct scattering map
	$\mathscr D$ given by 
	\begin{align*}
		\begin{array}{rcl}
			\mathscr D: B_X(0, c_0) &\to& L_\lambda^\infty(\mathbb R)  \\
			            u           &\mapsto& r
		\end{array}
	\end{align*}
	is well-defined.
\end{thm}
\begin{proof}
	Since $M_1^+$ and $N_1^+$ exist and are unique for each $u \in B_X(0, c_0)$, the 
	map $u \mapsto r$ is well-defined as a function. Moreover, it is easy to check 
	that $\frac{1-2\zeta(\lambda)}{2\zeta(-\lambda)-1}$ is both positive and uniformly
	bounded in $\lambda$ for all real $\lambda \ne 0$, which means that \eqref{eq4:rbnd} and Lemma 
	\ref{lma4:rBnd} implies both that $|a(\lambda)| \geq 1$ for all 
	$\lambda \ne 0$ and, as a consequence
	\[
		\left| \frac{b(\lambda)}{a(\lambda)} \right|^2
			= 
				\frac{1-2\zeta(\lambda)}{2\zeta(-\lambda)-1} 
				\left[
					1 - \left|\frac{1}{a(\lambda)}\right|^2 
				\right].
	\]
	Since $|a(\lambda)| \to \infty$ as $\lambda \to 0$ and a simple computation shows that
	\[
		\lim_{\lambda \to 0} \frac{1-2\zeta(\lambda)}{2\zeta(-\lambda)-1} 
			= 1,
	\]
	$r = b/a$ is at least essentially bounded near $\lambda = 0$. Moreover, 
	since
	\begin{align}
		\label{eq4:rNegLamLimit}
		\lim_{\lambda\to-\infty}\frac{1-2\zeta(\lambda)}{2\zeta(-\lambda)-1}
			= 0,
	\end{align}
	we need only prove $r(\lambda)$ stays bounded for large \textit{positive} 
	$\lambda$. Indeed, a straight forward computation shows
	\begin{align}\label{eq4:reslimits}
		\lim_{\lambda\to+\infty} |\beta(\lambda)| = 1,
		\qquad \text{and} \qquad
		\lim_{\lambda\to+\infty} \alpha(\lambda) = 0.
	\end{align}
	Recalling that
	\begin{align*}
		a(\lambda)
				&:= 1 + i \alpha(\lambda) \,
					\int_{\mathbb R} u(x) \, M_1^+(x; \lambda, u) \, \mathrm{d}x \\
			b(\lambda)
				&= i \beta(\lambda) \, 
					\int_{\mathbb R} 
						e^{-ix\lambda} \, u(x) \, M_1^+(x; \lambda, u) 
					\, \mathrm{d}x,
	\end{align*}
	we see that
	\[
		\left|\frac{b(\lambda)}{a(\lambda)}\right|
			\lesssim \nm{u\,M_1^+(\dotarg; \lambda)}_{L^1}
			\leq 
				\nm{M_1^+(\dotarg; \lambda)}_{\inn{\dotarg}L^\infty} 
				\nm{u}_{L^{1,1}}
	\]
	for $\lambda \gg 1$. Given $M_1^+ = \left(1 - T_{L, \lambda, u} \right)^{-1} 1$
	and the operator $T_{L, \lambda, u}$ is bounded uniformly in 
	$\lambda$\textemdash{}in fact, 
	$\nm{T_{L, \lambda, u}}_{\inn{\dotarg}L^\infty\toitself}
	<\frac{1}{2}$\textemdash{}we conclude by Neumann series that 
	$\nm{M_1^+(\dotarg; \lambda)}_{\inn{\dotarg}L^\infty}$ is also uniformly 
	bounded in $\lambda$. The result therefore follows.
\end{proof}

While, we do not yet have a proof that the ILW direct
scattering map is Lipschitz as a map from $B_X(0, c_0)$ into 
$L_\lambda^\infty$, we are able to prove that it is Lipschitz in 
more restrictive regimes.  


%%===================
%% Lipschitz Theorems
%%===================
\begin{thm}\label{thm4:DlipR}
	For $c_0>0$ from Proposition \ref{prop4:exist} and for all fixed 
	$k > 0$, the ILW direct scattering map 
	\[
		\mathscr D: B_X(0, c_0) \ni u \mapsto r \in
			L_\lambda^\infty\big((-\infty,-k]\cup[k,\infty)\big)
	\] 
	is Lipschitz continuous with Lipschitz constant depending on $k$. 
\end{thm}
\begin{proof}
	Let $u_1, u_2 \in B_X(0, c_0)$ be arbitrary and respectively denote by 
	$r_1 = b_1 / a_1$, $r_2 = b_2 / a_2$ the corresponding ILW direct scattering
	map $\mathscr D$ outputs. Since $|a_1(\lambda)| \geq 1$ for all $\lambda \in \mathbb R$
	by Lemma \ref{lma4:rBnd}, we find 
	\begin{align}\label{eq4:DlipPnT}
		\left| \frac{b_1}{a_1} - \frac{b_2}{a_2} \right|
			&\leq \left|\frac{b_1}{a_1} - \frac{b_2}{a_1}\right|
				+ \left|\frac{b_2}{a_1} - \frac{b_2}{a_2}\right| \\
			&= \frac{1}{|a_1|} |b_1 - b_2| 
				+ \frac{1}{|a_1|} 
					\left|\frac{b_2}{a_2}\right| 
					\left| a_1 - a_2 \right|.
					\nonumber
	\end{align}
	Proposition \ref{prop4:uMcont} implies the map $u \mapsto uM_1^+$ is Lipschitz as a map
	into $L^1_x$. As such, 
	\begin{align}\label{eq4:DlipB}
		\frac{1}{|a_1|} |b_1 - b_2| 
			&= \frac{|\beta(\lambda)|}{|a_1(\lambda)|}
				\left|
					\int_{\mathbb R} 
						e^{ix\lambda} 
						\big(
							u_1(x) \, M_1^+( x; \lambda, u_1) 
								- u_2(x) \, M_1^+( x; \lambda, u_2) 
						\big)
					\, \mathrm{d}x
				\right| \\
			&\leq 
				\frac{|\beta(\lambda)|}{|a_1(\lambda)|} 
				\nm{
					u_1\, M_1^+(\dotarg; \lambda, u_1) 
					- u_2\, M_1^+(\dotarg; \lambda, u_2)
				}_{L^1} 
				\nonumber \\
			&\lesssim 
				\frac{|\beta(\lambda)|}{|a_1(\lambda)|}
				\nm{u_1 - u_2}_X,
				\nonumber 
	\end{align}
	where the implied constant is uniform in $\lambda$. Similarly,
	\begin{align}\label{eq4:DlipA}
		|a_1 - a_2|
			& \leq 
				|\alpha(\lambda)|
				\nm{
					u_1 \, M_1^+(\dotarg; \lambda, u_1) 
					- u_2 \, M_1^+(\dotarg; \lambda, u_2)
				}_{L^1} 
				\\
			&\lesssim |\alpha(\lambda)| \nm{u_1 - u_2}_X,
				\nonumber
	\end{align}
	where the implied constant is again uniform in $\lambda$.
	Now, the proof of Theorem \ref{thm4:Dwelldefined} implies 
	that the term
	\[
		\frac{1}{|a_1|} \left| \frac{b_2}{a_2} \right| |a_1 - a_2|
	\]
	is bounded for $|\lambda| > 0$. Through direct computation,
	it is straightforward to show 
	\begin{align*}
		\lim_{\lambda \to -\infty} |\alpha(\lambda)| 
			= \lim_{\lambda \to +\infty} |\beta(\lambda)| 
			= 1, 
			\\
		\lim_{\lambda \to +\infty} |\alpha(\lambda)| 
			= \lim_{\lambda \to -\infty} |\beta(\lambda)| 
			= 0,  
	\end{align*}
	which implies $|\alpha(\lambda)|$ and $|\beta(\lambda)|$ are bounded for 
	$|\lambda| \gg 1$. Further, since $\alpha$ and $\beta$ have exactly one 
	singularity, namely $\lambda = 0$, we conclude by estimates 
	\eqref{eq4:DlipPnT} through \eqref{eq4:DlipA} that
	\begin{align}\label{eq4:DlipFinal}
		\nm{r_1 - r_2}_{L_\lambda^\infty}
			\lesssim_k \|u_1 - u_2\|_X
	\end{align}
	as $|a_1| \geq 1$ for $\lambda \ne 0$, where the implied constant depends 
	on $k$ but is otherwise independent of $\lambda$.
\end{proof}

\begin{rmk}\label{rmk4:Lip}
	The difficulty in extending Theorem \ref{thm4:DlipR} to all values of real 
	$\lambda$ is due to the possibility of the implied constant in 
	\eqref{eq4:DlipFinal} ``blowing-up'' as $k\to 0$\textemdash{}especially when 
	either $\int_{\mathbb R} u_1 \, M_1^+(x; \lambda = 0, u_1) \, \mathrm{d}x$ or 
	$\int_{\mathbb R} u_2 \, M_1^+(x; \lambda = 0, u_2) \, \mathrm{d}x$ are zero. To see 
	why this is so, note that 
	\[
		\frac{\alpha(\lambda)}{a(\lambda)}
			= \frac{\alpha(\lambda)}
				{
					1 
						+ \alpha(\lambda) 
							\int_{\mathbb R} 
								u \, M_1^+(x; \lambda) 
							\, \mathrm{d}x
				}
			=\frac{1}
				{
					\frac{1}{\alpha(\lambda)} 
						+ \int_{\mathbb R} u \, M_1^+(x; \lambda) \, \mathrm{d}x,
				}
	\]
	and
	\[
		\frac{\beta(\lambda)}{a(\lambda)}
			= \frac{\beta(\lambda)}
				{
					1 
						+ \alpha(\lambda) 
							\int_{\mathbb R} 
								u \, M_1^+(x; \lambda) 
							\, \mathrm{d}x
				}
			=\frac{1}
				{
					\frac{1}{\beta(\lambda)} 
						+ \frac{\alpha(\lambda)}{\beta(\lambda)}
							\int_{\mathbb R} 
								u \, M_1^+(x; \lambda) 
							\, \mathrm{d}x.
				}
	\]
	which means 
	\[
		\frac{\alpha(\lambda)}{a(\lambda)},~\frac{\beta(\lambda)}{a(\lambda)}
			\sim \mathcal O
				\left(
					\frac{1}
						{
							\int_{\mathbb R} 
								u \, M_1^+(x; \lambda) 
							\, \mathrm{d}x
						}
				\right)
	\]
	for $|\lambda| \ll 1$, as 
	$\lim_{\lambda\to0} \alpha(\lambda) / \beta(\lambda) = 1$.
	Thus, if either $\int_{\mathbb R} u_1 \, M_1^+(x; \lambda = 0, u_1) \, \mathrm{d}x$ or 
	$\int_{\mathbb R} u_2 \, M_1^+(x; \lambda = 0, u_2) \, \mathrm{d}x$ are zero, then the 
	approach in the proof of Theorem \ref{thm4:DlipR} fails miserably for 
	$|\lambda|$ that is not controlled below.
\end{rmk}

In light of Remark \ref{rmk4:Lip}, we obtain the following easy ``extension''
of Theorem \ref{thm4:DlipR}, which emphasizes the challenge in actually extending
Theorem \ref{thm4:DlipR} to all real values of $\lambda$. 

\begin{cor}\label{cor4:Lip}
	For every $u \in B_X(0, c_0)$ with the property that
	\[
		\int_{\mathbb R} u\,M_1^+(x; \lambda = 0, u) \, \mathrm{d}x \ne 0,
	\]
	there is a neighborhood $\mathcal N(u)$ in $B_X(0, c_0)$ about $u$ 
	for which the map $\mathscr D : \mathcal N(u) \mapsto 
	L_\lambda^\infty(\mathbb R)$ is Lipschitz continuous. 
\end{cor}
\begin{proof}
	Fix $\varepsilon > 0$ so that 
	$\left| \int_{\mathbb R} u\,M_1^+(x; \lambda = 0, u) \, \mathrm{d}x \right| > 2\varepsilon$
	Using the Lipschitz continuity of the map 
	$B_X(0, c_0) \ni w(x) \to  w(x)\, M_1^+(x; \lambda, w)
	\in L_x^1(\mathbb R)$, we may choose $\mathcal N_\varepsilon(u)$ so that 
	every $w$ in $\mathcal N(u)$ satisfies 
	\[
		\left| \int_{\mathbb R} w(x)\, M_1^+(x; \lambda=0, w)\,dx \right|
			\geq \varepsilon.
	\]
	Then, Corollary \ref{cor4:Lip} follows from the proof of Theorem 
	\ref{thm4:DlipR}, Remark \ref{rmk4:Lip}, and the Dominated Convergence
	Theorem.
\end{proof}

The following lemma, Lemma \ref{lma4:Mtexpansion}, was developed as part of an, as 
yet, unsuccessful bid to 
extend Theorem \ref{thm4:DlipR} to all real $r$. We include this lemma here in the 
hopes that it may eventually be useful in completing the proof that the ILW 
direct scattering map is Lipschitz continuous. 

\begin{lma}\label{lma4:Mtexpansion}
	For $u \in X\cap \inn{x}^{-5} L_x^\infty(\mathbb R)$, the Jost solution 
	boundary value
	$M_1^+$ has the following $\inn{x}^4L_x^\infty(\mathbb R)$
	linear approximation in $\lambda$ centered at $\lambda = 0$
	\begin{align}
		M_1^+(x; \lambda, u) 
			= M^{(0)}(x; u) + \lambda \, M^{(1)}(x; u) + o(\lambda),
	\end{align}
	where $M^{(0)}(x; u) = M_1^+(x; \lambda = 0, u)$ and
	$M^{(1)}$ is $(\partial M_1^+ / \partial \lambda)(x; 0)$
\end{lma}
\begin{proof}
	It suffices to prove that $M_1^+$ has an $\inn{x}^4L^\infty(\mathbb R)$ 
	derivative in $\lambda$ at $\lambda = 0$. To do so, we 
	define $M_h$ as the difference quotient
	\[
		M_h(\lambda; x) :=
			\frac{M_1^+(x; \lambda +h) - M_1^+(x; \lambda)}{h}.
	\]
	In order to simplify notation, throughout the rest of this proof,
	we denote $G_L^+$ by $G$, $M_1^+$ by $M$, and suppress $x$ dependency.
	That is, $G(\lambda):= G_L^+(x; \lambda)$ and $M(\lambda):=M_1^+(x; \lambda)$.
	Please note that while not explicitly indicated by the notation in this
	proof, all convolutions are with respect to the variable $x$.

	By linearity of convolution operators 
	\begin{align*}
		M_h(\lambda; x)
			&= 
				\frac{
					G(\lambda+h)* \big[u\,M(\lambda+h)\big]
					-G(\lambda) * \big[u\,M(\lambda)\big]
				}
				{h} \\[.1\baselineskip]
			&= 
				\frac{
					\big[
						G(\lambda+h) - G(\lambda)
					\big]
						* \big[u\,M(\lambda+h)\big]
			% &\qquad
					- G(\lambda) 
						* 
							\big\{
								u
								\big[
									M(\lambda+h)
									-M(\lambda)
								\big]
							\big\}
				}
				{h} \\
			&= 
				\left(
					\frac{G(\lambda+h)-G(\lambda)}{h}
				\right)
					* \big[u\,M(\lambda+h)\big]
				- G(\lambda) 
					* \big[ u \, M_h(\lambda) \big]
	\end{align*}
	Define $G_h$ to be the difference quotient 
	$G_h = \frac{1}{h}\big[G(\lambda+h) - G(\lambda)\big]$ and let 
	$T_\lambda$ denote the operator given by $T_\lambda f = G(\lambda)*(u\,f)$.
	Since, as we see in the proof of Proposition \ref{prop4:exist}, $I + T_\lambda$
	is invertible, the following formula for $M_h$ follows from the above 
	computation:
	\begin{align}\label{eq3:Mh}
		M_h(\lambda) := (I + T_\lambda)^{-1} \big(G_h * u\, M(\lambda+h)\big).
	\end{align}
	Given the continuity of $(I + T_\lambda)^{-1}$, equation \eqref{eq3:Mh}
	implies that $M_1^+$ is differentiable in $\lambda$ ($\lambda \in \mathbb R$)
	if and only if the limit 
	\[
		\lim_{h\to0} G_h * u\, M(\lambda+h)
	\]
	holds pointwise for each $x \in \mathbb R$. 

	Since a natural candidate for the limit of $G_h * u\, M(\lambda+h)$
	as $h\to 0$ is $\left(\frac{\partial}{\partial \lambda}G\right) * (uM)$,
	note that
	\begin{align}
		&G_h*\big(uM(\lambda+h)\big) 
				- \left(\frac{\partial}{\partial \lambda}G\right) 
					* \big(uM(\lambda)\big) \\
			&\qquad=
				\left(G_h - \frac{\partial}{\partial \lambda}G\right)
						* \big(uM(\lambda +h)\big)
					+ \left(\frac{\partial}{\partial \lambda}G\right)
						* u \big(M(\lambda+h) - M(\lambda)\big).
				\nonumber
	\end{align}
	
	By Technical Lemma \ref{tlma2:1}, 
	\begin{align*}
		\left\|
			\left(G_h - \frac{\partial}{\partial \lambda}G\right)
			* \big( uM(\lambda+h)\big)
		\right\|_{\inn{x}^4L_x^\infty}
			&\leq 
				\left\|
					G_h - \frac{\partial}{\partial \lambda}G
				\right\|_{\inn{x}^4 L_x^1}
				\left\|
					\inn{\dotarg}^4
					u\,M(\lambda+h)
				\right\|_{L_x^\infty}.
	\end{align*}
	Hence
	\begin{align} \label{eq3:??}
		\lim_{h\to\infty} 
				\left\|
					\left(G_h - \frac{\partial}{\partial \lambda}G\right)
					* \big( uM(\lambda+h)\big)
				\right\|_{\inn{x}^4 L_x^\infty} 
			= 0,
	\end{align}
	as 
	\begin{align*}
		\left\|
			\inn{\dotarg}^4
			u\,M(\lambda+h)
		\right\|_{L_x^\infty}
		=
			\esssup_{x\in\mathbb R} \big(\inn{x}^{-1} M\big) \big(\inn{x}^5 u\big)
		\leq \|M\|_{\inn{x}L_x^\infty} \|u\|_{\inn{x}^{-5}L_x^\infty}.
	\end{align*}

	Similarly, 
	\begin{align*}
		&
			\left\|
				\left(\frac{\partial}{\partial \lambda} G\right)
				*
				u\big(M(\lambda+h) - M(\lambda)\big)
			\right\|_{\inn{x}^4 L_x^\infty} \\
		&\qquad\leq
			\left\|
				\frac{\partial}{\partial \lambda} G
			\right\|_{\inn{x}^4 L^1}
			\left\|	
				\inn{\dotarg}^4 u 
				\big[
					M(\lambda + h) - M(\lambda)
				\big]
			\right\|_{L_x^\infty} 
			\nonumber \\
		&\qquad\leq
			\left\|
				\frac{\partial}{\partial \lambda} G
			\right\|_{\inn{x}^4 L^1}
			\|u\|_{\inn{x}^{-5} L_x^\infty}
			\|M(\lambda+h) - M(\lambda)\|_{\inn{x}L_x^\infty}
			\nonumber
	\end{align*}
	Hence, by Lemma \ref{lma4:Mlamcont}, we also find
	\begin{align}
		\lim_{h\to\infty}
			\left\|
				\left(\frac{\partial}{\partial \lambda} G\right)
				*
				u\big(M(\lambda+h) - M(\lambda)\big)
			\right\|_{\inn{x}^4 L_x^\infty} 
			=0,
	\end{align}
	from which the result follows.
\end{proof}


\end{document}