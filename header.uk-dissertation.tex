% \documentclass[final]{ukthesis}


\usepackage{amsthm, amsmath, amssymb}
% \usepackage{bm} %% Adds the \bm command for bolding certain math symbols
\usepackage{mathrsfs,  mathtools}
% \usepackage{dsfont, bbding, moresize}
\usepackage{setspace}
\usepackage{subfiles}
\usepackage{pgfplots}
\usepackage{float}
\usepackage{wrapfig}
\usepackage{subcaption}
\usepackage{longtable}


\DeclareMathAlphabet{\mathpzc}{OT1}{pzc}{m}{it}


\usepackage{tikz}
\usepackage{tkz-euclide}
\usetikzlibrary{decorations.markings}
\usetikzlibrary{arrows, calc, intersections}




\def\cs#1{\texttt{\char`\\#1}}


%%========
%% Algebra
%%========
\DeclareMathOperator{\ann}{Ann}
\DeclareMathOperator{\chr}{char}
\DeclareMathOperator{\GF}{GF}
\DeclareMathOperator{\rank}{rank}
\DeclareMathOperator{\Gal}{Gal}
\DeclareMathOperator{\ord}{ord}
\DeclareMathOperator{\Aut}{Aut}
\newcommand{\nsg}{\triangleleft}
\newcommand{\nsgeq}{\trianglelefteq}



%%================
%% Analysis Macros
%%================
\newcommand{\uc}{\rightrightarrows}
\newcommand{\grad}{\nabla}
\DeclareMathOperator{\diver}{div}
\DeclareMathOperator{\gradient}{grad}
\DeclareMathOperator{\curl}{curl}
\newcommand*\laplace{\mathop{}\!\mathbin\bigtriangleup}
\newcommand*\DAlambert{\mathop{}\!\mathbin\Box}
\DeclareMathOperator{\dist}{dist}
\DeclareMathOperator{\pv}{p.v.}
\DeclareMathOperator{\sign}{sign}
\newcommand{\widesim}[2][1.5]{
  \mathrel{\underset{#2}{\scalebox{#1}[1]{$\sim$}}}
}
% \DeclareMathOperator{\ae}{a.e.}
% \renewcommand{\liminf}{\mathop{\underline{\lim}}}
\def\Xint#1{\mathchoice
   {\XXint\displaystyle\textstyle{#1}}%
   {\XXint\textstyle\scriptstyle{#1}}%
   {\XXint\scriptstyle\scriptscriptstyle{#1}}%
   {\XXint\scriptscriptstyle\scriptscriptstyle{#1}}%
   \!\int}
\def\XXint#1#2#3{{\setbox0=\hbox{$#1{#2#3}{\int}$}
     \vcenter{\hbox{$#2#3$}}\kern-.5\wd0}}
\def\ddint{\Xint=}
\def\dint{\Xint-}



%%=======
%% Arrows
%%=======
\newcommand{\rarr}{\rightarrow}
\newcommand{\larr}{\leftarrow}
\newcommand{\darr}{\downarrow}
\newcommand{\uarr}{\uparrow}
\newcommand{\ra}{\Rightarrow}
\newcommand{\la}{\Leftarrow}





%%=====================
%% Combinatorics Macros
%%=====================
\def \bangle{ \atopwithdelims \langle \rangle}


%%========================================
%% Common Double Bar Symbols (i.e. Fields)
%%========================================
\newcommand{\bb}{\mathbb}
\newcommand{\NN}{\mathbb{N}}
\newcommand{\ZZ}{\mathbb{Z}}
\newcommand{\RR}{\mathbb{R}}
\newcommand{\QQ}{\mathbb{Q}}
\newcommand{\Rn}{{\mathbb{R}^n}}
\newcommand{\Rm}{\mathbb{R}^m}
\DeclareMathOperator{\RP}{\mathbb{R}P}
\newcommand{\CC}{\mathbb{C}}
\newcommand{\Cn}{\mathbb{C}^n}
\newcommand{\id}{\mathds{1}}



%%=====================
%% Common Greek Symbols
%%=====================
\newcommand{\al}{\alpha}
\newcommand{\be}{\beta}
\newcommand{\del}{\delta}
\newcommand{\gam}{\gamma}
\newcommand{\Ga}{\Gamma}
\newcommand{\ve}{\varepsilon}
\newcommand{\lam}{\lambda}
% \newcommand{\n}{\eta}
\newcommand{\Lam}{\Lambda}
% \renewcommand{\t}{\tau}
% \newcommand{\p}{\rho}
\newcommand{\sig}{\sigma}
\newcommand{\om}{\omega}
% \newcommand{\z}{\zeta}
\newcommand{\vphi}{\varphi}
\newcommand{\ze}{\zeta}






%%=============================
%% Common Set Theoretic Symbols
%%=============================
\newcommand{\se}{\subseteq}
\newcommand{\su}{\subset}
\newcommand{\csu}{\Subset}
\newcommand{\bd}{\partial}
\newcommand{\cl}{\overline}
% \newcommand{\bs}{\backslash}
\newcommand{\sm}{\backslash}
\newcommand{\es}{\emptyset}
\newcommand{\symdiff}{\bigtriangleup}
\DeclareMathOperator{\supp}{supp}
% \newcommand{\symdiff}{\triangle}






%%================
%% Complex Anaysis
%%================
\DeclareMathOperator{\Log}{Log}
\DeclareMathOperator{\Res}{Res}
\DeclareMathOperator{\re}{Re}
\DeclareMathOperator{\im}{Im}


%%============
%% Draft Tools
%%============
\def\pauseyn{1}
\def\draft{0}
\newcommand{\ifdraft}[1]
{
  \if\draft1
    #1
  \fi
}

\newcommand{\ifpause}
{
  \if\pauseyn1
    \pause
  \fi
}



%%============
%% Fancy Fonts
%%============
% \DeclareMathAlphabet{\mathpzc}{OT1}{pzc}{m}{it}
% \DeclareMathAlphabet{\mathcal}{OMS}{cmsy}{m}{n}
\newcommand{\mc}{\mathcal}
\newcommand{\ms}{\mathscr}



%%====================
%% Functional Analysis
%%====================
\newcommand{\inn}[1]{\left\langle #1 \right\rangle}
\newcommand{\nm}[1]{\left\| #1 \right\|}
\newcommand{\norm}[2][ ]{\left\Vert #2 \right\Vert_{#1}}
\newcommand{\nmop}[1]{\nm{ #1}_{\text{op}} }
\DeclareMathOperator{\coker}{coker}
\DeclareMathOperator{\spec}{spec}
\DeclareMathOperator{\diag}{diag}
\DeclareMathOperator{\ad}{ad}
\DeclareMathOperator{\proj}{proj}
\DeclareMathOperator{\ran}{ran}
\DeclareMathOperator{\spn}{span}




%%=========
%% Geometry
%%=========
\newcommand{\arc}{\measuredangle}
\DeclareMathOperator{\Ric}{Ric}
\DeclareMathOperator{\diam}{diam}




%%==========================
%% Hyperbolic Trig Functions
%%==========================
\DeclareMathOperator{\csch}{csch}
\DeclareMathOperator{\sech}{sech}




%%======================
%% Lebesgue Space Macros
%%======================
\newcommand{\lp}{{L^p}}
\newcommand{\lh}{{L^2}}
\newcommand{\lpu}{{L^p(U)}}
\newcommand{\lpv}{{L^p(V)}}
\newcommand{\lhu}{{L^2(U)}}
\newcommand{\lhv}{{L^2(V)}}
\newcommand{\lhr}{{L^2(\R)}}
\newcommand{\lpr}{{L^p(\R)}}
\newcommand{\lprn}{{L^p(\Rn)}}
\newcommand{\lhuinn}[1]{\inn{#1}_{\lhu}}         % L^2 inner product on U
\newcommand{\lhunm}[1]{\nm{#1}_{L^2(U)}}         % L^2 norm on U
\newcommand{\lhrinn}[1]{\inn{#1}_{L^2(\R)}}      % L^2 inner product on \R
\newcommand{\lhrnm}[1]{\nm{#1}_{L^2(\R)}}        % L^2 norm on \R
\newcommand{\suprnm}[1]{\nm{#1}_{L^\infty(\R)}}  % Supremum norm
\DeclareMathOperator*{\esssup}{ess\,sup}




%%=================================
%% Linear Algebra and Matrix Macros
%%=================================
\DeclareMathOperator{\adj}{adj}
\DeclareMathOperator{\Tr}{Tr}
\DeclareMathOperator{\tr}{tr}
\newcommand{\twovec}[2]{\begin{pmatrix} #1 \\ #2 \end{pmatrix}}
\newcommand{\twomat}[4]
{
  \begin{pmatrix}
    #1  & #2  \\
    #3  & #4
  \end{pmatrix}
}
\newcommand{\upmat}[1]
{
  \begin{pmatrix}
    1 & {#1}  \\
    0 & 1 
  \end{pmatrix}
}
\newcommand{\lowmat}[1]
{
  \begin{pmatrix}
    1 & 0   \\
    {#1}  &   1 
  \end{pmatrix}
}
\newcommand{\diagmat}[2]
{
  \begin{pmatrix}
    #1    & 0   \\
    0   & #2
  \end{pmatrix}
}
\makeatletter
\renewcommand*\env@matrix[1][\arraystretch]{%
  \edef\arraystretch{#1}%
  \hskip -\arraycolsep
  \let\@ifnextchar\new@ifnextchar
  \array{*\c@MaxMatrixCols c}}
\makeatother




%%==============
%% Number Theory
%%==============
\DeclareMathOperator{\lcm}{lcm}
\newcommand{\ceil}[1]{\left\lceil #1 \right\rceil}
\newcommand{\floor}[1]{\left\lfloor #1 \right\rfloor}




%%=====================
%% Sobolev Space Macros
%%=====================
\newcommand{\hone}{ {H^1\(\R^1\)} }
% \newcommand{\honen}{ {H^1\(\R^n\)} }
\newcommand{\honenm}[1]{\nm{#1}_{\hone}}
\newcommand{\honeinn}[1]{\inn{#1}_{\hone}}


%%=========================
%% Custom Formatting Macros
%%=========================
\newcommand{\dotarg}{\kern 0.5ex  \cdot \kern 0.5ex}
\newcommand{\noi}{\noindent}
\newcommand{\wt}[1]{\widetilde{#1}}
\newcommand{\wh}[1]{\widehat{#1}}
\newcommand{\wc}[1]{\widecheck{#1}}
\newcommand{\ch}[1]{\check{#1}}
\newcommand{\ol}{\overline}
\newcommand{\ul}{\underline}
\newcommand{\ds}{\displaystyle}
%\newcommand{\SquareShadowB}{\textifsymbol[ifgeo]{1}}
%\renewcommand{\qedsymbol}{\SquareShadowB}
% \renewcommand{\qedsymbol}{\SquareShadowBottomRight}
\renewcommand{\(}{\left(}
\renewcommand{\)}{\right)}

%%=============================================
%% Allow the use of specific symbols from mathb
%%=============================================
\DeclareFontFamily{U}{mathb}{\hyphenchar\font45}
\DeclareFontShape{U}{mathb}{m}{n}{
      <5> <6> <7> <8> <9> <10> gen * mathb
      <10.95> mathb10 <12> <14.4> <17.28> <20.74> <24.88> mathb12
      }{}
\DeclareSymbolFont{mathb}{U}{mathb}{m}{n}
\DeclareMathSymbol{\righttoleftarrow}{3}{mathb}{"FD}



%%========================
%% Project Specific Macros
%%========================
\newcommand{\earsdown}{
\begin{tikzpicture}[scale=0.1]
	\draw (-2.5, 0) -- (-1.5,0);
	\draw (-0.5, 0) -- (0.5,0);
	\draw (1.5, 0) -- (2.5,0);
	\draw (-1.5,0) arc (180:360:0.5);
	\draw (0.5,0) arc (180:360:0.5);
\end{tikzpicture}
}

\newcommand{\earsup}{
\begin{tikzpicture}[scale=0.1]
	\draw (-2.5, 0) -- (-1.5,0);
	\draw (-0.5, 0) -- (0.5,0);
	\draw (1.5, 0) -- (2.5,0);
	\draw (-1.5,0) arc (180:0:0.5);
	\draw (0.5,0) arc (180:0:0.5);
\end{tikzpicture}
}

\def\toitself{\righttoleftarrow}

%%===================
%% Margin Note Macros
%%===================
\usepackage{marginnote}
\def\marginnotetextwidth{1.5 in}
\usepackage{ulem}

\newcommand{\lsidenote}[1]{
  \reversemarginpar
  \marginnote{
    \scriptsize
    \color{red}
    \begin{OnehalfSpace}
      #1
    \end{OnehalfSpace}
  }
}


\newcommand{\rsidenote}[1]{
  \marginnote{
    \scriptsize
    \color{red}
    \begin{OnehalfSpace}
      #1
    \end{OnehalfSpace}
  }
}




%%=================
%% Wide Check Macro
%%=================
%% code from mathabx.sty and mathabx.dcl, needed to make \wc (\widecheck) command work
\DeclareFontFamily{U}{mathx}{\hyphenchar\font45}
\DeclareFontShape{U}{mathx}{m}{n}{
      <5> <6> <7> <8> <9> <10>
      <10.95> <12> <14.4> <17.28> <20.74> <24.88>
      mathx10
      }{}
\DeclareSymbolFont{mathx}{U}{mathx}{m}{n}
\DeclareFontSubstitution{U}{mathx}{m}{n}
\DeclareMathAccent{\widecheck}{0}{mathx}{"71}
\DeclareMathAccent{\wideparen}{0}{mathx}{"75}

%%=====================
%% Theorem Environments
%%=====================
\theoremstyle{plain}      \newtheorem{thm}{Theorem}[section]
% \numberwithin{thm}{section}
\theoremstyle{plain}      \newtheorem*{thm*}{Theorem}%[section]
\theoremstyle{plain}      \newtheorem{lma}[thm]{Lemma}%[section]
\theoremstyle{plain}      \newtheorem{tlma}[thm]{Technical Lemma}%[section]
\theoremstyle{plain}      \newtheorem*{lma*}{Lemma}
\theoremstyle{plain}      \newtheorem{cor}[thm]{Corollary}%[section]
\theoremstyle{plain}      \newtheorem*{cor*}{Corollary}%[section]
\theoremstyle{plain}      \newtheorem{claim}[thm]{Claim}%[section]
\theoremstyle{plain}      \newtheorem*{claim*}{Claim}
\theoremstyle{plain}      \newtheorem{prop}[thm]{Proposition}%[section]
\theoremstyle{plain}      \newtheorem*{prop*}{Proposition}%[section]
\theoremstyle{plain}      \newtheorem{observation}{Observation}
\theoremstyle{plain}      \newtheorem{property}{Property}
\theoremstyle{remark}     \newtheorem{rmk}{Remark}
\theoremstyle{remark}     \newtheorem*{rmk*}{Remark}
\theoremstyle{remark}     \newtheorem*{rcl}{Recall}
% \theoremstyle{remark}     \newtheorem*{fact}{Fact}
\theoremstyle{definition} \newtheorem{defn}[thm]{Definition}%[section]
\theoremstyle{definition} \newtheorem*{defn*}{Definition}%[section]
\theoremstyle{definition} \newtheorem{ex}{Example}[section]
\theoremstyle{definition} \newtheorem*{ex*}{Example}%[section]


\theoremstyle{plain}      \newtheorem{result}{Result}
\theoremstyle{plain}      \newtheorem{formResult}[result]{Formal Result}
\theoremstyle{plain}      \newtheorem{desResult}{Desired Result}


\newtheorem{manualtheoreminner}{Theorem}
\newenvironment{mthm}[1]{%
  \renewcommand\themanualtheoreminner{#1}%
  \manualtheoreminner
}{\endmanualtheoreminner}

\newtheorem{manualcorollaryinner}{Corollary}
\newenvironment{mcor}[1]{%
  \renewcommand\themanualcorollaryinner{#1}%
  \manualcorollaryinner
}{\endmanualcorollaryinner}

\newtheorem{manualpropositioninner}{Proposition}
\newenvironment{mprop}[1]{%
  \renewcommand\themanualpropositioninner{#1}%
  \manualpropositioninner
}{\endmanualpropositioninner}

\newtheorem{manuallemmainner}{Lemma}
\newenvironment{mlma}[1]{%
  \renewcommand\themanuallemmainner{#1}%
  \manuallemmainner
}{\endmanuallemmainner}

\theoremstyle{definition}
\newtheorem{manualdefinitioninner}{Definition}
\newenvironment{mdefn}[1]{%
  \renewcommand\themanualdefinitioninner{#1}%
  \manualdefinitioninner
}{\endmanualdefinitioninner}

\theoremstyle{definition}
\newtheorem{manualremarkinner}{Remark}
\newenvironment{mrmk}[1]{%
  \renewcommand\themanualremarkinner{#1}%
  \manualremarkinner
}{\endmanualremarkinner}


%%===============================
%% Longtable related environments
%%===============================
\newenvironment{talign}
{
  \newline
  \begin{center}
    $
      \begin{aligned}
}
{
      \end{aligned}
    $
  \end{center}
  % \vspace{-1\baselineskip}
}


\newenvironment{teqn}
{
  \newline
  \begin{center}
    $\displaystyle
}
{
    $
  \end{center}
}


\newcolumntype{R}[1]{>{\raggedleft\let\newline\\\arraybackslash\hspace{0pt}}p{#1}}
\newcolumntype{L}[1]{>{\raggedright\let\newline\\\arraybackslash\hspace{0pt}}p{#1}}
\newcolumntype{C}[1]{>{\centering\let\newline\\\arraybackslash\hspace{0pt}}p{#1}}

\newenvironment{indextable}[1]{
  \rowcolors{2}{gray!10}{white}
  \begin{longtable}
    {R{.66in}|L{4.29in}|C{.55in}}
    \rowcolor{white}
    \textbf{Symbol} & \textbf{Meaning} & \textbf{Ref.} \\
    \hline                                      
    \endfirsthead
    \multicolumn{3}{c}%
    {#1 -- \textit{Continued from previous page}} \\
    \hline
    \rowcolor{white}
    \textbf{Symbol} & \textbf{Meaning} & \textbf{Ref.} \\
    \hline
    \endhead
    \hline \multicolumn{3}{r}{\textit{Continued on next page}} \\
    \endfoot
    \hline
    \endlastfoot
}
{
  \end{longtable}
}



%%=============================
%% Additional Typesetting Tools
%%=============================

\def\namedlabel#1#2{\begingroup
    #2%
    \def\@currentlabel{#2}%
    \phantomsection\label{#1}\endgroup
}